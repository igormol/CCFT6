%% LyX 2.4.4 created this file.  For more info, see https://www.lyx.org/.
%% Do not edit unless you really know what you are doing.
\documentclass[british,nofootinbib, 11pt,a4paper]{revtex4-1}
\usepackage[T1]{fontenc}
\usepackage[latin9]{inputenc}
\setcounter{secnumdepth}{3}
\usepackage{booktabs}
\usepackage{mathrsfs}
\usepackage{mathtools}
\usepackage{amsmath}
\usepackage{amsthm}
\usepackage{amssymb}
\usepackage{esint}

\makeatletter

%%%%%%%%%%%%%%%%%%%%%%%%%%%%%% LyX specific LaTeX commands.
%% Because html converters don't know tabularnewline
\providecommand{\tabularnewline}{\\}

%%%%%%%%%%%%%%%%%%%%%%%%%%%%%% Textclass specific LaTeX commands.
\theoremstyle{plain}
\newtheorem*{conjecture*}{\protect\conjecturename}
\theoremstyle{plain}
\newtheorem{thm}{\protect\theoremname}
\theoremstyle{remark}
\newtheorem{rem}[thm]{\protect\remarkname}
\theoremstyle{remark}
\newtheorem{notation}[thm]{\protect\notationname}
\theoremstyle{definition}
\newtheorem{defn}[thm]{\protect\definitionname}
\theoremstyle{plain}
\newtheorem{lem}[thm]{\protect\lemmaname}

%%%%%%%%%%%%%%%%%%%%%%%%%%%%%% User specified LaTeX commands.
\usepackage{tikz-cd} % For elegant diagrams
\usepackage{mathtools}

\newcommand\bigw{\scalebox{1}[.875]{$\bigwedge$}}
\DeclareMathAlphabet{\mathpzc}{T1}{pzc}{b}{it}  % b for medium/bold series
\newcommand{\Cycle}{\mathpzc{C}}

\newcommand{\widesim}[2][1.5]{
  \mathrel{\overset{#2}{\scalebox{#1}[1]{$\sim$}}}
}

\makeatother

\usepackage{babel}
\providecommand{\conjecturename}{Conjecture}
\providecommand{\definitionname}{Definition}
\providecommand{\lemmaname}{Lemma}
\providecommand{\notationname}{Notation}
\providecommand{\remarkname}{Remark}
\providecommand{\theoremname}{Theorem}

\begin{document}
\title{A Celestial Description of Planar Yang-Mills Theory}
\author{Igor Mol}
\affiliation{Federal University of Juiz de Fora $-$ Departament of Mathematics}
\affiliation{Federal University of Minas Gerais $-$ Instituto de Ci�ncias Exatas
(ICEx)}
\email{igormol@ime.unicamp.br}

\begin{abstract}
Continuing our program of deriving aspects of celestial holography
from string theory, we extend the Roiban-Spradlin-Volovich-Witten
(RSVW) formalism to celestial amplitudes. We reformulate the tree-level
maximally-helicity-violating (MHV) celestial leaf amplitudes for gluons
in $\mathcal{N}=4$ supersymmetric Yang-Mills (SYM) theory and for
gravitons in $\mathcal{N}=8$ Supergravity in terms of \emph{minitwistor
wavefunctions}. These are defined as representatives of cohomology
classes on the minitwistor space $\mathbf{MT}$, associated to the
three-dimensional Euclidean anti-de Sitter space. In this framework,
celestial leaf amplitudes are expressed as integrals over the moduli
space of minitwistor lines. We construct a minitwistor generating
functional for MHV leaf amplitudes using the Quillen determinant line
bundle, extending the approach originally developed by Boels, Mason
and Skinner. Building on this formalism, we propose supersymmetric
celestial conformal field theories (CFTs) as $\sigma$-models, where
the worldsheet is given by the celestial supersphere $\mathbf{CP}^{1|2}$,
and the target space is the minitwistor superspace $\mathbf{MT}^{2|\mathcal{N}}$.
We demonstrate that the semiclassical effective action of these $\sigma$-models
reproduces the MHV gluonic and gravitational leaf amplitudes in $\mathcal{N}=4$
SYM theory and $\mathcal{N}=8$ Supergravity. This construction provides
a concrete realisation of the supersymmetric celestial CFT framework
recently introduced by Tropper (2024). \citet{adamo2013moduli}
\end{abstract}
\maketitle
%
\tableofcontents{}

\newpage{}

\section{Introduction}

An open problem in high-energy theoretical physics is the dynamical
derivation of the celestial holography dictionary, which posits that
the physics of asymptotically flat spacetimes can be encoded in a
conjectured celestial conformal field theory (CCFT) living on the
celestial sphere.

The simplicity of gauge-theory scattering amplitudes for gluons in
maximally-helicity-violating (MHV) configurations, described by the
Parke-Taylor formula, led Nair to reinterpret these amplitudes as
correlation functions of a Wess-Zumino-Novikov-Witten (WZNW) current
algebra on the Riemann sphere, $\mathbf{CP}^{1}$. This viewpoint
motivated the development of twistor and ambitwistor string theories
and, later, the Cachazo-He-Yuan (CHY) formalism. Here, we ask whether
a similar, bottom-up approach can yield concrete dynamical models
for CCFTs. Such models may describe realistic physical theories or,
at least, specific sectors thereof.

\section{Minitwistor Superwavefunctions}

\subsection{Review: Twistor Superwavefunctions}

The $\mathcal{N}=4$ supersymmetric twistor wavefunction for gluons,
labelled by the scaling weight $w$, defines a Dolbeault cohomology
class:
\[
[f^{w}]\,\in\,H^{0,1}\big(\mathbf{PT}^{3|4},\mathcal{O}\left(-w\right)\big).
\]
Introduce homogeneous coordinates:
\[
\mathcal{Z}^{I}\,\coloneqq\,\big(\lambda^{A},\mu_{\dot{A}},\psi^{\alpha}\big)
\]
on projective twistor superspace $\mathbf{PT}^{3|4}\subset\mathbf{CP}^{3|4}$.
The dual (momentum-twistor) coordinates are:
\[
\mathcal{W}^{I}\,\coloneqq\,\big(\nu^{A},\bar{\nu}_{\dot{A}},\eta^{\alpha}\big).
\]
We employ abstract index notation with $I,I',...\in\{A,\dot{A},\alpha\}$. 

The twistor superwavefunction is explicitly given by:
\begin{equation}
f^{w}\big(\mathcal{Z}^{I};\mathcal{W}^{I'}\big)\,\coloneqq\,\underset{\mathbf{C}^{*}\,\,\,}{\int}\,\frac{dt}{t}\,t^{w}\,\,\,\overline{\delta}^{2}\big(\nu^{A}-t\lambda^{A}\big)\exp\big(it\,([\mu\bar{\nu}]+\psi\cdot\eta)\big)\,.\label{eq:-27}
\end{equation}
Here the spinorial delta function is defined by:
\begin{equation}
\overline{\delta}^{2}\left(\sigma^{A}\right)\,\coloneqq\,\frac{1}{\left(2\pi i\right)^{2}}\,\bigwedge_{A\in\{1,2\}}\,\overline{\partial}\,\,\,\frac{1}{\sigma^{A}}\,\,\,,
\end{equation}
where $[\sigma^{A}]$ denote homogeneous coordinates on the projective
line $\mathbf{CP}^{1}$.

To obtain a more explicit representation, introduce an auxiliary spinor
$\iota^{A}$, chosen arbitrarily but non-vanishing. Using the Green's
function for the Dolbeault operator $\overline{\partial}$ on $\mathbf{CP}^{1}$,
one shows:
\begin{equation}
\overline{\delta}^{2}\left(z^{A}-t\lambda^{A}\right)\,=\,\frac{1}{\left(2\pi i\right)^{2}}\,\bigwedge_{A\in\{1,2\}}\,\overline{\partial}\,\,\,\frac{1}{z^{A}-t\lambda^{A}}\,=\,\overline{\delta}\left(t-\frac{\left\langle z\iota\right\rangle }{\left\langle \lambda\iota\right\rangle }\right)\,\overline{\delta}\left(\left\langle \lambda z\right\rangle \right).
\end{equation}
Substituting this into Eq. (\ref{eq:-27}) gives the explicit gluonic
superwavefunction:
\begin{equation}
f^{p}\big(\mathcal{Z}^{I};\mathcal{W}^{I'}\big)=\overline{\delta}\left(\left\langle \lambda z\right\rangle \right)\left(\frac{\left\langle z\iota\right\rangle }{\left\langle \lambda\iota\right\rangle }\right)^{p-1}\,\exp\left(i\frac{\left\langle z\iota\right\rangle }{\left\langle \lambda\iota\right\rangle }\,\left(\,s[\mu\bar{z}]+\psi\cdot\eta\,\right)\right).\label{eq:-28}
\end{equation}

Next, define the \emph{projective delta function} $\overline{\delta}_{\Delta}$
on $\mathbf{CP}^{1}$ with conformal weight $\Delta$:
\begin{equation}
\overline{\delta}_{\Delta}\big(z^{A},\lambda^{A}\big)\,\coloneqq\,\frac{1}{\left(2\pi i\right)^{2}}\,\underset{\mathbf{C}^{*}}{\int}\,\frac{dt}{t}\,\,\,t^{\Delta}\,\bigwedge_{A\in\{1,2\}}\,\overline{\partial}\,\,\,\frac{1}{z^{A}-t\lambda^{A}}\,=\,\overline{\delta}_{\Delta}\left(\left\langle \lambda z\right\rangle \right)\left(\frac{\left\langle z\iota\right\rangle }{\left\langle \lambda\iota\right\rangle }\right)^{\Delta-1}.\label{eq:-35}
\end{equation}
Using this identity, Eq. (\ref{eq:-28}) simplifies to:
\begin{equation}
f^{p}\big(\mathcal{Z}^{I};\mathcal{W}^{I'}\big)\,=\,\overline{\delta}_{p}\left(z,\lambda\right)\,\exp\left(i\frac{\left\langle z\iota\right\rangle }{\left\langle \lambda\iota\right\rangle }\,\left(s[\mu\bar{z}]+\psi\cdot\eta\right)\right).\label{eq:-154}
\end{equation}


\subsection{Minitwistor Superwavefunctions}

To construct the celestial superwavefunction for gluons in the minitwistor-superspace
formalism, we introduce a normalised spinor basis:
\begin{equation}
z^{A}\,\coloneqq\,\left(1,-\zeta\right),\,\,\,\bar{z}_{\dot{A}}\,\coloneqq\,(1,-\bar{\zeta}),
\end{equation}
where $\zeta$ and $\bar{\zeta}$ are holomorphic and antiholomorphic
coordinates on the celestial sphere $\mathcal{CS}\simeq\mathbf{CP}^{1}$.

In the gluonic superwavefunction (Eq. (\ref{eq:-154})), we make the
substitutions:
\[
\nu^{A}\,\mapsto\,z^{A},\,\,\,\bar{\nu}_{\dot{A}}\,\mapsto\,s\cdot\bar{z}_{\dot{A}},
\]
where $s$ is a real, nonnegative parameter encoding the gluon's frequency.

From the viewpoint of celestial CFT, a gluon state is entirely specified
by three data:
\begin{enumerate}
\item The celestial conformal weight $\Delta$.
\item The helicity, encoded by the Grassmann variables $\eta^{\alpha}$,
with $\alpha=1,...,4$ labelling the supersymmetry generators.
\item The insertion point on the celestial sphere $\mathcal{CS}$, given
by the normalised spinor basis $\{z^{A},\bar{z}_{\dot{A}}\}$.
\end{enumerate}
%
The \emph{minitwistor superwavefunction} associated with a gluon of
conformal weight $\Delta$ and configuration $\{z^{A},\bar{z}_{\dot{A}},\eta^{\alpha}\}$
is defined by the half-Mellin transform:
\begin{equation}
\Psi_{\Delta}^{p}\big(\lambda^{A},\mu_{\dot{A}},\psi^{\alpha};z^{A},\bar{z}_{A},\eta^{\alpha}\big)\,\coloneqq\,\underset{\mathbf{R}_{+}}{\int}\,\frac{ds}{s}\,\,\,s^{\Delta}\,f^{p}\big(\lambda^{A},\mu_{\dot{A}},\eta^{\alpha};z^{A},s\bar{z}_{\dot{A}},\eta^{\alpha}\big).
\end{equation}
Substituting the definition of $f^{p}$ from Eq. (\ref{eq:-27}) yields
the double-integral representation:
\begin{equation}
\Psi_{\Delta}^{p}\big(\lambda^{A},\mu_{\dot{A}},\psi^{\alpha};z^{A},\bar{z}_{\dot{A}},\eta^{\alpha}\big)\,=\,\underset{\mathbf{R}_{+}}{\int}\,\frac{ds}{s}\,s^{\Delta}\,\underset{\mathbf{C}^{*}\,\,\,}{\int}\,\frac{dt}{t}\,t^{p}\,\,\,\overline{\delta}^{2}\big(z^{A}-t\lambda^{A}\big)\,\exp\big(it\,(s[\mu\bar{z}]+\psi\cdot\eta)\big)\,\label{eq:}
\end{equation}


\paragraph*{Homogeneity.}

The affine and Mellin integrals make explicit the homogeneous transformation
properties of the superwavefunction $\Psi_{\Delta}^{p}$. For any
$t_{1},t_{2}\in\mathbf{C}^{*}$, one has:
\begin{equation}
\Psi_{\Delta}^{p}\big(t_{1}\cdot\lambda^{A},t_{2}\cdot\mu_{\dot{A}},t_{1}\cdot\psi^{\alpha};z^{A},\bar{z}_{\dot{A}},\eta^{\alpha}\big)\,=\,t_{1}^{\Delta-p}\,t_{2}^{-\Delta}\,\Psi_{\Delta}^{p}\big(\lambda^{A},\mu_{\dot{A}},\psi^{\alpha};z^{a},\bar{z}_{\dot{A}},\eta^{\alpha}\big),\label{eq:-29}
\end{equation}
\begin{equation}
\Psi_{\Delta}^{p}\big(\lambda^{A},\mu_{\dot{A}},\psi^{\alpha};t_{1}\cdot z^{A},t_{2}\cdot\bar{z}_{\dot{A}},t_{1}^{-1}\cdot\eta^{\alpha}\big)\,=\,t_{1}^{p-\Delta-2}\,t_{2}^{-\Delta}\,\Psi_{\Delta}^{p}\big(\lambda^{A},\mu_{\dot{A}},\psi^{\alpha};z^{A},\bar{z}_{\dot{A}},\eta^{\alpha}\big).\label{eq:-30}
\end{equation}
These homogeneity laws suggest that $z^{A},\bar{z}_{\dot{A}},\eta^{\alpha}$
may be regarded as dual to $\lambda^{A},\mu_{\dot{A}},\psi^{\alpha}$.
This interpretation will be justified when we introduce the minitwistor
transform.

In Eq. (\ref{eq:-29}), the Grassmann-odd coordinate $\psi^{\alpha}$
scales like $\lambda^{A}$. Hence $\lambda^{A},\mu_{\dot{A}},\psi^{\alpha}$
transform as:
\begin{equation}
\lambda^{A}\,\mapsto\,t_{1}\cdot\lambda^{A},\,\,\,\mu_{\dot{A}}\,\mapsto\,t_{2}\cdot\mu_{\dot{A}},\,\,\,\psi^{\alpha}\,\mapsto\,t_{1}\cdot\psi^{\alpha}.\label{eq:-1}
\end{equation}
By contrast, Eq. (\ref{eq:-30}) shows that $\eta^{\alpha}$ transforms
inversely to $z^{A}$. Thus $z^{A},\bar{z}_{\dot{A}},\eta^{\alpha}$
obey:
\begin{equation}
z^{A}\,\mapsto\,t_{1}\cdot z^{A},\,\,\,\bar{z}_{\dot{A}}\,\mapsto\,t_{2}\cdot\bar{z}_{\dot{A}},\,\,\,\eta^{\alpha}\,\mapsto\,t_{1}^{-1}\cdot\eta^{\alpha}.\label{eq:-31}
\end{equation}


\paragraph*{Explicit Form of $\Psi_{\Delta}^{p}$.}

Performing the affine and Mellin integrals in Eq. (\ref{eq:}) yields:
\begin{equation}
\Psi_{\Delta}^{p}\big(\mathsf{Z}^{I};\mathsf{W}^{I'}\big)\,=\,\overline{\delta}\left(\left\langle \lambda z\right\rangle \right)\left(\frac{\left\langle z\iota\right\rangle }{\left\langle \lambda\iota\right\rangle }\right)^{p-\Delta-1}\frac{\mathcal{C}\left(\Delta\right)}{[\mu\bar{z}]^{\Delta}}\exp\left(i\frac{\left\langle z\iota\right\rangle }{\left\langle \lambda\iota\right\rangle }\psi\cdot\eta\right).
\end{equation}
Alternatively, one may invoke the projective delta function $\overline{\delta}_{\Delta}$
on $\mathbf{CP}^{1}$ (Eq. (\ref{eq:-35})). In this notation, the
superwavefunction simplifies to:
\begin{equation}
\Psi_{\Delta}^{p}\big(\mathsf{Z}^{I};\mathsf{W}^{I'}\big)\,=\,\overline{\delta}_{p-\Delta}\left(z,\lambda\right)\frac{\mathcal{C}\left(\Delta\right)}{[\mu\bar{z}]^{\Delta}}\exp\left(i\frac{\left\langle z\iota\right\rangle }{\left\langle \lambda\iota\right\rangle }\psi\cdot\eta\right).\label{eq:-39}
\end{equation}
This form will prove useful when deriving the bulk-to-boundary propagator
on the $(3|8)$-dimensional Euclidean anti-de Sitter space. This derivation
proceeds via the Penrose integral-geometric transform of $\Psi_{\Delta}^{p}$.

To establish completeness and orthogonality of the family $\{\Psi_{\Delta}^{p}\}$,
we introduce the conjugate wavefunction:
\begin{equation}
\widetilde{\Psi}_{\Delta}^{p}\big(\mathsf{Z}^{I};\mathsf{W}^{I'}\big)\,\coloneqq\,\overline{\delta}_{p-\Delta}\left(z,\lambda\right)\frac{\mathcal{C}\left(\Delta\right)}{[\bar{z}\mu]^{\Delta}}\exp\left(-i\frac{\left\langle z\iota\right\rangle }{\left\langle \lambda\iota\right\rangle }\psi\cdot\eta\right).\label{eq:-41}
\end{equation}
Under conjugation, the phase in the exponential changes sign. The
denominator ordering also changes from $[\mu\bar{z}]$ to $[\bar{z}\mu]$.
The latter condition simplifies the proportionality factors in the
resulting completeness and orthogonality relations.

\subsection{Homogeneous Bundles on Minitwistor Superspace}

In the preceding subsection, we derived an explicit formula for the
minitwistor superwavefunction $\Psi_{\Delta}^{p}(\mathsf{W}^{I};\mathsf{Z}^{I'})$
by applying the Mellin transform to the corresponding $\mathcal{N}=4$
supersymmetric twistor wavefunction. We now interpret $\Psi_{\Delta}^{p}$
geometrically. It defines a section (more precisely a $\left(0,1\right)$-current)
on the minitwistor superspace $\mathbf{MT}_{s}$ (see Subsection \ref{subsec:Minitwistor-Superspace}).

There is a dual minitwistor superspace $\mathbf{MT}_{s}^{*}$, which
is canonically isomorphic to the holomorphic celestial supersphere
$\mathcal{CS}\simeq\mathbf{CP}^{1|4}$ (Subsection \ref{subsec:Dual-Minitwistor-Superspace}).
Equivalently, $\mathbf{MT}_{s}^{*}$ serves as a parameter space for
$\mathcal{CS}$. The minitwistor transform then carries sections of
holomorphic bundles over $\mathbf{MT}_{s}$ to sections over $\mathbf{MT}_{s}^{*}$.

This transform is our prescription for converting sectional (or leaf)
amplitudes into minitwistor amplitudes. It thereby provides a geometric
reinterpretation of celestial amplitudes as semiclassical expectation
values of Wilson line operators, or alternatively as correlation functions
in the minitwistor sigma-model. In this sense, the minitwistor superspace
geometry offers a dual description of the flat-space hologram on the
celestial supersphere, much as the Fourier transform relates position
and momentum representations in elementary quantum mechanics.


\subsubsection{Minitwistor Superspace\label{subsec:Minitwistor-Superspace}}

Let $\mathbf{MT}\subset\bigtimes^{2}\mathbf{CP}^{1}$ denote the bosonic
minitwistor space of three-dimensional Euclidean anti-de Sitter space.
We begin by defining the ordinary $(2|4)$-dimensional minitwistor
superspace $\mathbf{MT}_{s}$. This supermanifold is obtained by enforcing
the transformation laws of Eq. (\ref{eq:-1}) as a symmetry group
on its underlying (projective) geometry.

\textcompwordmark{}

\paragraph*{Definition of $\mathbf{MT}_{s}$.}

Introduce the vector superspace $\mathbf{V}\simeq\mathbf{C}^{4|4}$
with Cartesian coordinates $\mathsf{W}^{I}\,\coloneqq\,\big(\lambda^{A},\mu_{\dot{A}},\psi^{\alpha}\big)$.
On $\mathbf{U}\coloneqq\mathbf{V}-\{0\}$, impose the equivalence
relation:
\begin{equation}
\big(\lambda^{A},\mu_{\dot{A}},\psi^{\alpha}\big)\,\sim\,\big(t_{1}\cdot\lambda^{A},t_{2}\cdot\mu_{\dot{A}},t_{1}\cdot\psi^{\alpha}\big),\quad\forall\,t_{1},t_{2}\in\mathbf{C}^{*}.
\end{equation}
Define the $\mathcal{N}=4$ \emph{minitwistor superspace }$\mathbf{MT}_{s}$
as the set of equivalence classes $\mathsf{w}\coloneqq[\mathsf{W}^{I}]\in\mathbf{U}/\sim$
subject to the non-degeneracy condition:
\begin{equation}
[\lambda^{\flat}\mu]\neq0,\quad\big(\lambda^{\flat}\big)_{\dot{A}}\coloneqq\big(\lambda^{A}\big)^{*};
\end{equation}
and equip it with the quotient topology. This condition ensures nontrivial
Dolbeault cohomology on $\mathbf{MT}_{s}$, so that minitwistor superwavefunctions
are realised as its cohomology representatives.

\textcompwordmark{}

\paragraph*{Quotient Map.}

Denote by 
\begin{equation}
\pi_{0}:\mathbf{M}\longrightarrow\mathbf{MT}_{s},\quad\mathsf{W}^{I}\mapsto\mathsf{w}
\end{equation}
the natural quotient map, where $\mathsf{w}=[\mathsf{W}^{I}]$. Here
$\mathbf{M}\subset\mathbf{U}$ is the maximal open submanifold on
which $\pi_{0}$ is surjective. Its boundary $\partial_{\mathbf{U}}\mathbf{M}$
consists of those equivalence classes $[(\lambda^{A},\mu_{\dot{A}},\psi^{\alpha})]$
for which $[\lambda^{\flat}\mu]=0$. 

Elements of $\mathbf{MT}_{s}$ will henceforth be called\emph{ $\mathbf{Z}_{2}$-graded
minitwistors} and denoted by $\mathsf{w},\mathsf{w}'$, etc. For any
$\mathsf{w}\in\mathbf{MT}_{s}$, we refer to a choice of preimage
$\mathsf{W}^{I}\in\pi_{0}^{-1}(\mathsf{w})$ as a \emph{coordinate
representative.}

\textcompwordmark{}

\paragraph*{Orientation. }

To specify the orientation of $\mathbf{MT}_{s}$, first introduce
the holomorphic and antiholomorphic forms:
\begin{equation}
D\lambda\,\coloneqq\,\varepsilon_{AB}\lambda^{A}d\lambda^{B},\,\,\,D\mu\,\coloneqq\,\varepsilon^{\dot{A}\dot{B}}\mu_{\dot{A}}d\mu_{\dot{B}},
\end{equation}
and let $d^{0|4}\psi$ denote the Berezin measure on $\mathbf{C}^{0|4}$.
The natural volume form on $\mathbf{MT}_{s}$ is the $\mathbf{Z}_{2}$-graded
measure:
\begin{equation}
D^{2|4}\mathsf{W}\,\coloneqq\,D\lambda\wedge D\mu\wedge d^{0|4}\psi.
\end{equation}
Under the rescaling of Eq. (\ref{eq:-1}), one finds:
\begin{equation}
D^{2|4}\mathsf{W}\,\mapsto\,t_{1}^{-2}t_{2}^{2}\,D^{2|4}\mathsf{W}.
\end{equation}


\subsubsection{Dual Minitwistor Superspace\label{subsec:Dual-Minitwistor-Superspace}}

The second set of transformation rules, stated in Eq. (\ref{eq:-31}),
motivates the introduction of the \emph{dual minitwistor superspace}
$\mathbf{MT}_{s}^{*}$. We will show in our discussion of scattering
amplitudes that this dual space is intimately connected to the holomorphic
celestial supersphere. Indeed, each point of the celestial supersphere
corresponds to a point in the dual minitwistor superspace.

The minitwistor transform $\mathcal{MT}$ then carries sections over
the dual superspace $\mathbf{MT}_{s}^{*}$ to sections over the original
minitwistor supermanifold $\mathbf{MT}_{s}$. This operation is \emph{analogous
}to the ordinary Fourier transform in quantum mechanics, which maps
wavefunctions from the position representation to the momentum representation.
In our framework, \emph{the minitwistor superspace geometry provides
a dual description of the flat-space hologram.}

\textcompwordmark{}

\paragraph*{Definition of $\mathbf{MT}_{s}^{*}$.}

We begin by defining the dual vector superspace $\mathbf{V}^{*}\simeq\big(\mathbf{C}^{4|4}\big)^{*}$
with dual coordinates $\mathsf{Z}^{I}=(z^{A},\bar{z}_{\dot{A}},\eta^{\alpha})$.
Here $z^{A}$ and $\bar{z}_{\dot{A}}$ will serve as the van der Waerden
spinors parametrising the holomorphic celestial sphere.

Let $\mathbf{U}^{*}\coloneqq\mathbf{V}^{*}-\{0\}$. On $\mathbf{U}^{*}$
impose the equivalence relation:
\begin{equation}
\big(z^{A},\bar{z}_{\dot{A}},\eta^{\alpha}\big)\,\simeq\,\big(t_{1}\cdot z^{A},t_{2}\cdot\bar{z}_{\dot{A}},t_{1}^{-1}\cdot\eta^{\alpha}\big),\quad\forall\,t_{1},t_{2}\in\mathbf{C}^{*}.
\end{equation}
The \emph{dual minitwistor superspace} $\mathbf{MT}_{s}^{*}$ is then
the set of equivalence classes $\mathsf{z}\coloneqq[\mathsf{Z}^{I}]\in\mathbf{U}^{*}\boldsymbol{\big/}\simeq$
subject to the non-degeneracy condition
\begin{equation}
[z^{\flat}\bar{z}]\;\neq\;0,\quad\big(z^{\flat}\big)_{\dot{A}}\;\coloneqq\;\big(z^{A}\big)^{*};
\end{equation}
and endowed with the quotient topology.

\textcompwordmark{}

\paragraph*{Quotient map.}

In what follows, we denote by
\begin{equation}
\pi_{0}^{*}\colon\mathbf{M}^{*}\boldsymbol{\longrightarrow}\mathbf{MT}_{s}^{*},\quad\mathsf{Z}^{I}\mapsto\mathsf{z}
\end{equation}
the quotient map, where $\mathsf{z}=[\mathsf{Z}^{I}]$. Here $\mathbf{M}^{*}\subset\mathbf{U}^{*}$
is the maximal open submanifold on which $\pi_{0}^{*}$ is surjective.
Its boundary $\partial_{\mathbf{U}^{*}}\mathbf{M}^{*}$ consists of
those equivalence classes $[(z^{A},\bar{z}_{\dot{A}},\eta^{\alpha})]$
for which the non-degeneracy condition is violated, i.e. $[z^{\flat}\bar{z}]=0$.

Points of $\mathbf{MT}_{s}^{*}$ are denoted by $\mathsf{z},\mathsf{z}',\dots$
and are called \emph{dual $\mathbf{Z}_{2}$-graded minitwistors}.
Any lift $\mathsf{Z}^{I}\in(\pi_{0}^{*})^{-1}(\mathsf{z})$ is referred
to as a \emph{coordinate representative} of the dual minitwistor point
$\mathsf{z}$.

\textcompwordmark{}

\paragraph*{Volume superform.}

To fix the orientation of $\mathbf{MT}_{s}^{*}$, first introduce
the standard holomorphic and anti-holomorphic measures on $\mathbf{CP}^{1}$:
\begin{equation}
Dz\,\coloneqq\,\varepsilon_{AB}z^{A}dz^{B},\,\,\,D\bar{z}\,\coloneqq\,\varepsilon^{\dot{A}\dot{B}}\bar{z}_{\dot{A}}d\bar{z}_{\dot{B}}.
\end{equation}
Let $d^{0|4}\eta$ denote the Berezin measure on $\mathbf{C}^{0|4}$.
The resulting $\mathbf{Z}_{2}$-graded volume form is:
\begin{equation}
D^{2|4}\mathsf{Z}\,\coloneqq\,Dz\wedge D\bar{z}\wedge d^{0|4}\eta.
\end{equation}
Under the scaling of Eq. (\ref{eq:-31}), one finds:
\begin{equation}
D^{2|4}\mathsf{Z}\,\mapsto\,t_{1}^{6}t_{2}^{2}\,D^{2|4}\mathsf{Z}.\label{eq:-38}
\end{equation}


\subsubsection{$\mathcal{O}_{A}(p,q)$-bundle}

We now introduce a two-parameter family of holomorphic vector bundles
whose sections model the physical fields and the minitwistor superwavefunctions.

\textcompwordmark{}

\paragraph*{Definition of $\mathcal{O}_{A}(p,q)$.}

Fix $p,q\in\mathbf{Z}$ and a normed algebra $A$ selected according
to the desired background field or superwavefunction. Define the auxiliary
trivial bundle $\mathbf{E}\coloneqq\mathbf{M}\times A$.

On $\mathbf{E}$, impose the following equivalence relation:
\begin{equation}
\big(\lambda^{A},\mu_{\dot{A}},\psi^{\alpha},|a\rangle\big)\,\equiv_{p,q}\,\big(t_{1}\cdot\lambda^{A},t_{2}\cdot\mu_{\dot{A}},t_{1}\cdot\psi^{\alpha},t_{1}^{p}t_{2}^{q}\cdot|a\rangle\big),\quad\forall\,t_{1},t_{2}\in\mathbf{C}_{*}.
\end{equation}
The total space of our bundle is then the quotient:
\begin{equation}
\mathcal{O}_{A}(p,q)\,\coloneqq\,\mathbf{E}\boldsymbol{\big/}\equiv_{p,q},
\end{equation}
with the natural projection $Q:\mathbf{E}\longrightarrow\mathcal{O}_{A}(p,q)$. 

\textcompwordmark{}

\paragraph*{Fibration.}

Let $\text{pr}_{\mathbf{M}}:\mathbf{E}\longrightarrow\mathbf{M}$
be the projection onto the first factor and define the surjection
\begin{equation}
Q_{0}:\mathbf{E}\longrightarrow\mathbf{MT}_{s},\quad Q_{0}\coloneqq\pi_{0}\circ\text{pr}_{\mathbf{M}}.
\end{equation}
By construction, if $\mathsf{w}\in\mathbf{MT}_{s}$ is a $\mathbf{Z}_{2}$-graded
minitwistor point and $\mathsf{W}^{I}\in\pi_{0}^{-1}(\mathsf{w})$
any coordinate representative, then $Q_{0}\big(\mathsf{W}^{I},|a\rangle\big)=\mathsf{w}$
for all $|a\rangle\in A$.

Next, consider the quotient manifold $\mathcal{O}_{A}(p,q)$. We endow
it with the structure of a holomorphic vector bundle over $\mathbf{MT}_{s}$
by introducing the projection 
\begin{equation}
\pi:\mathcal{O}_{A}(p,q)\longrightarrow\mathbf{MT}_{s}.
\end{equation}
This bundle map is uniquely determined by the condition that the quotient
map $Q_{0}$ lifts to $Q$. Equivalently, the following diagram commutes:\begin{equation} \label{eq:comm-diagram-modified} \begin{tikzcd}[   row sep=2cm,   column sep=2cm,   arrows={shorten <=4pt, shorten >=4pt},   every label/.append style={font=} ]   & \mathbf{E}     \arrow[dl, "Q"']     \arrow[dr, "Q_0"]   & \\   \mathcal{O}_A(p,q)     \arrow[rr, "{\pi}"{swap,yshift=-5pt}]   & & \mathbf{MT} \end{tikzcd} \end{equation} Hence
one has the relation
\begin{equation}
\pi\circ Q=Q_{0}.
\end{equation}


\paragraph*{Module of Sections.}

We next characterise the sections of $\mathcal{O}_{A}(p,q)\stackrel{\pi}{\to}\mathbf{MT}_{s}$.
A map
\begin{equation}
a:\mathbf{M}\longrightarrow A,\quad\mathsf{W}^{I}\mapsto|a(\mathsf{W}^{I})\rangle
\end{equation}
is called an \emph{$A$-valued homogeneous function of bi-degree $(p,q)$}
if the following holds. Write $\mathsf{W}^{I}=(\lambda^{A},\mu_{\dot{A}},\psi^{\alpha})$
and, for any $t_{1},t_{2}\in\mathbf{C}_{*}$, define the rescaled
coordinates
\begin{equation}
\mathsf{W}'{}^{I}\coloneqq\big(t_{1}\cdot\lambda^{A},t_{2}\cdot\mu_{\dot{A}},t_{1}\cdot\psi^{\alpha}\big).
\end{equation}
Then homogeneity demands
\begin{equation}
|a(\mathsf{W}'{}^{I})\rangle\,=\,t_{1}^{p}t_{2}^{q}\cdot|a(\mathsf{W}^{I})\rangle,\quad t_{1},t_{2}\in\mathbf{C}_{*}.
\end{equation}
Since $A$ is a normed algebra, one may equip the space of $A$-valued
functions on $\mathbf{MT}_{s}$ with the corresponding Fr�chet topology.
We then define the complex vector space
\begin{equation}
\mathcal{S}_{A}(p,q)\,\coloneqq\,\left\{ a:\mathbf{MT}_{s}\longrightarrow A\big|a\text{ is smooth and homogeneous of bi-degree }(p,q)\right\} .
\end{equation}

The space $\mathcal{S}_{A}(p,q)$ admits the natural structure of
a module over the ring $\mathscr{C}^{\infty}(\mathbf{MT}_{s})$ of
complex-valued smooth functions on the minitwistor superspace. If
$a\in\mathcal{S}_{A}(p,q)$ and $\varphi\in\mathscr{C}^{\infty}(\mathbf{MT}_{s})$,
we set:
\begin{equation}
(\varphi\cdot a)(\mathsf{W}^{I})\,\coloneqq\,\varphi\big(\pi_{0}(\mathsf{W}^{I})\big)\,|a(\mathsf{W}^{I})\rangle.
\end{equation}


\paragraph*{Main Result.}

Our principal claim is that the module of smooth sections of $\mathcal{O}_{A}(p,q)\stackrel{\pi}{\to}\mathbf{MT}_{s}$
coincides with $\mathcal{S}_{A}(p,q)$,
\begin{equation}
\Gamma\big(\mathbf{MT}_{s};\mathcal{O}_{A}(p,q)\big)\,\boldsymbol{\simeq}\,\mathcal{S}_{A}(p,q).
\end{equation}
To prove the section/function correspondence, let $s\in\Gamma(\mathbf{MT}_{s};\mathcal{O}_{A}(p,q))$
be a smooth section. For any $\mathsf{W}^{I}\in\mathbf{M}$ projecting
to $\mathsf{w}\in\mathbf{MT}_{s}$, we have $\pi\circ s\circ\pi_{0}(\mathsf{W}^{I})=\mathsf{w}$.
Since $Q\colon\mathbf{E}\longrightarrow\mathcal{O}_{A}(p,q)$ is the
quotient map, there exists a (unique) element $|a(\mathsf{W}^{I})\rangle\in A$
such that
\begin{equation}
Q\big(\mathsf{W}^{I},|a(\mathsf{W}^{I})\rangle\big)=s\circ\pi_{0}(\mathsf{W}^{I}).
\end{equation}
Uniqueness follows because if also $\tilde{a}:\mathsf{W}^{I}\mapsto|\tilde{a}(\mathsf{W}^{I})\rangle$
satisfies $Q(\mathsf{W}^{I},\tilde{a})=s$, then $(\mathsf{W}^{I},a)\equiv_{p,q}(\mathsf{W}^{I},\tilde{a})$,
which forces $a=\tilde{a}$. Thus we obtain a well-defined map
\begin{equation}
a\colon\mathbf{M}\longrightarrow A,\quad\mathsf{W}^{I}\mapsto|a(\mathsf{W}^{I})\rangle.
\end{equation}
Smoothness of $a$ follows from that of $s$ together with the local
triviality of the quotient. It remains to verify homogeneity. Write
$\mathsf{W}^{I}=(\lambda^{A},\mu_{\dot{A}},\psi^{\alpha})$ and for
$t_{1},t_{2}\in\mathbf{C}_{*}$ set $\mathsf{W}'{}^{I}=(t_{1}\lambda^{A},t_{2}\mu_{\dot{A}},t_{1}\psi^{\alpha})$.
Since $\pi_{0}(\mathsf{W}'{}^{I})=\pi_{0}(\mathsf{W}^{I})$, the section
takes the same value, $s(\pi_{0}(\mathsf{W}'{}^{I}))=s(\pi_{0}(\mathsf{W}^{I}))$.
Hence
\begin{equation}
Q(\mathsf{W}'{}^{I},|a(\mathsf{W}'{}^{I})\rangle)=Q(\mathsf{W}^{I},|a(\mathsf{W}^{I})\rangle).
\end{equation}
By the defining equivalence on $\mathbf{E}$, this implies
\begin{equation}
\big(\mathsf{W}'{}^{I},|a(\mathsf{W}'{}^{I})\rangle\big)\equiv_{p,q}\big(\mathsf{W}'{}^{I},t_{1}^{p}t_{2}^{q}\cdot|a(\mathsf{W}^{I})\rangle\big),
\end{equation}
and uniqueness then yields
\begin{equation}
|a(\mathsf{W}'{}^{I})\rangle=t_{1}^{p}t_{2}^{q}\cdot|a(\mathsf{W}^{I})\rangle.
\end{equation}
Thus $a$ is homogeneous of bi-degree $(p,q)$ and so belongs to $\mathcal{S}_{A}(p,q)$.

Conversely, we construct a section from a homogeneous function $a\in\mathcal{S}_{A}(p,q)$.
For each $\mathbf{Z}_{2}$-graded minitwistor point $\mathsf{w}\in\mathbf{MT}_{s}$,
choose any coordinate representative $\mathsf{W}^{I}\in\pi_{0}^{-1}(\mathsf{w})$.
We then define
\begin{equation}
s(\mathsf{w})\coloneqq Q\big(\mathsf{W}^{I},|a(\mathsf{W}^{I})\rangle\big)\in\mathcal{O}_{A}(p,q).
\end{equation}
To see that $s$ is well-defined, suppose $\mathsf{W}'{}^{I}$ is
another lift of $\mathsf{w}$. Then there exist $t_{1},t_{2}\in\mathbf{C}_{*}$
with $\mathsf{W}'{}^{I}=(t_{1}\lambda^{A},t_{2}\mu_{\dot{A}},t_{1}\psi^{\alpha})$.
Homogeneity of $a$ gives $|a(\mathsf{W}'{}^{I})\rangle=t_{1}^{p}t_{2}^{q}\cdot|a(\mathsf{W}^{I})\rangle$.
Hence
\begin{equation}
Q\big(\mathsf{W}'{}^{I},|a(\mathsf{W}'{}^{I})\rangle\big)=Q\big(\mathsf{W}'{}^{I},t_{1}^{p}t_{2}^{q}\cdot|a(\mathsf{W}^{I})\rangle\big)=Q\big(\mathsf{W}^{I},|a(\mathsf{W}^{I})\rangle\big),
\end{equation}
so $s(\mathsf{w})$ is independent of the choice of representative.
Smoothness of $s$ follows from that of $a$ together with the local
triviality of the bundle $\mathcal{O}_{A}(p,q)\stackrel{\pi}{\to}\mathbf{MT}_{s}$.
Finally, $s$ is a section because
\begin{equation}
\pi\big(s(\mathsf{w})\big)=\pi\big(Q(\mathsf{W}^{I},|a(\mathsf{W}^{I})\rangle)\big)=Q_{0}\big(\mathsf{W}^{I},|a(\mathsf{W}^{I})\rangle\big)=\pi_{0}(\mathsf{W}^{I})=\mathsf{w},
\end{equation}
i.e. $\pi\circ s=\text{id}_{\mathbf{MT}_{s}}$. Thus $s\in\Gamma(\mathbf{MT}_{s};\mathcal{O}_{A}(p,q))$,
completing the correspondence.

\subsubsection{$\mathcal{O}_{A}^{*}(r,s)$-bundle}

A full understanding of minitwistor wavefunctions requires the introduction
of the dual vector bundle $\mathcal{O}_{A}^{*}(r,s)\stackrel{\pi^{*}}{\to}\mathbf{MT}_{s}^{*}.$ 

Our correspondence between the holomorphic celestial supersphere $\mathcal{CS}$
and the minitwistor superspace $\mathbf{MT}_{s}$ is mediated by the
minitwistor transform $\mathcal{MT}$. Unlike an ordinary Fourier
transform, $\mathcal{MT}$ carries sections of $\mathcal{O}_{A}(p,q)\stackrel{\pi}{\to}\mathbf{MT}_{s}$
to sections of $\mathcal{O}_{A}^{*}(r,s)\stackrel{\pi^{*}}{\to}\mathbf{MT}_{s}^{*}$. 

\textcompwordmark{}

\paragraph*{Definition of $\mathcal{O}_{A}^{*}(r,s)$.}

In the previous subsection, we fixed a normed algebra $A$ adapted
to the background field theory or wavefunction in question. We now
introduce its dual algebra $A^{*}$, whose elements we denote by $\langle a|$.
We also define the auxiliary trivial bundle $\mathbf{E}^{*}\coloneqq\mathbf{M}^{*}\times A^{*}$.

On the auxiliary bundle, we impose the equivalence:
\begin{equation}
\big(z^{A},\bar{z}_{\dot{A}},\eta^{\alpha},\langle a|\big)\cong_{r,s}\big(t_{1}\cdot z^{A},t_{2}\cdot\bar{z}_{\dot{A}},t_{1}^{-1}\cdot\eta^{\alpha},t_{1}^{r}t_{2}^{s}\cdot\langle a|\big),\quad\forall\,t_{1},t_{2}\in\mathbf{C}^{*}.
\end{equation}
The total space of the dual bundle is the quotient
\begin{equation}
\mathcal{O}_{A}^{*}(r,s)\,\coloneqq\,\mathbf{E}^{*}\boldsymbol{\big/}\cong_{r,s},
\end{equation}
with the natural projection $Q^{*}:\mathbf{E}^{*}\longrightarrow\mathcal{O}_{A}^{*}(r,s)$.

\textcompwordmark{}

\paragraph*{Dual Fibration.}

Next, let $\text{pr}_{\mathbf{M}^{*}}:\mathbf{E}^{*}\longrightarrow\mathbf{M}^{*}$
be the projection onto the first factor. Define
\begin{equation}
Q_{0}^{*}\colon\mathbf{E}^{*}\longrightarrow\mathbf{MT}_{s}^{*},\quad Q_{0}^{*}\coloneqq\pi_{0}^{*}\circ\text{pr}_{\mathbf{M}^{*}}.
\end{equation}
By construction, if $\mathsf{z}\in\mathbf{MT}_{s}^{*}$ has any lift
$\mathsf{Z}^{I}\in(\pi_{0}^{*})^{-1}(\mathsf{z})$, then $Q_{0}^{*}(\mathsf{Z}^{I},\langle a|)=\mathsf{z}$
for all $\langle a|\in A^{*}$.

Finally, we define the dual fibration:
\begin{equation}
\pi^{*}:\mathcal{O}_{A}^{*}(r,s)\longrightarrow\mathbf{MT}_{s}^{*}.
\end{equation}
It is uniquely specified by the requirement that the diagram\begin{equation} \label{eq:comm-diagram-dual} \begin{tikzcd}[   row sep=2cm,   column sep=2cm,   arrows={shorten <=4pt, shorten >=4pt},   every label/.append style={font=} ]   & \mathbf{E}^{*}     \arrow[dl, "Q^{*}"']     \arrow[dr, "Q_{0}^{*}"]   & \\   \mathcal{O}_{A}^{*}(r,s)     \arrow[rr, "{\pi^{*}}"{swap,yshift=-5pt}]   & & \mathbf{MT}_{s}^{*} \end{tikzcd} \end{equation} commutes,
i.e. $\pi^{*}\circ Q^{*}=Q_{0}^{*}$.

\textcompwordmark{}

\paragraph*{Sections.}

The description of sections of the dual bundle $\mathcal{O}_{A}^{*}(r,s)$
parallels that of $\mathcal{O}_{A}(p,q)$. A map
\begin{equation}
a^{*}:\mathbf{M}^{*}\longrightarrow A^{*},\quad\mathsf{Z}^{I}\mapsto\langle a(\mathsf{Z}^{I})|
\end{equation}
is called an \emph{$A^{*}$-valued homogeneous function of bi-degree
$(r,s)$} if the following holds. Write $\mathsf{Z}^{I}=(z^{A},\bar{z}_{\dot{A}},\eta^{\alpha})$
and, for any $t_{1},t_{2}\in\mathbf{C}_{*}$, set $\mathsf{Z}'{}^{I}=(t_{1}z^{A},t_{2}\bar{z}_{\dot{A}},t_{1}^{-1}\eta^{\alpha})$.
Then homogeneity demands:
\begin{equation}
\langle a(\mathsf{Z}'{}^{I})|=t_{1}^{r}t_{2}^{s}\cdot\langle a(\mathsf{Z}^{I})|.
\end{equation}
To discuss the section/function correspondence on $\mathcal{O}_{A}^{*}(r,s)\stackrel{\pi^{*}}{\to}\mathbf{MT}_{s}^{*}$,
define the dual function space:
\begin{equation}
\mathcal{S}_{A}^{*}(r,s)\;\coloneqq\;\big\{\,a^{*}:\mathbf{M}^{*}\longrightarrow A^{*}\,\big|\,a^{*}\text{ smooth and homogeneous of bi-degree }(r,s)\big\}.
\end{equation}
Here smoothness is understood in the Fr�chet sense, since $A$ (and
hence $A^{*}$) carries a norm making it a Banach space. We therefore
equip $\mathcal{S}_{A}^{*}(r,s)$ with the induced Fr�chet topology,
rendering it a locally convex topological vector space.

By exactly the same arguments as in the previous subsection, one shows
that each $A^{*}$-valued homogeneous function $a^{*}$ defines a
unique holomorphic section of the dual bundle $\mathcal{O}_{A}^{*}(r,s)$,
and vice versa. Hence there is a natural isomorphism
\begin{equation}
\Gamma\big(\mathbf{MT}_{s}^{*};\mathcal{O}_{A}^{*}(r,s)\big)\,\boldsymbol{\simeq}\,\mathcal{S}_{A}^{*}(r,s).
\end{equation}


\subsubsection{Superforms and Currents on Minitwistor Superspace}

The projective delta function $\overline{\delta}_{\Delta}$ appearing
in Eq. (\ref{eq:-39}) indicates that the minitwistor superwavefunction
$\Psi_{\Delta}^{p}$ is most naturally realised as a current\footnote{The theory of currents was initiated by \citet{schwartz1954espaces,schwartz1957theorie,schwartz1958theorie}
and \citet{de2012differentiable}. A measure-theoretic framework was
developed by \citet{federer1965some,federer1959curvature,federer1960normal,federer2014geometric}.
The complex-analytic aspects relevant to our discussion were reviewed
by \citet{king1971currents}. For a modern treatment emphasising positive
line bundles in the context of algebraic geometry, see the review
by \citet{demailly19962}.}. We now review the requisite differential-geometric framework.

\textcompwordmark{}

\paragraph*{Superforms.}

Fix integers $m,n,p,q\in\mathbf{Z}$ with $0\leq m,n\leq2$. Denote
by $\bigw^{m,n}\,\mathbf{MT}_{s}$ the exterior superbundle of $\mathbf{Z}_{2}$-graded
$(m,n)$-forms on minitwistor superspace. An element of this bundle
is called an \emph{$(m,n)$-superform}.

We consider the sheaf of \emph{$\mathcal{O}_{\mathbf{C}}(p,q)$-valued
differential $(m,n)$-superforms}. Explicitly, set
\begin{equation}
\Omega^{m,n}\big(\mathbf{MT}_{s};\mathcal{O}_{\mathbf{C}}(p,q)\big)\;\coloneqq\;\Gamma\big(\mathbf{MT}_{s};\bigw^{m,n}\,\mathbf{MT}_{s}\otimes\mathcal{O}_{\mathbf{C}}(p,q)\big).
\end{equation}
This is a module over the ring $\mathscr{C}^{\infty}(\mathbf{MT}_{s})$.
Equipped with the Whitney $\mathscr{C}^{\infty}$-topology\footnote{We equip the space $\Omega^{m,n}\big(\mathbf{MT}_{s};\mathcal{O}_{\mathbf{C}}(p,q)\big)$
with the following topology. A sequence $(\boldsymbol{\alpha}_{i})_{i\in\mathbf{N}}$
of $\mathcal{O}_{\mathbf{C}}(p,q)$-valued differential $(m,n)$-superforms
converges to $\boldsymbol{\alpha}$ if and only if, on each trivialising
neighbourhood $U\subset\mathbf{MT}_{s}$ of the bundle $\mathcal{O}_{\mathbf{C}}(p,q)$,
the following holds:
\begin{enumerate}
\item Write $(\boldsymbol{\alpha}_{i}-\boldsymbol{\alpha})_{i\in\mathbf{N}}$
in local coordinates as a finite collection of component functions.
\item For every multi-index $k$ and every compact set $K\subset U$, the
derivatives
\[
D^{k}\boldsymbol{\alpha}_{i}-D^{k}\boldsymbol{\alpha}
\]
converge uniformly to zero on $K$ as $i\to\infty$.
\end{enumerate}
In this way, all component functions of $(\boldsymbol{\alpha}_{i}-\boldsymbol{\alpha})_{i\in\mathbf{N}}$,
together with all their derivatives, vanish uniformly on compact subsets
on each trivialising patch. }, this space becomes a locally convex, complete topological vector
superspace over $\mathbf{C}$.

\textcompwordmark{}

\paragraph*{Supercurrents.}

A supercurrent of bi-degree $(2-m,2-n)$ (equivalently bi-dimension
$(m,n)$) over the bundle $\mathcal{O}_{\mathbf{C}}(2-p,2-q)\stackrel{\pi}{\to}\mathbf{MT}_{s}$
is a continuous, complex-linear functional
\begin{equation}
\mathcal{T}\colon\Omega^{m,n}\big(\mathbf{MT}_{s};\mathcal{O}_{\mathbf{C}}(p-2,q-2)\big)\longrightarrow\mathbf{C}.
\end{equation}
Continuity is understood with respect to the Whitney $\mathscr{C}^{\infty}$-topology.
The space of all such supercurrents is the strong-dual of the corresponding
module of superforms:
\begin{equation}
\mathscr{D}'_{2-m,2-n}\big(\mathbf{MT}_{s};\mathcal{O}_{\mathbf{C}}(2-p,2-q)\big)\;\coloneqq\;\big(\Omega^{m,n}\big(\mathbf{MT}_{s};\mathcal{O}_{\mathbf{C}}(p-2,q-2)\big)\big)'.
\end{equation}
One extends the boundary and contraction operators from the exterior
algebra of superforms to this space of currents. A natural wedge product
exists between superforms and supercurrents. However, an exterior
product of two supercurrents is not generically well defined. (See
\citet[Sec. 26]{simon1984lectures}.)

\textcompwordmark{}

\paragraph*{Minitwistor Superwavefunctions as Currents.}

With the preceding framework in place, $\Psi_{\Delta}^{p}$ is not
an ordinary differential $(0,1)$-form on $\mathbf{MT}_{s}$. Instead,
it defines a supercurrent of bi-degree $(0,1)$ over the bundle $\mathcal{O}_{\mathbf{C}}(\Delta-p,-\Delta)$.
To state this precisely, fix a dual $\mathbf{Z}_{2}$-graded minitwistor
point $\mathsf{z}\in\mathbf{MT}_{s}^{*}$ and choose a coordinate
representative $\mathsf{Z}^{I}\in(\pi_{0}^{*})^{-1}(\mathsf{z})$.
Then
\begin{equation}
\mathcal{T}_{\Delta}^{p}\in\mathscr{D}'_{0,1}\big(\mathbf{MT}_{s};\mathcal{O}_{\mathbf{C}}(\Delta-p,-\Delta)\big),\quad\mathcal{T}_{\Delta}^{p}\;\coloneqq\;\Psi_{\Delta}^{p}(\;\cdot\;;\mathsf{Z}^{I}).
\end{equation}
Next, let
\begin{equation}
\boldsymbol{\alpha}\in\Omega^{2,1}\big(\mathbf{MT}_{s};\mathcal{O}_{\mathbf{C}}(p-\Delta,\Delta)\big)
\end{equation}
be any differential $(2,1)$-superform with values in $\mathcal{O}_{\mathbf{C}}(p-\Delta,\Delta)$.
By the section/function correspondence, there exists
\begin{equation}
a\in\mathcal{S}_{\mathbf{C}}(p-\Delta+2,\Delta-2)
\end{equation}
such that on each trivialising neighbourhood $U\subset\mathbf{MT}_{s}$
one has
\begin{equation}
\boldsymbol{\alpha}\,\big|_{U}\;=\;a(\mathsf{W}^{I})\,D^{2|4}\mathsf{W}.
\end{equation}
Hence the action of the supercurrent $\mathcal{T}_{\Delta}^{p}$ on
$\boldsymbol{\alpha}$ is given by
\begin{equation}
\langle\mathcal{T}_{\Delta}^{p},\boldsymbol{\alpha}\rangle\;=\;\underset{\mathbf{MT}_{s}\,\,\,}{\int}\,D^{2|4}\mathsf{W}\,\,\,a(\mathsf{W}^{I})\wedge\Psi_{\Delta}^{p}(\mathsf{W}^{I};\mathsf{Z}^{I'}).
\end{equation}
In the next subsection, we shall use this pairing to formulate the
minitwistor transform.

\textcompwordmark{}

\paragraph*{Currents on the Dual Superspace.}

We adopt the viewpoint that the minitwistor supergeometry is dual
to the flat-space hologram on the holomorphic celestial supersphere.
Consistency then demands that the minitwistor transform $\mathcal{MT}$
be invertible. Equivalently, the family of superwavefunctions $\{\Psi_{\Delta}^{p}\}$
must also furnish a corresponding family of supercurrents on the \emph{dual}
minitwistor superspace. Accordingly, we extend the above discussion
to currents valued in the bundle $\mathcal{O}_{\mathbf{C}}^{*}(p,q)\stackrel{\pi^{*}}{\to}\mathbf{MT}_{s}^{*}$.

\textcompwordmark{}

\paragraph*{Differential Superforms on Dual Space.}

Let $\bigw^{m,n}\,\mathbf{MT}_{s}^{*}$ be the exterior superbundle
of complex $(m,n)$-superforms on the dual minitwistor superspace
$\mathbf{MT}_{s}^{*}$. We consider the module of smooth sections
\begin{equation}
\Omega^{m,n}\big(\mathbf{MT}_{s}^{*};\mathcal{O}_{\mathbf{C}}^{*}(p,q)\big)\;\coloneqq\;\Gamma\big(\mathbf{MT}_{s}^{*};\bigw^{m,n}\,\mathbf{MT}_{s}^{*}\otimes\mathcal{O}_{\mathbf{C}}^{*}(p,q)\big).
\end{equation}
Endowed with the Whitney $\mathscr{C}^{\infty}$-topology, this becomes
the locally convex, complete superspace of \emph{$\mathcal{O}_{\mathbf{C}}^{*}(p,q)$-valued
differential $(m,n)$-superforms on $\mathbf{MT}_{s}^{*}$}. 

\textcompwordmark{}

\paragraph*{Dual Supercurrents.}

A supercurrent of bi-degree $(2-m,2-n)$ (equivalently bi-dimension
$(m,n)$) over the bundle $\mathcal{O}_{\mathbf{C}}^{*}(2-p,2-q)\stackrel{\pi^{*}}{\to}\mathbf{MT}_{s}^{*}$
is a continuous, $\mathbf{C}$-linear functional
\begin{equation}
^{*}\mathcal{T}\colon\Omega^{m,n}\big(\mathbf{MT}^{*};\mathcal{O}_{\mathbf{C}}^{*}(p-2,q-2)\big)\longrightarrow\mathbf{C}.
\end{equation}
Continuity is again with respect to the Whitney $\mathscr{C}^{\infty}$-topology.

The space of supercurrents of bi-degree $(2-m,2-n)$ over the bundle
$\mathcal{O}_{\mathbf{C}}^{*}(2-p,2-q)$ is 
\begin{equation}
\mathscr{D}'_{2-m,2-n}\big(\mathbf{MT}_{s}^{*};\mathcal{O}_{\mathbf{C}}^{*}(2-p,2-q)\big)\;\coloneqq\;\big(\Omega^{m,n}\big(\mathbf{MT}_{s}^{*};\mathcal{O}_{\mathbf{C}}^{*}(p-2,q-2)\big)\big)'
\end{equation}
endowed with the strong-dual topology.

\textcompwordmark{}

\paragraph*{Supercurrents on the Dual Superspace.}

We now associate to each minitwistor superwavefunction $\Psi_{\Delta}^{p}$
a supercurrent on $\mathbf{MT}_{s}^{*}$. Fix a $\mathbf{Z}_{2}$-graded
minitwistor point $[\mathsf{W}^{I}]\in\mathbf{MT}_{s}$. Then define
\begin{equation}
^{*}\mathcal{T}_{\Delta}^{p}\colon\mathscr{D}'_{0,1}\big(\mathbf{MT}_{s}^{*};\mathcal{O}_{\mathbf{C}}^{*}(p-\Delta-2,-\Delta)\big),\quad^{*}\mathcal{T}_{\Delta}^{p}\;\coloneqq\;\Psi_{\Delta}^{p}(\mathsf{W}^{I};\;\cdot\;).
\end{equation}
Let
\begin{equation}
\boldsymbol{\beta}\in\Omega^{2,1}\big(\mathbf{MT}_{s}^{*};\mathcal{O}_{\mathbf{C}}^{*}(\Delta-p+2,\Delta)\big)
\end{equation}
be any differential $(2,1)$-superform. By the section/function correspondence,
there exists
\begin{equation}
b\in\mathcal{S}_{\mathbf{C}}^{*}(\Delta-p-4,\Delta-2)
\end{equation}
such that on each trivialising neighbourhood $U^{*}\subset\mathbf{MT}_{s}^{*}$,
\begin{equation}
\boldsymbol{\beta}\,\big|_{U^{*}}\;=\;b(\mathsf{W}^{I})\,D^{2|4}\mathsf{W}.
\end{equation}
Hence the action of the supercurrent on $\boldsymbol{\beta}$ is:
\begin{equation}
\langle^{*}\mathcal{T}_{\Delta}^{p},\boldsymbol{\beta}\rangle\;=\;\underset{\mathbf{MT}_{s}^{*}}{\int}\,\Psi_{\Delta}^{p}(\mathsf{W}^{I};\mathsf{Z}^{I'})\wedge b(\mathsf{Z}^{I'})\;D^{2|4}\mathsf{Z}.
\end{equation}
In the following subsection, this pairing will define the \emph{inverse
minitwistor transform} $\mathcal{MT}^{-1}$.


\subsection{Minitwistor Transform}

We now show that the family $\{\Psi_{\Delta}^{p}\}$ of minitwistor
wavefunctions is complete and orthogonal. Accordingly, we may interpret
\[
\Psi_{\Delta}^{p}\big(\mathsf{Z}^{I};z^{A},\bar{z}_{\dot{A}},\eta^{\alpha}\big)
\]
as the wavefunction of an external gluon with conformal weight $\Delta$
and quantum numbers $z^{A},\bar{z}_{\dot{A}},\eta^{\alpha}$. 

This interpretation follows from the existence of a minitwistor transform
$\mathcal{MT}$. The mapping $\mathcal{MT}$ carries holomorphic sections
on $\mathbf{MT}_{s}$ to those on its dual. Moreover, $\mathcal{MT}$
satisfies a Fourier-type inversion theorem.

\subsubsection{Completeness Relation\label{subsec:Completeness-Relation}}

To derive the completeness relation for the family $\{\Psi_{\Delta}^{p}\}$,
consider the differential form on the dual superspace $\mathbf{MT}_{s}^{*}$:
\begin{equation}
\boldsymbol{a}\,\coloneqq\,a(z^{A},\bar{z}_{\dot{A}},\eta^{\alpha})\,\,\,D^{2|4}\mathsf{W},
\end{equation}
where:
\begin{equation}
a\big(z^{A},\bar{z}_{\dot{A}},\eta^{\alpha}\big)\,\coloneqq\,\Psi_{\tilde{\Delta}}^{\tilde{p}}\big(\lambda^{A},\mu_{\dot{A}},\psi^{\alpha};z^{A},\bar{z}_{\dot{A}},\eta^{\alpha}\big)\widetilde{\Psi}_{\Delta}^{p}\big(\sigma^{A},\omega_{\dot{A}},\chi^{\alpha};z^{A},\bar{z}_{\dot{A}},\eta^{\alpha}\big).
\end{equation}
The integral
\[
\underset{\mathbf{MT}_{s}^{*}\,\,\,}{\int}\,\,\,\boldsymbol{a}
\]
exists when $\boldsymbol{a}$ is invariant under the transformation
of Eq. (\ref{eq:-31}).

Equations (\ref{eq:-30}) and (\ref{eq:-38}) imply that $\boldsymbol{a}$
defines a volume form when:
\begin{equation}
\Delta+\tilde{\Delta}=2,\qquad p+\tilde{p}=0.
\end{equation}
We therefore set:
\begin{equation}
\boldsymbol{a}\,=\,\Psi_{2-\Delta}^{-p}\big(\lambda^{A},\mu_{\dot{A}},\psi^{\alpha};z^{A},\bar{z}_{\dot{A}},\eta^{\alpha}\big)\,\widetilde{\Psi}_{\Delta}^{p}\big(\sigma^{A},\omega_{\dot{A}},\chi^{\alpha};z^{A},\bar{z}_{\dot{A}},\eta^{\alpha}\big)\,\,\,D^{2|4}\mathsf{W}.
\end{equation}
Substituting the expression for $\Psi_{\Delta}^{p}$ from Eq. (\ref{eq:-39})
gives:
\begin{equation}
\boldsymbol{a}\,=\,\overline{\delta}_{p-\Delta}\left(z,\sigma\right)\overline{\delta}_{\Delta-p-2}(z,\lambda)\frac{\mathcal{C}\left(\Delta\right)\mathcal{C}\left(2-\Delta\right)}{[\bar{z}\omega]^{\Delta}[\mu\bar{z}]^{2-\Delta}}\exp\left(i\frac{\left\langle z\iota\right\rangle }{\left\langle \lambda\iota\right\rangle }\left(\psi-\frac{\left\langle \lambda\iota\right\rangle }{\left\langle \sigma\iota\right\rangle }\chi\right)\cdot\eta\right)\,D^{2|4}\mathsf{W}
\end{equation}
Finally, integrating over $\mathbf{MT}_{s}^{*}$ yields:
\begin{align}
 & \underset{\mathbf{MT}_{s}^{*}\,\,\,}{\int}\,D^{2|4}\mathsf{W}\,\,\,\Psi_{2-\Delta}^{-p}\big(\lambda^{A},\mu_{\dot{A}},\psi^{\alpha};z^{A},\bar{z}_{\dot{A}},\eta^{\alpha}\big)\,\widetilde{\Psi}_{\Delta}^{p}\big(\sigma^{A},\omega_{\dot{A}},\chi^{\alpha};z^{A},\bar{z}_{\dot{A}},\eta^{\alpha}\big)\\
 & \,\,\,=\,\,\,\overline{\delta}_{p-\Delta}\left(\lambda,\sigma\right)\,\delta^{0|4}\left(\psi-\frac{\left\langle \lambda\iota\right\rangle }{\left\langle \sigma\iota\right\rangle }\chi\right)\,\underset{\mathbf{CP}^{1}\,\,\,\,\,}{\int}\,D\bar{z}\,\,\,\frac{\mathcal{C}\left(\Delta\right)\mathcal{C}\left(2-\Delta\right)}{[\bar{z}\omega]^{\Delta}[\mu\bar{z}]^{2-\Delta}}.\label{eq:-33}
\end{align}

We proceed by considering the integral:
\begin{equation}
\mathcal{I}\big(\mu_{\dot{A}},\omega_{\dot{B}}\big)\,\coloneqq\,\underset{\mathbf{CP}^{1}\,\,\,\,\,}{\int}\,D\bar{z}\,\,\,\frac{\mathcal{C}\left(\Delta\right)\mathcal{C}\left(2-\Delta\right)}{[\bar{z}\omega]^{\Delta}[\mu\bar{z}]^{2-\Delta}}.
\end{equation}
This expression is well-defined only in a distributional sense. Indeed,
if one assumes that $\mathcal{I}$ admits an analytic form, then:
\begin{equation}
\mathcal{I}\big(t_{1}\cdot\mu_{\dot{A}},t_{2}\cdot\omega_{\dot{B}}\big)\,=\,t_{1}^{\Delta-2}\,t_{2}^{-\Delta}\,\mathcal{I}\big(\mu_{\dot{A}},\omega_{\dot{B}}\big),\qquad\forall\,t_{1},t_{2}\in\mathbf{C}^{*}.\label{eq:-32}
\end{equation}
Lorentz invariance, by contrast, requires $\mathcal{I}$ to scale
as a power of $[\mu\omega]$. These two requirements are incompatible
unless the proportionality factors vanishes or diverges. This is precisely
the behaviour of the projective delta function $\overline{\delta}_{\Delta}$.

In \citet{sharma2022ambidextrous} this is confirmed by explicit integration:
\begin{equation}
\underset{\mathbf{CP}^{1}\,\,\,\,\,}{\int}\,D\bar{z}\,\,\,\frac{\mathcal{C}\left(\Delta\right)\mathcal{C}\left(2-\Delta\right)}{[\bar{z}\omega]^{\Delta}[\mu\bar{z}]^{2-\Delta}}\,=\,4\pi^{2}\overline{\delta}_{\Delta}\left(\mu,\omega\right).\label{eq:-43}
\end{equation}
Substitution into Eq. (\ref{eq:-33}) then yields:
\begin{align}
 & \underset{\mathbf{MT}_{s}^{*}\,\,\,}{\int}\,D^{2|4}\mathsf{W}\,\,\,\Psi_{2-\Delta}^{-p}\left(\lambda,\mu,\psi;z,\bar{z},\eta\right)\,\widetilde{\Psi}_{\Delta}^{p}\left(\sigma,\omega,\chi;z,\bar{z},\eta\right)\label{eq:-36}\\
 & \,\,\,=\,\,\,4\pi^{2}\,\overline{\delta}_{p-\Delta}\left(\lambda,\sigma\right)\,\overline{\delta}_{\Delta}\left(\mu,\omega\right)\,\delta^{0|4}\left(\psi^{\alpha}-\frac{\left\langle \lambda\iota\right\rangle }{\left\langle \sigma\iota\right\rangle }\chi^{\alpha}\right).
\end{align}
This establishes the completeness relation for minitwistor wavefunctions. 

\textcompwordmark{}

\paragraph*{Minitwistor Delta Function.}

Equation (\ref{eq:-36}) is rather involved. We seek a concise reformulation
that makes its homogeneity properties manifest. To this end, we extend
the projective delta function $\overline{\delta}_{\Delta}$ on $\mathbf{CP}^{1}$
to the minitwistor superspace $\mathbf{MT}_{s}$. This extension should
preserve covariance under the transformation law of Eq. (\ref{eq:-1}).

Define the \emph{minitwistor delta function} with homogeneities $\Delta_{1},\Delta_{2}$
by:
\begin{equation}
\overline{\delta}_{\Delta_{1},\Delta_{2}}^{2|4}\big(\mathsf{Z}^{I};\mathsf{Z}'{}^{J}\big)\,=\,\underset{\mathbf{C}^{*}\,\,\,}{\int}\,\frac{dt_{1}}{t_{1}}\,t_{1}^{\Delta_{1}}\,\underset{\mathbf{C}^{*}\,\,\,}{\int}\,\frac{dt_{2}}{t_{2}}\,t_{2}^{\Delta_{2}}\,\,\,\overline{\delta}^{2}\big(\lambda^{A}-t_{1}\sigma^{A}\big)\,\overline{\delta}^{2}\big(\mu_{\dot{A}}-t_{2}\omega_{\dot{A}}\big)\,\delta^{0|4}\big(\psi^{\alpha}-t_{1}\chi^{\alpha}\big),\label{eq:-34}
\end{equation}
where:
\[
\mathsf{Z}^{I}\,\coloneqq\,\big(\lambda^{A},\mu_{\dot{A}},\psi^{\alpha}\big),\,\,\,\mathsf{Z}'{}^{I}\,\coloneqq\,\big(\sigma^{A},\omega_{\dot{A}},\chi^{\alpha}\big)\in\mathbf{MT}_{s}.
\]

Now let $\iota^{A}$ be an auxiliary non-vanishing spinor. Using the
fundamental solution of the Dolbeault operator $\overline{\partial}$
on $\mathbf{CP}^{1}$, we find:
\begin{equation}
\overline{\delta}^{2}\big(\lambda^{A}-t_{1}\sigma^{A}\big)\,=\,\frac{1}{\left(2\pi i\right)^{2}}\,\bigwedge_{A\in\{1,2\}}\,\overline{\partial}\,\,\,\frac{1}{\lambda^{A}-t_{1}\sigma^{A}}\,=\,\overline{\delta}\left(t_{1}-\frac{\left\langle \lambda\iota\right\rangle }{\left\langle \sigma\iota\right\rangle }\right)\overline{\delta}\left(\left\langle \sigma\lambda\right\rangle \right),\label{eq:-45}
\end{equation}
\begin{equation}
\overline{\delta}^{2}\big(\mu_{\dot{A}}-t_{2}\omega_{\dot{A}}\big)\,=\,\frac{1}{\left(2\pi i\right)^{2}}\,\bigwedge_{\dot{A}\in\{\dot{1},\dot{2}\}}\,\overline{\partial}\,\,\,\frac{1}{\mu_{\dot{A}}-t_{2}\omega_{\dot{A}}}\,=\,\overline{\delta}\left(t_{2}-\frac{[\mu\bar{\iota}]}{[\omega\bar{\iota}]}\right)\overline{\delta}\left([\omega\mu]\right).\label{eq:-46}
\end{equation}
By substituting into Eq. (\ref{eq:-34}) and invoking the definition
of $\overline{\delta}_{\Delta}$ from Eq. (\ref{eq:-35}), we obtain:
\begin{equation}
\overline{\delta}_{\Delta_{1},\Delta_{2}}^{2|4}\big(\mathsf{Z};\mathsf{Z}'\big)\,=\,\overline{\delta}_{\Delta_{1}}\big(\lambda^{A},\sigma^{A}\big)\,\overline{\delta}_{\Delta_{2}}\big(\mu_{\dot{A}},\omega_{\dot{A}}\big)\,\delta^{0|4}\left(\psi^{\alpha}-\frac{\left\langle \lambda\iota\right\rangle }{\left\langle \sigma\iota\right\rangle }\chi^{\alpha}\right).\label{eq:-179}
\end{equation}


\paragraph*{Canonical Form.}

Using the minitwistor delta function, the completeness relation in
Eq. (\ref{eq:-36}) can be written as:
\begin{equation}
\underset{\mathbf{MT}_{s}^{*}\,\,\,}{\int}\,D^{2|4}\mathsf{W}\,\,\,\Psi_{2-\Delta}^{-p}\left(\mathsf{Z};\mathsf{W}\right)\,\widetilde{\Psi}_{\Delta}^{p}\left(\mathsf{Z}';\mathsf{W}\right)\,=\,4\pi^{2}\,\overline{\delta}_{p-\Delta,\Delta}^{2|4}\left(\mathsf{Z};\mathsf{Z}'\right).\label{eq:-37}
\end{equation}
We now adopt a simple convention for the conjugate wavefunction. Define:
\begin{equation}
\Psi_{\Delta}^{p}\left(\mathsf{W};\mathsf{Z}\right)\,\coloneqq\,\widetilde{\Psi}_{\Delta}^{p}\left(\mathsf{Z};\mathsf{W}\right).
\end{equation}
With this definition, Eq. (\ref{eq:-37}) takes the canonical form:
\begin{equation}
\underset{\mathbf{MT}_{s}^{*}\,\,\,}{\int}\,D^{2|4}\mathsf{W}\,\,\,\Psi_{2-\Delta}^{-p}\left(\mathsf{Z};\mathsf{W}\right)\,\Psi_{\Delta}^{p}\left(\mathsf{W};\mathsf{Z}'\right)\,=\,4\pi^{2}\,\overline{\delta}_{p-\Delta,\Delta}^{2|4}\left(\mathsf{Z};\mathsf{Z}'\right).\label{eq:-40}
\end{equation}


\subsubsection{Orthogonality}

To prove that the minitwistor transform $\mathcal{MT}$ is invertible,
we derive an orthogonality relation for the family $\{\Psi_{\Delta}^{p}\}$
of minitwistor wavefunctions. Whereas completeness (Eq. (\ref{eq:-40}))
follows from integrating a differential form over the dual superspace
$\mathbf{MT}_{s}^{*}$, orthogonality is obtained by integrating over
the superspace $\mathbf{MT}_{s}$. 

First, define the $\mathbf{Z}_{2}$-graded differential form on $\mathbf{MT}_{s}$,
\begin{equation}
\boldsymbol{b}\,\coloneqq\,b\big(\lambda^{A},\mu_{\dot{A}},\psi^{\alpha}\big)\,\,\,D^{2|4}\mathsf{Z},
\end{equation}
where:
\begin{equation}
b\big(\lambda^{A},\mu_{\dot{A}},\psi^{\alpha}\big)\,\coloneqq\,\widetilde{\Psi}_{\tilde{\Delta}}^{\tilde{p}}\big(\lambda^{A},\mu_{\dot{A}},\psi^{\alpha};z^{A},\bar{z}_{\dot{A}},\eta^{\alpha}\big)\,\Psi_{\Delta}^{p}\big(\lambda^{A},\mu_{\dot{A}},\psi^{\alpha};z'{}^{A},\bar{z}'_{\dot{A}},\eta'{}^{\alpha}\big).
\end{equation}
The integral
\[
\underset{\mathbf{MT}_{s}\,\,\,}{\int}\,\,\,\boldsymbol{b}
\]
is well-defined only if $\boldsymbol{b}$ is invariant under the transformations
of Eq. (\ref{eq:-1}).

Equation (\ref{eq:-29}) implies that $\boldsymbol{b}$ is a volume
form on $\mathbf{MT}_{s}$ when:
\begin{equation}
\Delta+\tilde{\Delta}=2,\qquad p+\tilde{p}=0.
\end{equation}
Accordingly, we set:
\begin{equation}
\boldsymbol{b}\,=\,\widetilde{\Psi}_{2-\Delta}^{-p}\big(\lambda^{A},\mu_{\dot{A}},\psi^{\alpha};z^{A},\bar{z}_{\dot{A}},\eta^{\alpha}\big)\,\Psi_{\Delta}^{p}\big(\lambda^{A},\mu_{\dot{A}},\psi^{\alpha};z'{}^{A},\bar{z}'_{\dot{A}},\eta'{}^{\alpha}\big)\,\,\,D^{2|4}\mathsf{Z}.
\end{equation}
Next, substitute the explicit forms of $\Psi_{\Delta}^{p}$ and $\widetilde{\Psi}_{\Delta}^{p}$
from Eqs. (\ref{eq:-39}) and (\ref{eq:-41}). One obtains:
\begin{equation}
\boldsymbol{b}\,=\,\overline{\delta}_{p-\Delta}\left(z',\lambda\right)\,\overline{\delta}_{\Delta-p-2}\left(z,\lambda\right)\,\frac{\mathcal{C}\left(\Delta\right)\mathcal{C}\left(2-\Delta\right)}{[\mu\bar{z}']^{\Delta}[\bar{z}\mu]^{2-\Delta}}\,\exp\left(-i\frac{\left\langle z\iota\right\rangle }{\left\langle \lambda\iota\right\rangle }\psi\cdot\left(\eta-\frac{\left\langle z'\iota\right\rangle }{\left\langle z\iota\right\rangle }\eta'\right)\right)\,\,\,D^{2|4}\mathsf{Z}.
\end{equation}
We then integrate over the minitwistor superspace:
\begin{align}
 & \underset{\mathbf{MT}_{s}\,\,\,}{\int}\,D^{2|4}\mathsf{Z}\,\,\,\widetilde{\Psi}_{2-\Delta}^{-p}\left(\lambda,\mu,\psi;z,\bar{z},\eta\right)\,\Psi_{\Delta}^{p}\left(\lambda,\mu,\psi;z',\bar{z}',\eta'\right)\\
 & \,=\,\left(\frac{\left\langle z\iota\right\rangle }{\left\langle z'\iota\right\rangle }\right)^{4}\,\overline{\delta}_{\Delta-p-2}\left(z,z'\right)\,\delta^{0|4}\left(\eta^{\alpha}-\frac{\left\langle z'\iota\right\rangle }{\left\langle z\iota\right\rangle }\eta'{}^{\alpha}\right)\,\underset{\mathbf{CP}^{1}\,\,\,}{\int}\,D\mu\,\,\,\frac{\mathcal{C}\left(\Delta\right)\mathcal{C}\left(2-\Delta\right)}{[\mu\bar{z}']^{\Delta}[\bar{z}\mu]^{2-\Delta}}.\label{eq:-42}
\end{align}
Using Eq. (\ref{eq:-35}) for the projective delta function on $\mathbf{CP}^{1}$,
we have:
\begin{equation}
\overline{\delta}_{\Delta-p+2}\left(z,z'\right)\,=\,\overline{\delta}\left(\left\langle z'z\right\rangle \right)\,\left(\frac{\left\langle z\iota\right\rangle }{\left\langle z'\iota\right\rangle }\right)^{\left(\Delta-p+2\right)-1}\,=\,\left(\frac{\left\langle z\iota\right\rangle }{\left\langle z'\iota\right\rangle }\right)^{4}\,\overline{\delta}_{\Delta-p-2}\left(z,z'\right).
\end{equation}
Substituting into (\ref{eq:-42}) and using the integral identity
of Eq. (\ref{eq:-43}) gives the final result:
\begin{align}
 & \underset{\mathbf{MT}_{s}\,\,\,}{\int}\,D^{2|4}\mathsf{Z}\,\,\,\widetilde{\Psi}_{2-\Delta}^{-p}\left(\lambda,\mu,\psi;z,\bar{z},\eta\right)\,\Psi_{\Delta}^{p}\left(\lambda,\mu,\psi;z',\bar{z}',\eta'\right)\\
 & \,=\,4\pi^{2}\,\overline{\delta}_{\Delta-p+2}\left(z,z'\right)\,\overline{\delta}_{\Delta}\left(\bar{z},\bar{z}'\right)\,\delta^{0|4}\left(\eta^{\alpha}-\frac{\left\langle z'\iota\right\rangle }{\left\langle z\iota\right\rangle }\eta'{}^{\alpha}\right).\label{eq:-44}
\end{align}


\paragraph*{Dual Delta Function.}

Equation (\ref{eq:-44}) is impractical for explicit calculations.
To remedy this, we extend the projective delta function $\overline{\delta}_{\Delta}$
on $\mathbf{CP}^{1}$ to a dual-minitwistor delta function on $\mathbf{MT}_{s}^{*}$.
This new distribution must transform homogeneously under rescalings,
in accordance with Eq. (\ref{eq:-31}).

Define the \emph{dual-minitwistor delta function} with homogeneity
degrees $\Delta_{1}$ and $\Delta_{2}$ by:
\begin{equation}
\widetilde{\delta}_{\Delta_{1},\Delta_{2}}^{2|4}\left(\mathsf{W}^{I},\mathsf{W}'{}^{J}\right)\,\coloneqq\,\underset{\mathbf{C}^{*}\,\,\,}{\int}\,\frac{dt_{1}}{t_{1}}\,t_{1}^{\Delta_{1}}\,\underset{\mathbf{C}^{*}\,\,\,}{\int}\,\frac{dt_{2}}{t_{2}}\,t_{2}^{\Delta_{2}}\,\overline{\delta}^{2}\big(z^{A}-t_{1}z'{}^{A}\big)\,\overline{\delta}^{2}\big(\bar{z}_{\dot{A}}-t_{2}\bar{z}'_{\dot{A}}\big)\delta^{0|4}\left(\eta^{\alpha}-t_{1}^{-1}\eta'{}^{\alpha}\right),\label{eq:-47}
\end{equation}
where:
\[
\mathsf{W}^{I}\,\coloneqq\,\big(z^{A},\bar{z}_{\dot{A}},\eta^{\alpha}\big),\,\,\,\mathsf{W}'{}^{I}\,\coloneqq\,\big(z'{}^{A},\bar{z}'_{\dot{A}},\eta'{}^{\alpha}\big)\in\mathbf{MT}_{s}^{*}.
\]
Substituting Eqs. (\ref{eq:-45}) and (\ref{eq:-46}) into Eq. (\ref{eq:-47})
yields:
\begin{equation}
\widetilde{\delta}_{\Delta_{1},\Delta_{2}}^{2|4}\left(\mathsf{W};\mathsf{W}'\right)\,=\,\overline{\delta}_{\Delta_{1}}\left(z,z'\right)\,\overline{\delta}_{\Delta_{2}}\left(\bar{z},\bar{z}'\right)\,\delta^{0|4}\left(\eta^{\alpha}-\frac{\left\langle z'\iota\right\rangle }{\left\langle z\iota\right\rangle }\eta'{}^{\alpha}\right).
\end{equation}
Finally, using the convention:
\begin{equation}
\Psi_{2-\Delta}^{-p}\left(\mathsf{W};\mathsf{Z}\right)\,=\,\widetilde{\Psi}_{2-\Delta}^{-p}\left(\mathsf{Z};\mathsf{W}\right),
\end{equation}
the orthogonality relation becomes:
\begin{equation}
\underset{\mathbf{MT}_{s}\,\,\,}{\int}\,D^{2|4}\mathsf{Z}\,\,\,\Psi_{2-\Delta}^{-p}\left(\mathsf{W};\mathsf{Z}\right)\,\Psi_{\Delta}^{p}\left(\mathsf{Z};\mathsf{W}'\right)\,=\,4\pi^{2}\,\widetilde{\delta}_{\Delta-p+2,\Delta}^{2|4}\left(\mathsf{W};\mathsf{W}'\right).
\end{equation}


\subsubsection{Minitwistor Fourier Transform\label{subsec:Minitwistor-Fourier-Transform}}

Having established the completeness and orthogonality of the family
$\{\Psi_{\Delta}^{p}\}$, we now introduce the minitwistor transform
$\mathcal{MT}$ and its inverse. Unlike the ordinary Fourier transform,
$\mathcal{MT}$ sends sections of holomorphic vector bundles over
the minitwistor superspace $\mathbf{MT}_{s}$ to sections over the
dual superspace $\mathbf{MT}_{s}^{*}$.

\paragraph*{Preliminaries.}

Let
\begin{equation}
\varphi\coloneqq\varphi\big(\lambda^{A},\mu_{\dot{A}},\psi^{\alpha}\big)
\end{equation}
be a holomorphic section of the bundle
\begin{equation}
\mathcal{O}\left(w_{1}\right)\oplus\mathcal{O}\left(w_{2}\right)\longrightarrow\mathbf{MT}_{s}.
\end{equation}
Define the $\mathbf{Z}_{2}$-graded differential form on $\mathbf{MT}_{s}$:
\begin{equation}
\boldsymbol{c}\,\coloneqq\,\varphi\big(\lambda^{A},\mu_{\dot{A}},\eta^{\alpha}\big)\,\widetilde{\Psi}_{\Delta}^{p}\big(\lambda^{A},\mu_{\dot{A}},\psi^{\alpha};z^{A},\bar{z}_{\dot{A}},\eta^{\alpha}\big)\,\,\,D^{2|4}\mathsf{Z}.
\end{equation}
For the integral
\[
\underset{\mathbf{MT}_{s}\,\,\,}{\int}\,\,\,\boldsymbol{c}
\]
to be well-defined, $\boldsymbol{c}$ must be invariant under the
rescalings of Eq. (\ref{eq:-1}).

Equations (\ref{eq:-29}) and (\ref{eq:-44}) show that $\boldsymbol{c}$
is a volume form precisely when:
\begin{equation}
w_{1}=p-\Delta+2,\qquad w_{2}=\Delta-2.
\end{equation}
Hence we take\footnote{Let $\pi:E\longrightarrow B$ be a holomorphic vector superbundle
over the base $B$. We denote by $\Gamma(E;B)$ the $\mathcal{O}(B)$-module
of holomorphic sections of $E$.}:
\begin{equation}
\varphi\in\Gamma\big(\mathcal{O}(p-\Delta+2)\oplus\mathcal{O}(\Delta-2);\mathbf{MT}_{s}\big).\label{eq:-49}
\end{equation}

We define the \emph{minitwistor transform} of a section $\varphi$
:
\begin{equation}
\Phi\left(\mathsf{W}^{I}\right)\coloneqq\mathcal{MT}\big[\varphi(\mathsf{Z}^{J})\big](\mathsf{W}^{I}),
\end{equation}
where:
\begin{equation}
\Phi\left(z^{A},\bar{z}_{\dot{A}},\eta^{\alpha}\right)\,\coloneqq\,\underset{\mathbf{MT}_{s}\,\,\,}{\int}\,D^{2|4}\mathsf{Z}\,\,\,\varphi\left(\lambda^{A},\mu_{\dot{A}},\eta^{\alpha}\right)\,\widetilde{\Psi}_{\Delta}^{p}\big(\lambda^{A},\mu_{\dot{A}},\eta^{\alpha};z^{A},\bar{z}_{\dot{A}},\eta^{\alpha}\big).\label{eq:-48}
\end{equation}
From the homogeneity laws in Eqs. (\ref{eq:-30}) and (\ref{eq:-31}),
one finds:
\begin{equation}
\Phi\big(t_{1}\cdot z^{A},t_{2}\cdot\bar{z}_{\dot{A}},t_{1}^{-1}\cdot\eta^{\alpha}\big)\,=\,t_{1}^{p-\Delta-2}\,t_{2}^{-\Delta}\,\Phi\big(z^{A},\bar{z}_{\dot{A}},\eta^{\alpha}\big),\qquad\forall\,t_{1},t_{2}\in\mathbf{C}^{*}.\label{eq:-155}
\end{equation}
Thus $\Phi$ is a section of:
\begin{equation}
\mathcal{O}\left(p-\Delta-2\right)\oplus\mathcal{O}\left(-\Delta\right)\longrightarrow\mathbf{MT}_{s}^{*}.
\end{equation}
Accordingly, the transform acts as:
\begin{equation}
\mathcal{MT}:\Gamma\big(\mathcal{O}(p-\Delta+2)\oplus\mathcal{O}(\Delta-2);\mathbf{MT}_{s}\big)\longrightarrow\Gamma\big(\mathcal{O}\left(p-\Delta-2\right)\oplus\mathcal{O}\left(-\Delta\right);\mathbf{MT}_{s}^{*}\big).
\end{equation}


\paragraph*{Inversion Theorem.}

We now derive the inversion of the minitwistor transform using the
completeness relation of Subsec. \ref{subsec:Completeness-Relation}
and Fubini's theorem.

Define the superform on $\mathbf{MT}_{s}^{*}$:
\begin{equation}
\boldsymbol{d}\,\coloneqq\,\Psi_{\tilde{\Delta}}^{\tilde{p}}\big(\lambda'{}^{A},\mu'_{\dot{A}},\psi'{}^{\alpha};z^{A},\bar{z}_{\dot{A}},\eta^{\alpha}\big)\,\Phi\big(z^{A},\bar{z}_{\dot{A}},\eta^{\alpha}\big)\,\,\,D^{2|4}\mathsf{W}.
\end{equation}
Equations (\ref{eq:-29}) and (\ref{eq:-155}) imply that $\boldsymbol{d}$
is a volume form when:
\begin{equation}
\tilde{p}=-p,\qquad\tilde{\Delta}=2-\Delta.
\end{equation}
Accordingly, we define the \emph{inverse transform }$\mathcal{MT}^{-1}$
by:
\begin{equation}
\mathcal{MT}^{-1}[\Phi]\big(\lambda'{}^{A},\mu'_{\dot{A}},\psi'{}^{\alpha}\big)\,\coloneqq\,\frac{1}{4\pi^{2}}\underset{\mathbf{MT}_{s}^{*}\,\,\,}{\int}\,D^{2|4}\mathsf{W}\,\,\,\Psi_{2-\Delta}^{-p}\big(\lambda'{}^{A},\mu'_{\dot{A}},\psi'{}^{\alpha}|z^{A},\bar{z}_{\dot{A}},\eta^{\alpha}\big)\,\Phi\big(z^{A},\bar{z}_{\dot{A}},\eta^{\alpha}\big).\label{eq:-51}
\end{equation}
Substitute $\Phi$ from Eq. (\ref{eq:-48}) and apply Fubini's theorem.
One finds:
\begin{align}
 & \mathcal{MT}^{-1}[\Phi]\left(\lambda',\mu',\psi'\right)\,=\,\frac{1}{4\pi^{2}}\,\underset{\mathbf{MT}_{s}\,\,\,}{\int}\,D^{2|4}\mathsf{Z}\,\,\,\varphi\left(\lambda,\mu,\eta\right)\,\,\,\\
 & \underset{\mathbf{MT}_{s}^{*}\,\,\,}{\int}\,D^{2|4}\mathsf{W}\,\,\,\Psi_{2-\Delta}^{-p}\left(\lambda',\mu',\psi';z,\bar{z},\eta\right)\,\widetilde{\Psi}_{\Delta}^{p}\left(\lambda,\mu,\eta;z,\bar{z},\eta\right).
\end{align}
Invoking the completeness relation (Eq. (\ref{eq:-40})) yields:
\begin{equation}
\mathcal{MT}^{-1}[\Phi]\left(\lambda',\mu',\psi'\right)\,=\,\varphi\left(\lambda',\mu',\psi'\right).
\end{equation}
Hence the inversion formula is:
\begin{equation}
\varphi\big(\lambda^{A},\mu_{\dot{A}},\eta^{\alpha}\big)\,=\,\frac{1}{4\pi^{2}}\,\underset{\mathbf{MT}_{s}^{*}\,\,\,}{\int}\,D^{2|4}\mathsf{W}\,\,\,\Psi_{2-\Delta}^{-p}\left(\lambda,\mu,\psi;z,\bar{z},\eta\right)\,\Phi\left(z,\bar{z},\eta\right).\label{eq:-50}
\end{equation}

Finally, if one takes this as the defining relation for $\varphi$
given $\Phi$ and applies orthogonality, then Eq. (\ref{eq:-51})
follows from Eq. (\ref{eq:-50}). We summarise our results below.

\textcompwordmark{}

\paragraph*{Summary.}

Let
\[
\mathsf{Z}^{I}\,\coloneqq\,\big(\lambda^{A},\mu_{\dot{A}},\psi^{\alpha}\big),\qquad\mathsf{W}^{I'}\,\coloneqq\,\big(z^{A},\bar{z}_{\dot{A}},\eta^{\alpha}\big)
\]
parametrise the minitwistor superspace $\mathbf{MT}_{s}$ and its
dual $\mathbf{MT}_{s}^{*}$, respectively. Denote by $D^{2|4}\mathsf{Z}$
and $D^{2|4}\mathsf{W}$ the corresponding measures.

Let
\[
\varphi\,\in\,\Gamma\big(\mathcal{O}\left(p-\Delta+2\right)\oplus\mathcal{O}\left(\Delta-2\right);\mathbf{MT}_{s}\big)
\]
and let
\[
\Phi\,\in\,\Gamma\big(\mathcal{O}\left(p-\Delta-2\right)\oplus\mathcal{O}\left(-\Delta\right);\mathbf{MT}_{s}^{*}\big).
\]
Then $\varphi$ and $\Phi$ are related by the integral transform:
\begin{equation}
\Phi\left(\mathsf{W}\right)\,=\,\mathcal{MT}\big[\varphi\left(\mathsf{Z}\right)\big]\left(\mathsf{W}\right)\,=\,\underset{\mathbf{MT}_{s}\,\,\,}{\int}\,D^{2|4}\mathsf{Z}\,\,\,\varphi\left(\mathsf{Z}\right)\,\widetilde{\Psi}_{\Delta}^{p}\left(\mathsf{Z};\mathsf{W}\right),
\end{equation}
if and only if its inverse holds:
\begin{equation}
\varphi\left(\mathsf{Z}\right)\,=\,\mathcal{MT}^{-1}\big[\Phi\left(\mathsf{W}\right)\big]\left(\mathsf{Z}\right)\,=\,\underset{\mathbf{MT}_{s}^{*}\,\,\,}{\int}\,D^{2|4}\mathsf{W}\,\,\,\Psi_{2-\Delta}^{-p}\left(\mathsf{Z};\mathsf{W}\right)\,\Phi\left(\mathsf{W}\right).
\end{equation}


\subsection{Celestial BMSW Identity}

One of the principal results of this work is the geometric and dynamical
reformulations of the celestial leaf amplitudes for gluons in $\mathcal{N}=4$
SYM theory.

In the geometric formulation, we will express the celestial amplitudes
as expectation values of holomorphic Wilson loops on minitwistor superspace
$\mathbf{MT}_{s}$. These Wilson loops are supported on a family of
nodal minitwistor lines. Expanding the path-ordered exponential that
defines the holonomy of the pseudoholomorphic connection will produce
the $n$-fold Penrose transform of the minitwistor superwavefunctions.

In the dynamical formulation, we will realise the celestial amplitudes
as semiclassical expectation values of correlators in the minitwistor
sigma-model. These correlators will be encoded by the Quillen determinant
of a gauge potential on $\mathbf{MT}_{s}$. Their evaluation again
reduces to an $n$-fold Penrose transform.

A key ingredient in both reformulations is the supersymmetric generalisation
of the celestial Boels-Mason-Skinner-Witten (BMSW) identity. The bosonic
version was derived in Mol (2025). This identity states that the $n$-fold
Penrose transform of the minitwistor superwavefunctions $\Psi_{\Delta}^{p}$
coincides with the integral kernel of the Mellin-transformed Parke-Taylor
factors appearing in the gluon amplitudes. We shall now present its
derivation.

\subsubsection{Preliminaries\label{subsec:Preliminaries}}

The central result of this subsection rests on the minitwistor Penrose
transform\footnote{\citet{jones1984minitwistors,jones1985minitwistor}.}.
We therefore recall the geometric structures on which it is defined.

The supersymmetric extension of the Hitchin correspondence\footnote{\citet{hitchin1982monopoles,hitchin1982twistor}.}
establishes a bijection between points of the minitwistor superspace
$\mathbf{MT}_{s}$ and totally geodesic null hypersurfaces in an Einstein-Weyl
supermanifold\footnote{\citet{leites2002einstein,dewitt1992supermanifolds,rogers2007supermanifolds,leites1980introduction,manin1997introduction}.}
$\mathbf{H}_{s}$. Conversely, each point of $\mathbf{H}_{s}$ corresponds
to a distinguished curve in $\mathbf{MT}_{s}$, known as a \emph{minitwistor
line}.

In our case, $\mathbf{MT}_{s}$ is the $\mathcal{N}=4$ supersymmetric
extension of an open subset of the quadric $\mathbf{CP}^{1}\times\mathbf{CP}^{1}$.
One finds that $\mathbf{H}_{s}$ is then the complexification of the
$(3|8)$-dimensional anti-de Sitter superspace\footnote{\citet{koning2025anti}.}.
The precise interplay among $\mathbf{MT}_{s}$, $\mathbf{H}_{s}$
and the projective spinor superbundle $\mathbf{P}(\mathcal{S})$ will
emerge in our definition of the double fibration below. Before that,
however, we review the projective model for the hyperbolic supergeometry
of $\mathbf{H}_{s}$.

\textcompwordmark{}

\paragraph*{Projective Model of Hyperbolic Space.}

We give a concise construction of the three-dimensional hyperbolic
model in $\mathbf{CP}^{3}$. For the $n$-dimensional case, see \citet{bailey1998twistor}.

Let $X_{A\dot{A}}$ be homogeneous coordinates on $\mathbf{CP}^{3}$.
In abstract index notation, the statement $X_{A\dot{A}}\in\mathbf{CP}^{3}$
is to be interpreted as the equivalence class $[X_{A\dot{A}}]$ in
$\mathbf{CP}^{3}$.

Define the bilinear form and its associated norm by
\begin{equation}
(X,Y)\;\coloneqq\;\varepsilon_{\overset{}{A}\overset{}{B}}\varepsilon_{\dot{A}\dot{B}}X^{A\dot{A}}Y^{B\dot{B}},\quad\left\Vert X\right\Vert ^{2}\;\coloneqq\;-(X,X),\quad\forall\,X_{A\dot{A}},Y_{A\dot{A}}\in\mathbf{CP}^{3}.
\end{equation}
Let $\mathscr{C}$ denote the complexified null cone,
\begin{equation}
\mathscr{C}\;\coloneqq\;\big\{\,X_{A\dot{A}}\in\mathbf{CP}^{3}\,\big|\,\Vert X\Vert=0\,\big\}.
\end{equation}
The complex hyperbolic space is then the open submanifold \textbf{$\mathbf{H}\coloneqq\mathbf{CP}^{3}\backslash\mathscr{C}$.}

We define a metric tensor $\boldsymbol{g}_{A\dot{A}B\dot{B}}$ on
the hyperbolic space $\mathbf{H}$ by requiring that its line element
in the dual coordinate basis $\{dX^{A\dot{A}}\}$ takes the form
\begin{equation}
ds^{2}\;\coloneqq\;\boldsymbol{g}_{A\dot{A}B\dot{B}}\,dX^{A\dot{A}}dX^{B\dot{B}}\;=\;-\frac{1}{\Vert X\Vert^{2}}\,\left(\Vert dX\Vert^{2}-\frac{(X,dX)^{2}}{\Vert X\Vert^{2}}\right).
\end{equation}
By construction, this metric is invariant under overall rescaling
of $X_{A\dot{A}}$, and has no component along the radial (scale)
direction.

To formalise these properties, let $\xi\coloneqq X^{A\dot{A}}\nabla_{A\dot{A}}$
be the Euler vector field on $\mathbf{CP}^{3}$, where
\begin{equation}
\nabla_{A\dot{A}}\;\coloneqq\;\frac{\partial}{\partial X^{A\dot{A}}}.
\end{equation}
The metric then satisfies:
\begin{equation}
\mathscr{L}_{\xi}\;\boldsymbol{g}_{A\dot{A}B\dot{B}}\;=\;0,\quad\xi^{A\dot{A}}\boldsymbol{g}_{A\dot{A}B\dot{B}}\;=\;0,
\end{equation}
where $\mathscr{L}_{\xi}$ denotes the Lie derivative along $\xi$.

The natural orientation on $\mathbf{H}$ is specified by the $\mathcal{O}_{\mathbf{C}}(4)$-valued
differential $3$-form:
\begin{equation}
D^{3}X\;\coloneqq\;\varepsilon_{[\overset{}{A}\overset{}{B}}\varepsilon_{\overset{}{C}\overset{}{D}]}\varepsilon_{[\dot{A}\dot{B}}\varepsilon_{\dot{C}\dot{D}]}\;X^{A\dot{A}}\,dX^{B\dot{B}}\wedge dX^{C\dot{C}}\wedge dX^{D\dot{D}}.
\end{equation}

We assert that the triple $\big(\mathbf{H},\mathbf{g}_{A\dot{A}B\dot{B}},D^{3}X\big)$
realises three-dimensional hyperbolic space. To see this, define the
weightless coordinate function:
\begin{equation}
\mathcal{R}_{A\dot{A}}:\mathbf{H}\longrightarrow\mathbf{C}^{4},\quad\mathcal{R}_{A\dot{A}}\coloneqq\frac{X_{A\dot{A}}}{\Vert X\Vert}.
\end{equation}
A direct computation shows
\begin{equation}
ds^{2}\;=\;\varepsilon_{\overset{}{A}\overset{}{B}}\varepsilon_{\dot{A}\dot{B}}\,d\mathcal{R}^{A\dot{A}}d\mathcal{R}^{B\dot{B}}\quad\text{and}\quad\Vert\mathcal{R}\Vert^{2}\coloneqq-\mathcal{R}_{A\dot{A}}\mathcal{R}^{A\dot{A}}=1.
\end{equation}
Thus the map $e\colon X_{A\dot{A}}\mapsto\mathcal{R}_{A\dot{A}}$
embeds $\mathbf{H}$ isometrically onto the hyperboloid:
\begin{equation}
H_{3}\;\coloneqq\;\big\{\,\mathcal{R}_{A\dot{A}}\in\mathbf{C}^{4}\,\big|\,\Vert\mathcal{R}\Vert^{2}=1\big\},
\end{equation}
the standard model of three-dimensional hyperbolic space.

Finally, one checks that the pullback (via $e^{*}$) of the standard
volume form on $H_{3}$ lies in the same orientation class as $D^{3}X$.
This completes the identification of $\mathbf{H}$ with the classical
hyperbolic geometry of $H_{3}$.

\textcompwordmark{}

To formulate $\mathcal{N}=4$ SYM theory, we must employ the supersymmetric
Hitchin correspondence. In the forthcoming discussion of the double
fibration on the projective spinor bundle $\mathbf{P}(\mathcal{S})$,
we will introduce the minitwistor incidence relations. These relations
identify the \emph{minitwistor lines} in $\mathbf{MT}_{s}$ as distinguished
curves. It then follows that \emph{the moduli space of these lines
in the $(2|4)$-dimensional minitwistor superspace is diffeomorphic
to the $(3|8)$-dimensional hyperbolic superspace.}

\textcompwordmark{}

\paragraph*{Hyperbolic Superspace.}

Before presenting the supersymmetric Hitchin correspondence in detail,
we extend our hyperbolic model $\mathbf{H}$ by adjoining fermionic
directions. The appropriate mathematical framework is that of a vector
superbundle, as discussed by \citet{manin1997introduction} and \citet{rogers2007supermanifolds}.

We define the\emph{ $(3|8)$-dimensional hyperbolic superspace} as
the trivial vector superbundle $\mathbf{H}_{s}\coloneqq\mathbf{H}\times\mathbf{C}^{0|8}$.
Its fibre is the vector superspace $\mathbf{C}^{0|8}$ spanned by
the Grassmann-valued van der Waerden spinors $\theta_{A}^{\alpha}$.
Each fibre carries the orientation provided by Berezin's measure $d^{0|8}\theta$.

A global trivialisation is provided by the superchart
\begin{equation}
\mathsf{X}^{K}\colon\mathbf{H}_{s}\longrightarrow\mathbf{CP}^{3}\times\mathbf{C}^{0|8},\quad\mathsf{X}^{K}\;\coloneqq\;\big(X_{A\dot{A}},\theta_{A}^{\alpha}\big),
\end{equation}
where $K$ indexes both bosonic and fermionic dimensions.

The canonical orientation measure on $\mathbf{H}_{s}$ is
\begin{equation}
D^{3|8}\mathsf{X}\;\coloneqq\;\frac{D^{3}X}{\Vert X\Vert^{4}}\wedge d^{0|8}\theta.
\end{equation}
By defining $\mathbf{H}_{s}$ as a trivial superbundle, we have imposed
projective invariance solely along the bosonic (horizontal) directions.
Therefore, under the scale transformation $X_{A\dot{A}},\theta_{A}^{\alpha}\mapsto t\cdot X_{A\dot{A}},\theta_{A}^{\alpha}$
the measure $D^{3|8}\mathsf{X}$ remains invariant.

\textcompwordmark{}

\paragraph*{Double Fibration.}

The Penrose transform is most naturally formulated via a double fibration.
We now define the fibration that realises the supersymmetric Hitchin
correspondence for the minitwistor superspace.

Let $\mathbf{P}(\mathcal{S})\coloneqq\mathbf{CP}^{3|8}\times\mathbf{CP}^{1}$
be the complex projective spinor superbundle over $\mathbf{CP}^{3|8}\coloneqq\mathbf{CP}^{3}\times\mathbf{C}^{0|8}$.
Define the open submanifold
\begin{equation}
\mathbf{P}'(\mathcal{S})\;\coloneqq\;\big\{\,\big(\mathsf{X}^{K},[\lambda^{A}]\big)\in\mathbf{P}(\mathcal{S})\,\big|\,\Vert X\Vert\neq0\,\big\}.
\end{equation}
We have two projections from $\mathbf{P}'(\mathcal{S})$:
\begin{equation}
\tau\colon\mathbf{P}'(\mathcal{S})\longrightarrow\mathbf{H}_{s},\quad\tau\big(\mathsf{X}^{K},[\lambda^{A}]\big)\coloneqq\big(\mathsf{X}^{K}\big),
\end{equation}
and
\begin{equation}
\upsilon\colon\mathbf{P}'(\mathcal{S})\longrightarrow\mathbf{MT}_{s}\quad\upsilon\big(\mathsf{X}^{K},[\lambda^{A}]\big)\coloneqq\pi_{0}\big(\lambda^{A},\lambda^{A}X_{A\dot{A}},\lambda^{A}\theta_{A}^{\alpha}\big).
\end{equation}
The product map
\begin{equation}
\tau\times\upsilon\colon\mathbf{P}'(\mathcal{S})\longrightarrow\mathbf{H}_{s}\times\mathbf{MT}_{s}
\end{equation}
is an embedding, and we denote its image by $\widetilde{\mathbf{P}'(\mathcal{S})}\subset\mathbf{H}_{s}\times\mathbf{MT}_{s}$.

Fix a point $\mathsf{X}^{K}\in\mathbf{H}_{s}$. Its $\tau$-fibre
is $\mathscr{F}_{\mathsf{X}^{K}}\coloneqq\tau^{-1}(\mathsf{X}^{K})$.
Under the embedding $\tau\times\upsilon$, $\mathscr{F}_{\mathsf{X}^{K}}$
maps to a submanifold of $\mathbf{MT}_{s}$. Thus we obtain a family
$\{\mathscr{F}_{\mathsf{X}^{K}}\}_{\mathsf{X}^{K}\in\mathbf{H}_{s}}$
of submanifolds in $\mathbf{MT}_{s}$ parametrised by $\mathbf{H}_{s}$.
Each $\mathscr{F}_{\mathsf{X}^{K}}$ is precisely the \emph{minitwistor
line} corresponding to $\mathsf{X}^{K}$.

Similarly, fix a minitwistor point $\mathsf{w}\in\mathbf{MT}_{s}$.
Its $\upsilon$-fibre is $\mathscr{G}_{\mathsf{w}}\coloneqq\upsilon^{-1}(\mathsf{w})$.
Under $\tau\times\upsilon$, $\mathscr{G}_{\mathsf{w}}$ embeds as
a submanifold of $\mathbf{H}_{s}$. Hence there is a family $\{\mathscr{G}_{\mathsf{w}}\}_{\mathsf{w}\in\mathbf{MT}_{s}}$
of submanifolds in $\mathbf{H}_{s}$ parametrised by $\mathbf{MT}_{s}$.
In Hitchin's correspondence, $\mathscr{G}_{\mathsf{w}}$ is the totally
geodesic null hypersurface associated to $\mathsf{w}$.

Therefore, the correspondence is summarised by the double fibration:\begin{equation}\label{eq:comm-diagram-tau-upsilon-small} \begin{tikzcd}[row sep=1.732cm, column sep=2cm, arrows={shorten <=4pt, shorten >=4pt}]   & \mathbf{P}'(\mathcal{S})     \arrow[dl, "\tau"']     \arrow[dr, "\upsilon"]   & \\   \mathbf{H}_{s}   &    & \mathbf{MT}_{s} \end{tikzcd} \end{equation} The
families of fibres $\{\mathscr{F}_{\mathsf{X}^{K}}\}$ and $\{\mathscr{G}_{\mathsf{w}}\}$
are related by the \emph{incidence relation}:
\begin{equation}
\big(\mathsf{X}^{K},\mathsf{w}\big)\in\widetilde{\mathbf{P}'(\mathcal{S})}\quad\iff\quad\mathsf{w}\in\mathscr{F}(\mathsf{X}^{K})\quad\iff\quad\mathsf{X}^{K}\in\mathscr{G}(\mathsf{w}).
\end{equation}
To describe this explicitly, choose a representative $\mathsf{W}^{I}=(\lambda^{A},\mu_{\dot{A}},\psi^{\alpha})\in(\pi_{0})^{-1}(\mathsf{w})$
and write $\mathsf{X}^{K}=(X_{A\dot{A}},\theta_{A}^{\alpha})$. Then
\begin{equation}
\big(\mathsf{X}^{K},[\mathsf{W}^{I}]\big)\in\widetilde{\mathbf{P}'(\mathcal{S})}\quad\iff\quad\mu_{\dot{A}}=\lambda^{A}X_{A\dot{A}},\;\psi^{\alpha}=\lambda^{A}\theta_{A}^{\alpha}.
\end{equation}
These conditions depend only on the projective class $[\mathsf{W}^{I}]$.

\textcompwordmark{}

\paragraph*{Minitwistor Lines.}

The planar Yang-Mills amplitudes on minitwistor space localise precisely
on certain rational curves called minitwistor lines\@. Similarly,
the worldsheet of our minitwistor sigma-model embeds into $\mathbf{MT}_{s}$
as such a line. We now characterise these special curves.

Following \citet{hitchin1982twistor} and \citet{jones1984minitwistors},
a \emph{minitwistor line} is defined to be a rational curve whose
normal bundle is isomorphic to $\mathcal{O}(2)$. The hyperbolic superspace
$\mathbf{H}_{s}$ parametrises all such lines. To each point $\mathsf{x}\in\mathbf{H}_{s}$
we associate a unique minitwistor line $\mathcal{L}_{\mathsf{x}}\subset\mathbf{MT}_{s}$.
Denote its normal bundle by $\text{Nor}\,(\mathcal{L}_{\mathsf{x}})$.
Kodaira's theorem\footnote{\citet{kodaira1963structure}.} then identifies
the tangent space of $\mathbf{H}_{s}$ at $\mathsf{x}$ with the space
of global sections of $\text{Nor\,}(\mathcal{L}_{\mathsf{x}})$:
\begin{equation}
T_{\mathsf{x}}(\mathbf{H}_{s})\;\simeq\;\Gamma\big(\mathcal{L}(X,\theta);\text{Nor\,}(\mathcal{L}_{\mathsf{x}})\big).
\end{equation}

Let $\mathsf{X}^{K}=(X_{A\dot{A}},\theta_{A}^{\alpha})$ be a coordinate
representative of a point $\mathsf{x}\in\mathbf{H}_{s}$. We define
its associated minitwistor line by
\begin{equation}
\mathcal{L}(X,\theta)\coloneqq\upsilon\big(\mathscr{F}_{\mathsf{X}^{K}}\big)=\big\{\,\pi_{0}\big(\lambda^{A},\lambda^{A}X_{A\dot{A}},\lambda^{A}\theta_{A}^{\alpha}\big)\,\big|\,[\lambda^{A}]\in\mathbf{CP}^{1}\,\big\}.\label{eq:-160}
\end{equation}
By construction, $\mathcal{L}(X,\theta)$ is a rational curve in $\mathbf{MT}_{s}$.
We now check that its normal bundle is $\mathcal{O}(2)$, in accordance
with the Hitchin-Jones definition.

First, project onto the bosonic component via
\begin{equation}
p_{b}:\mathbf{MT}_{s}\longrightarrow\mathbf{MT}\subset\mathbf{CP}^{1}\times\mathbf{CP}^{1},\quad p_{b}(\mathsf{w})\;\coloneqq\;\big([\lambda^{A}],[\mu_{\dot{A}}]\big).
\end{equation}
The image of $\mathcal{L}(X,\theta)$ under $p_{b}$ is the bosonic
minitwistor line $L(X)\coloneqq p_{b}\big(\mathcal{L}(X,\theta)\big)$.

Consider the Veronese-type embedding
\begin{equation}
V\colon\mathbf{MT}\longrightarrow\mathbf{CP}^{3},\quad V\big([\lambda^{A}],[\mu_{\dot{A}}]\big)\;\coloneqq\;[\lambda^{A}\mu_{\dot{A}}].
\end{equation}
Under $V$, the curve $L(X)$ becomes a nonsingular conic in $\mathbf{CP}^{3}$.
Any two such conics intersect in precisely two points, which implies
that the normal bundle of $L(X)$ is $\mathcal{O}(2)$. It follows
that the normal bundle of the full supersymmetric line $\mathcal{L}(X,\theta)$
in $\mathbf{MT}_{s}$ is also $\mathcal{O}(2)$, in agreement with
the Hitchin-Jones definition.

We now demonstrate that $\mathbf{H}_{s}$ indeed parametrises all
minitwistor lines in $\mathbf{MT}_{s}$. We begin with the bosonic
projection. On $\mathbf{CP}^{1}\times\mathbf{CP}^{1}$, the bosonic
incidence relation $\mu_{\dot{A}}=\lambda^{A}X_{A\dot{A}}$ defines
the intersection of the quadric $V(\mathbf{MT})$ with a hyperplane
in $\mathbf{CP}^{3}$. A plane section is tangent to $V(\mathbf{MT})$
precisely when $\det(X^{A\dot{A}})=0$. If instead the section is
non-tangent, then the matrix $X^{A\dot{A}}$ is determined only up
to an overall scale.

Let $W$ denote the set of \emph{non-tangent} hyperplane sections
of $V(\mathbf{MT})$. Equivalently, $W$ is the space of non-null
rays through the origin in complexified Minkowski space $\mathbf{C}^{4}$.
Thus $W$ is diffeomorphic to the projective model $\mathbf{H}$ of
complex hyperbolic space. It follows that, upon adjoining the fermionic
dimensions, the full moduli superspace of minitwistor lines in $\mathbf{MT}_{s}$
is precisely $\mathbf{H}_{s}\simeq\mathbf{H}\times\mathbf{C}^{0|8}$.

\textcompwordmark{}

\paragraph*{The Penrose Integrand.}

In the minitwistor Penrose transform (Subsection \ref{subsec:Minitwistor-Correspondence}),
one treats the minitwistor line $\mathcal{L}(X,\theta)$ as a fibration
over the Riemann sphere $\mathbf{CP}^{1}$. This formulation simplifies
the construction of the top-forms on $\mathcal{L}(X,\theta)$ needed
in the Penrose integral formula.

Every point $\mathsf{w}\in\mathcal{L}(X,\theta)$ arises from a unique
homogeneous coordinate $[\lambda^{A}]\in\mathbf{CP}^{1}$ via $\mathsf{w}=\pi_{0}(\lambda^{A},\lambda^{A}X_{A\dot{A}},\lambda^{A}\theta_{A}^{\alpha})$.
Hence there is a natural projection
\begin{equation}
\mathrm{pr}_{\mathcal{L}}\colon\mathcal{L}(X,\theta)\longrightarrow\mathbf{CP}^{1},\quad\mathsf{w}\;\longmapsto\;[\lambda^{A}].
\end{equation}
This map realises $\mathcal{L}(X,\theta)$ as a holomorphic fibration
over the Riemann sphere. An embedding of the celestial sphere $\mathcal{CS}$
into $\mathcal{L}(X,\theta)$ is then equivalent to a section $s\in\Gamma(\mathbf{CP}^{1};\mathcal{L}(X,\theta))$
such that $\mathrm{d}s:T(\mathbf{CP}^{1})\to T(\mathcal{L}(X,\theta))$
is an isomorphism.

Fix integers $0\leq m,n\leq2$. We define the restriction homomorphism
\begin{equation}
\rho_{\mathcal{L}(X,\theta)}\colon\quad H^{m,n}\big(\mathbf{MT}_{s};\,\mathcal{O}_{\mathbf{C}}(p,q)\big)\;\longrightarrow\;H^{m,n}\big(\mathcal{L}(X,\theta);\,\mathcal{O}_{\mathbf{C}}(p+q)\big)
\end{equation}
by
\begin{equation}
\rho_{\mathcal{L}(X,\theta)}(\varphi)\coloneqq\varphi|_{\mathcal{L}(X,\theta)}\;\coloneqq\;s^{*}(\varphi),
\end{equation}
where $s\colon\mathbf{CP}^{1}\to\mathcal{L}(X,\theta)$ is any holomorphic
embedding. One checks easily that $\rho_{\mathcal{L}(X,\theta)}$
is independent of the choice of $s$. Thus $\rho_{\mathcal{L}(X,\theta)}$
carries a Dolbeault class on $\mathbf{MT}_{s}$ to the corresponding
class on the line. Concretely, take the standard parametrisation
\begin{equation}
s\colon\mathbf{CP}^{1}\longrightarrow\mathcal{L}(X,\theta),\quad[\lambda^{A}]\;\mapsto\;\pi_{0}\big(\lambda^{A},\lambda^{A}X_{A\dot{A}},\lambda^{A}\theta_{A}^{\alpha}\big).
\end{equation}
Then a representative $\varphi$ restricts as:
\begin{equation}
\varphi|_{\mathcal{L}(X,\theta)}(\lambda^{A})\;=\;\varphi(\lambda^{A},\lambda^{A}X_{A\dot{A}},\lambda^{A}\theta_{A}^{\alpha}).
\end{equation}

To construct the Penrose integrand, let $\varphi$ be a $\mathcal{O}_{\mathbf{C}}(p,q)$-valued
differential $(0,1)$-form on $\mathbf{MT}_{s}$. We wish to build
a top-form $\boldsymbol{f}[\varphi]\in\Omega^{1,1}(\mathcal{L}(X,\theta))$
to serve as the integrand in the Penrose formula. Use the homogeneous
coordinate $[\lambda^{A}]$ on each fibre of $\mathcal{L}(X,\theta)\stackrel{\mathrm{pr}_{\mathcal{L}}}{\longrightarrow}\mathbf{CP}^{1}$
to define the holomorphic measure:
\begin{equation}
D\lambda\;\coloneqq\;\varepsilon_{AB}\,\lambda^{A}d\lambda^{B}\;\in\Omega^{1,0}\big(\mathcal{L}(X,\theta);\,\mathcal{O}_{\mathbf{C}}(2)\big).
\end{equation}
Restrict $\varphi$ to $\mathcal{L}(X,\theta)$ via the restriction
homomorphism. Then set:
\begin{equation}
\boldsymbol{f}[\varphi]\;\coloneqq\;D\lambda\,\wedge\,\varphi|_{\mathcal{L}(X,\theta)}(\lambda^{A}).\label{eq:-161}
\end{equation}
Under the rescaling $\lambda^{A}\mapsto t\,\lambda^{A}$ ($t\in\mathbf{C}_{*}$),
one finds $\boldsymbol{f}[\varphi]\mapsto t^{p+q+2}\,\boldsymbol{f}[\varphi]$.
Hence $\boldsymbol{f}[\varphi]$ is a volume form precisely when $p+q+2=0$. 

Finally, define
\begin{equation}
\boldsymbol{f}\colon\Omega^{0,1}\big(\mathbf{MT}_{s};\,\mathcal{O}_{\mathbf{C}}(\Delta-2,-\Delta)\big)\,\longrightarrow\,\Omega^{1,1}\big(\mathcal{L}(X,\theta)\big),\quad\varphi\;\longmapsto\;\boldsymbol{f}[\varphi].
\end{equation}
This map sends a $\mathcal{O}_{\mathbf{C}}(\Delta-2,-\Delta)$-valued
differential $(0,1)$-form to the Penrose integrand on $\mathcal{L}(X,\theta)$.

\subsubsection{Minitwistor Correspondence\label{subsec:Minitwistor-Correspondence}}

The minitwistor Penrose transform provides an isomorphism between
Dolbeault cohomology classes on the homogeneous bundles over $\mathbf{MT}_{s}$
and solutions of the covariant wave equation on the hyperbolic superspace
$\mathbf{H}_{s}$. In particular, it encodes bulk-to-boundary propagators
on $\mathbf{H}_{s}$ in terms of cohomology data on $\mathbf{MT}_{s}$.

Our primary aim in this section is to establish the Penrose machinery
on $\mathbf{MT}_{s}$. This setup leads directly to the celestial
BMSW identity. That identity will serve as the bridge between celestial
amplitudes and holomorphic Wilson lines on minitwistor superspace.
Equivalently, it allows us to generate celestial amplitudes as correlation
functions of the minitwistor sigma-model.

As a further application, we employ the minitwistor Penrose transform
to construct the celestial superwavefunction $\Phi_{\Delta}(\mathsf{X}^{K};\mathsf{Z}^{I})$
for gluons in the spacetime representation. By the minitwistor correspondence,
$\Phi_{\Delta}$ automatically satisfies the covariant wave equation
on the hyperboloid $\mathbf{H}_{s}$. Importantly, the argument $\mathsf{X}^{K}$
lies in $\mathbf{H}_{s}$, not in complexified Minkowski superspace
$\mathbf{C}^{4|8}$. This reflects the fact that $\Phi_{\Delta}$
is defined on the leaves of the hyperbolic foliation of Minkowski
superspace used in the leaf amplitude formalism. In other words, $\Phi_{\Delta}$
is the\emph{ dimensionally reduced }wavefunction\emph{.}

\textcompwordmark{}

\paragraph*{Definition.}

The \emph{minitwistor Penrose transform} is the map
\begin{equation}
\mathcal{P}\colon\quad H^{0,1}\big(\mathbf{MT}_{s};\,\mathcal{O}_{\mathbf{C}}(\Delta-2,-\Delta)\big)\;\longrightarrow\;\Gamma\big(\mathbf{H}_{s};\,\mathcal{O}_{\mathbf{C}}(-\Delta)\big)
\end{equation}
defined by
\begin{equation}
\mathcal{P}\varphi\;\coloneqq\;\int_{\mathcal{L}(X,\theta)}\;\boldsymbol{f}\,[\varphi].\label{eq:-167}
\end{equation}
Here $\boldsymbol{f}[\varphi]\in\Omega^{1,1}(\mathcal{L}(X,\theta))$
is the Penrose integrand introduced in Eq. (\ref{eq:-161}). Equivalently,
writing $\mathsf{X}^{K}=(X_{A\dot{A}},\theta_{A}^{\alpha})$ and using
the explicit form of $\boldsymbol{f}[\varphi]$, one has
\begin{equation}
\mathcal{P}\varphi\,\big(X_{A\dot{A}},\theta_{A}^{\alpha}\big)\;=\;\int_{\mathcal{L}(X,\theta)}\;D\lambda\wedge\varphi\,|_{\mathcal{L}(X,\theta)}\,(\lambda^{A}).
\end{equation}


\paragraph*{Consistency.}

We now verify consistency of the definition of $\mathcal{P}\varphi$.
By construction (cf. end of Subsection \ref{subsec:Preliminaries}),
$\boldsymbol{f}[\varphi]$ is a top-form on $\mathcal{L}(X,\theta)$
that is invariant under the fibre rescaling $\lambda^{A}\mapsto t\,\lambda^{A}$.
Under the base rescaling $X_{A\dot{A}}\mapsto t\,X_{A\dot{A}}$, one
finds $\boldsymbol{f}[\varphi]\mapsto t^{-\Delta}\,\boldsymbol{f}[\varphi]$.
This shows 
\begin{equation}
\mathcal{P}\varphi\,\big(t\,X_{A\dot{A}},\theta_{A}^{\alpha}\big)=t^{-\Delta}\,\mathcal{P}\varphi\,\big(X_{A\dot{A}},\theta_{A}^{\alpha}\big)
\end{equation}
for all $t\in\mathbf{C}_{*}$, so $\mathcal{P}\varphi\in\Gamma\big(\mathbf{H}_{s};\,\mathcal{O}_{\mathbf{C}}(-\Delta)\big)$.

Next, we check independence of representative. Let $\varphi_{1}$
and $\varphi_{2}$ represent the same class in $H^{0,1}\big(\mathbf{MT}_{s};\mathcal{O}_{\mathbf{C}}(\Delta-2,-\Delta)\big)$.
Then $\varphi_{1}=\varphi_{2}+\overline{\partial}\Lambda$ for some
$\Lambda\in\Omega^{0,0}(\mathbf{MT};\mathcal{O}_{\mathbf{C}}(\Delta-2,-\Delta))$.
It follows that
\begin{equation}
\mathcal{P}(\varphi_{1}-\varphi_{2})\;=\;\int_{\mathcal{L}(X,\theta)}\;D\lambda\wedge\overline{\partial}\Lambda|_{\mathcal{L}(X,\theta)}(\lambda^{A})=0,
\end{equation}
since $\overline{\partial}\Lambda$ is exact on each fibre. This shows
that $\mathcal{P}$ is well-defined on Dolbeault cohomology classes.

\textcompwordmark{}

\paragraph*{Differential Equation.}

A simple yet important consequence of the definition of $\mathcal{P}$
is the following differential identity. Let $[\varphi]\in H^{0,1}\big(\mathbf{MT}_{s};\mathcal{O}_{\mathbf{C}}(\Delta-2,-\Delta)\big)$
be any representative. Since $\varphi$ is holomorphic ($\overline{\partial}\,\varphi=0$),
the chain rule on the restriction to the line $\mathcal{L}(X,\theta)$
gives
\begin{equation}
\nabla_{A\dot{A}}\,\varphi|_{\mathcal{L}(X,\theta)}\,(\lambda^{A})\;=\;\lambda_{A}\,\frac{\partial\varphi}{\partial\mu^{\dot{A}}}\Bigg|_{\mathcal{L}(X,\theta)}.
\end{equation}
Moreover, dominated convergence and the mean-value theorem justify
exchanging $\nabla_{A\dot{A}}$ with the integral defining the Penrose
transform. Hence
\begin{equation}
\nabla_{A\dot{A}}\,\mathcal{P}\varphi\;=\;\int_{\mathcal{L}(X,\theta)}\;D\lambda\wedge\lambda_{A}\,\frac{\partial\varphi}{\partial\mu^{\dot{A}}}\Bigg|_{\mathcal{L}(X,\theta)}.
\end{equation}
Acting once more with $\nabla_{A\dot{A}}$ then yields the partial
differential equation:
\begin{equation}
\nabla^{A\dot{A}}\nabla_{A\dot{A}}\,\mathcal{P}\varphi\;=\;0.\label{eq:-162}
\end{equation}


\paragraph*{Proper Functions.}

Our aim is to reformulate Eq. (\ref{eq:-162}) as a covariant wave
equation on hyperbolic superspace $\mathbf{H}_{s}$. To that end,
we introduce the following definition. A section $\Phi$ of the homogeneous
bundle over $\mathbf{H}_{s}$ is said to define a \emph{proper function}
on $\mathbf{H}_{s}$ iff $\mathscr{L}_{\xi}\,\Phi=0$. 

The Penrose transform then lifts to a map
\begin{equation}
\mathcal{P}_{*}\colon\quad H^{0,1}\big(\mathbf{MT}_{s};\,\mathcal{O}_{\mathbf{C}}(\Delta-2,-\Delta)\big)\;\longrightarrow\;\mathscr{C}^{\infty}(\mathbf{H}_{s})
\end{equation}
defined by
\begin{equation}
\mathcal{P}_{*}[\varphi]\;\coloneqq\;\Vert X\Vert^{\Delta}\,\int_{\mathcal{L}(X,\theta)}\;\boldsymbol{f}[\varphi].
\end{equation}
Set $\Phi_{\Delta}\coloneqq\mathcal{P}_{*}[\varphi]$. A direct computation
shows
\begin{equation}
\nabla^{A\dot{A}}\nabla_{A\dot{A}}\,\big(\Vert X\Vert^{-\Delta}\,\Phi_{\Delta}(X_{B\dot{B}},\theta_{C}^{\alpha})\big)\;=\;0.\label{eq:-163}
\end{equation}
The remaining task is to prove that this equation is equivalent to
the eigenvalue problem for the Beltrami-Laplace operator $\square_{\mathbf{H}}$
on the hyperbolic space $\mathbf{H}$.

\textcompwordmark{}

\paragraph*{A Simple Lemma.}

The link between Eq. (\ref{eq:-163}) and the spectral theory of the
wave operator on hyperbolic space is introduced by the following result.
Let
\begin{equation}
\mathcal{J}_{A\dot{A}B\dot{B}}\;\coloneqq\;-i\left(X_{A\dot{A}}\frac{\partial}{\partial X^{B\dot{B}}}-X_{B\dot{B}}\frac{\partial}{\partial X^{A\dot{A}}}\right)\label{eq:-164}
\end{equation}
be the generator of the Lie algebra of isometries of $\mathbf{H}$.
Here $\{\partial/\partial X^{A\dot{A}}\}$ is the coordinate frame.
Define the quadratic Casimir operator by:
\begin{equation}
\mathcal{Q}\;\coloneqq\;\frac{1}{2}\mathcal{J}_{A\dot{A}B\dot{B}}\mathcal{J}^{A\dot{A}B\dot{B}}.\label{eq:-165}
\end{equation}
Then for every proper function $\Phi_{\Delta}\in\mathscr{C}^{\infty}(\mathbf{H}_{s})$
one has the equivalence:
\begin{equation}
\nabla^{A\dot{A}}\nabla_{A\dot{A}}\,\big(\Vert X\Vert^{-\Delta}\,\Phi_{\Delta}\big)\;=\;0\quad\iff\quad\square_{\mathbf{H}}\Phi_{\Delta}\;=\;\Delta(\Delta-2)\Phi_{\Delta}.
\end{equation}

To establish this result, we first expand the left-hand side of Eq.
(\ref{eq:-163}) as:
\begin{equation}
\Vert X\Vert^{2}\,\nabla^{A\dot{A}}\nabla_{A\dot{A}}\,\Phi_{\Delta}+2\Delta\,\mathscr{L}_{\xi}\Phi_{\Delta}\;=\;\Delta(\Delta-2)\,\Phi_{\Delta}.
\end{equation}
Since $\Phi_{\Delta}$ is assumed proper ($\mathscr{L}_{\xi}\,\Phi_{\Delta}=0$),
this reduces to:
\begin{equation}
\Vert X\Vert^{2}\,\nabla^{A\dot{A}}\nabla_{A\dot{A}}\Phi_{\Delta}\;=\;\Delta(\Delta-2)\,\Phi_{\Delta}.\label{eq:-166}
\end{equation}

On the other hand, a direct computation of the Casimir operator $\mathcal{Q}$
using Eqs. (\ref{eq:-164}) and (\ref{eq:-165}) gives:
\begin{equation}
\mathcal{Q}\,\Phi_{\Delta}\;=\;\Vert X\Vert^{2}\,\nabla^{A\dot{A}}\nabla_{A\dot{A}}\,\Phi_{\Delta}+2\,X^{A\dot{A}}\nabla_{A\dot{A}}\,\Phi_{\Delta}+X^{A\dot{A}}\nabla_{A\dot{A}}\big(X^{B\dot{B}}\nabla_{B\dot{B}}\Phi_{\Delta}\big).
\end{equation}
It follows that Eq. (\ref{eq:-166}) is equivalent to:
\begin{equation}
\mathcal{Q}\,\Phi_{\Delta}\;=\;\Delta(\Delta-2)\,\Phi_{\Delta}.
\end{equation}
Finally, the lemma follows from the well-known result (e.g. \citet{fronsdal1974elementary})
that on homogeneous spaces $\mathcal{Q}=\square_{\mathbf{H}}$.

\textcompwordmark{}

\paragraph*{Main Result.}

The preceding lemma implies that the modified minitwistor Penrose
transform $\mathcal{P}_{*}$ sends $\overline{\partial}$-cohomology
classes on $\mathbf{MT}_{s}$ to solutions of the eigenvalue problem
for the Beltrami-Laplace operator on $\mathbf{H}_{s}$. We now formalise
this statement.

Let $U$ be any differentiable manifold and let $\mathcal{T\colon\mathscr{C}^{\infty}}(U)\to\mathscr{C}^{\infty}(U)$
be a linear differential operator. We define its kernel by:
\begin{equation}
\text{ker}\,(U;\mathcal{T})\;\coloneqq\;\big\{\,\Phi\in\mathscr{C}^{\infty}(U)\,\big|\,\mathcal{T}\Phi=0\,\big\}.
\end{equation}
It follows that
\begin{equation}
\mathcal{P}_{*}\big(H^{0,1}\big(\mathbf{MT}_{s};\,\mathcal{O}_{\mathbf{C}}(\Delta-2,-\Delta)\big)\big)\;\subseteq\;\mathrm{ker}\big(\mathbf{H}_{s};\,\square_{\mathbf{H}}-\Delta(\Delta-2)\big).
\end{equation}
Furthermore, the invertibility of the X-ray transform on projective
spaces (\citet{gel_fand2003selected}) and a homological argument
yield a stronger result\footnote{See \citet{jones1984minitwistors} and \citet{bailey1998twistor}.}:
\begin{equation}
H^{0,1}\big(\mathbf{MT}_{s};\,\mathcal{O}_{\mathbf{C}}(\Delta-2,-\Delta)\big)\;\simeq\;\mathrm{ker}\big(\mathbf{H}_{s};\,\square_{\mathbf{H}}-\Delta(\Delta-2)\big).
\end{equation}

We note that the integral transform in Eq. (\ref{eq:-167}) admits
an explicit inversion in the form of a Leray residue formula. This
construction builds on Gindikin's analyses of the Radon and John's
transforms (\citet{gindikin1990cauchy,gindikin2007second,gindikin2014inversion}),
the Cauchy-Fantappi� formula, and Leray's theory of multidimensional
residues.

\textcompwordmark{}

\paragraph*{Celestial Superwavefunction.}

The simplest non-trivial application of the minitwistor correspondence
is the derivation of the celestial superwavefunction $\Phi_{\Delta}(\mathsf{X}^{K};\mathsf{Z}^{I})$
from the minitwistor wavefunction $\Psi_{\Delta}^{p}(\mathsf{W}^{I};\mathsf{Z}^{I})$.
Here $\Phi_{\Delta}$ is a distribution on hyperbolic superspace $\mathbf{H}_{s}$,
and $\Psi_{\Delta}^{p}$ is a current on minitwistor space $\mathbf{MT}_{s}$.
These are related by a suitable modification of the Penrose transform.

\citet{d1996radon} generalised the Penrose transform from its standard
double-fibration formulation over flag manifolds to the setting of
$\mathscr{D}$-modules. \citet{david2003x} then applied their theory
to currents of differential forms valued in vector bundles. \citet{voronov1991geometric}
provided a detailed extension of geometric integration theory to supermanifolds\footnote{For a review, cf. \citet{witten2012notes}.}.
Using this framework, we lift the minitwistor Penrose transform to
the supercurrent category and denote the resulting map by:
\begin{equation}
\mathcal{P}'\colon\quad\mathscr{D}'_{0,1}\big(\mathbf{MT}_{s};\,\mathcal{O}_{\mathbf{C}}(\Delta-2,-\Delta)\big)\;\longrightarrow\;\mathscr{D}'(\mathbf{H}_{s}).
\end{equation}
Here $\mathscr{D}'(\mathbf{H}_{s})$ is the module of distributions
on the supermanifold $\mathbf{H}_{s}$. The covariant wave operator
acts on $\mathscr{D}'(\mathbf{H}_{s})$ by duality, and we denote
its kernel by:
\begin{equation}
\mathrm{ker}'\big(\mathbf{H}_{s};\,\square_{\mathbf{H}}-\Delta(\Delta-2)\big)\;\coloneqq\;\big\{\,\omega\in\mathscr{D}'(\mathbf{H}_{s})\,\big|\,\square_{\mathbf{H}}\,\omega\;=\;\Delta(\Delta-2)\,\omega\,\big\}.
\end{equation}
The minitwistor correspondence in the current-distribution category
now reads:
\begin{equation}
H_{0,1}\big(\mathbf{MT}_{s};\,\mathcal{O}_{\mathbf{C}}(\Delta-2,-\Delta)\big)\;\simeq\;\mathrm{ker}'\big(\mathbf{H}_{s};\,\square_{\mathbf{H}}-\Delta(\Delta-2)\big).
\end{equation}

Fix a dual minitwistor $[\mathsf{Z}^{I}]\in\mathbf{MT}_{s}^{*}$.
The representative superwavefunction
\begin{equation}
\Psi_{\Delta}^{2}\big(\,\cdot\,;\mathsf{Z}^{I}\big)\in\mathscr{D}'_{0,1}\big(\mathbf{MT}_{s};\,\mathcal{O}_{\mathbf{C}}(\Delta-2,-\Delta)\big)
\end{equation}
lies in the domain of $\mathcal{P}'$. We therefore define the \emph{celestial
superwavefunction} for gluons of conformal weight $\Delta$ by:
\begin{equation}
\Phi_{\Delta}\big(\,\cdot\,;\mathsf{Z}^{I}\big)\in\mathscr{D}'(\mathbf{H}_{s}),\quad\Phi_{\Delta}\big(\,\cdot\,;\mathsf{Z}^{I}\big)\;\coloneqq\;\mathcal{P}'\big[\Psi_{\Delta}^{2}(\,\cdot\,;\mathsf{Z}^{I})\big].
\end{equation}
Explicitly, one has:
\begin{equation}
\Phi_{\Delta}\big(X_{A\dot{A}},\theta_{A}^{\alpha};\mathsf{Z}^{I}\big)\;=\;\Vert X\Vert^{\Delta}\,\int_{\mathcal{L}(X,\theta)}\;D\lambda\wedge\Psi_{\Delta}^{2}\,\big|_{\mathcal{L}(X,\theta)}\big(\lambda^{A};\mathsf{Z}^{I}\big).\label{eq:-168}
\end{equation}
The restriction of a minitwistor superwavefunction $\Psi_{\Delta}^{p}$
to the line $\mathcal{L}(X,\theta)$ is the supercurrent
\begin{equation}
\Psi_{\Delta}^{p}\,\big|_{\mathcal{L}(X,\theta)}\big(\,\cdot\,;\mathsf{Z}^{I}\big)\in\mathscr{D}'_{0,1}\big(\mathcal{L}(X,\theta);\,\mathcal{O}_{\mathbf{C}}(-p)\big)
\end{equation}
given by:
\begin{equation}
\Psi_{\Delta}^{p}\,\big|_{\mathcal{L}(X,\theta)}\big(\lambda^{A};\mathsf{Z}^{I}\big)\;=\;\overline{\delta}_{p-\Delta}(z^{A},\lambda^{A})\,\frac{\mathcal{C}(\Delta)}{\langle\lambda|X|\bar{z}]^{\Delta}}\,\exp\left(i\frac{\left\langle z\iota\right\rangle }{\left\langle \lambda\iota\right\rangle }\langle\lambda|\theta\cdot\eta\rangle\right).\label{eq:-169}
\end{equation}
Substituting into the Penrose integral (Eq. (\ref{eq:-168})) yields:
\begin{equation}
\Phi_{\Delta}\big(X_{A\dot{A}},\theta_{A}^{\alpha};\mathsf{Z}^{I}\big)\;=\;K_{\Delta}\big(X_{A\dot{A}};z^{A},\bar{z}_{\dot{A}}\big)\,e^{i\langle z|\theta\cdot\eta\rangle},
\end{equation}
where the bulk-to-boundary propagator $K_{\Delta}$ on $\mathbf{H}$
is
\begin{equation}
K_{\Delta}\big(X_{A\dot{A}};z^{A},\bar{z}_{\dot{A}}\big)\;=\;\frac{\mathcal{C}(\Delta)}{\langle z|\mathcal{R}|\bar{z}]^{\Delta}},\quad\mathcal{R}_{A\dot{A}}\;=\;\frac{X_{A\dot{A}}}{\Vert X\Vert}.
\end{equation}

The physical interpretation of $\Phi_{\Delta}$ proceeds via its relation
to the $\mathcal{N}=4$ conformal primary wavefunction $\phi_{\Delta}$.
This wavefunction is given by:
\begin{equation}
\phi_{\Delta}(x^{\mu},\theta_{A}^{\alpha};z^{A},\bar{z}_{\dot{A}},\eta^{\alpha})\;=\;\frac{\Gamma(\Delta)}{(\varepsilon-iq(z,\bar{z})\cdot x)^{\Delta}}\,e^{i\langle z|\theta\cdot\eta\rangle},
\end{equation}
where $q^{\mu}(z,\bar{z})\coloneqq z^{A}(\sigma^{\mu})_{A\dot{A}}\bar{z}^{\dot{A}}$
is the standard null four-vector. The superwavefunction $\phi_{\Delta}$
describes a gluon of conformal weight $\Delta$ and helicity state
$\eta^{\alpha}$. It is obtained by extending the analyses of \citet{banerjee2019null,banerjee2019symmetries,banerjee2020conformal,banerjee2020conformal1}
to the Lorentz \emph{supergroup}.

Dimensional reduction of $\phi_{\Delta}$ onto the leaves of the hyperbolic
foliation of Klein superspace $\mathbf{K}^{4|8}\subset\mathbf{C}^{4|8}$
yields the celestial superwavefunction $\Phi_{\Delta}$. In other
words, $\Phi_{\Delta}$ is obtained by restricting $\phi_{\Delta}$
to each hyperbolic slice in the leaf amplitude formalism\footnote{\citet{banerjee2025worldsheet}.}.
An important property of the family $\{\Phi_{\Delta}\}$ follows from
the analysis of the spectral theory for primary fields in the $H_{3}^{+}$-WZNW
model\footnote{\citet{teschner1997conformal,teschner1999mini,teschner1999structure,teschner2000operator,ribault2005h3+}.}.
The set $\{\Phi_{\Delta}\}$ is both complete and $\delta$-function
orthonormal. 

\subsubsection{Boels-Mason-Skinner-Witten \emph{(BMSW)} Identity}

Building on the constructions of \citet{boels2007supersymmetric}
and \citet{witten2004perturbative}, we now derive the central result
of this subsection. We refer to this as the celestial Boels-Mason-Skinner-Witten
(BMSW) identity. It takes the form of an integral formula equating
the $n$-fold minitwistor Penrose transform of the superwavefunctions
$\Psi_{\Delta}^{p}$, weighted by a Lie-algebra-valued logarithmic
differential form, with the kernel of the Mellin-transformed Parke-Taylor
factors appearing in the celestial leaf amplitudes for gluonic scattering.
This identity will serve as our key formula for both geometric and
dynamical interpretations of the $\mathcal{N}=4$ SYM celestial amplitudes.

To derive the integral identity, fix a dual minitwistor point $[\mathsf{Z}^{I}]\in\mathbf{MT}_{s}^{*}$,
and let $\mathsf{W}^{I}=(\lambda^{A},\mu_{\dot{A}},\psi^{\alpha})$
parametrise minitwistor superspace. Let $\mathbf{G}$ be a compact,
simply-connected and semisimple Lie group with Lie algebra $\mathfrak{g}$.
Choose a Lie-algebra-valued section
\begin{equation}
g^{\mathsf{a}}\in\Gamma\big(\mathbf{CP}^{1};\mathfrak{g}\otimes\mathcal{O}_{\mathbf{C}}(-2)\big).
\end{equation}
In the BMSW identity, we use the minitwistor wavefunction
\begin{equation}
\Psi_{\Delta}^{0}\big(\,\cdot\,;\mathsf{Z}^{I}\big)\in\mathscr{D}'_{0,1}\big(\mathbf{MT}_{s};\,\mathcal{O}_{\mathbf{C}}(\Delta,-\Delta)\big),
\end{equation}
which we abbreviate by $\Psi_{\Delta}$.

Define the Lie-algebra-valued supercurrent
\begin{equation}
\mathcal{K}_{\Delta,g}^{\mathsf{a}}\in\mathscr{D}'_{0,1}\big(\mathbf{MT}_{s};\,\mathfrak{g}\otimes\mathcal{O}_{\mathbf{C}}(\Delta-2,-\Delta)\big)
\end{equation}
by
\begin{equation}
\mathcal{K}_{\Delta,g}^{\mathsf{a}}\big(\mathsf{W}^{I}\big)\;\coloneqq\;g^{\mathsf{a}}(\lambda^{A})\,\Psi_{\Delta}\big(\mathsf{W}^{I};\mathsf{Z}^{I'}\big).
\end{equation}
Here and below we omit the symbol $\otimes$ for simple juxtaposition.

Since $\mathcal{K}_{\Delta,g}^{\mathsf{a}}$ lies in the domain of
$\mathcal{P}'$, we may write the Penrose integral as follows:
\begin{equation}
\mathcal{P}'\big[\mathcal{K}_{\Delta,g}^{\mathsf{a}}\big]\,\big(X_{A\dot{A}},\theta_{A}^{\alpha}\big)\;=\;\int_{\mathcal{L}(X,\theta)}\;D\lambda\wedge g^{\mathsf{a}}(\lambda^{A})\,\Psi_{\Delta}\,\big|_{\mathcal{L}(X,\theta)}(\lambda^{A};\mathsf{Z}^{I}).
\end{equation}
Using the restriction formula in Eq. (\ref{eq:-169}), we find that
the Penrose integral evaluates to:
\begin{equation}
\int_{\mathcal{L}(X,\theta)}\;D\lambda\wedge g^{\mathsf{a}}(\lambda^{A})\,\Psi_{\Delta}\,|_{\mathcal{L}(X,\theta)}(\lambda^{A};\mathsf{Z}^{I})\;=\;\frac{\mathcal{C}(\Delta)}{\langle z|X|\bar{z}]^{\Delta}}\,e^{i\langle z|\theta\cdot\eta\rangle}\,g^{\mathsf{a}}(z^{A}).
\end{equation}
Define the $n$-fold Cartesian product of the minitwistor line by
$\mathcal{L}_{n}\coloneqq\bigtimes{}^{n}\;\mathcal{L}(X,\theta)$.
An inductive argument then shows that:
\begin{equation}
\underset{\mathcal{L}_{n}\,\,\,}{\int}\;\bigwedge_{i=1}^{n}\;D\lambda_{i}\wedge g^{\mathsf{a}_{i}}(\lambda_{i}^{A})\,\Psi_{\Delta_{i}}\,\big|_{\mathcal{L}(X,\theta)}\big(\lambda^{A};\mathsf{Z}^{I}\big)\;=\;\prod_{i=1}^{n}\;\frac{\mathcal{C}(\Delta_{i})}{\langle z_{i}|X|\bar{z}_{i}]^{\Delta_{i}}}\,e^{i\langle z_{i}|\theta\cdot\eta_{i}\rangle}\,g^{\mathsf{a}_{i}}(\lambda_{i}^{A}).\label{eq:-170}
\end{equation}
Next, let $\{\mathsf{T}^{\mathsf{a}}\}$ be a basis for $\mathfrak{g}$.
Define
\begin{equation}
g^{\mathsf{a}_{i}}(\lambda_{i}^{A})\;=\;\frac{\mathsf{T}^{\mathsf{a}_{i}}}{\lambda_{i}\cdot\lambda_{i+1}}.
\end{equation}
Introduce the Lie-algebra-valued logarithmic differential form:
\begin{equation}
\boldsymbol{\omega}^{\mathsf{a}_{i}}(\lambda_{i}^{A})\;\coloneqq\;\mathsf{T}^{\mathsf{a}_{i}}\,\frac{D\lambda_{i}}{\lambda_{i}\cdot\lambda_{i+1}}.\label{eq:-178}
\end{equation}
Then Eq. (\ref{eq:-170}) reduces to the \emph{supersymmetric} \emph{celestial
BMSW identity}:
\begin{equation}
\underset{\mathcal{L}_{n}\,\,\,}{\int}\;\bigwedge_{i=1}^{n}\;\boldsymbol{\omega}^{\mathsf{a}_{i}}(\lambda_{i}^{A})\wedge\Psi_{\Delta_{i}}\,\big|_{\mathcal{L}(X,\theta)}\big(\lambda_{i}^{A};\mathsf{Z}_{i}^{I}\big)\;=\;\prod_{i=1}^{n}\;\frac{\mathcal{C}(\Delta_{i})}{\langle z_{i}|X|\bar{z}_{i}]^{\Delta_{i}}}\,e^{i\langle z_{i}|\theta\cdot\eta_{i}\rangle}\,\frac{\mathsf{T}^{\mathsf{a}_{i}}}{z_{i}\cdot z_{i+1}}.
\end{equation}


\subsection{Supersymmetric Celestial RSVW Identity}

In the preceding subsection, we derived the integral kernel for the
Mellin-transformed Parke-Taylor factors, namely
\begin{equation}
\prod_{i=1}^{n}\;\frac{\mathcal{C}(\Delta_{i})}{\langle z_{i}|X|\bar{z}_{i}]^{\Delta_{i}}}\,e^{i\langle z_{i}|\theta\cdot\eta_{i}\rangle}\,\frac{\mathsf{T}^{\mathsf{a}_{i}}}{z_{i}\cdot z_{i+1}}\,,\label{eq:-182}
\end{equation}
as an $n$-fold Penrose transform of the minitwistor superwavefunctions
$\Psi_{\Delta}$. We called this the supersymmetric celestial BMSW
identity. It provides a direct map from celestial leaf amplitudes
to correlation functions in the minitwistor sigma-model, via the power-series
expansion of the Quillen determinant line bundle.

We now develop a complementary formulation by translating celestial
leaf amplitudes into minitwistor amplitudes. Direct evaluation of
the minitwistor transform on distributional data is challenging. Instead,
we will derive a convenient rewriting of the identity resolution.
Substituting this into the leaf amplitudes then produces their minitwistor
counterparts in a systematic manner.

Our approach follows \citet{roiban2004googly,roiban2004tree,roiban2005dissolving}
and \citet{witten2004perturbative}. In contrast to the BMSW identity,
where one integrates the superwavefunction $\Psi_{\Delta}$ over the
minitwistor line $\mathcal{L}(X,\theta)$, we impose the supersymmetric
incidence relations directly on $\Psi_{\Delta}$:
\begin{equation}
\mu_{\dot{A}}=\lambda^{A}\,X_{A\dot{A}},\quad\psi^{\alpha}=\lambda^{A}\,\theta_{A}^{\alpha}.
\end{equation}
These relations are enforced via delta ``functions'' on $\mathbf{MT}_{s}$.
The resulting multidimensional minitwistor transform reproduces the
Parke-Taylor kernel of Eq. (\ref{eq:-182}). We term this the \emph{supersymmetric
celestial Roiban-Spradlin-Volovich-Witten} (RSVW) \emph{identity}.
Its bosonic version first appeared in Mol (2025).

\subsubsection{Preliminaries}

Imposing the incidence relations on minitwistor superspace $\mathbf{MT}_{s}$
via distributions (or, more precisely, via currents of differential
forms) is subtler than in standard projective twistor superspace $\mathbf{PT}^{3|4}$.
In $\mathbf{MT}_{s}$, the spinor coordinates $\lambda^{A}$ and $\mu_{\dot{A}}$
carry independent scaling weights. We must however construct the integrand
as a \emph{legitimate} differential form on $\mathbf{MT}_{s}$, rather
than merely as a section of the homogeneous bundle $\mathcal{O}_{\mathbf{C}}(p,q)$.
Moreover, the form must have the correct bi-degree in the exterior
algebra of $\mathbf{MT}_{s}$.

We therefore dedicate this first subsection to a careful construction
of the integral over $\mathbf{MT}_{s}$ of the superwavefunction $\Psi_{\Delta}$
that properly implements the incidence relations.

\textcompwordmark{}

We again consider a compact, simply-connected, semisimple gauge group
$\mathbf{G}$ with Lie algebra $\mathfrak{g}$. Choose a Lie-algebra-valued
section $g^{\mathsf{a}}\in\Gamma(\mathbf{CP}^{1};\mathfrak{g}\otimes\mathcal{O}_{\mathbf{C}}(-2)).$
Let $\boldsymbol{k}_{\Delta,g}^{\mathsf{a}}$ be the $(2,1)$-supercurrent
on $\mathbf{MT}_{s}$ defined by:
\begin{equation}
\boldsymbol{k}_{\Delta,g}^{\mathsf{a}}\;\coloneqq\;g^{\mathsf{a}}(\lambda^{A})\;\overline{\delta}_{\Delta}\big(\mu_{\dot{A}},\lambda^{A}X_{A\dot{A}}\big)\wedge\delta^{0|4}\big(\psi^{\alpha}-\lambda^{A}\theta_{A}^{\alpha}\big)\wedge D^{2|4}\mathsf{W}.
\end{equation}
Under the rescaling
\begin{equation}
\mathsf{W}^{I}\;\longmapsto\;\mathsf{W}'{}^{I}\coloneqq(t_{1}\,\lambda^{A},t_{2}\,\mu_{\dot{A}},t_{1}\,\psi^{\alpha}),
\end{equation}
one checks
\begin{equation}
\boldsymbol{k}_{\Delta,g}^{\mathsf{a}}\;\longmapsto\;t_{1}^{-\Delta}t_{2}^{\Delta}\,\boldsymbol{k}_{\Delta,g}^{\mathsf{a}}.
\end{equation}
Hence $\boldsymbol{k}_{\Delta,g}^{\mathsf{a}}$ defines an $\mathcal{O}_{\mathfrak{g}}(-\Delta,\Delta)$-valued
supercurrent of bi-degree $(2,1)$.

Now fix a dual minitwistor $[\mathsf{Z}^{I}]\in\mathbf{MT}_{s}^{*}$.
The superwavefunction $\Psi_{\Delta}\coloneqq\Psi_{\Delta}^{0}$ transforms
under $\mathsf{W}^{I}\mapsto\mathsf{W}'{}^{I}$ as
\begin{equation}
\Psi_{\Delta}\big(\mathsf{W}^{I};\mathsf{Z}^{I}\big)\;\longmapsto\;t_{1}^{\Delta}t_{2}^{-\Delta}\,\Psi_{\Delta}\big(\mathsf{W}^{I};\mathsf{Z}^{I}\big).
\end{equation}
Then
\begin{equation}
\mathcal{V}_{\Delta,g}^{\mathsf{a}}\;\coloneqq\,\Psi_{\Delta}\big(\mathsf{W}^{I};\mathsf{Z}^{I}\big)\wedge\boldsymbol{k}_{\Delta,g}^{\mathsf{a}}\big(\mathsf{W}^{I};\mathsf{X}^{K}\big)\label{eq:-180}
\end{equation}
defines a bi-degree $(2,2)$ supercurrent on $\mathbf{MT}_{s}$ that
is invariant under $\mathsf{W}^{I}\mapsto\mathsf{W}'{}^{I}$.

Since
\begin{equation}
\mathbf{MT}_{s}\subset\mathbf{CP}^{1|4}\times\mathbf{CP}^{1}
\end{equation}
and $\mathbf{CP}^{1|4}\times\mathbf{CP}^{1}$ is compact, any smooth
proper function $\Phi\colon\mathbf{MT}_{s}\to\mathbf{C}$ (i.e. $\mathscr{L}_{\xi}\Phi=0$)
serves as a test form of bi-degree $(0,0)$. In particular, choosing
$\Phi$ to be the characteristic function $\boldsymbol{\chi}$ of
the quadric, we regard $\mathcal{V}_{\Delta,g}^{\mathsf{a}}$ as a
\emph{volume form} on $\mathbf{MT}_{s}$. Hence the integral
\begin{equation}
\langle\mathcal{V}_{\Delta,g}^{\mathsf{a}},\boldsymbol{\chi}\rangle=\underset{\mathbf{MT}_{s}\,\,\,}{\int}\mathcal{V}_{\Delta,g}^{\mathsf{a}}\;=\;\underset{\mathbf{MT}_{s}\,\,\,}{\int}\Psi_{\Delta}\big(\mathsf{W}^{I};\mathsf{Z}^{I'}\big)\wedge\boldsymbol{k}_{\Delta,g}^{\mathsf{a}}\big(\mathsf{W}^{I};\mathsf{X}^{K}\big)\label{eq:-181}
\end{equation}
is well-defined. 

Substituting the explicit form of the superwavefunction $\Psi_{\Delta}$
(Eq. (\ref{eq:-39})) into the definition of $\mathcal{V}_{\Delta,g}^{\mathsf{a}}$
(Eq. (\ref{eq:-180})), we obtain:
\begin{align}
\mathcal{V}_{\Delta,g}^{\mathsf{a}} & =g^{\mathsf{a}}(\lambda^{A})\;\frac{\mathcal{C}(\Delta)}{[\mu\bar{z}]^{\Delta}}\,\exp\left(i\frac{\left\langle z\iota\right\rangle }{\left\langle \lambda\iota\right\rangle }\psi\cdot\eta\right)\;\overline{\delta}_{-\Delta}\big(z^{A},\lambda^{A}\big)\wedge\overline{\delta}_{\Delta}\big(\mu_{\dot{A}},\lambda^{A}\,X_{A\dot{A}}\big)\\
 & \quad\wedge\delta^{0|4}\big(\psi^{\alpha}-\lambda^{A}\,\theta_{A}^{\alpha}\big)\wedge D^{2|4}\mathsf{W}.
\end{align}
The integral in Eq. (\ref{eq:-181}) then evaluates to:
\begin{align}
 & \underset{\mathbf{MT}_{s}\,\,\,}{\int}D^{2|4}\mathsf{W}\quad\Psi_{\Delta}\big(\mathsf{W}^{I};\mathsf{Z}^{I'}\big)\;\overline{\delta}_{\Delta}\big(\mu_{\dot{A}},\lambda^{A}\,X_{A\dot{A}}\big)\,\delta^{0|4}\big(\psi^{\alpha}-\lambda^{A}\,\theta_{A}^{\alpha}\big)\;g^{\mathsf{a}}(\lambda^{A})\label{eq:-176}\\
 & \quad=\;\frac{\mathcal{C}(\Delta)}{\langle z|X|\bar{z}]^{\Delta}}\,e^{i\langle z|\theta\cdot\eta\rangle}\,g^{\mathsf{a}}(z^{A}).\label{eq:-175}
\end{align}
We have denoted wedge products by juxtaposition for compactness. By
construction of $\mathcal{V}_{\Delta,g}^{\mathsf{a}}$, this integral
is projectively well-defined.

\subsubsection{A Simple Lemma}

The next step is to unify the delta-currents in Eq. (\ref{eq:-176})
into the minitwistor delta $\overline{\delta}_{\Delta_{1},\Delta_{2}}^{2|4}$
introduced in Subsection \ref{subsec:Completeness-Relation} (Eq.
(\ref{eq:-179})). We achieve this via the following result.

\textcompwordmark{}

\paragraph*{Lemma.}

Let $\mathsf{Y}^{I}\colon\mathbf{CP}^{1}\to\mathcal{L}(X,\theta)\subset\mathbf{MT}_{s}$
be the embedding of the holomorphic celestial sphere into the minitwistor
line, defined by
\begin{equation}
\mathsf{Y}^{I}\big(\sigma^{A}\big)\;\coloneqq\;\big(\sigma^{A},\sigma^{A}\,X_{A\dot{A}},\sigma^{A}\,\theta_{A}^{\alpha}\big).
\end{equation}
By construction, $\mathsf{Y}^{I}$ is a global section of $\mathcal{L}(X,\theta)\to\mathbf{CP}^{1}$,
and its differential $\text{d}\mathsf{Y}^{I}$ is an isomorphism on
tangent spaces. Hence each point $\mathsf{w}\in\mathcal{L}(X,\theta)$
corresponds to a unique projective spinor $[\sigma^{A}]\in\mathbf{CP}^{1}$.
We use these homogeneous coordinates to orient $\mathcal{L}(X,\theta)$.
Explicitly, we define the holomorphic measure:
\begin{equation}
D\sigma\;\in\;\Omega^{1,0}\big(\mathcal{L}(X,\theta);\,\mathcal{O}_{\mathbf{C}}(2)\big),\quad D\sigma\coloneqq\varepsilon_{AB}\,\sigma^{A}\,d\sigma^{B}.
\end{equation}
With these preparations, the following equality of currents holds
on $\mathcal{L}(X,\theta)$:
\begin{equation}
\overline{\delta}_{\Delta}\big(\mu_{\dot{A}},\lambda^{A}\,X_{A\dot{A}}\big)\,\delta^{0|4}\big(\psi^{\alpha}-\lambda^{A}\,\theta_{A}^{\alpha}\big)\;g^{\mathsf{a}}(\lambda^{A})=\underset{\mathbf{CP}^{1}\,\,\,}{\int}D\sigma\;g^{\mathsf{a}}(\sigma^{A})\;\wedge\;\overline{\delta}_{-\Delta,\Delta}^{2|4}\big(\mathsf{W}^{I};\,\mathsf{Y}^{I'}(\sigma^{A})\big).\label{eq:-177}
\end{equation}


\paragraph*{Proof of Lemma.}

The first step uses the resolution of identity for the holomorphic
delta $\overline{\delta}^{2}$ on $\mathbf{C}^{2}$. We rewrite the
left-hand side as:
\begin{align}
 & \overline{\delta}_{\Delta}\big(\mu_{\dot{A}},\lambda^{A}\,X_{A\dot{A}}\big)\,\delta^{0|4}\big(\psi^{\alpha}-\lambda^{A}\,\theta_{A}^{\alpha}\big)\;g^{\mathsf{a}}(\lambda^{A})\label{eq:-171}\\
 & \quad=\;\underset{\mathbf{C}^{2}\,\,\,}{\int}d^{2}s\;\overline{\delta}^{2}\big(\lambda^{A}-s^{A}\big)\,\overline{\delta}_{\Delta}\big(\mu_{\dot{A}},s^{A}\,X_{A\dot{A}}\big)\,\delta^{0|4}\big(\psi^{\alpha}-s^{A}\,\theta_{A}^{\alpha}\big)\;g^{\mathsf{a}}(s^{A}).\label{eq:-172}
\end{align}
Next we parametrise $s^{A}=t\,\sigma^{A}$ with $t\in\mathbf{C}_{*}$
and $[\sigma^{A}]\in\mathbf{CP}^{1}$. The measure decomposes as
\begin{equation}
d^{2}s\;=\;D\sigma\,\wedge\,\frac{dt}{t}\;t^{2},
\end{equation}
so Eq. (\ref{eq:-172}) becomes:
\begin{align}
 & \overline{\delta}_{\Delta}\big(\mu_{\dot{A}},\lambda^{A}\,X_{A\dot{A}}\big)\,\delta^{0|4}\big(\psi^{\alpha}-\lambda^{A}\,\theta_{A}^{\alpha}\big)\;g^{\mathsf{a}}(\lambda^{A})\\
 & =\underset{\mathbf{CP}^{1}\,\,\,}{\int}D\sigma\;\underset{\mathbf{C}_{*}\,\,\,}{\int}\frac{dt}{t}\;t^{-\Delta}\,\overline{\delta}^{2}\big(\lambda^{A}-t\sigma^{A}\big)\,\overline{\delta}_{\Delta}\big(\mu_{\dot{A}},\sigma^{A}\,X_{A\dot{A}}\big)\,\delta^{0|4}\big(\psi^{\alpha}-t\sigma^{A}\,\theta_{A}^{\alpha}\big)\;g^{\mathsf{a}}(\sigma^{A}).
\end{align}
Carrying out the $t$-integral yields the minitwistor delta $\overline{\delta}_{-\Delta,\Delta}^{2|4}$.
Hence
\begin{align}
 & \overline{\delta}_{\Delta}\big(\mu_{\dot{A}},\lambda^{A}\,X_{A\dot{A}}\big)\,\delta^{0|4}\big(\psi^{\alpha}-\lambda^{A}\,\theta_{A}^{\alpha}\big)\;g^{\mathsf{a}}(\lambda^{A})\label{eq:-173}\\
 & =\underset{\mathbf{CP}^{1}\,\,\,}{\int}D\sigma\;g^{\mathsf{a}}(\sigma^{A})\;\wedge\;\overline{\delta}_{-\Delta,\Delta}^{2|4}\big(\lambda^{A},\mu_{\dot{A}},\psi^{\alpha}\big|\sigma^{A},\sigma^{A}\,X_{A\dot{A}},\sigma^{A}\,\theta_{A}^{\alpha}\big).\label{eq:-174}
\end{align}

Finally, substituting $\mathsf{W}^{I}=(\lambda^{A},\mu_{\dot{A}},\psi^{\alpha})$
and $\mathsf{Y}^{I}(\sigma^{A})$ into Eq. (\ref{eq:-174}) completes
the proof of the lemma.

\subsubsection{Main Result}

We are now ready to present the main result of this subsection.

Substituting the lemma's identity (Eq. (\ref{eq:-177})) into the
integral of the top-form $\mathcal{V}_{\Delta,g}^{\mathsf{a}}$ (Eq.
(\ref{eq:-176})) yields:
\begin{align}
 & \underset{\mathbf{MT}_{s}\,\,\,}{\int}D^{2|4}\mathsf{W}\;\Psi_{\Delta}\big(\mathsf{W}^{I};\mathsf{Z}^{I}\big)\;\underset{\mathbf{CP}^{1}\,\,\,}{\int}D\sigma\;g^{\mathsf{a}}(\sigma^{A})\;\wedge\;\overline{\delta}_{-\Delta,\Delta}^{2|4}\big(\mathsf{W}^{I};\mathsf{Y}^{I'}(\sigma^{A})\big)\\
 & =\;\frac{\mathcal{C}(\Delta)}{\langle z|X|\bar{z}]^{\Delta}}\,e^{i\langle z|\theta\cdot\eta\rangle}\,g^{\mathsf{a}}(z^{A}).
\end{align}

We extend this to $n$ insertions by defining $\mathbf{X}_{n}\coloneqq\bigtimes^{n}\mathbf{MT}_{s}$
and $\mathbf{L}_{n}\coloneqq\bigtimes^{n}\mathbf{CP}^{1}$. An inductive
argument then gives:
\begin{align}
 & \underset{\mathbf{X}_{n}\,}{\int}\bigwedge_{i=1}^{n}\,D^{2|4}\mathsf{W}_{i}\;\Psi_{\Delta_{i}}\big(\mathsf{W}_{i}^{I};\mathsf{Z}_{i}^{I'}\big)\;\underset{\mathbf{L}_{n}\,}{\int}\bigwedge_{j=1}^{n}\,D\sigma_{j}\;g^{\mathsf{a}_{j}}(\sigma_{j}^{A})\;\wedge\;\overline{\delta}_{-\Delta_{j},\Delta_{j}}^{2|4}\big(\mathsf{W}_{j}^{J};\,\mathsf{Y}^{J'}(\sigma_{j}^{A})\big)\\
 & =\;\prod_{i=1}^{n}\,\frac{\mathcal{C}(\Delta_{i})}{\langle z_{i}|X|\bar{z}_{i}]^{\Delta_{i}}}\,e^{i\langle z_{i}|\theta\cdot\eta_{i}\rangle}\,g^{\mathsf{a}_{i}}(z_{i}^{A}).
\end{align}

Finally, choosing a basis $\{\mathsf{T}^{\mathsf{a}_{i}}\}$ of $\mathfrak{g}$
and setting
\begin{equation}
g^{\mathsf{a}_{i}}(\sigma^{A})\;=\;\frac{\mathsf{T}^{\mathsf{a}_{i}}}{\sigma_{i}\cdot\sigma_{i+1}}
\end{equation}
we find, via the logarithmic form $\boldsymbol{\omega}^{\mathsf{a}_{i}}(\sigma_{i}^{A})$
of Eq. (\ref{eq:-178}), the $\mathcal{N}=4$ \emph{supersymmetric
celestial RSVW identity}:
\begin{align}
 & \underset{\mathbf{X}_{n}\,}{\int}\bigwedge_{i=1}^{n}\,D^{2|4}\mathsf{W}_{i}\;\Psi_{\Delta_{i}}\big(\mathsf{W}_{i}^{I};\mathsf{Z}_{i}^{I'}\big)\;\underset{\mathbf{L}_{n}\,}{\int}\bigwedge_{j=1}^{n}\,\boldsymbol{\omega}^{\mathsf{a}_{j}}(\sigma_{j}^{A})\;\wedge\;\overline{\delta}_{-\Delta_{j},\Delta_{j}}^{2|4}\big(\mathsf{W}_{j}^{J};\,\mathsf{Y}^{J'}(\sigma_{j}^{A})\big)\\
 & =\;\prod_{i=1}^{n}\,\frac{\mathcal{C}(\Delta_{i})}{\langle z_{i}|X|\bar{z}_{i}]^{\Delta_{i}}}\,e^{i\langle z_{i}|\theta\cdot\eta_{i}\rangle}\,\frac{\mathsf{T}^{\mathsf{a}_{i}}}{z_{i}\cdot z_{i+1}}.
\end{align}

This formula serves as the key entry in our dictionary between celestial
and minitwistor amplitudes for gluons in planar Yang-Mills theory.
It provides the link between gluonic amplitudes and the holonomies
of the Knizhnik-Zamolodchikov connection.

\section{Tree-level $S$-Matrix}

The on-shell BCFW recursion relations\footnote{See \citet{britto2005direct,britto2005new,bianchi2008generating,brandhuber2008note,arkani2010simplest,elvang2009recursion}.}
in $\mathcal{N}=4$ SYM theory admit an explicit solution known as
the Drummond-Henn representation\footnote{Cf. \citet{drummond2009all,dixon2011all,korchemsky2010twistor}. }.
This solution provides a compact, manifestly supersymmetric expression
for the tree-level superamplitude. From it one may extract purely
gluonic amplitudes valid in any gauge theory. Remarkably, the Drummond-Henn
formula is written in terms of nested sums of dual-conformal $R$-invariants,
thus making both standard and dual superconformal symmetries manifest.

In this subsection, we employ the Drummond-Henn solution for tree-level
$N^{k}$-MHV superamplitudes to derive the corresponding sectional/leaf
amplitudes in terms of minitwistor superwavefunctions. Our strategy
consists of two main steps:
\begin{enumerate}
\item We write an integral representation of each $R$-invariant via the
Fadde'ev-Popov method, such that all frequency dependence appears
as exponential factors. The resulting representation admits a straightforward
Mellin transform.
\item We combine the Fadde'ev-Popov representation of the $R$-invariant
with the celestial RSVW identity. The resulting expression is the
tree-level $N^{k}$-MHV minitwistor amplitude, given by an integral
over the moduli space parametrising configurations of $2k+1$ minitwistor
lines.
\end{enumerate}
We begin by deriving in detail the $N^{1}$- and $N^{2}$-MHV amplitudes.
We then extend the construction inductively to obtain the full tree-level
celestial $\mathcal{S}$-matrix.

\subsection{$N^{1}$-MHV Scattering Amplitude\label{subsec:N-MHV-Scattering-Amplitude}}

Consider $\mathcal{N}=4$ supersymmetric Yang-Mills (SYM) theory on
four-dimensional Klein space\footnote{The Kleinian metric in global rectangular coordinates is: 
\[
h_{\mu\nu}\coloneqq\text{diag}\left(-1,-1,+1,+1\right).
\]
Klein space is reviewed in \citet{barrett1994kleinian,bhattacharjee2022celestial,crawley2022black,duary2024spectral,klein1870theorie,plucker1865xvii,penrose1984spinors}.} $\mathbf{K}^{4}\coloneqq\big(\mathbf{R}^{4},h_{\mu\nu}\big)$. We
endow this background with a gauge group $\mathbf{G}$. Its Lie algebra
$\mathfrak{g}\simeq\left(T_{e}\left(\mathbf{G}\right),[\cdot,\cdot]\right)$
is assumed compact and semi-simple. Let $\{\mathsf{T}^{\mathsf{a}}\}$
be a basis of generators satisfying $[\mathsf{T}^{\mathsf{a}},\mathsf{T}^{\mathsf{b}}]=if^{\mathsf{abc}}\mathsf{T}^{\mathsf{c}}$,
with $f^{\mathsf{abc}}$ the structure constants. We choose the trace
normalisation $\mathsf{Tr}\left(\mathsf{T}^{\mathsf{a}}\mathsf{T}^{\mathsf{b}}\right)=2^{-1}\boldsymbol{k}^{\mathsf{ab}}$,
where $\boldsymbol{k}^{\mathsf{ab}}$ is the Cartan-Killing form on
$\mathfrak{g}$. 

We study tree-level scattering of $n$ gluons in a next-to-maximal-helicity-violating
($N^{1}$-MHV) configuration. The $N^{1}$-MHV superamplitude depends
on spinor momenta $\nu_{i}^{A},\bar{\nu}_{i\dot{A}}$ and Grassmann
variables $\eta_{i}^{\alpha}$ $\left(\alpha=1,2,3,4\right)$ encoding
helicity degrees of freedom. It takes the form\footnote{For a review, see \citet{brandhuber2011tree,elvang2013scattering,badger2024scattering}.}:
\begin{equation}
\mathcal{A}_{n}^{\mathsf{a_{1}...a_{n}}}\left(\nu_{i}^{A},\bar{\nu}_{i\dot{A}},\eta_{i}^{\alpha}\right)\,=\,\left(2\pi\right)^{4}\delta^{4|0}\big(P^{A\dot{A}}\big)\delta^{0|8}\big(Q^{\alpha\dot{A}}\big)\,A_{n}^{\mathsf{a_{1}...a_{n}}}\left(\nu_{i}^{A},\bar{\nu}_{i\dot{A}},\eta_{i}^{\alpha}\right).\label{eq:-3}
\end{equation}
Here $i=1,...,n$ labels the external gluons. The total four-momentum
and supercharge are:
\begin{equation}
P^{A\dot{A}}\,\coloneqq\,\sum_{i=1}^{n}\,\nu_{i}^{A}\bar{\nu}_{i}^{\dot{A}},\,\,\,Q^{\alpha A}\,\coloneqq\,\sum_{i=1}^{n}\,\nu_{i}^{A}\,\eta_{i}^{\alpha}.
\end{equation}


\paragraph*{Dual Coordinates.}

To express the $N^{1}$-MHV superamplitude, we introduce dual (zone)
coordinates $y_{i}^{A\dot{A}}$. These solve the momentum-conserving
delta function $\delta^{4|0}\big(P^{A\dot{A}}\big)$ via:
\begin{equation}
y_{i}^{A\dot{A}}-y_{i+1}^{A\dot{A}}\coloneqq\nu_{i}^{A}\bar{\nu}_{i}^{\dot{A}}.\label{eq:-9}
\end{equation}
For $1\leq i<j\leq n$, set:
\begin{equation}
y_{ij}^{A\dot{A}}\coloneqq y_{i}^{A\dot{A}}-y_{j}^{A\dot{A}},\,\,\,y_{ji}^{A\dot{A}}\coloneqq-y_{ij}^{A\dot{A}}.
\end{equation}


\subsubsection{Dual Conformal Invariant}

The $N^{1}$-MHV superamplitude admits a representation in terms of
a dual-conformal $R$-invariant\footnote{For a discussion of dual conformal symmetry, see \citet{drummond2008planar,mason2010complete,alday2008scattering,henn2009duality,korchemsky2010superconformal,alday2007gluon,brandhuber2008mhv}. }.
This invariant takes the form:
\begin{equation}
R_{n;ab}\big(\nu_{i}^{A},\bar{\nu}_{i\dot{A}},\eta_{i}^{\alpha}\big)\,\,=\,\frac{\left\langle \nu_{a-1},\nu_{a}\right\rangle \,\left\langle \nu_{b-1},\nu_{b}\right\rangle \,\delta^{0|4}\big(\varXi^{\alpha}\big)}{y_{ab}^{2}\,\langle\nu_{n}|y_{nb}y_{ba}^{-1}|\nu_{a-1}\rangle\,\langle\nu_{n}|y_{nb}y_{ba}^{-1}|\nu_{a}\rangle\,\langle\nu_{n}|y_{na}y_{ab}^{-1}|\nu_{b-1}\rangle\,\langle\nu_{n}|y_{na}y_{ab}^{-1}|\nu_{b}\rangle},\label{eq:-2}
\end{equation}
where:
\begin{equation}
\Xi^{\alpha}\big(\nu_{i}^{A},\bar{\nu}_{i\dot{A}},\eta_{i}^{\alpha}\big)\,\coloneqq\,\sum_{i=1}^{a-1}\,\langle\nu_{n}|y_{nb}y_{ba}^{-1}|\nu_{i}\rangle\,\eta_{i}^{\alpha}\,+\,\sum_{i=1}^{b-1}\,\langle\nu_{n}|y_{na}y_{ab}^{-1}|\nu_{i}\rangle\,\eta_{i}^{\alpha}.
\end{equation}


\paragraph*{Celestial Description.}

To pass from momentum-space amplitudes to their celestial-basis form,
we introduce a normalised set of van der Waerden spinors $\{z_{i}^{A},\bar{z}_{i\dot{A}}\}$
on the celestial torus $\mathcal{CT}$. These are defined by:
\begin{equation}
z_{i}^{A}\,\coloneqq\,\left(1,-\zeta_{i}\right),\,\,\,\bar{z}_{i\dot{A}}\,\coloneqq\,\left(1,-\bar{\zeta}_{i}\right),\label{eq:-98}
\end{equation}
where $\zeta_{i},\bar{\zeta}_{i}$ are planar coordinates on $\mathcal{CT}$.

Denote by $s_{i}>0$ the frequency of the $i$-th gluon. We reparametrise
the spinor momenta via:
\begin{equation}
\nu_{i}^{A}\,\mapsto\,z_{i}^{A},\,\,\,\bar{\nu}_{i\dot{A}}\,\mapsto\,s_{i}\,\bar{z}_{i\dot{A}}.\label{eq:-24}
\end{equation}
Applying these replacements to the $R$-invariant (\ref{eq:-2}) yields
its celestial form:
\begin{equation}
R_{n;ab}=\frac{\langle z_{a-1},z_{a}\rangle\langle z_{b-1},z_{b}\rangle\delta^{0|4}\big(\Theta^{\alpha}\big)}{y_{ab}^{2}\langle z_{n}|y_{nb}y_{ba}^{-1}|z_{a-1}\rangle\langle z_{n}|y_{nb}y_{ba}^{-1}|z_{a}\rangle\langle z_{n}|y_{na}y_{ab}^{-1}|z_{b-1}\rangle\langle z_{n}|y_{na}y_{ab}^{-1}|z_{b}\rangle},\label{eq:-4}
\end{equation}
where: 
\begin{equation}
\Theta^{\alpha}\left(u^{A},v^{B}\right)\,\coloneqq\,\sum_{i=1}^{a-1}\,\langle u,z_{i}\rangle\,\eta_{i}^{\alpha}\,+\,\sum_{j=1}^{b-1}\,\langle v,z_{j}\rangle\,\eta_{j}^{\alpha}.\label{eq:-5}
\end{equation}


\subsubsection{Fadde'ev-Popov Representation\label{subsec:Fadde'ev-Popov-Method-N-MHV}}

Our objective is to find the Mellin transform of the gluonic superamplitude
(Eq. (\ref{eq:-3})). This yields the celestial amplitude:
\[
\widehat{\mathcal{A}}_{n}^{\mathsf{a_{1}...a_{n}}}\big(z_{i}^{A},\bar{z}_{i\dot{A}},\eta_{i}^{\alpha}\big).
\]
The $R$-invariant in Eq. (\ref{eq:-3}) depends nonlinearly on the
frequencies $s_{i}$ through the dual coordinates $y_{i}^{A\dot{A}}$.
This nonlinearity obstructs a direct Mellin transform. To overcome
this difficulty, we recast the $R$-invariant via the Fadde'ev-Popov
procedure.

Our aim is an integral representation in which all $s_{i}$-dependence
is isolated into delta functions. Let $\mathcal{I}\,\coloneqq\,\mathbf{R}^{2}\times\mathbf{R}^{2}$
be the integration domain. We introduce coordinates $U^{A'}\,\coloneqq\,\big(u^{A},v^{B}\big)$
on $\mathcal{I}$, where $u^{A}$ and $v^{A}$ are van der Waerden
spinors and $A'\in\{A,B\}$. The orientation on $\mathcal{I}$ is
given by the Lebesgue measure:
\begin{equation}
d^{4}U\,\coloneqq\,d^{2}u\wedge d^{2}v.
\end{equation}
Observe that the reality of $\mathcal{I}$ follows from the Kleinian
signature.

We localise the Fadde'ev-Popov integral on a constraint subset $\mathscr{C}\subset\mathcal{I}$.
To define $\mathscr{C}$, consider the auxiliary spinor functions:
\begin{equation}
f_{n;ab}^{A}\big(y_{i}^{B\dot{B}}\big)\,\coloneqq\,z_{n}^{B}\,\left(y_{nb}\right)_{B\dot{B}}\,(y_{ba}^{-1})^{A\dot{B}},\label{eq:-6}
\end{equation}
\begin{equation}
g_{n;ab}^{A}\big(y_{i}^{B\dot{B}}\big)\,\coloneqq\,z_{n}^{B}\,(y_{na})_{B\dot{B}}\,(y_{ab}^{-1})^{A\dot{B}}.\label{eq:-90}
\end{equation}
Then $\mathscr{C}$ is the set of points $U^{A'}\in\mathcal{I}$ satisfying:
\begin{equation}
u^{A}=f_{n;ab}^{A}\big(y_{i}^{B\dot{B}}\big),\,\,\,v^{A}=g_{n;ab}^{A}\big(y_{i}^{B\dot{B}}\big).
\end{equation}
The corresponding constraint delta distribution is:
\begin{equation}
\delta_{\mathscr{C}}\big(U^{A'}\big)\,\coloneqq\,\overline{\delta}^{2}\big(u^{A}-f_{n;ab}^{A}\big(y_{i}^{B\dot{B}}\big)\big)\,\overline{\delta}^{2}\big(v^{A}-g_{n;ab}^{A}\big(y_{i}^{B\dot{B}}\big)\big).
\end{equation}

With these ingredients, the Fadde'ev-Popov representation of the $R$-invariant
reads:
\begin{equation}
R_{n;ab}\,=\,\frac{1}{y_{ab}^{2}}\,\underset{\mathcal{I}\,\,\,\,\,}{\int}\,d^{4}U\,\,\,\mathcal{F}_{ab}\big(U^{A'}\big)\,\delta^{0|4}\big(\Theta^{\alpha}\big)\,\delta_{\mathscr{C}}\big(U^{A'}\big),\label{eq:-88}
\end{equation}
where:
\begin{equation}
\mathcal{F}_{ab}\big(U^{A'}\big)\,\coloneqq\,\frac{\langle z_{a-1},z_{a}\rangle\langle z_{b-1},z_{b}\rangle}{\langle z_{a-1},u\rangle\langle u,z_{a}\rangle\langle z_{b-1},v\rangle\langle v,z_{b}\rangle}.\label{eq:-145}
\end{equation}
This representation makes the $s_{i}$-dependence factor through the
constraint delta function $\delta_{\mathscr{C}}$.

\subsubsection{Delta Functions\label{subsec:Delta-Functions-N-MHV}}

We now expand the delta functions in Eq. (\ref{eq:-88}) using their
integral representations. We express them in terms of the celestial
coordinates $z_{i}^{A}$, $\bar{z}_{i\dot{A}}$ and $\eta_{i}^{\alpha}$.

\paragraph*{Fermionic Delta Function.}

Let $\alpha=1,...,4$ index the supersymmetry generators, and let
$\varepsilon^{\alpha}$ be a Grassmann variable. In superanalysis,
one defines:
\begin{equation}
\delta^{0|4}\big(\varepsilon^{\alpha}\big)\,\coloneqq\,\bigwedge_{\alpha=1}^{4}\,\varepsilon^{\alpha}.
\end{equation}
In terms of a Berezin integral\footnote{See \citet{berezin2013introduction}, \citet{dewitt1992supermanifolds},
\citet{leites1980introduction} and \citet{manin1997introduction}.},
\begin{equation}
\delta^{0|4}\big(\varepsilon^{\alpha}\big)\,=\,\underset{\mathbf{R}^{0|4}\,\,\,}{\int}\,d^{0|4}\chi\,\,\,\exp\left(i\chi\cdot\varepsilon\right),\,\,\,\chi\cdot\varepsilon\,\coloneqq\,\chi_{\alpha}\varepsilon^{\alpha}.\label{eq:-89}
\end{equation}
Substituting Eq. (\ref{eq:-5}) into Eq. (\ref{eq:-89}) then yields:
\begin{equation}
\delta^{0|4}\left(\Theta^{\alpha}\right)\,=\,\underset{\mathbf{R}^{0|4}\,\,\,}{\int}\,d^{0|4}\chi\,\,\,\bigwedge_{i=1}^{a-1}\,\,\,\exp\left(i\left\langle u,z_{i}\right\rangle \chi\cdot\eta_{i}\right)\,\,\,\bigwedge_{j=a}^{b-1}\,\,\,\exp\left(i\left\langle v,z_{j}\right\rangle \chi\cdot\eta_{j}\right).\label{eq:-14}
\end{equation}


\paragraph*{Bosonic Delta Function.}

Let $\lambda^{A}$ be a real van der Waerden spinor. The spinor delta
function admits the integral representation:
\begin{equation}
\overline{\delta}^{2}\left(\lambda^{A}\right)\,=\,\underset{\mathbf{R}^{2}\,\,\,}{\int}\,\frac{d^{2}\sigma}{\left(2\pi\right)^{2}}\,\exp\left(i\left\langle \lambda\sigma\right\rangle \right).\label{eq:-102}
\end{equation}
By substituting Eq. (\ref{eq:-6}) into this representation, one obtains:
\begin{equation}
\overline{\delta}^{2}\big(u^{A}-f_{n;ab}^{A}\big(y_{i}^{B\dot{B}}\big)\big)\,=\,\underset{\mathbf{R}^{2}\,\,\,}{\int}\,\frac{d^{2}\hat{u}}{\left(2\pi\right)^{2}}\,\,\,\exp\big(i\,\langle z_{n}|y_{nb}y_{ba}^{-1}|\hat{u}\rangle\big)\,e^{-i\langle u|\hat{u}\rangle}.\label{eq:-7}
\end{equation}
Similarly, substituting Eq. (\ref{eq:-90}) into Eq. (\ref{eq:-102})
yields:
\begin{equation}
\overline{\delta}^{2}\big(v^{A}-g_{n;ab}^{A}\big(y_{i}^{B\dot{B}}\big)\big)\,=\,\underset{\mathbf{R}^{2}\,\,\,}{\int}\frac{d^{2}\hat{v}}{\left(2\pi\right)^{2}}\,\,\,\exp\left(i\,\langle z_{n}|y_{na}y_{ab}^{-1}|\hat{v}\rangle\right)\,e^{-i\langle v|\hat{v}\rangle}.\label{eq:-8}
\end{equation}
Next, introduce the change of integration variables:
\begin{equation}
\tilde{u}^{\dot{A}}\,\coloneqq\,\hat{u}_{A}\left(y_{ba}^{-1}\right)^{A\dot{A}},\,\,\,\tilde{v}^{\dot{A}}\,\coloneqq\,\hat{v}_{A}\left(y_{ab}^{-1}\right)^{A\dot{A}}.
\end{equation}
It follows that:
\begin{equation}
\overline{\delta}^{2}\big(u^{A}-f_{n;ab}^{A}\big(y_{i}^{B\dot{B}}\big)\big)\,=\,\left|y_{ba}^{2}\right|\,\underset{\mathbf{R}^{2}\,\,\,}{\int}\,\frac{d^{2}\tilde{u}}{\left(2\pi\right)^{2}}\,\,\,\exp\left(i\,\langle z_{n}|y_{nb}|\tilde{u}]\right)\,\exp\left(-i\,\langle u|y_{ba}|\tilde{u}]\right),\label{eq:-10}
\end{equation}
\begin{equation}
\overline{\delta}^{2}\big(v^{A}-g_{n;ab}^{A}\big(y_{i}^{B\dot{B}}\big)\big)\,=\,\left|y_{ab}^{2}\right|\,\underset{\mathbf{R}^{2}\,\,\,}{\int}\frac{d^{2}\tilde{v}}{\left(2\pi\right)^{2}}\,\,\,\exp\left(i\,\langle z_{n}|y_{na}|\tilde{v}]\right)\,\exp\left(-i\,\langle v|y_{ab}|\tilde{v}]\right).\label{eq:-11}
\end{equation}

Finally, employing the definition of $y_{i}^{A\dot{A}}$ in Eq. (\ref{eq:-9}),
one expands the bosonic delta functions in the celestial parametrisation
$z_{i}^{A},\bar{z}_{i\dot{A}},\eta_{i}^{\alpha}$: 
\begin{equation}
\overline{\delta}^{2}\big(u^{A}-f_{n;ab}^{A}\big(y_{i}^{B\dot{B}}\big)\big)\,=\,\left|y_{ba}^{2}\right|\,\underset{\mathbf{R}^{2}\,\,\,}{\int}\,\frac{d^{2}\tilde{u}}{\left(2\pi\right)^{2}}\,\,\,\prod_{i=a}^{b-1}\,\exp\left(-is_{i}\,\langle z_{i}|u\tilde{u}|\bar{z}_{i}]\right)\,\,\,\prod_{i=b}^{n}\,\exp\left(is_{i}\,\langle z_{i}|z_{n}\tilde{u}|\bar{z}_{i}]\right),\label{eq:-12}
\end{equation}
\begin{equation}
\overline{\delta}^{2}\big(v^{A}-g_{n;ab}^{A}\big(y_{i}^{B\dot{B}}\big)\big)\,=\,\left|y_{ab}^{2}\right|\,\underset{\mathbf{R}^{2}\,\,\,}{\int}\,\frac{d^{2}\tilde{v}}{\left(2\pi\right)^{2}}\,\,\,\prod_{i=a}^{b-1}\,\exp\left(is_{i}\langle z_{i}|z_{n}\tilde{v}+v\tilde{v}|\bar{z}_{i}]\right)\,\,\,\prod_{i=b}^{n}\,\exp\left(is_{i}\langle z_{i}|z_{n}\tilde{v}|\bar{z}_{i}]\right).\label{eq:-13}
\end{equation}


\subsubsection{Integral Representation for the $R$-invariant\label{subsec:Integral-Representation-for}}

In Subsection \ref{subsec:Fadde'ev-Popov-Method-N-MHV} we applied
the Fadde'ev-Popov method to the $R$-invariant, noting that all dependence
on the frequency parameters $s_{i}$ is carried by delta functions.
In Subsection \ref{subsec:Delta-Functions-N-MHV} we then expanded
each delta function in the spinor basis $\{z_{i}^{A},\bar{z}_{i\dot{A}}\}$
and the Grassmann variables $\eta_{i}^{\alpha}$. By combining these
two steps, we arrive at the final form of the $R$-invariant, which
we now discuss just before performing the Mellin transform.

\paragraph*{Integration Superdomain.}

We begin by defining the \emph{parameter superspace} as:
\begin{equation}
\mathcal{P}\,\coloneqq\,\mathbf{R}^{8|4}.
\end{equation}
This supermanifold is globally charted by the coordinates:
\begin{equation}
\tau^{M}\,\coloneqq\,\big(u^{A},v^{B},\tilde{u}_{\dot{A}},\tilde{v}_{\dot{B}},\chi^{\alpha}\big),
\end{equation}
where the abstract index $M$ ranges over $\{A,B,\dot{A},\dot{B},\chi^{\alpha}\}$.
We shall refer to $\tau^{M}$ as the \emph{moduli parameters.}

The canonical orientation on $\mathcal{P}$ is provided by the $\mathbf{Z}_{2}$-graded
volume form:
\begin{equation}
d^{8|4}\tau\,\coloneqq\,d^{2}u\wedge d^{2}v\wedge d^{2}\tilde{u}\wedge d^{2}\tilde{v}\wedge d^{0|4}\chi.
\end{equation}


\paragraph*{Embedding Coordinates.}

In the following subsections, we shall construct the moduli superspace
$\mathscr{M}_{3}$, which parametrises the configuration of three
minitwistor lines on which the amplitude localises. For now, we regard
$\mathscr{M}_{3}$ as an abstract supermanifold.

For each gluon $i=1,...,n$ participating in the scattering process,
we associate a copy $\mathcal{P}_{i}$ of the parameter superspace,
which may be viewed as a submanifold of $\mathscr{M}_{3}$. The embedding
coordinates adapted to $\mathcal{P}_{i}$ are defined as follows.

Let the index $\ell$ run over the set $\{1,2,3\}$. We introduce
the family of coordinate functions on the parameter superspace:
\begin{equation}
\mathsf{Q}_{\ell}^{K}\coloneqq\big(\mathcal{Q}_{\ell}^{A\dot{A}},q_{\ell}^{\alpha A}\big):\mathcal{P}\longrightarrow\mathbf{R}^{4|8},
\end{equation}
whose components are listed in Table \ref{tab:Embedding-coordinates--N-MHV}.

\begin{table}
\begin{centering}
\begin{tabular}{ccc}
\toprule 
Index $\ell$ & $\mathcal{Q}_{\ell}^{A\dot{A}}$ & $q_{\ell}^{\alpha A}$\tabularnewline
\midrule
\midrule 
$1$ & $-\big(u^{A}+v^{A}\big)\chi^{\alpha}$ & $0$\tabularnewline
\midrule
\midrule 
$2$ & $-v^{A}\chi^{\alpha}$ & $z_{n}^{A}\tilde{v}^{\dot{A}}-u^{A}\tilde{u}^{\dot{A}}+v^{A}\tilde{v}^{\dot{A}}$\tabularnewline
\midrule 
$3$ & $0$ & $z_{n}^{A}\big(\tilde{u}^{A}+\tilde{v}^{\dot{A}}\big)$\tabularnewline
\bottomrule
\end{tabular}
\par\end{centering}
\caption{Embedding coordinates on parameter superspace $\mathcal{P}$.\label{tab:Embedding-coordinates--N-MHV}}

\end{table}
Next, define the cluster-indicator function $c(i)$ by:
\begin{equation}
c\left(i\right)\,\coloneqq\,\begin{cases}
1, & 1\leq i\leq a-1;\\
2, & a\leq i\leq b-1;\\
3, & b\leq i\leq n.
\end{cases}\label{eq:-144}
\end{equation}
This assigns to the $i$-th gluon the cluster to which it belongs.
The \emph{embedding coordinates} on each copy $\mathcal{P}_{i}$ are
then: 
\begin{equation}
\widetilde{\mathsf{Q}}_{i}^{K}\coloneqq\big(\widetilde{\mathcal{Q}}_{i}^{A\dot{A}},\tilde{q}_{i}^{\alpha A}\big):\mathcal{P}_{i}\longrightarrow\mathbf{R}^{4|8},
\end{equation}
with the identification:
\begin{equation}
\widetilde{\mathcal{Q}}_{i}^{A\dot{A}}\,\coloneqq\,\mathcal{Q}_{c\left(i\right)}^{A\dot{A}},\,\,\,\tilde{q}_{i}^{\alpha A}\,\coloneqq\,q_{c\left(i\right)}^{\alpha A}.\label{eq:-120}
\end{equation}


\paragraph*{Comment.}

Under the rescaling of the moduli parameters:
\begin{equation}
\tau^{M}=\big(u^{A},v^{B},\tilde{u}_{\dot{A}},\tilde{v}_{\dot{B}},\chi^{\alpha}\big)\mapsto\tilde{\tau}^{M}=\big(u^{A},v^{B},r\tilde{u}_{\dot{A}},r\tilde{v}_{\dot{B}},\chi^{\alpha}\big),
\end{equation}
the embedding coordinates transform homogeneously:
\begin{equation}
\widetilde{\mathcal{Q}}_{i}^{A\dot{A}}\,\mapsto\,r\,\widetilde{\mathcal{Q}}_{i}^{A\dot{A}},\,\,\,\tilde{q}_{i}^{\alpha A}\mapsto\tilde{q}_{i}^{\alpha A}.
\end{equation}


\paragraph{Main Result.}

By substituting Eqs. (\ref{eq:-14}), (\ref{eq:-12}) and (\ref{eq:-13})
into Eq. (\ref{eq:-4}), the $R$-invariant can be written as:
\begin{equation}
R_{n;ab}\,=\,\mathcal{N}_{ab}\,\underset{\mathcal{I}\,\,\,\,\,}{\int}\,d^{8|4}\tau\,\,\,\mathcal{F}_{ab}\left(\tau\right)\,\bigwedge_{i=1}^{n}\,\exp\big(i\langle z_{i}|\tilde{q}_{i}\cdot\eta_{i}\rangle+is_{i}\langle z_{i}|\widetilde{\mathcal{Q}}_{i}|\bar{z}_{i}]\big),\label{eq:-15}
\end{equation}
with the normalisation factor defined by:
\begin{equation}
\mathcal{N}_{ab}\,\coloneqq\,\frac{1}{\left(2\pi\right)^{4}}\,y_{ab}^{2}.
\end{equation}
In the representation (\ref{eq:-15}), the entire dependence on the
frequency parameters $s_{i}$ resides within the exponential factors.
This structure renders the formula ideally suited for the Mellin transform
needed to obtain the celestial amplitude.

\subsubsection{Celestial Superamplitude\label{subsec:Celestial-Superamplitude}}

The tree-level $N^{1}$-MHV superamplitude in $\mathcal{N}=4$ SYM
theory is obtained by summing its partial amplitudes:
\begin{equation}
\mathcal{A}_{n}^{\mathsf{a_{1}...\mathsf{a}_{n}}}\big(\nu_{i}^{A},\bar{\nu}_{i\dot{A}},\eta_{i}^{\alpha}\big)\,=\,\sum_{a,b}\,\mathcal{A}_{n;ab}^{\mathsf{a}_{1}...\mathsf{a}_{n}}\big(\nu_{i}^{A},\bar{\nu}_{i\dot{A}},\eta_{i}^{\alpha}\big).\label{eq:-16}
\end{equation}
The sub-amplitudes takes the form:
\begin{equation}
\mathcal{A}_{n;ab}^{\mathsf{a_{1}...a_{n}}}\big(\nu_{i}^{A},\bar{\nu}_{i\dot{A}},\eta_{i}^{\alpha}\big)\,=\,\left(2\pi\right)^{4}\,\delta^{4|0}\big(P^{A\dot{A}}\big)\,\delta^{0|8}\big(Q^{\alpha A}\big)\,R_{n;ab}\,\mathsf{Tr}\,\prod_{i=1}^{n}\,\frac{\mathsf{T}^{\mathsf{a}_{i}}}{\nu_{i}\cdot\nu_{i+1}}.\label{eq:-25}
\end{equation}

To express this amplitude in celestial coordinates, we implement the
substitutions $\nu_{i}^{A}\mapsto z_{i}^{A}$ and $\bar{\nu}_{i\dot{A}}\mapsto s_{i}\bar{z}_{i\dot{A}}$.
Then, using the integral representation of the $R$-invariant obtained
in the preceding section (refer to Eq. (\ref{eq:-15})), we now proceed
to derive the Mellin transform of the partial superamplitude $\mathcal{A}_{n;ab}^{\mathsf{a_{1}...a_{n}}}$.

\textcompwordmark{}

\paragraph*{Preliminaries.}

We first derive an integral representation for the distributional
prefactor:
\[
\left(2\pi\right)^{4}\,\delta^{4|0}\big(P^{A\dot{A}}\big)\,\delta^{0|8}\big(Q^{\alpha A}\big).
\]
This term enforces four-momentum and supercharge conservation in the
scattering process. 

The four-momentum delta function is:
\begin{equation}
\delta^{4|0}\big(P^{A\dot{A}}\big)\,=\,\frac{1}{\left(2\pi\right)^{4}}\,\underset{\mathbf{R}^{4}\,\,\,}{\int}\,d^{4}x\,\,\,\exp\left(i\,x\cdot P\right),\,\,\,x\cdot P\coloneqq x_{A\dot{A}}P^{A\dot{A}}.\label{eq:-26}
\end{equation}
Now, the supercharge $Q^{\alpha A}$ is a Grassmann-valued van der
Waerden spinor. Its fermionic delta function is defined by:
\begin{equation}
\delta^{0|8}\big(Q^{\alpha A}\big)\,\coloneqq\,\frac{1}{2^{4}}\,\bigwedge_{\alpha=1}^{4}\,\varepsilon^{AB}\,Q_{A}^{\alpha}\wedge Q_{B}^{\alpha}.
\end{equation}
Equivalently, via a Berezin integral:
\begin{equation}
\delta^{0|8}\big(Q^{\alpha A}\big)\,=\,\underset{\mathbf{R}^{0|8}}{\int}\,d^{0|8}\theta\,\,\,\exp\left(i\,\theta_{\alpha A}Q^{\alpha A}\right).\label{eq:-91}
\end{equation}
To combine Eqs. (\ref{eq:-26}) and (\ref{eq:-91}), introduce superspace
coordinates $\mathsf{x}^{K}\coloneqq\big(x_{A\dot{A}},\theta_{A}^{\alpha}\big)$
on $\mathbf{R}^{4|8}$ with abstract index $K$. The standard orientation
is given by the Berezin-de Witt volume superform:
\begin{equation}
d^{4|8}\mathsf{x}\,\coloneqq\,d^{4}x\wedge d^{0|8}\theta.\label{eq:-94}
\end{equation}
The explicit forms of $P^{A\dot{A}}$ and $Q^{\alpha A}$ appear in
Eqs. (\ref{eq:-26}) and (\ref{eq:-91}). Substituting these equations
into the bosonic and fermionic delta functions yields:
\begin{equation}
\left(2\pi\right)^{4}\,\delta^{4|0}\big(P^{A\dot{A}}\big)\,\delta^{0|8}\big(Q^{\alpha A}\big)\,=\,\underset{\mathbf{R}^{4|8}\,\,\,}{\int}\,d^{4|8}\mathsf{x}\,\,\,\bigwedge_{i=1}^{n}\,\exp\big(i\langle z_{i}|\theta\cdot\eta_{i}\rangle+is_{i}\langle z_{i}|x|\bar{z}_{i}]\big).\label{eq:-111}
\end{equation}


\paragraph*{Pre-moduli Superspace.}

We now use the integral formula for the $R$-invariant in Eq. (\ref{eq:-15})
and the expansion of the distributional prefactor in Eq. (\ref{eq:-111})
to recast the partial amplitude $\mathcal{A}_{n;ab}^{\mathsf{a_{1}...a_{n}}}$
as follows.

Define the \emph{pre-moduli superspace}:
\begin{equation}
\widehat{\mathscr{M}}_{3}\,\coloneqq\,\mathbf{R}^{4|8}\times\mathcal{P}.
\end{equation}
Its global chart is given by the moduli coordinates:
\begin{equation}
\hat{\gamma}^{Q}\,\coloneqq\,\big(\mathsf{x}^{K},\tau^{M}\big),
\end{equation}
with abstract index $Q\in\{K,M\}$. The standard orientation on $\widehat{\mathscr{M}}_{3}$
is fixed by the $\mathbf{Z}_{2}$-graded volume form:
\begin{equation}
\mathcal{D}\hat{\boldsymbol{\gamma}}\,\coloneqq\,d^{4|8}\mathsf{x}\wedge d^{8|4}\tau.
\end{equation}


\paragraph*{Comment.}

In the next subsections we will show that the leaf amplitude arises
from a dimensional reduction of $\widehat{\mathscr{M}}_{3}$ to the
moduli superspace $\mathscr{M}_{3}$. The latter parametrises three
minitwistor lines on which the amplitude localises. This construction
motivates the names ``pre-moduli'' superspace and ``moduli'' coordinates.

\textcompwordmark{}

Finally, substituting Eqs. (\ref{eq:-15}) and (\ref{eq:-111}) into
(\ref{eq:-25}) yields:
\begin{equation}
\mathcal{A}_{n;ab}^{\mathsf{a_{1}...a_{n}}}\big(z_{i}^{A},s_{i}\bar{z}_{i\dot{A}},\eta_{i}^{\alpha}\big)\,=\,\mathcal{N}_{ab}\,\underset{\widehat{\mathscr{M}}_{3}\,\,\,}{\int}\,\mathcal{D}\hat{\boldsymbol{\gamma}}\,\,\,\mathcal{F}_{ab}\left(\tau\right)\,\mathsf{Tr}\,\bigwedge_{i=1}^{n}\,e^{i\langle z_{i}|\left(\theta+\tilde{q}_{i}\right)\cdot\eta_{i}\rangle+is_{i}\langle z_{i}|x+\widetilde{\mathcal{Q}}_{i}|\bar{z}_{i}]}\,\frac{\mathsf{T}^{a_{i}}}{z_{i}\cdot z_{i+1}}.\label{eq:-92}
\end{equation}


\paragraph*{Mellin Transform.}

We conclude by computing the Mellin transform of the amplitude (\ref{eq:-16}).
This yields the $N^{1}$-MHV \emph{celestial }superamplitude:
\[
\widehat{\mathcal{A}}_{n}^{\mathsf{a_{1}...a_{n}}}\big(z_{i}^{A},\bar{z}_{i\dot{A}},\eta_{i}^{\alpha}\big).
\]

Let $\mathscr{R}\coloneqq\big(\mathbf{R}_{+},\cdot\big)$ denote the
multiplicative group of positive real numbers. We regard the frequencies
$s_{i}$ as affine coordinates on $\mathscr{R}$. Consider the $n$-fold
product group $\mathscr{R}^{n}\coloneqq\bigtimes^{n}\,\mathscr{R}$
with global coordinates $(s_{i})$ and Haar measure:
\begin{equation}
d\rho_{s_{i}}\,=\,\bigwedge_{i=1}^{n}\,d\log s_{i}.\label{eq:-112}
\end{equation}

For each external gluon $i=1,...,n$, let $\Delta_{i}$ be its celestial
conformal weight and $\epsilon_{i}$ its helicity expectation value.
Define the scaling dimension:
\begin{equation}
h_{i}\,\coloneqq\,\frac{\Delta_{i}+\epsilon_{i}}{2}.
\end{equation}
Furthermore, denote by:
\begin{equation}
\mathsf{W}_{i}^{I}\,\coloneqq\,\big(z_{i}^{A},\bar{z}_{i\dot{A}},\eta_{i}^{\alpha}\big)
\end{equation}
the dual real minitwistor encoding the insertion point $\{z_{i}^{A},\bar{z}_{i\dot{A}}\}$
on the celestial torus $\mathcal{CT}$ and the helicity state $\eta_{i}^{\alpha}$.

The \emph{tree-level $N^{1}$-MHV celestial superamplitude} is then
defined by the $n$-dimensional Mellin transform:
\begin{equation}
\widehat{\mathcal{A}}_{n}^{\mathsf{a_{1}...a_{n}}}\big(\mathsf{W}_{i}^{I}\big)\,\coloneqq\,\underset{\mathscr{R}^{n}\,\,\,\,\,}{\int}\,d\rho_{s_{i}}\,\,\,\mathcal{A}_{n}^{\mathsf{a_{1}...a_{n}}}\big(z_{i}^{A},s_{i}\bar{z}_{i\dot{A}},\eta_{i}^{\alpha}\big)\,\prod_{i=1}^{n}s_{i}^{2h_{i}}.
\end{equation}
Substituting Eq. (\ref{eq:-92}) into this definition shows that the
celestial amplitude decomposes as:
\begin{equation}
\widehat{\mathcal{A}}_{n}^{\mathsf{a_{1}...a_{n}}}\big(\mathsf{W}_{i}^{I}\big)\,=\,\sum_{a,b}\,\mathscr{P}_{ab}\,\widehat{\mathcal{A}}_{n;ab}^{\mathsf{a_{1}...a_{n}}}\big(\mathsf{W}_{i}^{I}\big),
\end{equation}
where the \emph{partial celestial superamplitude} takes the form:
\begin{equation}
\widehat{\mathcal{A}}_{n;ab}^{\mathsf{a_{1}...a_{n}}}\big(\mathsf{W}_{i}^{I}\big)\,=\,\underset{\widehat{\mathscr{M}}_{3}\,\,\,}{\int}\,\mathcal{D}\hat{\boldsymbol{\gamma}}\,\,\,\mathcal{F}_{ab}\left(\tau\right)\,\mathsf{Tr}\,\bigwedge_{i=1}^{n}\,\frac{\mathcal{C}\left(2h_{i}\right)}{\langle z_{i}|x+\widetilde{\mathcal{Q}}_{i}|\bar{z}_{i}]^{2h_{i}}}\,\exp\left(i\langle z_{i}|\left(\theta+\tilde{q}_{i}\right)\cdot\eta_{i}\rangle\right)\,\frac{\mathsf{T}^{a_{i}}}{z_{i}\cdot z_{i+1}}.\label{eq:-93}
\end{equation}
For simplicity, we henceforth focus on $\widehat{\mathcal{A}}_{n;ab}^{\mathsf{a_{1}...a_{n}}}$
and refer to it simply as the $N^{1}$-MHV celestial amplitude.

\subsubsection{Sectional Amplitude\label{subsec:Sectional-Amplitude}}

We now dimensionally reduce the integral over $\mathbf{R}^{4|8}$
in Eq. (\ref{eq:-93}). This reduction yields the sectional (or leaf)
amplitude
\[
\mathcal{M}_{n;ab}^{\mathsf{a_{1}...a_{n}}}\big(\mathsf{W}_{i}^{I}\big).
\]
It is defined by an integral over the real projective superspace $\mathbf{RP}^{3|8}$.
We will show below that its minitwistor transform localises on a family
of three minitwistor lines in $\mathbf{MT}_{s}$.

\textcompwordmark{}

\paragraph*{Klein and Projective Spaces.}

We begin by defining the geometric framework on which the sectional
amplitude is built. First, we introduce a coordinate chart on the
timelike wedge $W^{-}$ of Klein space $\mathbf{K}^{4}$. This chart
is related to homogeneous coordinates on $\mathbf{RP}^{3}$ and is
adapted to the standard foliation of $\mathbf{K}^{4}$ by Lorentzian
hyperbolic leaves. We then extend this construction to Klein superspace
$\mathbf{K}^{4|8}$ to accommodate the celestial and minitwistor superamplitudes. 

\textcompwordmark{}

Klein space can be partitioned into the lightcone $\Lambda$ and the
timelike and spacelike wedges, denoted $W^{-}$ and $W^{+}$, respectively.
The timelike wedge $W^{-}$ is defined as the set of all $x_{A\dot{A}}\in\mathbf{K}^{4}$
such that $x^{2}\coloneqq x_{A\dot{A}}x^{A\dot{A}}<0$.

Let $X_{A\dot{A}}$ be homogeneous coordinates on $\mathbf{RP}^{3}$.
We define the projective coordinates:
\begin{equation}
\mathcal{R}_{A\dot{A}}\,\coloneqq\,\left|X\right|^{-1}X_{A\dot{A}}.
\end{equation}
By construction, $\mathcal{R}_{A\dot{A}}$ is invariant under rescalings
$X_{A\dot{A}}\mapsto tX_{A\dot{A}}$ with $t>0$.

Let $r$ be an affine parameter on the multiplicative group of positive
real numbers $\mathscr{R}$. The coordinate system $\mathcal{X}\coloneqq\big(r,\mathcal{R}_{A\dot{A}}\big)$
charts $W^{-}$ via the bijection:
\begin{equation}
p\in W^{-}\mapsto\mathcal{X}(p)=\big(r(p),\mathcal{R}_{A\dot{A}}(p)\big)\in\mathbf{R}_{+}\times\mathbf{RP}^{3}.
\end{equation}
The map from $\mathcal{X}(p)$ to spacetime coordinates $x_{A\dot{A}}(p)$
is given by:
\begin{equation}
x_{A\dot{A}}\left(p\right)\,=\,r\left(p\right)\mathcal{R}_{A\dot{A}}\left(p\right).
\end{equation}
In terms of the coordinate system $\mathcal{X}$, the Lebesgue measure
decomposes on $W^{-}$ as:
\begin{equation}
d^{4}x\,\big|_{W^{-}}\,=\,r^{4}\,d\rho_{r}\wedge\frac{D^{3}X}{\left|X\right|^{4}}.\label{eq:-95}
\end{equation}
Here, $D^{3}X$ is the canonical volume form on $\mathbf{RP}^{3}$,
and $d\rho_{r}\coloneqq d\log r$ is the Haar measure on $\mathscr{R}$.

\textcompwordmark{}

\paragraph*{Klein and Projective Superspaces.}

We extend the foregoing construction to its supersymmetric analogue.
Define the $(3|8)$-dimensional projective superspace as the trivial
superbundle:
\begin{equation}
\mathbf{RP}^{3|8}\simeq\mathbf{RP}^{3}\times\mathbf{R}^{0|8}.
\end{equation}
Its typical fibre is the vector superspace spanned by the Grassmann
coordinates $\theta_{A}^{\alpha}$, equipped with the Berezin measure
$d^{0|8}\theta$. Introduce global coordinates:
\begin{equation}
\mathbb{X}^{K}\,\coloneqq\,\big(X_{A\dot{A}},\theta_{A}^{\alpha}\big),
\end{equation}
and fix the orientation by the $\mathbf{Z}_{2}$-graded volume form:
\begin{equation}
D^{3|8}\mathbb{X}\,\coloneqq\,\frac{D^{3}X}{\left|X\right|^{4}}\wedge d^{0|8}\theta.\label{eq:-113}
\end{equation}

In complete analogy, define Klein superspace as:
\begin{equation}
\mathbf{K}^{4|8}\simeq\mathbf{K}^{4}\times\mathbf{R}^{0|8}.
\end{equation}
It is charted by $\mathsf{x}^{K}=\big(x_{A\dot{A}},\theta_{A}^{\alpha}\big)$
and oriented by the Berezin-de Witt volume superform $d^{4|8}\mathsf{x}$
(refer to Eq. (\ref{eq:-94})).

Finally, let $W_{s}^{-}\subset\mathbf{K}^{4|8}$ denote the supersymmetric
timelike wedge. By definition,
\begin{equation}
\mathsf{x}^{K}\in W_{s}^{-}\iff x_{A\dot{A}}\in W^{-}.
\end{equation}
Upon restriction to $W_{s}^{-}$, the Berezin-de Witt superform decomposes
as:
\begin{equation}
d^{4|8}\mathsf{x}\,|_{W_{s}^{-}}\,=\,r^{4}\,d\rho_{r}\wedge D^{3|8}\mathbb{X}.\label{eq:-146}
\end{equation}


\paragraph*{Partial Amplitudes.}

Having introduced the necessary geometric structures on Klein superspace,
we now implement the leaf amplitude formalism. 

Recall the normalised basis of van de Waerden spinors $\{z_{i}^{A},\bar{z}_{i\dot{A}}\}$,
where $z_{i}^{A}=(1,-\zeta_{i})$ and $\bar{z}_{i\dot{A}}=(1,-\bar{\zeta}_{i})$.
Here $\left(\zeta_{i},\bar{\zeta}_{i}\right)$ are planar coordinates
on the celestial torus $\mathcal{CT}$. This spinor basis parametrises
the insertion point of the $i$-th gluon on $\mathcal{CT}$.

We introduce an involution $\sharp$ on dotted spinors by:
\begin{equation}
\bar{z}_{i\dot{A}}\mapsto\,\bar{z}_{i\dot{A}}^{\sharp}\,\coloneqq\,\left(1,\bar{\zeta}_{i}\right).
\end{equation}
Equivalently, on planar coordinates it acts as $\left(\zeta_{i},\bar{\zeta}_{i}\right)\mapsto\left(\zeta_{i},-\bar{\zeta}_{i}\right).$
This involution extends to dual minitwistors via:
\begin{equation}
\mathsf{W}_{i}^{I}\coloneqq\big(z_{i}^{A},\bar{z}_{i\dot{A}},\eta_{i}^{\alpha}\big)\,\mapsto\,\mathsf{W}_{i}^{\sharp I}\,\coloneqq\,\big(z_{i}^{A},\bar{z}_{i\dot{A}}^{\sharp},\eta_{i}^{\alpha}\big).\label{eq:-149}
\end{equation}

The first step in the leaf-amplitude algorithm is the decomposition:
\begin{equation}
\widehat{\mathcal{A}}_{n;ab}^{\mathsf{a_{1}...a_{n}}}\big(\mathsf{W}_{i}^{I}\big)\,=\,\mathcal{B}_{n;ab}^{\mathsf{a_{1}...a_{n}}}\big(\mathsf{W}_{i}^{I}\big)\,+\,\mathcal{B}_{n;ab}^{\mathsf{a_{1}...a_{n}}}\big(\mathsf{W}_{i}^{\sharp I}\big).
\end{equation}
To write the partial amplitude $\mathcal{B}_{n;ab}^{\mathsf{a_{1}...a_{n}}}$,
we next specify the integration domain. Define the \emph{moduli superspace}
for $N^{1}$-MHV sectional amplitudes as the supermanifold:
\begin{equation}
\mathscr{M}_{3}\,\coloneqq\,\mathbf{RP}^{3|8}\times\mathbf{R}^{8|4}.
\end{equation}
It is globally charted by the coordinates:
\begin{equation}
\gamma^{Q}\,\coloneqq\,\big(\mathbb{X}^{K},\tau^{M}\big),
\end{equation}
and oriented by the measure:
\begin{equation}
\mathcal{D}\boldsymbol{\gamma}\,\coloneqq\,D^{3|8}\mathbb{X}\wedge d^{8|4}\tau.
\end{equation}
The partial amplitude then takes the form:
\begin{equation}
\mathcal{B}_{n;ab}^{\mathsf{a_{1}...a_{n}}}\big(\mathsf{W}_{i}^{I}\big)\,=\,\underset{\mathscr{R}\,\,\,\,\,}{\int}\,d\rho_{r}\,r^{4}\,\underset{\mathscr{M}_{3}\,\,\,\,\,}{\int}\,\mathcal{D}\boldsymbol{\gamma}\,\,\,\mathcal{F}_{ab}\left(\tau\right)\,\mathsf{Tr}\,\bigwedge_{i=1}^{n}\,\frac{\mathcal{C}\left(2h_{i}\right)}{\langle z_{i}|r\mathcal{R}+\widetilde{\mathcal{Q}}_{i}|\bar{z}_{i}]^{2h_{i}}}\,e^{i\langle z_{i}|\left(\theta+\tilde{q}_{i}\right)\cdot\eta_{i}\rangle}\,\frac{\mathsf{T}^{a_{i}}}{z_{i}\cdot z_{i+1}}.\label{eq:-96}
\end{equation}


\paragraph*{Dimensional Reduction.}

The final step in reducing the celestial superamplitude to an integral
over the moduli superspace $\mathscr{M}_{3}$ is the integration over
the affine parameter $r$ in Eq. (\ref{eq:-96}).

Under the rescaling:
\begin{equation}
\tau^{M}=\big(u^{A},v^{B},\tilde{u}_{\dot{A}},\tilde{v}_{\dot{B}},\chi^{\alpha}\big)\,\mapsto\,\tilde{\tau}^{M}\coloneqq\big(u^{A},v^{B},r\tilde{u}_{\dot{A}},r\tilde{v}_{\dot{B}},\chi^{\alpha}\big),
\end{equation}
the measure on moduli superspace transforms as:
\begin{equation}
\mathcal{D}\boldsymbol{\gamma}\,\mapsto\,r^{4}\,\mathcal{D}\boldsymbol{\gamma}.
\end{equation}
Similarly, the embedding coordinates of Eq. (\ref{eq:-120}) rescale
according to:
\begin{equation}
\widetilde{\mathcal{Q}}_{i}^{A\dot{A}}\,\mapsto\,r\,\widetilde{\mathcal{Q}}_{i}^{A\dot{A}},\,\,\,\tilde{q}_{i}^{\alpha A}\,\mapsto\,\tilde{q}_{i}^{\alpha A}.
\end{equation}
Substituting these into Eq. (\ref{eq:-96}) shows that all factors
of $r$ decouple and can be integrated explicitly. One thus obtains:
\begin{equation}
\mathcal{B}_{n;ab}^{\mathsf{a_{1}...a_{n}}}\big(\mathsf{W}_{i}^{I}\big)\,=\,2\pi\delta(\beta_{1})\,\mathcal{M}_{n;ab}^{\mathsf{a_{1}...a_{n}}}\big(\mathsf{W}_{i}^{I}\big),
\end{equation}
where the scaling parameter:
\begin{equation}
\beta_{1}\,\coloneqq\,8-2\sum_{i=1}^{n}h_{i},
\end{equation}
encodes the total scaling dimension of the process.

Finally, the sectional amplitude for $n$-gluon scattering in an $N^{1}$-MHV
configuration reads:
\begin{equation}
\mathcal{M}_{n;ab}^{\mathsf{a_{1}...a_{n}}}\big(\mathsf{W}_{i}^{I}\big)=\underset{\mathscr{M}_{3}\,\,\,\,\,}{\int}\,\mathcal{D}\boldsymbol{\gamma}\,\,\,\mathcal{F}_{ab}\left(\tau\right)\,\mathsf{Tr}\,\bigwedge_{i=1}^{n}\,\frac{\mathcal{C}\left(2h_{i}\right)}{\langle z_{i}|\mathcal{R}+\widetilde{\mathcal{Q}}_{i}|\bar{z}_{i}]^{2h_{i}}}\,\exp\left(i\langle z_{i}|\left(\theta+\tilde{q}_{i}\right)\cdot\eta_{i}\rangle\right)\,\frac{\mathsf{T}^{a_{i}}}{z_{i}\cdot z_{i+1}}.\label{eq:-17}
\end{equation}
 

\subsubsection{Geometrical Formulation\label{subsec:Geometrical-Formulation}}

Now we derive the main result of this section. By virtue of the celestial
RSVW identity, the \emph{minitwistor amplitude}
\[
\widetilde{\mathcal{M}}_{n;ab}^{\mathsf{a_{1}...a_{n}}}\big(\mathsf{Z}_{i}^{I}\big)
\]
admits an elegant geometric and physical interpretation. It is realised
as a volume integral over the moduli superspace $\mathscr{M}_{3}$.
This observation suggests that the minitwistor amplitude may play
a more fundamental role than the original celestial amplitude in the
construction of a holographic dual to perturbative gauge theory (and
perhaps $\mathcal{N}=8$ Supergravity) in asymptotically flat spacetimes.

\textcompwordmark{}

\paragraph*{Celestial RSVW Identity.}

The derivation begins by reformulating the celestial RSVW identity
in terms of minitwistor geometry. Recall that $\mathbf{RP}^{3|8}$
is the superspace parametrising minitwistor lines in $\mathbf{MT}_{s}$.
For each point $p\in\mathbf{RP}^{3|8}$, the minitwistor line $\mathscr{L}(p)$
is the set of minitwistors:
\[
\mathsf{Z}^{I}\,\coloneqq\,\big(\lambda^{A},\mu_{\dot{A}},\psi^{\alpha}\big)\in\mathbf{MT}_{s}
\]
obeying the incidence relations:
\begin{equation}
\begin{cases}
\mu_{\dot{A}}=\lambda^{A}\mathcal{R}_{A\dot{A}}(p),\\
\psi^{\alpha}=\lambda^{A}\theta_{A}^{\alpha}\left(p\right).
\end{cases}
\end{equation}

Now, let $\pi_{p}:\mathscr{L}(p)\longrightarrow\mathbf{RP}^{1}$ be
the canonical projection and choose homogeneous coordinates $[\sigma^{A}]$
on $\mathscr{L}(p)$. These coordinates trivialise the fibration $\pi_{p}$.
A smooth section of $\pi_{p}$ embeds $\mathbf{RP}^{1}$ into $\mathbf{MT}_{s}$
as the minitwistor line $\mathscr{L}(p)$. Therefore, define $\mathsf{Y}_{p}^{I}:\mathbf{RP}^{1}\longrightarrow\mathscr{L}\left(p\right)$
via:
\begin{equation}
\mathsf{Y}_{p}^{I}\big(\sigma^{A}\big)\,\coloneqq\,\big(\sigma^{A},\sigma^{A}\mathcal{R}_{A\dot{A}}(p),\sigma^{A}\theta_{A}^{\alpha}(p)\big).
\end{equation}
By construction, $\pi\circ\mathsf{Y}_{p}^{I}\big(\sigma^{A}\big)=\sigma^{A}$.
So $\mathsf{Y}_{p}^{I}$ provides the desired embedding.

We next consider integration over $\mathscr{L}(p)$. In the trivialisation
$[\sigma^{A}]$, the natural measure is $D\sigma\coloneqq\varepsilon_{AB}\sigma^{A}d\sigma^{B}$.
Let $f$ be a smooth section of $\mathcal{O}(-2)$ on $\mathscr{L}(p)$,
so that for any $t>0$,
\begin{equation}
f\big(t\sigma^{A}\big)=t^{-2}f\big(\sigma^{A}\big).
\end{equation}
Then the differential form:
\begin{equation}
\boldsymbol{\omega}_{f}\big(\sigma^{A}\big)\,\coloneqq\,f\big(\sigma^{A}\big)\,D\sigma,\label{eq:-18}
\end{equation}
defines a well-posed integration measure on $\mathscr{L}(p)$.

Finally, over the dual minitwistor superspace $\widehat{\mathbf{MT}}^{2|4}$,
define the section:
\begin{equation}
\Phi_{\Delta,p}\big(\mathsf{W}^{I}\big)\,\coloneqq\,\frac{\mathcal{C}\left(\Delta\right)}{\langle z|\mathcal{R}|\bar{z}]^{\Delta}}\,e^{i\langle z|\theta\cdot\eta\rangle}\,f\big(z^{A}\big),\label{eq:-19}
\end{equation}
where $\mathsf{W}^{I}=\big(z^{A},\bar{z}_{\dot{A}},\eta^{\alpha}\big)$.
The celestial RSVW identity then takes the form:
\begin{equation}
\Phi_{\Delta,p}\big(\mathsf{W}^{I}\big)\,=\,\underset{\mathbf{MT}_{\mathbf{}}^{2|4}\,\,\,}{\int}\,D^{2|4}\mathsf{Z}\,\,\,\Psi_{\Delta}\big(\mathsf{Z}^{I};\mathsf{W}^{I'}\big)\,\underset{\mathbf{RP}^{1}\,\,\,}{\int}\,\boldsymbol{\omega}_{f}\big(\sigma^{A}\big)\,\overline{\delta}_{\left(-\Delta,\Delta\right)}^{2|4}\big(\mathsf{Z}^{I};\mathsf{Y}^{I'}\big(\sigma^{A}\big)\big).\label{eq:-20}
\end{equation}


\paragraph*{Minitwistor Amplitude.}

Substituting the reformulated celestial RSVW identity (Eq. (\ref{eq:-20}))
into the sectional amplitude (Eq. (\ref{eq:-17})) yields the following
representation.

Let $\ell=1,2,3$. Recall that $\mathcal{P}\simeq\mathbf{R}^{8|4}$
is the parameter superspace charted by moduli $\tau^{M}$. The embedding
coordinates $\mathcal{Q}_{\ell}^{A\dot{A}}$ and $q_{\ell}^{\alpha A}$
are defined in Subsection \ref{subsec:Integral-Representation-for}.

For each point $\gamma^{Q}=\big(\mathbb{X}^{K},\tau^{M}\big)\in\mathscr{M}_{3}$,
the minitwistor line $\mathscr{L}_{\ell}\big(\gamma^{Q}\big)$ is
the set of minitwistors $\mathsf{Z}^{I}$ obeying the incidence relations:
\begin{equation}
\begin{cases}
\mu_{\dot{A}}=\lambda^{A}\big(\mathcal{R}_{A\dot{A}}+\mathcal{Q}_{\ell A\dot{A}}\big),\\
\psi^{\alpha}=\lambda^{A}\big(\theta_{A}^{\alpha}+q_{\ell A}^{\alpha}\big).
\end{cases}
\end{equation}
At fixed $\gamma^{Q}$, the conics $\big\{\mathscr{L}_{\ell}\big(\gamma^{Q}\big)\big\}_{\ell=1}^{3}$
form a triplet of minitwistor lines. Varying $\gamma^{Q}$ over $\mathscr{M}_{3}$
sweeps out all such triplets. Thus $\mathscr{M}_{3}$ serves as the
moduli superspace of three-line configurations.

Next, embed $\mathbf{RP}^{1}$ into each line $\mathscr{L}_{\ell}\big(\gamma^{Q}\big)$
via $\mathsf{Y}_{\ell}^{I}:\mathbf{RP}^{1}\longrightarrow\mathscr{L}_{\ell}\big(\mathbb{X}^{K}\big)$,
such that:
\begin{equation}
\mathsf{Y}_{\ell}^{I}\big(\sigma^{A}\big)\,\coloneqq\,\big(\sigma^{A},\sigma^{A}\big(\mathcal{R}_{A\dot{A}}+\mathcal{Q}_{\ell A\dot{A}}\big),\sigma^{A}\big(\theta_{A}^{\alpha}+q_{\ell A}^{\alpha}\big)\big).
\end{equation}
This map is a smooth section of the fibration $\pi_{\ell}:\mathscr{L}_{\ell}\big(\gamma^{Q}\big)\longrightarrow\mathbf{RP}^{1}$. 

The sectional amplitude then becomes a multi-dimensional minitwistor
transform. Writing $\mathbf{M}^{n}\coloneqq\bigtimes^{n}\mathbf{MT}_{s}$,
one has:
\begin{equation}
\mathcal{M}_{n;ab}^{\mathsf{a_{1}...a_{n}}}\big(\mathsf{W}_{i}^{I}\big)\,=\,\underset{\mathbf{M}^{n}\,\,\,}{\int}\,\bigwedge_{i=1}^{n}\,D^{2|4}\mathsf{Z}_{i}\,\Psi_{2h_{i}}\big(\mathsf{Z}_{i}^{I};\mathsf{W}_{i}^{I'}\big)\,\,\,\widetilde{\mathcal{M}}_{n;ab}^{\mathsf{a_{1}...a_{n}}}\big(\mathsf{Z}_{i}^{I}\big).
\end{equation}
Finally, introduce the Lie-algebra--valued logarithmic one-form on
$\mathbf{RP}^{1}$:
\begin{equation}
\boldsymbol{\omega}^{a_{i}}\big(\sigma_{i}^{A}\big)\,\coloneqq\,\mathsf{T}^{a_{i}}\,\frac{D\sigma_{i}}{\sigma_{i}\cdot\sigma_{i+1}}.
\end{equation}
With this definition, the \emph{$N^{1}$-MHV minitwistor amplitude}
is:
\begin{equation}
\widetilde{\mathcal{M}}_{n;ab}^{\mathsf{a_{1}...a_{n}}}\big(\mathsf{Z}_{i}^{I}\big)\,=\,\underset{\mathscr{M}_{3}\,\,\,\,\,}{\int}\,\mathcal{D}\boldsymbol{\gamma}\,\,\,\mathcal{F}_{ab}\big(\gamma^{Q}\big)\,\mathsf{Tr}\,\,\,\bigwedge_{i=1}^{n}\,\,\,\underset{\mathbf{RP}^{1}\,\,\,}{\int}\,\,\,\boldsymbol{\omega}^{a_{i}}\big(\sigma_{i}^{A}\big)\,\overline{\delta}_{\left(-2h_{i},2h_{i}\right)}^{2|4}\big(\mathsf{Z}^{I};\mathsf{Y}_{c(i)}^{I'}\big(\sigma_{i}^{A}\big)\big).\label{eq:-143}
\end{equation}


\paragraph*{Interpretation.}

The function $\mathcal{F}_{ab}$ (Eq. (\ref{eq:-144})) lifts to a
probability distribution on the moduli superspace. In our discussion
of minitwistor celestial CFT, we shall interpret Eq. (\ref{eq:-143})
as the semiclassical expectation value of the observable $\mathcal{F}_{ab}$.

The minitwistor amplitude $\widetilde{\mathcal{M}}_{n;ab}^{\mathsf{a_{1}...a_{n}}}$
derived in Eq. (\ref{eq:-143}) computes a volume integral over $\mathscr{M}_{3}$
weighted by the distribution $\mathcal{F}_{ab}$. The volume form
is localised on the minitwistor lines $\mathscr{L}_{\ell}\big(\gamma^{Q}\big)$
via the delta functions:
\[
\overline{\delta}_{\left(-2h_{i},2h_{i}\right)}^{2|4}\big(\mathsf{Z}^{I};\mathsf{Y}_{c(i)}^{I'}\big(\sigma_{i}^{A}\big)\big).
\]
These factors are supported precisely on the triplet $\big\{\mathscr{L}_{\ell}\big(\gamma^{Q}\big)\big\}_{\ell=1}^{3}$.
They are modulated by the celestial scaling dimensions $h_{i}$ of
the external gluons. 

Furthermore, the amplitude vanishes whenever the insertion point of
the $i$-th gluon does not lie on the conic $\mathscr{L}_{\ell}\big(\gamma^{Q}\big)$,
for $\ell=c(i)$ its cluster assignment (see Subsection \ref{subsec:Integral-Representation-for},
Eq. (\ref{eq:-144})).

\subsection{$N^{2}$-MHV Scattering Amplitude\label{subsec:N2-MHV-Scattering-Amplitude}}

We now construct the celestial and minitwistor superamplitudes for
next-to-next-MHV ($N^{2}$-MHV) gluon scattering. Our immediate goal
is to derive the explicit $N^{2}$-MHV \emph{celestial} amplitude.
More importantly, we aim to extend this approach to the full tree-level
$\mathcal{S}$-matrix of $\mathcal{N}=4$ SYM theory in Subsection
\ref{subsec:General-Case}.

\subsubsection{Order-$2$ $R$-Invariant}

To write the tree-level $N^{2}$-MHV superamplitude, we introduce
the \emph{order-$2$ $R$-invariant}, $R_{n;a_{1}b_{1},a_{2}b_{2}}$.
We begin by defining two auxiliary spinors in terms of the dual coordinates
$y_{i}^{A\dot{A}}$. Let:
\begin{equation}
u_{1}^{A}\,\coloneqq\,z_{n}^{B}\,\big(y_{nb_{1}}\big)_{B\dot{B}}\,\big(y_{b_{1}a_{1}}^{-1}\big)^{A\dot{B}},\,\,\,v_{1}^{A}\,\coloneqq\,z_{n}^{B}\,\big(y_{na_{1}}\big)_{B\dot{B}}\,\big(y_{a_{1}b_{1}}^{-1}\big)^{A\dot{B}}.\label{eq:-23}
\end{equation}
From $u_{1}^{A}$ we then define two spinor-valued functions:
\begin{equation}
\tilde{f}_{a_{1}a_{2}b_{2}}^{A}\big(u_{1}^{A}\big)\,\coloneqq\,u_{1}^{B}\,\big(y_{a_{1}b_{2}}\big)_{B\dot{B}}\,\big(y_{b_{2}a_{2}}^{-1}\big)^{A\dot{B}},\label{eq:-101}
\end{equation}
\begin{equation}
\tilde{g}_{a_{1}a_{2}b_{2}}^{A}\big(u_{1}^{A}\big)\,\coloneqq\,u_{1}^{B}\,\big(y_{a_{1}a_{2}}\big)_{B\dot{B}}\,\big(y_{a_{2}b_{2}}^{-1}\big)^{A\dot{B}}.
\end{equation}

Next, introduce a second pair of spinors $\big(u_{2}^{A},v_{2}^{B}\big)$
to parametrise $\mathcal{I}\coloneqq\mathbf{R}^{2}\times\mathbf{R}^{2}$.
Their reality follows from the Kleinian signature. The \emph{constraint
hypersurface} $\mathscr{C}\subset\mathcal{I}$ is then given by:
\begin{equation}
u_{2}^{A}\,=\,\tilde{f}_{a_{1}a_{2}b_{2}}^{A}\big(u_{1}^{B}\big),\,\,\,v_{2}^{A}\,=\,\tilde{g}_{a_{1}a_{2}b_{2}}^{A}\big(u_{1}^{B}\big).\label{eq:-22}
\end{equation}

Finally, in the normalised spinor basis $\{z_{i}^{A},\bar{z}_{i\dot{A}}\}$
that labels insertion points on the celestial torus $\mathcal{CT}$,
the order-$2$ $R$-invariant is:
\begin{equation}
R_{n;a_{1}b_{1},a_{2}b_{2}}\big(u_{2}^{A},v_{2}^{B},y_{i}^{C\dot{C}},z_{i}^{D}\big)\,\coloneqq\,\frac{\langle z_{a_{2}-1},z_{a_{2}}\rangle\langle z_{b_{2}-1},z_{b_{2}}\rangle\delta^{0|4}\big(\Theta_{2}^{\alpha}\big)}{y_{a_{2}b_{2}}^{2}\langle z_{a_{2}-1},u_{2}\rangle\langle u_{2},z_{a_{2}}\rangle\langle z_{b_{2}-1},v_{2}\rangle\langle v_{2},z_{b_{2}}\rangle},\label{eq:-21}
\end{equation}
for $\big(u_{2}^{A},v_{2}^{B}\big)\in\mathscr{C}$. Here,
\begin{equation}
\Theta_{2}^{\alpha}\big(u_{2}^{A},v_{2}^{B}\big)\,\coloneqq\,\sum_{i=a_{1}}^{a_{2}-1}\,\langle u_{2},z_{i}\rangle\,\eta_{i}^{\alpha}+\sum_{j=a_{1}}^{b_{2}-1}\,\langle v_{2},z_{j}\rangle\,\eta_{j}^{\alpha}.\label{eq:-99}
\end{equation}


\subsubsection{Fadde'ev-Popov Representation\label{subsec:Fadde'ev-Popov-Representation-2}}

We now apply the Fadde'ev-Popov method to derive an integral representation
of the order-$2$ $R$-invariant. This representation is tailored
for the subsequent Mellin transform of the $N^{2}$-MHV superamplitude.

First, we impose Eq. (\ref{eq:-22}) for the spinor variables $u_{2}^{A}$
and $v_{2}^{A}$ by inserting delta functions that localise their
integration to the constraint hypersurface $\mathscr{C}$. We write:
\begin{equation}
R_{n;a_{1}b_{1},a_{2}b_{2}}\,=\,\frac{1}{y_{a_{2}b_{2}}^{2}}\,\,\,\underset{\mathcal{I}\,\,\,\,\,}{\int}\,d^{2}u_{2}\wedge d^{2}v_{2}\,\,\,\,\,\mathcal{F}_{a_{2}b_{2}}\big(u_{2}^{A},v_{2}^{B}\big)\,\delta^{0|4}\big(\Theta_{2}^{\alpha}\big)\,\delta_{\mathscr{C}}\big(u_{2}^{A},v_{2}^{B}\big).\label{eq:-97}
\end{equation}
Here,
\begin{equation}
\delta_{\mathscr{C}}\big(u_{2}^{A},v_{2}^{B}\big)\,\coloneqq\,\overline{\delta}^{2}\big(u_{2}^{A}-\tilde{f}_{a_{1}a_{2}b_{2}}^{A}\big(u_{1}^{B}\big)\big)\,\overline{\delta}^{2}\big(v_{2}^{A}-\tilde{g}_{a_{1}a_{2}b_{2}}^{A}\big(u_{1}^{B}\big)\big)\label{eq:-100}
\end{equation}
enforces $u_{2}^{A}=\tilde{f}_{a_{1}a_{2}b_{2}}^{A}\big(u_{1}^{B}\big)$
and $v_{2}^{A}=\tilde{g}_{a_{1}a_{2}b_{2}}^{A}\big(u_{1}^{B}\big)$
on the integration variables. The function appearing under the integral
(\ref{eq:-97}) is:
\begin{equation}
\mathcal{F}_{a_{2}b_{2}}\big(u_{2}^{A},v_{2}^{B}\big)\,\coloneqq\,\frac{\langle z_{a_{2}-1},z_{a_{2}}\rangle\langle z_{b_{2}-1},z_{b_{2}}\rangle}{\langle z_{a_{2}-1},u_{2}\rangle\langle u_{2},z_{a_{2}}\rangle\langle z_{b_{2}-1},v_{2}\rangle\langle v_{2},z_{b_{2}}\rangle}.\label{eq:-108}
\end{equation}


\paragraph*{Comment.}

We have not imposed the definitions of $u_{1}^{A}$ and $v_{1}^{A}$
from Eq. (\ref{eq:-23}) as delta-function constraints here. Since
in the $N^{2}$-MHV superamplitude only the product:
\[
R_{n;a_{1}b_{1}}R_{n;a_{1}b_{1},a_{2}b_{2}},
\]
appears, the Fadde'ev-Popov representation of the order-$1$ $R$-invariant,
$R_{n;a_{1}b_{1}}$, already enforces those spinor definitions.


\subsubsection{Fermionic and Bosonic Delta-functions\label{subsec:Fermionic-and-Bosonic}}

Equation (\ref{eq:-97}) expresses the order-$2$ $R$-invariant such
that all dependence on the frequency parameters $s_{i}$ factorises
through the constraint delta function $\delta_{\mathscr{C}}$. Our
next task is to find an integral representation for the delta functions
in Eq. (\ref{eq:-97}).

We begin with the Grassmann delta function $\delta^{0|4}\big(\Theta_{2}^{\alpha}\big)$.
Recall from Subsection \ref{subsec:Fadde'ev-Popov-Method-N-MHV} that
the Berezin-integral representation of the fermionic delta function
takes the form:
\begin{equation}
\delta^{0|4}\big(\varepsilon^{\alpha}\big)\,=\,\int\,d^{0|4}\chi\,\,\,\exp\left(i\chi_{\alpha}\varepsilon^{\alpha}\right).
\end{equation}
Substituting Eq. (\ref{eq:-99}) into this form yields:
\begin{equation}
\delta^{0|4}\big(\Theta_{2}^{\alpha}\big)\,=\,\int\,d^{0|4}\chi_{2}\,\,\,\bigwedge_{i=a_{1}}^{a_{2}-1}\,\exp\big(i\langle u_{2},z_{i}\rangle\,\chi_{2}\cdot\eta_{i}\big)\,\bigwedge_{i=a_{1}}^{b_{2}-1}\,\exp\big(i\langle v_{2},z_{i}\rangle\,\chi_{2}\cdot\eta_{i}\big).\label{eq:-105}
\end{equation}

Next, we analyse the holomorphic delta functions in Eq. (\ref{eq:-100}).
Substituting Eq. (\ref{eq:-100}) into the integral form of $\overline{\delta}^{2}$,
as defined in Eq. (\ref{eq:-102}), we obtain:
\begin{equation}
\overline{\delta}^{2}\big(u_{2}^{A}-\tilde{f}_{a_{1}a_{2}b_{2}}^{A}\big(u_{1}^{A}\big)\big)\,=\,\underset{\mathbf{R}^{2}\,\,\,\,\,}{\int}\,\frac{d^{2}\hat{u}_{2}}{\left(2\pi\right)^{2}}\,\,\,e^{i\langle u_{2}\hat{u}_{2}\rangle}\,\exp\big(-i\langle u_{1}|y_{a_{1}b_{2}}y_{b_{2}a_{2}}^{-1}|\hat{u}_{2}]\big),\label{eq:-103}
\end{equation}
and similarly,
\begin{equation}
\overline{\delta}^{2}\big(v_{2}^{A}-\tilde{g}_{a_{1}a_{2}b_{2}}^{A}\big(u_{1}^{A}\big)\big)\,=\,\underset{\mathbf{R}^{2}\,\,\,\,\,}{\int}\,\frac{d^{2}\hat{v}_{2}}{\left(2\pi\right)^{2}}\,\,\,e^{i\langle v_{2}\hat{v}_{2}\rangle}\,\exp\big(-i\langle u_{1}|y_{a_{1}a_{2}}y_{a_{2}b_{2}}^{-1}|\hat{v}_{2}]\big).\label{eq:-104}
\end{equation}
We now change integration variables by:
\begin{equation}
\tilde{u}_{2}^{\dot{A}}\,\coloneqq\,\big(y_{b_{2}a_{2}}^{-1}\big)^{A\dot{A}}\,\hat{u}_{2A},\,\,\,\tilde{v}_{2}^{\dot{A}}\,\coloneqq\,\big(y_{a_{2}b_{2}}^{-1}\big)^{A\dot{A}}\,\hat{v}_{2A}.
\end{equation}
In these variables, Eqs. (\ref{eq:-103}) and (\ref{eq:-104}) become:
\begin{equation}
\overline{\delta}^{2}\big(u_{2}^{A}-\tilde{f}_{a_{1}a_{2}b_{2}}^{A}\big(u_{1}^{A}\big)\big)\,=\,\big|y_{b_{2}a_{2}}^{2}\big|\,\underset{\mathbf{R}^{2}\,\,\,}{\int}\,\frac{d^{2}\tilde{u}_{2}}{\left(2\pi\right)^{2}}\,\,\,\prod_{i=a_{2}}^{b_{2}-1}\,e^{-is_{i}\langle z_{i}|u_{2}\tilde{u}_{2}|\bar{z}_{i}]}\,\prod_{j=a_{1}}^{b_{2}-1}\,e^{-is_{i}\langle z_{j}|u_{1}\tilde{u}_{2}|\bar{z}_{j}]},\label{eq:-107}
\end{equation}
and:
\begin{equation}
\overline{\delta}^{2}\big(v_{2}^{A}-\tilde{g}_{a_{1}a_{2}b_{2}}^{A}\big(u_{1}^{A}\big)\big)\,=\,\big|y_{a_{2}b_{2}}^{2}\big|\,\underset{\mathbf{R}^{2}}{\int}\,\frac{d^{2}\tilde{v}_{2}}{\left(2\pi\right)^{2}}\,\prod_{i=a_{2}}^{b_{2}-1}\,e^{is_{i}\langle z_{i}|v_{2}\tilde{v}_{2}|\bar{z}_{i}]}\,\prod_{j=a_{1}}^{a_{2}-1}\,e^{-is_{j}\langle z_{j}|u_{1}\tilde{v}_{2}|\bar{z}_{j}]}.\label{eq:-106}
\end{equation}


\subsubsection{Degree-$2$ $R$-monomial\label{subsec:Integral-Representation-Second-Order}}

\paragraph*{Recap. }

In Subsection \ref{subsec:Fadde'ev-Popov-Representation-2}, the Fadde'ev-Popov
method was used to derive the integral formula for the order-$2$
$R$-invariant (see Eq. (\ref{eq:-97})). In that formula, the dependence
on the frequency parameters $s_{i}$ is completely factorised into
delta functions. In Subsection \ref{subsec:Fermionic-and-Bosonic},
those delta functions were then expanded via their integral representations
in terms of the celestial coordinates $z_{i}^{A}$, $\bar{z}_{i\dot{A}}$
and $\eta_{i}^{\alpha}$.

\textcompwordmark{}

We next consider the \emph{degree-$2$ $R$-monomial}, defined by:
\begin{equation}
R_{n}^{\left(2\right)}\,\coloneqq\,R_{n;a_{1}b_{1}}R_{n;a_{1}b_{1},a_{2}b_{2}}.
\end{equation}
To represent $R_{n}^{\left(2\right)}$ as an integral, we proceed
in two steps. First, substitute Eqs. (\ref{eq:-105}) and (\ref{eq:-106})
into the order-$2$ $R$-invariant, Eq. (\ref{eq:-97}). Second, employ
the result for $R_{n;a_{1}b_{1}}$ from Eq. (\ref{eq:-15}). This
then yields the following integral representation. 

\textcompwordmark{}

\paragraph*{Integration Superdomain.}

The parameter superspace is defined by: 
\begin{equation}
\mathcal{P}\coloneqq\mathbf{R}^{8|4}\times\mathbf{R}^{8|4}.
\end{equation}
For each $k=1,2$, define the \emph{moduli parameters}:
\begin{equation}
\tau_{k}^{M}\,\coloneqq\,\big(u_{k}^{A},v_{k}^{B},\tilde{u}_{k\dot{A}},\tilde{v}_{k\dot{B}},\chi_{k}^{\alpha}\big)\,\in\mathbf{R}^{8|4},
\end{equation}
where the abstract index $M$ ranges over $\{A,B,\dot{A},\dot{B},\alpha\}$.
Then $\mathcal{P}$ is globally charted by $\boldsymbol{\tau}^{P}\,\coloneqq\,\big(\tau_{1}^{M},\tau_{2}^{M'}\big)$,
with $P\in\{M,M'\}$. In Subsection \ref{subsec:Geometrical-Formulation-1}
we will identify each component of $\tau_{k}^{M}$ with the moduli
parameters of five minitwistor lines on which the $N^{2}$-MHV minitwistor
superamplitude localises. This identification justifies reffering
to $\mathcal{P}$ as the \emph{parameter superspace} and to $\tau_{k}^{M}$
as its \emph{moduli coordinates}.

The standard orientation on each copy of $\mathbf{R}^{8|4}$ is provided
by the $\mathbf{Z}_{2}$-graded volume form: 
\begin{equation}
d^{8|4}\tau_{k}\,\coloneqq\,d^{2}u_{k}\wedge d^{2}v_{k}\wedge d^{2}\tilde{u}_{k}\wedge d^{2}\tilde{v}_{k}\wedge d^{0|4}\chi_{k}.
\end{equation}
Consequently, the integration measure on the parameter superspace
$\mathcal{P}$ is: 
\begin{equation}
d^{16|8}\boldsymbol{\tau}\,\coloneqq\,d^{8|4}\tau_{1}\wedge d^{8|4}\tau_{2}.
\end{equation}


\paragraph*{Embedding Coordinates.}

Let $\mathscr{M}_{5}$ denote the moduli superspace characterising
the configuration of a system comprising five minitwistor lines. We
will specify $\mathscr{M}_{5}$ explicitly in the forthcoming subsection.
For now, let us regard $\mathscr{M}_{5}$ as an abstract supermanifold.

For each gluon $i$ participating in the $N^{2}$-MHV scattering process,
we assign a copy $\mathcal{P}_{i}$ of the parameter superspace. A
natural question then arises: what are the embedding coordinates of
$\mathcal{P}_{i}$ in $\mathscr{M}_{5}$? The answer is given by Eqs.
(\ref{eq:-105}), (\ref{eq:-107}) and (\ref{eq:-106}). 

Let the index $\ell$ range over $\{1,...,5\}$, and define the auxiliary
coordinate function:
\begin{equation}
\mathcal{Q}^{A\dot{A}}\,\coloneqq\,z_{n}^{A}\tilde{v}_{1}^{\dot{A}}-u_{1}^{A}\tilde{u}_{1}^{\dot{A}}+v_{1}^{A}\tilde{v}_{1}^{\dot{A}}.
\end{equation}
Using this, define the coordinate maps:
\begin{equation}
\mathsf{Q}_{\ell}^{K}\,\coloneqq\,\big(\mathcal{Q}_{\ell}^{A\dot{A}},q_{\ell}^{\alpha A}\big):\mathcal{P}\longrightarrow\mathbf{R}^{4|8},
\end{equation}
with components given in Table \ref{tab:Embedding-coordinates--1}. 

\begin{table}
\begin{centering}
\begin{tabular}{ccc}
\toprule 
Interval for $\ell$ & $\mathcal{Q}_{\ell}^{A\dot{A}}$ & $q_{\ell}^{\alpha A}$\tabularnewline
\midrule
\midrule 
$1$ & $0$ & $-\big(u_{1}^{A}+v_{1}^{A}\big)\,\chi_{1}^{\alpha}$\tabularnewline
\midrule
\midrule 
$2$ & $\mathcal{Q}^{A\dot{A}}-u_{1}^{A}\,\big(\tilde{u}_{2}^{\dot{A}}+\tilde{v}_{2}^{\dot{A}}\big)$ & $-v_{1}^{A}\,\chi_{1}^{\alpha}-\big(u_{2}^{A}+v_{2}^{A}\big)\,\chi_{2}^{\alpha}$\tabularnewline
\midrule
\midrule 
$3$ & $\mathcal{Q}^{A\dot{A}}-\big(u_{1}^{A}+u_{2}^{A}\big)\,\tilde{u}_{2}^{\dot{A}}+v_{2}^{A}\tilde{v}_{2}^{\dot{A}}$ & $-v_{1}^{A}\chi_{1}^{\alpha}-v_{2}^{A}\chi_{2}^{\alpha}$\tabularnewline
\midrule 
$4$ & $\mathcal{Q}^{A\dot{A}}$ & $-v_{1}^{A}\chi_{1}^{\alpha}$\tabularnewline
\midrule 
$5$ & $z_{n}^{A}\,\big(\tilde{u}_{1}^{\dot{A}}+\tilde{v}_{1}^{\dot{A}}\big)$ & $0$\tabularnewline
\bottomrule
\end{tabular}
\par\end{centering}
\caption{Embedding coordinates $\mathsf{Q}_{\ell}^{K}\protect\coloneqq\big(\mathcal{Q}_{\ell}^{A\dot{A}},q_{\ell}^{\alpha A}\big)$
on parameter superspace $\mathcal{P}$.\label{tab:Embedding-coordinates--1}}

\end{table}
Next, let $c\left(i\right)$ denote the \emph{indicator function}
for the $N^{2}$-MHV scattering process. This map assigns to the $i$-th
gluon its corresponding cluster. As we shall demonstrate in the following,
each cluster lies within one of the five minitwistor lines described
by $\mathscr{M}_{5}$. The clustering is defined by the prescription:
\begin{equation}
c(i)\,\coloneqq\,\begin{cases}
1, & 1\leq i\leq a_{1}-1;\\
2, & a_{1}\leq i\leq a_{2}-1;\\
3, & a_{2}\leq i\leq b_{2}-1;\\
4, & b_{2}\leq i\leq b_{1}-1;\\
5 & b_{1}\leq i\leq n.
\end{cases}\label{eq:-123}
\end{equation}
Consequently, the embedding coordinates of the copy $\mathcal{P}_{i}$
of the parameter superspace assigned to the $i$-th gluon are defined
by:
\begin{equation}
\widetilde{\mathcal{Q}}_{i}^{A\dot{A}}\coloneqq\mathcal{Q}_{c(i)}^{A\dot{A}},\,\,\,\tilde{q}_{i}^{\alpha A}\coloneqq q_{c(i)}^{\alpha A}.\label{eq:-121}
\end{equation}


\paragraph{Integral Representation.}

The degree-$2$ $R$-monomial admits a representation as an integral
over the parameter superspace $\mathcal{P}$ given by: 
\begin{equation}
R_{n}^{\left(2\right)}\,=\,\mathcal{N}_{a_{1}b_{1},a_{2}b_{2}}\,\underset{\mathcal{I}\,\,\,\,\,}{\int}\,d^{16|8}\boldsymbol{\tau}\,\,\,\mathcal{F}_{a_{1}b_{1},a_{2}b_{2}}\big(\tau_{1}^{M},\tau_{2}^{M'}\big)\,\,\,\bigwedge_{i=1}^{n}\,e^{is_{i}\langle z_{i}|\widetilde{\mathcal{Q}}_{i}|\bar{z}_{i}]+i\langle z_{i}|\tilde{q}_{i}\cdot\eta_{i}\rangle}.\label{eq:-109}
\end{equation}
The normalisation factor is defined by:
\begin{equation}
\mathcal{N}_{a_{1}b_{1},a_{2}b_{2}}\,\coloneqq\,\frac{1}{\left(2\pi\right)^{8}}\,y_{a_{1}b_{1}}^{2}y_{a_{2}b_{2}}^{2},
\end{equation}
and the integrand $\mathcal{F}_{a_{1}b_{1},a_{2}b_{2}}$ takes the
form:
\begin{equation}
\mathcal{F}_{a_{1}b_{1},a_{2}b_{2}}\big(\tau_{1}^{M},\tau_{2}^{M'}\big)\,\coloneqq\,\mathcal{F}_{a_{1}b_{1}}\big(u_{1}^{A},v_{1}^{B}\big)\,\mathcal{F}_{a_{2}b_{2}}\big(u_{2}^{A},v_{2}^{A}\big).
\end{equation}
Recall that $\mathcal{F}_{a_{1}b_{1}}$ was introduced in Eq. (\ref{eq:-108}).

\subsubsection{$N^{2}$-MHV Celestial Superamplitude}

In the preceding subsection, we obtained an integral formula for the
degree-$2$ $R$-monomial (refer to Eq. (\ref{eq:-109})). Its frequency
dependence is entirely described by exponential functions. This representation
is ready for the Mellin transform required to derive the celestial
amplitude, which we now proceed to analyse.

\paragraph*{Partial Amplitudes.}

The general solution for the tree-level $N^{2}$-MHV scattering amplitude
in $\mathcal{N}=4$ SYM theory takes the form:
\begin{equation}
\mathcal{A}_{2,n}^{\mathsf{a_{1}...a_{n}}}\big(\lambda_{i}^{A},\bar{\lambda}_{i\dot{A}},\eta_{i}^{\alpha}\big)\,=\,\sum_{a_{1},b_{1}}\sum_{a_{2},b_{2}}\,\mathcal{A}_{n;a_{1}b_{1},a_{2}b_{2}}^{\mathsf{a_{1}...a_{n}}}\big(\lambda_{i}^{A},\bar{\lambda}_{i\dot{A}},\eta_{i}^{\alpha}\big).\label{eq:-110}
\end{equation}
Here, the partial amplitudes corresponding to each sequence of indices:
\begin{equation}
1\leq a_{1}\leq a_{2}\leq b_{2}\leq b_{1}\le n,
\end{equation}
are given by:
\begin{equation}
\mathcal{A}_{n;a_{1}b_{1},a_{2}b_{2}}^{\mathsf{a_{1}...a_{n}}}\big(\lambda_{i}^{A},\bar{\lambda}_{i\dot{A}},\eta_{i}^{\alpha}\big)\,=\,\left(2\pi\right)^{4}\,\delta^{4|0}\big(P^{A\dot{A}}\big)\,\delta^{0|8}\big(Q^{\alpha A}\big)\,R_{n;a_{1}b_{1}}R_{n;a_{1}b_{1},a_{2}b_{2}}\,\mathsf{Tr}\,\prod_{i=1}^{n}\,\frac{\mathsf{T}^{a_{i}}}{\lambda_{i}\cdot\lambda_{i+1}}.\label{eq:-119}
\end{equation}


\paragraph*{Celestial Parameterisation.}

The first step in constructing the celestial amplitude consists of
expressing the partial amplitudes (\ref{eq:-119}) in terms of celestial
coordinates. This parameterisation is obtained by setting $\lambda_{i}^{A}=z_{i}^{A}$
and $\bar{\lambda}_{i\dot{A}}=s_{i}\bar{z}_{i\dot{A}}$.

\textcompwordmark{}

\paragraph*{Integral Representation.}

Our next task is to derive an integral formula for the $N^{2}$-MHV
scattering amplitude. In Subsection \ref{subsec:Celestial-Superamplitude}
we showed that
\[
\left(2\pi\right)^{4}\,\delta^{4|0}\big(P^{A\dot{A}}\big)\,\delta^{0|8}\big(Q^{\alpha A}\big)
\]
admits a representation as a superspace integral (see Eq. (\ref{eq:-111})).
This distributional factor enforces total momentum and supercharge
conservation in the scattering process. By combining that result with
the form of the degree-$2$ $R$-monomial from Eq. (\ref{eq:-109}),
we arrive at the following formulation.

The partial amplitude in Eq. (\ref{eq:-119}) can be written as an
integral over the \emph{pre-moduli superspace}:
\begin{equation}
\widehat{\mathscr{M}}_{5}\,\coloneqq\,\mathbf{R}^{4|8}\times\mathbf{R}^{8|4}\times\mathbf{R}^{8|4}.
\end{equation}
We chart $\widehat{\mathscr{M}}_{5}$ by:
\begin{equation}
\hat{\gamma}^{Q}\,\coloneqq\,\big(\mathsf{x}^{K},\tau_{1}^{M},\tau_{2}^{M'}\big),
\end{equation}
where the abstract index $Q\in\{K,M,M'\}$ labels the superspace coordinates.
In the next subsection, we will apply the leaf-amplitude formalism
to reduce $\widehat{\mathscr{M}}_{5}$ to the moduli superspace $\mathscr{M}_{5}$.
That space governs the configuration of five minitwistor lines on
which the (minitwistor) amplitude localises. This justifies calling
$\widehat{\mathscr{M}}_{5}$ the pre-moduli superspace.

The orientation on $\widehat{\mathscr{M}}_{5}$ is specified by the
$\mathbf{Z}_{2}$-graded volume form:
\begin{equation}
\mathcal{D}\hat{\boldsymbol{\gamma}}\,\coloneqq\,d^{4|8}\mathsf{x}\wedge d^{8|4}\tau_{1}\wedge d^{8|4}\tau_{2}.
\end{equation}
Here $\mathsf{x}^{K}=\big(x_{A\dot{A}},\theta_{A}^{\alpha}\big)\in\mathbf{R}^{4|8}$
denotes the standard coordinates on Klein superspace, equipped with
the Berezin-de Witt measure $d^{4|8}\mathsf{x}$ defined in Eq. (\ref{eq:-94}).

With these preliminaries in place, the $N^{2}$-MHV partial superamplitude
becomes:
\begin{equation}
\mathcal{A}_{n;a_{1}b_{1},a_{2}b_{2}}^{\mathsf{a_{1}...a_{n}}}\big(z_{i}^{A},s_{i}\bar{z}_{i\dot{A}},\eta_{i}^{\alpha}\big)\,=\,\underset{\widehat{\mathscr{M}}_{5}\,\,\,\,\,}{\int}\mathcal{D}\hat{\boldsymbol{\gamma}}\,\,\,\mathcal{F}_{a_{1}b_{1},a_{2}b_{2}}\big(\hat{\gamma}^{Q}\big)\,\mathcal{T}^{\mathsf{a_{1}...a_{n}}}\big(z_{i}^{A},\bar{z}_{i\dot{A}},\eta_{i}^{\alpha};\hat{\gamma}^{Q}\big),
\end{equation}
where the trace factor is given by:
\begin{equation}
\mathcal{T}^{\mathsf{a_{1}...a_{n}}}\big(z_{i}^{A},\bar{z}_{i\dot{A}},\eta_{i}^{\alpha};\hat{\gamma}^{Q}\big)\,=\,\mathsf{Tr}\,\,\,\bigwedge_{i=1}^{n}\,\exp\big(is_{i}\langle z_{i}|x+\widetilde{\mathcal{Q}}_{i}|\bar{z}_{i}]+i\langle z_{i}|(\theta+\tilde{q}_{i})\cdot\eta_{i}\rangle\big)\,\,\,\frac{\mathsf{T}^{a_{i}}}{z_{i}\cdot z_{i+1}}.
\end{equation}


\paragraph*{Celestial Superamplitude.}

We now derive $N^{2}$-MHV celestial amplitude. To set notation, define
the multiplicative group of positive reals $\mathscr{R}\coloneqq\big(\mathbf{R}_{+},\cdot\big)$,
and its $n$-fold product $\mathscr{R}^{n}\coloneqq\bigtimes_{i=1}^{n}\mathscr{R}$.
We regard the frequency parameters $\left(s_{i}\right)$ as affine
coordinates on $\mathscr{R}^{n}$. This space carries the Haar measure
$d\rho_{s_{i}}$, as in Eq. (\ref{eq:-112}).

Next, recall that each gluon insertion on the celestial torus $\mathcal{CT}$
is parametrised by the spinors $z_{i}^{A}$ and $\bar{z}_{i}^{A}$.
The Grassmann variables $\eta_{i}^{\alpha}$ encode the helicity degrees
of freedom. Together, these define the dual real minitwistor $\mathsf{W}_{i}^{I}\,\coloneqq\,\big(z_{i}^{A},\bar{z}_{i\dot{A}},\eta_{i}^{\alpha}\big)$.

For each gluon $i$, let $\Delta_{i}$ be its celestial conformal
weight and $\epsilon_{i}$ its helicity expectation value. We then
define the scaling dimension:
\begin{equation}
h_{i}\,\coloneqq\,\frac{\Delta_{i}+\epsilon_{i}}{2}.
\end{equation}

With these preliminaries, we introduce the celestial (partial) superamplitude
by a multidimensional Mellin transform:
\begin{equation}
\widehat{\mathcal{A}}_{n;a_{1}b_{1},a_{2}b_{2}}^{\mathsf{a_{1}...a_{n}}}\big(\mathsf{W}_{i}^{I}\big)\,\coloneqq\,\underset{\mathscr{R}^{n}\,\,\,\,\,}{\int}\,d\rho_{s_{i}}\,\,\,\mathcal{A}_{n;a_{1}b_{1},a_{2}b_{2}}^{\mathsf{a_{1}...a_{n}}}\big(z_{i}^{A},s_{i}\bar{z}_{i\dot{A}},\eta_{i}^{\alpha}\big)\,\prod_{i=1}^{n}\,s_{i}^{2h_{i}}.
\end{equation}
Performing the $s_{i}$ integrals yields an explicit expression as
an integral over the pre-moduli superspace:
\begin{equation}
\widehat{\mathcal{A}}_{n;a_{1}b_{1},a_{2}b_{2}}^{\mathsf{a_{1}...a_{n}}}\big(\mathsf{W}_{i}^{I}\big)\,=\,\underset{\widehat{\mathscr{M}}_{5}\,\,\,\,\,}{\int}\,\mathcal{D}\hat{\boldsymbol{\gamma}}\,\,\,\mathcal{F}_{a_{1}b_{1},a_{2}b_{2}}\big(\hat{\gamma}^{Q}\big)\,\,\,\widehat{\mathcal{T}}^{\mathsf{a_{1}...a_{n}}}\big(\mathsf{W}_{i}^{I};\hat{\gamma}^{Q}\big),\label{eq:-115}
\end{equation}
where the new trace factor is:
\begin{equation}
\widehat{\mathcal{T}}^{\mathsf{a_{1}...a_{n}}}\big(\mathsf{W}_{i}^{I};\hat{\gamma}^{Q}\big)\,=\,\mathsf{Tr}\,\bigwedge_{i=1}^{n}\,\frac{\mathcal{C}\big(2h_{i}\big)}{\langle z_{i}|x+\widetilde{\mathcal{Q}}_{i}|\bar{z}_{i}]^{2h_{i}}}\,e^{i\langle z_{i}|(\theta+\tilde{q})\cdot\eta_{i}\rangle}\,\frac{\mathsf{T}^{a_{i}}}{z_{i}\cdot z_{i+1}}.\label{eq:-114}
\end{equation}


\subsubsection{Sectional Amplitude}

Having obtained the celestial amplitude, we now invoke the leaf amplitude
formalism. The sectional (or leaf) amplitude arises via a dimensional
reduction of $\widehat{\mathscr{M}}_{5}$ to the moduli superspace
$\mathscr{M}_{5}$, which parametrises a configuration of five minitwistor
lines in $\mathbf{MT}_{s}$. 

In this perspective, the minitwistor amplitude:
\[
\widetilde{\mathcal{M}}_{n;a_{1}b_{1},a_{2}b_{2}}^{\mathsf{a_{1}...a_{n}}}\big(\mathsf{Z}_{i}^{I}\big),
\]
is viewed as more fundamental than the original celestial amplitude.

Before proceeding, the reader should review Subsection \ref{subsec:Sectional-Amplitude}.
That section recapitulates the geometric background on Klein and projective
superspaces needed for the leaf amplitude formalism. In particular,
one should pay special attention to the measure decomposition stated
in Eq. (\ref{eq:-146}).

\textcompwordmark{}

\paragraph*{Leaf Formalism.}

We now derive the leaf amplitude by decomposing the celestial amplitude
(cf. Eq. (\ref{eq:-115})) into a sum of partial amplitudes. 

To this end, we employ the involution operator $\sharp$ introduced
in Subsection \ref{subsec:Sectional-Amplitude}. For a van der Waerden
spinor $\bar{z}_{\dot{A}}=(1,-\bar{\zeta})$, define its involute
by $\bar{z}_{\dot{A}}^{\sharp}\coloneqq(1,\bar{\zeta})$. Then extend
$\sharp$ to dual minitwistor superspace via:
\begin{equation}
\sharp:\mathsf{W}^{I}=\big(z^{A},\bar{z}_{\dot{A}},\eta^{\alpha}\big)\mapsto\mathsf{W}^{\sharp I}=\big(z^{A},\bar{z}_{\dot{A}}^{\sharp},\eta^{\alpha}\big).
\end{equation}

To define each partial amplitude, introduce the \emph{moduli superspace}:
\begin{equation}
\mathscr{M}_{5}\,\coloneqq\,\mathbf{RP}^{3|8}\times\mathbf{R}^{8|4}\times\mathbf{R}^{8|4}.
\end{equation}
This supermanifold parametrises the configuration of five minitwistor
lines supporting the $N^{2}$-MHV amplitude. A natural coordinate
chart on $\mathscr{M}_{5}$ combines the projective superspace coordinates
$\mathbb{X}^{K}$ with the two sets of moduli parameters $\tau_{1}^{M}$
and $\tau_{2}^{M'}$. We assemble these into a single coordinate map:
\begin{equation}
\gamma^{Q}\,\coloneqq\,\big(\mathbb{X}^{K},\tau_{1}^{M},\tau_{2}^{M'}\big),
\end{equation}
where the abstract index $Q\in\{K,M,M'\}$ labels the superspace coordinates.
The canonical orientation on $\mathscr{M}_{5}$ is then given by the
$\mathbf{Z}_{2}$-graded volume form:
\begin{equation}
\mathcal{D}\boldsymbol{\gamma}\,\coloneqq\,D^{3|8}\mathbb{X}\wedge d^{8|4}\tau_{1}\wedge d^{8|4}\tau_{2}.
\end{equation}

As established in \citet{melton2023celestial}, the celestial amplitude
admits the decomposition:
\begin{equation}
\widehat{\mathcal{A}}_{n;a_{1}b_{1},a_{2}b_{2}}^{\mathsf{a_{1}...a_{n}}}\big(\mathsf{W}_{i}^{I}\big)\,=\,\mathcal{B}_{n;a_{1}b_{1},a_{2}b_{2}}^{\mathsf{a_{1}...a_{n}}}\big(\mathsf{W}_{i}^{I}\big)+\mathcal{B}_{n;a_{1}b_{1},a_{2}b_{2}}^{\mathsf{a_{1}...a_{n}}}\big(\mathsf{W}_{i}^{\sharp I}\big).
\end{equation}
The partial amplitude is given by:
\begin{equation}
\mathcal{B}_{n;a_{1}b_{1},a_{2}b_{2}}^{\mathsf{a_{1}...a_{n}}}\big(\mathsf{W}_{i}^{I}\big)\,=\,\underset{\mathbf{R}_{+}\,\,\,\,\,}{\int}\,dH_{r}\,\,\,r^{4}\,\underset{\mathscr{M}_{5}\,\,\,\,\,}{\int}\,\mathcal{D}\boldsymbol{\gamma}\,\,\,\mathcal{F}_{a_{1}b_{1},a_{2}b_{2}}\big(\tau_{1}^{M},\tau_{2}^{M'}\big)\,\widehat{\mathcal{T}}^{\mathsf{a_{1},...,a_{n}}}\big(\mathsf{W}_{i}^{I};\gamma^{Q}\big),
\end{equation}
and the trace factor takes the form:
\begin{equation}
\widehat{\mathcal{T}}^{\mathsf{a_{1},...,a_{n}}}\big(\mathsf{W}_{i}^{I};\gamma^{Q}\big)\,=\,\mathsf{Tr}\,\bigwedge_{i=1}^{n}\,\frac{\mathcal{C}\left(2h_{i}\right)}{\langle z_{i}|r\mathcal{R}+\widetilde{\mathcal{Q}}_{i}|\bar{z}_{i}]^{2h_{i}}}\,e^{i\langle z_{i}|(\theta+\tilde{q}_{i})\cdot\eta_{i}\rangle}\,\frac{\mathsf{T^{a_{i}}}}{z_{i}\cdot z_{i+1}}.
\end{equation}

Consider now the rescaling of the moduli parameters:
\begin{equation}
\tau_{\ell}^{M}=\big(u_{\ell}^{A},v_{\ell}^{B},\tilde{u}_{\ell\dot{A}},\tilde{v}_{\ell\dot{B}},\chi_{\ell}^{\alpha}\big)\,\mapsto\,\tilde{\tau}_{\ell}^{M}=\big(u_{\ell}^{A},v_{\ell}^{B},r\tilde{u}_{\ell\dot{A}},r\tilde{v}_{\ell\dot{B}},\chi_{\ell}^{\alpha}\big).
\end{equation}
Under this map, the measure on $\mathscr{M}_{5}$ and the embedding
coordinates scale as:
\begin{equation}
\mathcal{D}\boldsymbol{\gamma}\,\mapsto\,r^{8}\,\mathcal{D}\boldsymbol{\gamma},\,\,\,\widetilde{\mathcal{Q}}_{i}^{A\dot{A}}\,\mapsto\,r\,\widetilde{\mathcal{Q}}_{i}^{A\dot{A}},\,\,\,\tilde{q}_{i}^{\alpha A}\,\mapsto\,\tilde{q}_{i}^{\alpha A}.
\end{equation}
Performing these substitutions in Eq. (\ref{eq:-115}) allows the
affine parameter $r$ to factor out and be integrated. One finds:
\begin{equation}
\mathcal{B}_{n;a_{1}b_{1},a_{2}b_{2}}^{\mathsf{a_{1}...a_{n}}}\big(\mathsf{W}_{i}^{I}\big)\,=\,2\pi\delta(\beta_{2})\,\mathcal{M}_{n;a_{1}b_{1},a_{2}b_{2}}^{\mathsf{a_{1}...a_{n}}}\big(\mathsf{W}_{i}^{I}\big),
\end{equation}
where the overall conformal weight parameter is defined by:
\begin{equation}
\beta_{2}\,\coloneqq\,12-2\,\sum_{i=1}^{n}h_{i}.
\end{equation}

Finally, the tree-level $N^{2}$-MHV sectional (or leaf) amplitude
assumes the form:
\begin{equation}
\mathcal{M}_{n;a_{1}b_{1},a_{2}b_{2}}^{\mathsf{a_{1}...a_{n}}}\big(\mathsf{W}_{i}^{I}\big)\,=\,\underset{\mathscr{M}_{5}\,\,\,\,\,}{\int}\,\mathcal{D}\boldsymbol{\gamma}\,\,\,\mathcal{F}_{a_{1}b_{1},a_{2}b_{2}}\big(\tau_{1}^{M},\tau_{2}^{M'}\big)\,\widetilde{\mathcal{T}}^{\mathsf{a_{1}...a_{n}}}\big(\mathsf{W}_{i}^{I};\gamma^{Q}\big),\label{eq:-116}
\end{equation}
where the trace factor is given by:
\begin{equation}
\widetilde{\mathcal{T}}^{\mathsf{a_{1}...a_{n}}}\big(\mathsf{W}_{i}^{I};\gamma^{Q}\big)\,=\,\mathsf{Tr}\,\bigwedge_{i=1}^{n}\,\frac{\mathcal{C}\left(2h_{i}\right)}{\langle z_{i}|\mathcal{R}+\widetilde{\mathcal{Q}}_{i}|\bar{z}_{i}]^{2h_{i}}}\,e^{i\langle z_{i}|(\theta+\tilde{q}_{i})\cdot\eta_{i}\rangle}\,\frac{\mathsf{T^{a_{i}}}}{z_{i}\cdot z_{i+1}}.\label{eq:-118}
\end{equation}
Hence, the sectional amplitude reduces to an integral over the moduli
superspace $\mathscr{M}_{5}$.

\subsubsection{Geometrical Formulation\label{subsec:Geometrical-Formulation-1}}

We now turn to the final task. We determine the minitwistor transform
of the sectional amplitude (refer to Eq. (\ref{eq:-116})). Using
the celestial RSVW identity, as reformulated in Subsection \ref{subsec:Geometrical-Formulation},
we deduce an expression for the $N^{2}$-MHV minitwistor amplitude
as a volume integral over the moduli superspace $\mathscr{M}_{5}$.

\paragraph*{Preliminaries.}

Let the index $\ell$ range over $\{1,...,5\}$. We work in real minitwistor
superspace $\mathbf{MT}_{s}$. Its homogeneous coordinates are:
\[
\mathsf{Z}^{I}\coloneqq\big(\lambda^{A},\mu_{\dot{A}},\psi^{\alpha}\big).
\]
Now consider the family of real minitwistor lines $\big\{\mathscr{L}_{\ell}\big(\gamma^{Q}\big)\big\}$,
parametrised by the superspace coordinates $\gamma^{Q}$. Each line
$\mathscr{L}_{\ell}\big(\gamma^{Q}\big)$ is defined by the locus
of points $\mathsf{Z}^{I}$ satisfying the supersymmetric incidence
relations:
\begin{equation}
\begin{cases}
\mu_{\dot{A}}=\lambda^{A}\big(\mathcal{R}_{A\dot{A}}+\mathcal{Q}_{\ell A\dot{A}}\big),\\
\psi^{\alpha}=\lambda^{A}\big(\theta_{A}^{\alpha}+q_{\ell A}^{\alpha}\big).
\end{cases}\label{eq:-117}
\end{equation}

For a fixed point $p\in\mathscr{M}_{5}$ with coordinates $\gamma_{*}^{Q}\coloneqq\gamma^{Q}\left(p\right)$,
the set $\big\{\mathscr{L}_{\ell}\big(\gamma_{*}^{Q}\big)\big\}$
uniquely determines a configuration of five real minitwistor lines.
As $p$ varies over $\mathscr{M}_{5}$, these configurations sweep
out all possible quintets of lines defined by the incidence relations
in Eq. (\ref{eq:-117}). Hence, $\mathscr{M}_{5}$ is identified as
the moduli superspace for these quintet families.

Next, let:
\[
\pi_{\ell}:\mathscr{L}_{\ell}\big(\gamma^{Q}\big)\longrightarrow\mathbf{RP}^{1}
\]
denote the canonical surjection. We trivialise the fibration $\pi_{\ell}$
by introducing homogeneous coordinates $[\sigma^{A}]$ on $\mathbf{RP}^{1}$.
This trivialisation allows one to define the natural measure on each
minitwistor line. We set:
\begin{equation}
D\sigma\,\coloneqq\,\varepsilon_{AB}\sigma^{A}d\sigma^{B}.
\end{equation}
An embedding of $\mathbf{RP}^{1}$ into $\mathbf{MT}_{s}$ is simply
a smooth nonsingular section of $\pi_{\ell}$. In particular, define
$\mathsf{Y}_{\ell}^{I}:\mathbf{RP}^{1}\longrightarrow\mathscr{L}_{\ell}\big(\gamma^{Q}\big)$
via:
\begin{equation}
\mathsf{Y}_{\ell}^{I}\big(\sigma^{A}\big)\,\coloneqq\,\big(\sigma^{A},\sigma^{A}\big(\mathcal{R}_{A\dot{A}}+\mathcal{Q}_{\ell A\dot{A}}\big),\sigma^{A}\big(\theta_{A}^{\alpha}+q_{\ell A}^{\alpha}\big)\big).
\end{equation}
By construction, $\pi_{\ell}\circ\mathsf{Y}_{\ell}^{I}\big(\sigma^{A}\big)=\sigma^{A}$.
Therefore, $\mathsf{Y}_{\ell}^{I}$ constitutes an embedding of $\mathbf{RP}^{1}$
into the real minitwistor line $\mathscr{L}_{\ell}\big(\gamma^{Q}\big)$.

\textcompwordmark{}

\paragraph*{Minitwistor Amplitude.}

Let $\mathbf{M}^{n}\coloneqq\bigtimes^{n}\mathbf{MT}_{s}$ be our
integration superdomain. From the celestial RSVW identity (Eq. (\ref{eq:-20})),
it follows that the sectional amplitude admits an expression as an
$n$-fold minitwistor transform:
\begin{equation}
\mathcal{M}_{n;a_{1}b_{1},a_{2}b_{2}}^{\mathsf{a_{1}...a_{n}}}\big(\mathsf{W}_{i}^{I}\big)\,=\,\underset{\mathbf{M}^{n}\,\,\,\,\,}{\int}\,\bigwedge_{i=1}^{n}D^{2|4}\mathsf{Z}_{i}\,\Psi_{2h_{i}}\big(\mathsf{Z}_{i}^{I};\mathsf{W}_{i}^{I'}\big)\,\,\,\widetilde{\mathcal{M}}_{n;a_{1}b_{1},a_{2}b_{2}}^{\mathsf{a_{1}...a_{n}}}\big(\mathsf{Z}_{i}^{I}\big).
\end{equation}
The $N^{2}$-MHV minitwistor superamplitude is given by:
\begin{equation}
\widetilde{\mathcal{M}}_{n;a_{1}b_{1},a_{2}b_{2}}^{\mathsf{a_{1}...a_{n}}}\big(\mathsf{Z}_{i}^{I}\big)\,=\,\underset{\mathscr{M}_{5}\,\,\,\,\,}{\int}\,\mathcal{D}\boldsymbol{\gamma}\,\,\,\mathcal{F}_{a_{1}b_{1},a_{2}b_{2}}\big(\gamma^{Q}\big)\,\,\,\mathsf{Tr}\,\bigwedge_{i=1}^{n}\,\underset{\mathbf{RP}^{1}\,\,\,}{\int}\,\,\,\boldsymbol{\omega}^{a_{i}}\big(\sigma^{A}\big)\,\overline{\delta}_{\left(-2h_{i},2h_{i}\right)}^{2|4}\big(\mathsf{Z}_{i}^{I};\mathsf{Y}_{i}^{I'}\big(\sigma_{i}^{A}\big)\big).\label{eq:-122}
\end{equation}
The logarithmic form $\boldsymbol{\omega}^{a_{i}}\big(\sigma_{i}^{A}\big)$
on the minitwistor line $\mathscr{L}_{\ell}\big(\gamma^{Q}\big)$
is defined as:
\begin{equation}
\boldsymbol{\omega}^{a_{i}}\big(\sigma_{i}^{A}\big)\,\coloneqq\,\mathsf{T}^{a_{i}}\,\frac{D\sigma_{i}}{\sigma_{i}\cdot\sigma_{i+1}}.
\end{equation}


\paragraph*{Conclusion.}

The minitwistor delta-functions under the integral of Eq. (\ref{eq:-122}),
\[
\overline{\delta}_{\left(-2h_{i},2h_{i}\right)}^{2|4}\big(\mathsf{Z}_{i}^{I};\mathsf{Y}_{i}^{I'}\big(\sigma_{i}^{A}\big)\big),
\]
localise the integration measure over the moduli superspace $\mathscr{M}_{5}$
onto the support defined by the family of minitwistor lines $\big\{\mathscr{L}_{\ell}\big\}$.
The celestial scaling dimensions $h_{i}$ (associated with the gluons
involved in the scattering process) appear as weights in the construction
of the volume form on $\mathscr{M}_{5}$. Thus, the minitwistor amplitude
computes a weighted volume on the moduli superspace corresponding
to a quintuple of minitwistor lines. 

Furthermore, Eq. (\ref{eq:-122}) implies that the minitwistor amplitude
vanishes whenever the $i$-th gluon does not lie on the minitwistor
line $\mathscr{L}_{c\left(i\right)}$, where $c(i)$ denotes the cluster
assignment of the $i$-th gluon, as defined in Eq. (\ref{eq:-123}).

\subsection{General Case\label{subsec:General-Case}}

Let $p\coloneqq2k+3$. We now extend our analysis to the full tree-level
celestial $\mathcal{S}$-matrix. For an $N^{1}$-MHV configuration,
we will show that the minitwistor amplitude localises on $p$ distinct
minitwistor lines. It is then computed as a volume integral over the
moduli superspace $\mathscr{M}_{p}$, which parametrises all admissible
configurations of these $p$ lines.

\subsubsection{Dual Conformal Invariant}

We begin our analysis by defining the order-$\left(k+1\right)$ $R$-invariant.
Fix a family of indices:
\begin{equation}
1\leq a_{1}<a_{2}<...<a_{k}<a_{k+1}<b_{k+1}<b_{k}<...<b_{2}\leq b_{1}\leq n.\label{eq:-138}
\end{equation}
Define the sequences of van der Warden spinors $\big\{ u_{\ell}^{A}\big\}_{1\leq\ell\leq k}$
and $\big\{ v_{k}^{A}\big\}_{1\leq\ell\leq k}$ inductively. For the
first cases, we set:
\begin{equation}
u_{1}^{A}\,\coloneqq\,z_{n}^{B}\,\big(y_{nb_{1}}\big)_{B\dot{B}}\,\big(y_{b_{1}a_{1}}^{-1}\big)^{A\dot{B}},\,\,\,v_{1}^{A}\,\coloneqq\,z_{n}^{B}\,\big(y_{na_{1}}\big)_{B\dot{B}}\,\big(y_{a_{1}b_{1}}^{-1}\big).
\end{equation}
For all $1\leq k\leq n-1$, the recursion relations are given by:
\begin{equation}
u_{k+1}^{A}\,\coloneqq\,u_{k}^{B}\,\big(y_{a_{k}b_{k+1}}\big)_{B\dot{B}}\,\big(y_{b_{k+1}a_{k+1}}\big)^{A\dot{B}},
\end{equation}
\begin{equation}
v_{k+1}^{A}\,\coloneqq\,u_{k}^{B}\,\big(y_{a_{k}a_{k+1}}\big)_{B\dot{B}}\,\big(y_{a_{k+1}b_{k+1}}^{-1}\big)^{A\dot{B}}.
\end{equation}

The \emph{order-$\left(k+1\right)$ $R$-invariant}, expressed in
terms of the celestial supercoordinates $z_{i}^{A},\bar{z}_{i\dot{A}},\eta_{i}^{\alpha}$,
is defined by:
\begin{equation}
R_{n;a_{1}b_{1},...,a_{k}b_{k}}\,\coloneqq\,\frac{\langle z_{a_{k+1}-1},z_{a_{k+1}}\rangle\langle z_{b_{k+1}-1},z_{b_{k+1}}\rangle\delta^{0|4}\big(\Theta_{k+1}^{\alpha}\big)}{y_{a_{k+1}b_{k+1}}^{2}\langle z_{a_{k+1}-1},u_{k+1}\rangle\langle u_{k+1},z_{a_{k+1}}\rangle\langle z_{b_{k+1}-1},v_{k+1}\rangle\langle v_{k+1},z_{b_{k+1}}\rangle}.\label{eq:-124}
\end{equation}
Here, $\Theta_{k+1}^{\alpha}$ is the Grassmann-valued function entering
the fermionic delta distribution, defined as:
\begin{equation}
\Theta_{k+1}^{\alpha}\big(u_{k+1}^{A},v_{k+1}^{B}\big)\,\coloneqq\,\sum_{i=a_{k}}^{a_{k+1}-1}\langle u_{k+1},z_{i}\rangle\,\eta_{i}^{\alpha}+\sum_{j=a_{k}}^{b_{k+1}-1}\langle v_{k+1},z_{j}\rangle\,\eta_{j}^{\alpha}.\label{eq:-126}
\end{equation}
The order-$\left(k+1\right)$ $R$-invariant is one of the ingredients
of the partial amplitude: 
\begin{equation}
\mathcal{A}_{n;a_{1}b_{1},...,a_{k+1}b_{k+1}}^{\mathsf{a_{1}...a_{n}}}\big(\lambda_{i}^{A},\bar{\lambda}_{i\dot{A}},\eta_{i}^{\alpha}\big),\label{eq:-142}
\end{equation}
which will be discussed in Subsection \ref{subsec:-MHV-Celestial-Amplitude}.

To compute the tree-level $N^{k+1}$-MHV celestial superamplitude,
one must perform a half-Mellin transform of the partial amplitude
(\ref{eq:-142}). However, the structure of Eq. (\ref{eq:-124}) proves
unsuitable for a direct computation of the Mellin transform. To address
this difficulty, we invoke the Fadde'ev-Popov procedure, thereby expressing
the order-$\left(k+1\right)$ $R$-invariant as an integral over auxiliary
spinor variables $u_{k+1}^{A}$ and $v_{k+1}^{A}$.

In Kleinian signature, we take the integration domain to be $\mathcal{I}\coloneqq\mathbf{R}^{2}\times\mathbf{R}^{2}$,
parametrised by $U_{k+1}^{A'}\coloneqq\big(u_{k+1}^{A},v_{k+1}^{B}\big)$.
The standard orientation of $\mathcal{I}$ is given by the Lebesgue
measure:
\begin{equation}
d^{4}U_{k+1}\,\coloneqq\,d^{2}u_{k+1}\wedge d^{2}v_{k+1}.
\end{equation}
Our aim is to factorise all dependence on the $s_{i}$ into delta-functions.
To this end, we define the spinor-valued mappings:
\begin{equation}
f_{a_{k}a_{k+1}b_{k+1}}^{A}\big(u_{k}^{B},y_{i}^{C\dot{C}}\big)\,\coloneqq\,u_{k}^{B}\,\big(y_{a_{k}b_{k+1}}\big)_{B\dot{B}}\,\big(y_{b_{k+1}a_{k+1}}^{-1}\big)^{A\dot{B}},\label{eq:-127}
\end{equation}
\begin{equation}
g_{a_{k}a_{k+1}b_{k+1}}^{A}\big(u_{k}^{B},y_{i}^{C\dot{C}}\big)\,\coloneqq\,u_{k}^{B}\,\big(y_{a_{k}a_{k+1}}\big)_{B\dot{B}}\,\big(y_{a_{k+1}b_{k+1}}^{-1}\big)^{A\dot{B}}.\label{eq:-128}
\end{equation}
The \emph{constraint hypersurface} $\mathscr{C}$ is defined by the
locus of points $U_{k+1}^{A'}\in\mathcal{I}$ satisfying:
\begin{equation}
u_{k+1}^{A}\,=\,f_{a_{k}a_{k+1}b_{k+1}}^{A}\big(u_{k}^{B},y_{i}^{C\dot{C}}\big),\,\,\,v_{k+1}^{A}\,=\,g_{a_{k}a_{k+1}b_{k+1}}^{A}\big(u_{k}^{B},y_{i}^{C\dot{C}}\big).
\end{equation}
We define the delta-distribution on $\mathcal{I}$, supported on $\mathscr{C}$,
by:
\begin{equation}
\delta_{\mathscr{C}}\big(u_{k+1}^{A},v_{k+1}^{B}\big)\,\coloneqq\,\overline{\delta}^{2}\big(u_{k+1}^{A}-f_{a_{k}a_{k+1}b_{k+1}}^{A}\big(u_{k}^{B},y_{i}^{C\dot{C}}\big)\big)\,\overline{\delta}^{2}\big(v_{k+1}^{A}-\,g_{a_{k}a_{k+1}b_{k+1}}^{A}\big(u_{k}^{B},y_{i}^{C\dot{C}}\big)\big).
\end{equation}
We may then express Eq. (\ref{eq:-124}) as a Fadde'ev-Popov integral:
\begin{equation}
R_{n;a_{1}b_{1},...,a_{k+1}b_{k+1}}\,=\,\frac{1}{y_{a_{k+1}b_{k+1}}^{2}}\,\underset{\mathcal{I}\,\,\,\,\,}{\int}\,d^{4}U_{k+1}\,\,\,\mathcal{F}_{a_{k+1}b_{k+1}}\big(u_{k+1}^{A},v_{k+1}^{B}\big)\,\delta^{0|4}\big(\Theta_{k+1}^{\alpha}\big)\,\delta_{\mathscr{C}}\big(u_{k+1}^{A},v_{k+1}^{B}\big).\label{eq:-125}
\end{equation}
where:
\begin{equation}
\mathcal{F}_{a_{k+1}b_{k+1}}\big(u_{k+1}^{A},v_{k+1}^{B}\big)\,\coloneqq\,\frac{\langle z_{a_{k+1}-1},z_{a_{k+1}}\rangle\langle z_{b_{k+1}-1},z_{b_{k+1}}\rangle}{\langle z_{a_{k+1}-1},u_{k+1}\rangle\langle u_{k+1},z_{a_{k+1}}\rangle\langle z_{b_{k+1}-1},v_{k+1}\rangle\langle v_{k+1},z_{b_{k+1}}\rangle}.
\end{equation}


\subsubsection{Fermionic and Constraint Delta Functions}

The next step in deriving the Fadde'ev-Popov representation of the
order-$k$ $R$-invariant is to expand the Grassmann and spinor delta
functions in Eq. (\ref{eq:-125}).

\textcompwordmark{}

\paragraph*{Fermionic Delta-Function.}

Recall that the fermionic delta function $\delta^{0|4}\big(\varepsilon^{\alpha}\big)$
for a Grassmann variable $\varepsilon^{\alpha}$ admits the Berezin
integral representation (see Eq. (\ref{eq:-89})). We introduce:
\[
\varepsilon^{\alpha}=\Theta_{k+1}^{\alpha}\big(u_{k+1}^{A},v_{k+1}^{B}\big),
\]
using the definition of $\Theta_{k+1}^{\alpha}$ from Eq. (\ref{eq:-126}).
It follows that:
\begin{equation}
\delta^{0|4}\big(\Theta_{k+1}^{\alpha}\big)\,=\,\underset{\mathbf{R}^{0|4}\,\,\,}{\int}\,d^{0|4}\chi_{k+1}\,\,\,\bigwedge_{i=a_{k}}^{a_{k+1}-1}\,e^{i\langle u_{k+1},z_{i}\rangle\,\chi_{k+1}\cdot\eta_{i}}\,\bigwedge_{j=a_{k}}^{b_{k+1}-1}\,e^{i\langle v_{k+1},z_{j}\rangle\,\chi_{k+1}\cdot\eta_{j}}.\label{eq:-130}
\end{equation}


\paragraph*{Constraint Delta-Function.}

Consider the constraint delta-function $\delta_{\mathscr{C}}$. For
a real van der Waerden spinor $\lambda^{A}$, the two-component delta-distribution
$\overline{\delta}^{2}\big(\lambda^{A}\big)$ is given in Eq. (\ref{eq:-102}).
Using the definition of $f_{a_{k}a_{k+1}b_{k+1}}^{A}$ from Eq. (\ref{eq:-127}),
the $u_{k+1}^{A}$-component of $\delta_{\mathscr{C}}$ admits the
Fourier representation:
\begin{align}
 & \overline{\delta}^{2}\big(u_{k+1}^{A}-f_{a_{k}a_{k+1}b_{k+1}}^{A}\big(u_{k}^{B},y_{i}^{C\dot{C}}\big)\big)\\
 & =\underset{\mathbf{R}^{2}\,\,\,}{\int}\,\frac{d^{2}\hat{u}_{k+1}}{\left(2\pi\right)^{2}}\,\,\,e^{i\langle u_{k+1},\hat{u}_{k+1}\rangle}\,\exp\big(-iu_{k}^{B}\,\big(y_{a_{k}b_{k+1}}\big)_{B\dot{B}}\,\big(y_{b_{k+1}a_{k+!}}^{-1}\big)^{A\dot{B}}\hat{u}_{k+1,A}\big).
\end{align}
Under the change of variables:
\begin{equation}
\hat{u}_{k+1,A}\,\mapsto\,\tilde{u}_{k+1}^{\dot{A}}\,\coloneqq\,\big(y_{b_{k+1}a_{k+1}}^{-1}\big)^{A\dot{A}}\,\hat{u}_{k+1,A},
\end{equation}
one finds:
\begin{align}
 & \delta\big(u_{k+1}^{A}-f_{a_{k}a_{k+1}b_{k+1}}^{A}\big(u_{k}^{B},y_{i}^{C\dot{C}}\big)\big)\label{eq:-131}\\
 & =\,\big|y_{b_{k+1}a_{k+1}}^{2}\big|\,\underset{\mathbf{R}^{2}\,\,\,}{\int}\,\frac{d^{2}\tilde{u}_{k+1}}{\left(2\pi\right)^{2}}\,\,\,\prod_{i=a_{k+1}}^{b_{k+1}-1}\,e^{-is_{i}\langle z_{i}|u_{k+1}\tilde{u}_{k+1}|\bar{z}_{i}]}\,\prod_{j=a_{k}}^{b_{k+1}-1}\,e^{-is_{j}\langle z_{i}|u_{k}\tilde{u}_{k+1}|\bar{z}_{i}]}.\label{eq:-132}
\end{align}

Analogously, with $g_{a_{k}a_{k+1}b_{k+1}}^{A}$ as in Eq. (\ref{eq:-128}),
the $v_{k+1}^{A}$-component is:
\begin{align}
 & \overline{\delta}^{2}\big(v_{k+1}^{A}-u_{k}^{B}\,\big(y_{a_{k}a_{k+1}}\big)_{B\dot{B}}\,\big(y_{a_{k+1}b_{k+1}}^{-1}\big)^{A\dot{B}}\big)\\
 & =\,\underset{\mathbf{R}^{2}\,\,\,}{\int}\,\frac{d^{2}\hat{v}_{k+1}}{\left(2\pi\right)^{2}}\,\,\,e^{i\langle v_{k+1},\hat{v}_{k+1}\rangle}\,\exp\big(-iu_{k}^{B}\,\big(y_{a_{k}a_{k+1}}\big)_{B\dot{B}}\,\big(y_{a_{k+1}b_{k+1}}^{-1}\big)^{A\dot{B}}\,\hat{v}_{k+1,A}\big).\label{eq:-129}
\end{align}
With the substitution:
\begin{equation}
\hat{v}_{k+1,A}\,\mapsto\,\tilde{v}_{k+1}^{\dot{A}}\,\coloneqq\,\big(y_{a_{k+1}b_{k+1}}^{-1}\big)^{A\dot{A}}\,\hat{v}_{k+1,A},
\end{equation}
we obtain:
\begin{align}
 & \overline{\delta}^{2}\big(v_{k+1}^{A}-\,g_{a_{k}a_{k+1}b_{k+1}}^{A}\big(u_{k}^{B},y_{i}^{C\dot{C}}\big)\big)\label{eq:-133}\\
 & =\big|y_{a_{k+1}b_{k+1}}^{2}\big|\,\underset{\mathbf{R}^{2}\,\,\,}{\int}\,\frac{d^{2}\tilde{v}_{k+1}}{\left(2\pi\right)^{2}}\,\,\,\prod_{i=a_{k+1}}^{b_{k+1}-1}\,e^{is_{i}\langle z_{i}|v_{k+1}\tilde{v}_{k+1}|\bar{z}_{i}]}\,\prod_{j=a_{k}}^{a_{k+1}-1}\,e^{-is_{j}\langle z_{j}|u_{k}\tilde{v}_{k+1}|\bar{z}_{j}]}.\label{eq:-134}
\end{align}


\subsubsection{Fadde'ev-Popov Representation\label{subsec:Fadde'ev-Popov-Representation-General-Case}}

In the preceding subsections, we derived an expression for the order-$k$
$R$-invariant as an integral over the domain $\mathcal{I}$ (see
Eq. (\ref{eq:-125})). This integral is localised on the constraint
hypersurface $\mathscr{C}\subset\mathcal{I}$ via the Dirac delta
distribution $\delta_{\mathscr{C}}$. We then expanded the Grassmann
and spinor delta functions in the integrand in terms of the celestial
coordinates $z_{i}^{A}$, $\bar{z}_{i\dot{A}}$ and $\eta_{i}^{\alpha}$.

Substituting the expansions of Eqs. (\ref{eq:-130}), (\ref{eq:-132})
and (\ref{eq:-134}) into Eq. (\ref{eq:-125}) for the $R$-invariant
yields the following formulation.

\textcompwordmark{}

\paragraph*{Integration Superdomain.}

The Fadde'ev-Popov representation of the order-$k$ $R$-invariant
is given by an integral over the parameter superspace:
\begin{equation}
\mathcal{P}_{k+1}\,\coloneqq\,\mathbf{R}^{8|4}.
\end{equation}
Let the abstract index $M$ range over $\{A,B,\dot{A},\dot{B},\alpha\}$.
The parameter superspace is globally charted by the coordinates:
\begin{equation}
\tau_{k+1}^{M}\,\coloneqq\,\big(u_{k+1}^{A},v_{k+1}^{B},\tilde{u}_{k+1,\dot{A}},\tilde{v}_{k+1,\dot{B}},\chi_{k+1}^{\alpha}\big).
\end{equation}
Moreover, the orientation of $\mathcal{P}_{k+1}$ is provided by the
measure:
\begin{equation}
d^{8|4}\tau_{k+1}\,\coloneqq\,d^{2}u_{k+1}\wedge d^{2}v_{k+1}\wedge d^{2}\tilde{u}_{k+1}\wedge d^{2}\tilde{v}_{k+1}\wedge d^{0|4}\chi_{k+1}.
\end{equation}

In subsequent subsections, we will explicitly define the moduli superspace
$\mathscr{M}_{2k+3}$ that fully characterises the configuration of
a system consisting of $2k+3$ minitwistor lines. For now, we regard
$\mathscr{M}_{2k+3}$ as an abstract supermanifold. In this context,
the coordinate functions $\tau_{k+1}^{M}$ parametrise a supersymmetric
submanifold of $\mathscr{M}_{2k+3}$. Indeed, for each gluon $i$
participating in the $N^{k+1}$-MHV scattering process, there exists
a corresponding copy of this submanifold, denoted by $\mathcal{P}_{i}$,
and parametrised by the coordinate functions $\tau_{i}^{M}:\mathcal{P}_{i}\longrightarrow\mathbf{R}^{8|4}$.

\textcompwordmark{}

\paragraph*{Embedding Coordinates.}

Now, if $\mathscr{M}_{2k+3}$ is regarded as an abstract supermanifold
and each parameter superspace $\mathcal{P}_{i}$ is identified with
a submanifold thereof, how does one define the natural embedding coordinates
of $\mathcal{P}_{i}$ in $\mathscr{M}_{2k+3}$? The answer is provided
by Eqs. (\ref{eq:-130}), (\ref{eq:-131}) and (\ref{eq:-133}).

Examining the arguments within the exponential functions of these
expansions, we introduce the coordinate maps:
\[
\big(p_{i}^{A\dot{A}},\xi_{i}^{\alpha A}\big):\mathcal{P}_{i}\subset\mathscr{M}_{2k+1}\longrightarrow\mathbf{R}^{4}\times\mathbf{R}^{0|8},
\]
defined in Table \ref{tab:Embedding-coordinates-}. The quantities
$p_{i}^{A\dot{A}}$ and $\xi_{i}^{\alpha A}$ shall henceforth be
referred to as the \emph{embedding coordinates} of the parameter superspace
$\mathcal{P}_{i}$ associated with the $i$-th gluon.

\textcompwordmark{}

\begin{table}
\begin{centering}
\begin{tabular}{ccc}
\toprule 
Interval for $i$ & $p_{i}^{A\dot{A}}$ & $\xi_{i}^{\alpha A}$\tabularnewline
\midrule
\midrule 
$a_{k}\leq i\leq a_{k+1}-1$ & $-u_{k}^{A}\,\big(\tilde{u}_{k+1}^{\dot{A}}+\tilde{v}_{k+1}^{\dot{A}}\big)$ & $-\big(u_{k+1}^{A}+v_{k+1}^{A}\big)\,\chi_{k+1}^{\alpha}$\tabularnewline
\midrule
\midrule 
$a_{k+1}\leq i\leq b_{k+1}-1$ & $-\big(u_{k}^{A}+u_{k+1}^{A}\big)\,\tilde{u}_{k+1}^{\dot{A}}+v_{k+1}^{A}\tilde{v}_{k+1}^{\dot{A}}$ & $-v_{k+1}^{A}\chi_{k+1}^{\alpha}$\tabularnewline
\midrule 
otherwise & $0$ & $0$\tabularnewline
\bottomrule
\end{tabular}
\par\end{centering}
\caption{Embedding coordinates $\big(p_{i}^{A\dot{A}},\xi_{i}^{\alpha A}\big)$
of the parameter superspace $\mathcal{P}_{i}$.\label{tab:Embedding-coordinates-}}

\end{table}


\paragraph*{Integral Representation.}

By substituting Eqs. (\ref{eq:-130}), (\ref{eq:-131}) and (\ref{eq:-133})
into Eq. (\ref{eq:-125}), we obtain the Fadde'ev-Popov representation
of the order-$\left(k+1\right)$ $R$-invariant:
\begin{equation}
R_{n;a_{1}b_{1},...,a_{k+1}b_{k+1}}\,=\,\mathcal{N}_{a_{k+1}b_{k+1}}\,\underset{\mathcal{P}\,\,\,}{\int}\,d^{8|4}\tau_{k+1}\,\,\,\mathcal{F}_{a_{k+1}b_{k+1}}\big(\tau_{k+1}^{M}\big)\,\bigwedge_{i=1}^{n}\,e^{is_{i}\langle z_{i}|p_{i}|\bar{z}_{i}]+i\langle z_{i}|\xi_{i}\cdot\eta_{i}\rangle}.\label{eq:-137}
\end{equation}
The normalisation factor is defined by:
\begin{equation}
\mathcal{N}_{a_{k+1}b_{k+1}}\,\coloneqq\,\frac{1}{\left(2\pi\right)^{4}}\,y_{a_{k+1}b_{k+1}}^{2}.
\end{equation}


\subsubsection{Induction Hypothesis\label{subsec:Induction-Hypothesis}}

Our next objective is to generalise the method used for the $N^{1}$-
and $N^{2}$-MHV celestial amplitudes by formulating an induction
hypothesis for the $N^{k+1}$-MHV case. As before, we regard the moduli
superspace $\mathscr{M}_{2k+3}$ as an abstract supermanifold; its
detailed structure will be specified in the following subsection.

Let the index $m$ range over $1,...,k$. For each $m$, we postulate
a parameter superspace $\mathcal{P}_{m}$ with global coordinates:
\begin{equation}
\tau_{m}^{M}\,\coloneqq\,\big(u_{m}^{A},v_{m}^{B},\tilde{u}_{m\dot{A}},\tilde{v}_{m\dot{B}},\chi_{m}^{\alpha}\big).
\end{equation}
In later subsections, we will demonstrate that these $\tau^{M}$ parametrise
a supersymmetric submanifold of $\mathscr{M}_{2k+3}$.

We further postulate the existence of embedding coordinates:
\[
\mathsf{Q}_{i}^{K}\coloneqq\,\big(\mathcal{Q}_{i}^{A\dot{A}},q_{i}^{\alpha A}\big):\mathcal{P}_{1}\times...\times\mathcal{P}_{k}\longrightarrow\mathbf{R}^{4|8}.
\]
These functions depend on the moduli parameters $\tau^{M}$ and are
assumed to satisfy the axioms listed below.

\textcompwordmark{}

\paragraph*{Axiom 1. Moduli Reparametrization.}

Under a reparametrization of the moduli superspaces $\mathcal{P}_{m}$
given by the transformations:
\begin{equation}
\tau_{m}^{M}\,=\,\big(u_{m}^{A},v_{m}^{B},\tilde{u}_{m\dot{A}},\tilde{v}_{m\dot{B}},\chi_{m}^{\alpha}\big)\,\mapsto\,\tilde{\tau}_{m}^{M}\,=\,\big(u_{m}^{A},v_{m}^{B},r\tilde{u}_{m\dot{A}},r\tilde{v}_{m\dot{B}},\chi_{m}^{\alpha}\big),
\end{equation}
the embedding coordinates transform according to:
\begin{equation}
\mathcal{Q}_{i}^{A\dot{A}}\mapsto r\mathcal{Q}_{i}^{A\dot{A}},\,\,\,q_{i}^{\alpha A}\mapsto q_{i}^{\alpha A}.
\end{equation}

\textcompwordmark{}

Before stating the next axiom, we introduce the integration superdomain:
\begin{equation}
\mathscr{E}^{\left(k\right)}\,\coloneqq\,\bigtimes_{m=1}^{k}\,\mathcal{P}_{m}.
\end{equation}
This supermanifold is charted by:
\begin{equation}
\boldsymbol{\tau}^{P}\,\coloneqq\,\big(\tau_{1}^{M_{1}},\tau_{2}^{M_{2}},...,\tau_{k}^{M_{k}}\big),
\end{equation}
where the abstract index $P$ runs over $M_{1},...,M_{k}$. Its canonical
$\mathbf{Z}_{2}$-graded volume form is:
\begin{equation}
\mathcal{D}^{\left(k\right)}\boldsymbol{\tau}\,\coloneqq\,\bigwedge_{m=1}^{k}\,d^{8|4}\tau_{m}.
\end{equation}


\paragraph*{Axiom 2. Integral Representation.}

Define the order-$k$ $R$-monomial by:
\begin{equation}
R_{n}^{\left(k\right)}\,\coloneqq\,\bigwedge_{m=1}^{k}\,R_{n;a_{1}b_{1},...,a_{m}b_{m}}.\label{eq:-136}
\end{equation}
We postulate that the embedding coordinates $\mathcal{Q}_{i}^{A\dot{A}}$
and $q_{i}^{\alpha A}$ are such that:
\begin{equation}
R_{n}^{\left(k\right)}\,=\,\underset{\mathscr{E}^{\left(k\right)}\,\,\,}{\int}\,\mathcal{D}^{\left(k\right)}\boldsymbol{\tau}\,\,\,\mathcal{F}_{a_{1}b_{1},...,a_{k}b_{k}}\big(\tau^{P}\big)\,\bigwedge_{i=1}^{n}\,e^{is_{i}\langle z_{i}|\mathcal{Q}_{i}|\bar{z}_{i}]+i\langle z_{i}|q_{i}\cdot\eta_{i}\rangle},\label{eq:-135}
\end{equation}
where:
\begin{equation}
\mathcal{F}_{a_{1}b_{1},...,a_{k}b_{k}}\big(\tau^{P}\big)\,\coloneqq\,\prod_{\ell=1}^{k}\,\mathcal{F}_{a_{\ell}b_{\ell}}\big(\tau_{\ell}^{M}\big).
\end{equation}

\textcompwordmark{}

Axioms 1 and 2 are motivated by our explicit constructions of the
$N^{1}$- and $N^{2}$-MHV celestial amplitudes. In each of these
cases, one finds a consistent set of moduli parameters satisfying
Eq. (\ref{eq:-135}). For the $N^{1}$-MHV case, see Subsection \ref{subsec:Integral-Representation-for},
especially Eq. (\ref{eq:-15}). Likewise, the $N^{2}$-MHV construction
is reviewed in Subsection \ref{subsec:Integral-Representation-Second-Order},
especially Eq. (\ref{eq:-109}). Together, these lower-order examples
demonstrate that the integral-representation postulate (Axiom 2) naturally
extends to the general order-$k$ $R$-invariant.

\subsubsection{Outline of the Argument \label{subsec:-MHV-Celestial-Amplitude}}

In Subsection \ref{subsec:Fadde'ev-Popov-Representation-General-Case},
we applied the Fadde'ev-Popov method to derive an integral representation
of the order-$(k+1)$ $R$-invariant. That representation is written
as an integral over the parameter superspace $\mathcal{P}\simeq\mathbf{R}^{8|4}$. 

In Subsection \ref{subsec:Induction-Hypothesis}, we introduced our
induction hypothesis. We assumed the existence of embedding coordinates
$\mathsf{Q}_{i}^{K}$ that chart each superspace $\mathcal{P}_{m}$
for $1\leq m\leq k$. Using these coordinates, we then postulated
the integral formula of Eq. (\ref{eq:-135}) for the degree-$k$ $R$-monomial.
This formula is defined over the integration superdomain $\mathscr{E}^{\left(k\right)}\coloneqq\bigtimes_{m=1}^{k}\mathcal{P}_{m}$.

\textcompwordmark{}

To derive the $N^{k+1}$-MHV celestial amplitude, we begin by analysing
the partial amplitudes. Let $\{a_{\ell},b_{\ell}\}$ be a family of
indices satisfying $a_{\ell}<b_{\ell}$ for all $\ell=1,...,k+1$.
The corresponding partial amplitude is\footnote{See \citet{drummond2010dual}.}:
\begin{equation}
\mathcal{A}_{n;a_{1}b_{1},...,a_{k+1}b_{k+1}}^{\mathsf{a_{1}...a_{n}}}\big(\lambda_{i}^{A},\bar{\lambda}_{i\dot{A}},\eta_{i}^{\alpha}\big)\,=\,\left(2\pi\right)^{4}\delta^{4|0}\big(P^{A\dot{A}}\big)\delta^{0|8}\big(Q^{\alpha A}\big)\,A_{n;a_{1}b_{1},...,a_{k+1}b_{k+1}}^{\mathsf{a_{1}...a_{n}}}\big(\lambda_{i}^{A},\bar{\lambda}_{i\dot{A}},\eta_{i}^{\alpha}\big).\label{eq:-139}
\end{equation}
The reduced amplitude is defined by:
\begin{equation}
A_{n;a_{1}b_{1},...,a_{k+1}b_{k+1}}^{\mathsf{a_{1}...a_{n}}}\big(\lambda_{i}^{A},\bar{\lambda}_{i\dot{A}},\eta_{i}^{\alpha}\big)\,\coloneqq\,\bigwedge_{\ell=1}^{k+1}\,R_{n;a_{1}b_{1},...,a_{\ell}b_{\ell}}\,\mathsf{Tr}\,\prod_{i=1}^{n}\,\frac{\mathsf{T}^{a_{i}}}{\lambda_{i}\cdot\lambda_{i+1}}.\label{eq:-141}
\end{equation}

The full tree-level $N^{k+1}$-MHV superamplitude $\mathcal{A}_{n}^{\mathsf{a_{1}...a_{n}}}$
is obtained by summing two classes of contributions. The first class
consists of the partial amplitudes just defined. The second class
comprises degenerate cases. A detailed classification appears in \citet{drummond2009all,korchemsky2010twistor,korchemsky2010superconformal}.

To show that the minitwistor amplitude localises on configurations
of $2k+3$ minitwistor lines, it suffices to consider the ``canonical''
partial amplitude in Eq. (\ref{eq:-139}). We assume the index families
$\{a_{\ell},b_{\ell}\}$ satisfy the inequalities of Eq. (\ref{eq:-138}).
All other configurations then follow by relabelling or by taking degenerate
sub-amplitudes. 

Our derivation proceeds in four steps:
\begin{enumerate}
\item \emph{Integral Representations. }We merge the integral formula for
the degree-$k$ $R$-monomial (Eq. (\ref{eq:-135})) with the Fadde'ev-Popov
representation of the order-$(k+1)$ $R$-invariant (Eq. (\ref{eq:-137})).
\\
This construction expresses the partial amplitude as an integral over
the supermanifold $\widehat{\mathscr{M}}_{2k+3}$, which we term the
``pre-moduli'' superspace.
\item \emph{Mellin Transform.} We perform a Mellin transform on the resulting
integral. This yields the $N^{k+1}$-MVH celestial amplitude.
\item \emph{Dimensional Reduction.} Invoking the leaf amplitude formalism,
we carry out a dimensional reduction of $\widehat{\mathscr{M}}_{2k+3}$.
The result is the moduli superspace $\mathscr{M}_{2k+3}$, which parametrises
$2k+3$ minitwistor lines.
\item \emph{Minitwistor Amplitude.} Finally, we apply the celestial RSVW
identity. This step produces the corresponding minitwistor amplitude.
\end{enumerate}
%

\subsubsection{Celestial Amplitude}

In this subsection, we first derive an integral representation for
the degree-$(k+1)$ $R$-monomial. We then compute the Mellin transform
of the canonical partial amplitude introduced in Eq. (\ref{eq:-139}).
The outcome of this computation is the tree-level $N^{k+1}$-MHV celestial
superamplitude.

\textcompwordmark{}

\paragraph*{Notation.}

For brevity, we let the index $\ell$ run over $1,...,k+1$. We introduce
the compact label:
\[
(ab)_{\ell}\coloneqq(a_{1}b_{1},...,a_{k+1}b_{k+1}),
\]
and denote the corresponding $N^{k+1}$-MHV partial amplitude by:
\[
\mathcal{A}_{n;(ab)_{\ell}}^{\mathsf{a_{1}...a_{n}}}.
\]
This notation is unambiguous: the index structure $(ab)_{\ell}$ singles
out the ``canonical'' sub-amplitude in Eq. (\ref{eq:-139}) whose
sum reproduces the full scattering amplitude.

\textcompwordmark{}

\paragraph*{Preliminaries.}

We begin by introducing the \emph{degree-$(k+1)$ $R$-monomial},
which plays a central role in our construction of the $N^{k+1}$-MHV
celestial amplitude. It is defined by:
\begin{equation}
R_{n}^{\left(k+1\right)}\,\coloneqq\,\bigwedge_{\ell=1}^{k+1}\,R_{n;a_{1}b_{1},...,a_{\ell}b_{\ell}}.
\end{equation}
Our first task is to derive an integral representation for $R_{n}^{\left(k+1\right)}$.

The \emph{integration superdomain} for this representation is defined
by:
\begin{equation}
\mathscr{E}_{k+1}\coloneqq\bigtimes_{\ell=1}^{k+1}\mathcal{P}_{\ell}.
\end{equation}
We chart this supermanifold by the coordinates:
\begin{equation}
\tau^{P}\,\coloneqq\,\big(\tau_{1}^{M_{1}},...,\tau_{k+1}^{M_{k+1}}\big),
\end{equation}
where the abstract index $P$ runs over the list $M_{1},...,M_{k+1}$. 

The canonical $\mathbf{Z}_{2}$-graded volume form on $\mathscr{E}_{k+1}$
is:
\begin{equation}
\mathcal{D}^{\left(k+1\right)}\boldsymbol{\tau}\,\coloneqq\,\bigwedge_{\ell=1}^{k+1}\,d^{8|4}\tau_{\ell}.
\end{equation}

Each factor $\mathcal{P}_{\ell}$ is parametrised by embedding coordinates
$\mathsf{Q}_{i}^{K}=\big(\mathcal{Q}_{i}^{A\dot{A}},q_{i}^{\alpha A}\big)$.
From these, we define embedding coordinates on $\mathscr{E}_{k+1}$:
\[
\widetilde{\mathsf{Q}}_{i}^{K}\,\coloneqq\,\big(\widetilde{\mathcal{Q}}_{i}^{A\dot{A}},\tilde{q}_{i}^{\alpha A}\big):\mathscr{E}_{k+1}\longrightarrow\mathbf{R}^{4|8}.
\]
Explicit expressions for $\widetilde{\mathsf{Q}}_{i}^{K}$ appear
in Table \ref{tab:Supercoordinates}.

\begin{table}
\begin{centering}
\begin{tabular}{ccc}
\toprule 
Interval for $i$ & $\widetilde{\mathcal{Q}}_{i}^{A\dot{A}}$ & $\tilde{q}_{i}^{\alpha A}$\tabularnewline
\midrule
\midrule 
$a_{k}\leq i\leq a_{k+1}-1$ & $\mathcal{Q}_{i}^{A\dot{A}}-u_{k}^{A}\,\big(\tilde{u}_{k+1}^{\dot{A}}+\tilde{v}_{k+1}^{\dot{A}}\big)$ & $q_{i}^{\alpha A}-\big(u_{k+1}^{A}+v_{k+1}^{A}\big)\,\chi_{k+1}^{\alpha}$\tabularnewline
\midrule
\midrule 
$a_{k+1}\leq i\leq b_{k+1}-1$ & $\mathcal{Q}_{i}^{A\dot{A}}-\big(u_{k}^{A}+u_{k+1}^{A}\big)\,\tilde{u}_{k+1}^{\dot{A}}+v_{k+1}^{A}\tilde{v}_{k+1}^{\dot{A}}$ & $q_{i}^{\alpha A}-v_{k+1}^{A}\chi_{k+1}^{\alpha}$\tabularnewline
\midrule 
otherwise & $\mathcal{Q}_{i}^{A\dot{A}}$ & $q_{i}^{\alpha A}$\tabularnewline
\bottomrule
\end{tabular}
\par\end{centering}
\caption{Embedding coordinates $\widetilde{\mathsf{Q}}_{i}^{K}=\big(\widetilde{\mathcal{Q}}_{i}^{A\dot{A}},\tilde{q}_{i}^{\alpha A}\big)$
on the integration superdomain $\mathscr{E}_{k+1}$.\label{tab:Supercoordinates}}

\end{table}

We now invoke the second postulate of Subsection \ref{subsec:Induction-Hypothesis}.
By multiplying Eqs. (\ref{eq:-137}) and (\ref{eq:-135}), one obtains
an integral formula for the degree-$(k+1)$ $R$-monomial:
\begin{equation}
R_{n}^{\left(k+1\right)}\,=\,\mathcal{N}_{(ab)_{\ell}}\,\underset{\mathscr{E}_{k+1}\,\,\,}{\int}\,\mathcal{D}^{\left(k+1\right)}\boldsymbol{\tau}\,\,\,\mathcal{F}_{(ab)_{\ell}}\big(\tau^{\dot{P}}\big)\,\bigwedge_{i=1}^{n}\,\exp\big(is_{i}\langle z_{i}|\widetilde{\mathcal{Q}}_{i}|\bar{z}_{i}]+i\langle z_{i}|\tilde{q}_{i}\cdot\eta_{i}\rangle\big).\label{eq:-140}
\end{equation}
Here the \emph{weight function} is:
\begin{equation}
\mathcal{F}_{(ab)_{\ell}}\big(\tau^{\dot{P}}\big)\,\coloneqq\,\prod_{\ell=1}^{k+1}\,\mathcal{F}_{a_{\ell}b_{\ell}}\big(\tau_{\ell}^{\dot{P}}\big),
\end{equation}
and the overall normalisation factor is:
\begin{equation}
\mathcal{N}_{a_{1}b_{1},...,a_{k+1}b_{k+1}}\,\coloneqq\,\frac{1}{\left(2\pi\right)^{4\left(k+1\right)}}\,\prod_{\ell=1}^{k+1}\,y_{a_{\ell}b_{\ell}}^{2}.
\end{equation}

\textcompwordmark{}

\paragraph*{Celestial Reparametrization.}

We proceed by expressing the partial amplitude (\ref{eq:-139}) in
terms of the celestial coordinates $z_{i}^{A},\bar{z}_{i\dot{A}},\eta_{i}^{\alpha}$.
For each gluon $i$, the normalised spinor basis $\{z_{i}^{A},\bar{z}_{i\dot{A}}\}$
marks its insertion on the celestial torus $\mathcal{CT}$. The Grassmann
variables $\eta_{i}^{\alpha}$ encode helicity. Thus we set:
\[
\lambda_{i}^{A}=z_{i}^{A},\,\,\,\bar{\lambda}_{i\dot{A}}=s_{i}\bar{z}_{i\dot{A}}.
\]

Next, we derive an integral representation for the $N^{k+1}$-MHV
partial amplitude in terms of the embedding coordinates $\widetilde{\mathsf{Q}}_{i}^{K}$.
The integration domain is the \emph{pre-moduli superspace}:
\begin{equation}
\widehat{\mathscr{M}}_{2k+3}\,\coloneqq\,\mathbf{R}^{4|8}\times\mathscr{E}_{k+1}.
\end{equation}
Under the leaf amplitude formalism, $\widehat{\mathscr{M}}_{2k+3}$
reduces to the moduli superspace parametrising $2k+3$ minitwistor
lines.

The supermanifold $\widehat{\mathscr{M}}_{2k+3}$ is globally charted
by:
\begin{equation}
\hat{\gamma}^{Q}\coloneqq\big(\mathsf{x}^{K},\tau_{1}^{M_{1}},...,\tau_{k+1}^{M_{k+1}}\big),
\end{equation}
with abstract index $Q\in\{K,M_{1},...,M_{k+1}\}$. Its natural $\mathbf{Z}_{2}$-graded
volume form is:
\begin{equation}
\mathcal{D}\hat{\boldsymbol{\gamma}}\,\coloneqq\,d^{4|8}\mathsf{x}\wedge\mathcal{D}^{\left(k+1\right)}\boldsymbol{\tau}\,=\,d^{4|8}\mathsf{x}\wedge d^{8|4}\tau_{1}\wedge...\wedge d^{8|4}\tau_{k+1},
\end{equation}
where $d^{4|8}\mathsf{x}$ is the Berezin-de Witt measure on $\mathbf{R}^{4|8}$.

Substituting the integral form of the degree-$(k+1)$ $R$-monomial
(Eq. (\ref{eq:-135})) into the $N^{k+1}$-MHV partial amplitude yields:
\begin{equation}
\mathcal{A}_{n;(a_{\ell}b_{\ell})}^{\mathsf{a_{1}...a_{n}}}\big(z_{i}^{A},s_{i}\bar{z}_{i\dot{A}},\eta_{i}^{\alpha}\big)\,=\,\mathcal{N}_{(a_{\ell}b_{\ell})}\underset{\widehat{\mathscr{M}}_{2k+3}\,\,\,}{\int}\,\mathcal{D}\hat{\boldsymbol{\gamma}}\,\,\,\mathcal{F}_{(a_{\ell}b_{\ell})}\big(\hat{\gamma}^{Q}\big)\,\mathcal{T}^{\mathsf{a_{1}...a_{n}}}\big(\hat{\gamma}^{Q};z_{i}^{A},s_{i}\bar{z}_{i\dot{A}},\eta_{i}^{\alpha}\big).\label{eq:-147}
\end{equation}
Here, the trace factor is given by:
\begin{equation}
\mathcal{T}^{\mathsf{a_{1}...a_{n}}}\big(\hat{\gamma}^{Q};z_{i}^{A},s_{i}\bar{z}_{i\dot{A}},\eta_{i}^{\alpha}\big)\,=\,\mathsf{Tr}\,\,\,\bigwedge_{i=1}^{n}\,\exp\big(is_{i}\langle z_{i}|x+\widetilde{\mathcal{Q}}_{i}|\bar{z}_{i}]+i\langle z_{i}|(\theta+\tilde{q}_{i})\cdot\eta_{i}\rangle\big)\,\frac{\mathsf{T^{a_{i}}}}{z_{i}\cdot z_{i+1}}.
\end{equation}


\paragraph*{Mellin Transform.}

We now compute the Mellin transform of the integral formula derived
in Eq. (\ref{eq:-147}) for the $N^{k+1}$-MHV superamplitude. In
that formula, all dependence on the frequency parameters $s_{i}$
appears in exponential factors. The Mellin transform then produces
the desired celestial amplitude.

Let $\mathscr{R}$ be the multiplicative group of positive real numbers,
and denote its $n$-fold direct product by $\mathscr{R}^{n}\coloneqq\bigtimes^{n}\mathscr{R}$.
We regard the frequency parameters $s_{i}$ as affine coordinates
on $\mathscr{R}$, so that $(s_{i})$ defines a Cartesian chart on
$\mathscr{R}^{n}$. The natural orientation on $\mathscr{R}^{n}$
is given by the Haar measure $d\rho_{s_{i}}$ (see Eq. (\ref{eq:-112})).

We combine the normalised spinor basis $\{z_{i}^{A},\bar{z}_{i\dot{A}}\}$,
which locates the insertion point of the $i$-th gluon on the celestial
torus, with the Grassmann variables $\eta_{i}^{\alpha}$ encoding
its helicity, into the dual real minitwistor:
\[
\mathsf{W}^{I}\,\coloneqq\,\big(z_{i}^{A},\bar{z}_{i\dot{A}},\eta_{i}^{\alpha}\big).
\]

Therefore, we define the tree-level $N^{k+1}$-MHV celestial superamplitude
as the $n$-dimensional Mellin transform over $\mathscr{R}^{n}$:
\begin{equation}
\widehat{\mathcal{A}}_{n;(ab)_{\ell}}^{\mathsf{a_{1}...a_{n}}}\big(\mathsf{W}_{i}^{I}\big)\,\coloneqq\,\underset{\mathscr{R}^{n}\,\,\,}{\int}\,d\rho_{s_{i}}\,\,\,\mathcal{A}_{n;(a_{\ell}b_{\ell})}^{\mathsf{a_{1}...a_{n}}}\big(z_{i}^{A},s_{i}\bar{z}_{i\dot{A}},\eta_{i}^{\alpha}\big)\,\prod_{i=1}^{n}\,s_{i}^{2h_{i}}.
\end{equation}
Substituting Eq. (\ref{eq:-147}) into this definition and performing
the integrals over $s_{i}$ yields:
\begin{equation}
\widehat{\mathcal{A}}_{n;(ab)_{\ell}}^{\mathsf{a_{1}...a_{n}}}\big(\mathsf{W}_{i}^{I}\big)\,=\,\boldsymbol{P}_{(ab)_{\ell}}\,\widehat{A}_{n;(ab)_{\ell}}^{\mathsf{a_{1}...a_{n}}}\big(\mathsf{W}_{i}^{I}\big).
\end{equation}
The reduced celestial amplitude is given by:
\begin{equation}
\widehat{A}_{n;(ab)_{\ell}}^{\mathsf{a_{1}...a_{n}}}\big(\mathsf{W}_{i}^{I}\big)\,=\,\underset{\widehat{\mathscr{M}}_{2k+3}\,\,\,}{\int}\,\mathcal{D}\hat{\boldsymbol{\gamma}}\,\,\,\mathcal{F}_{(ab)_{\ell}}\big(\hat{\gamma}^{Q}\big)\,\widehat{\mathcal{T}}^{\mathsf{a_{1}...a_{n}}}\big(\mathsf{W}^{I};\hat{\gamma}^{Q}\big),\label{eq:-148}
\end{equation}
with the celestial trace factor:
\begin{equation}
\widehat{\mathcal{T}}^{\mathsf{a_{1}...a_{n}}}\big(\mathsf{W}^{I};\hat{\gamma}^{Q}\big)\,\coloneqq\,\mathsf{Tr}\,\bigwedge_{i=1}^{n}\,\frac{\mathcal{C}(2h_{i})}{\langle z_{i}|x+\widetilde{\mathcal{Q}}_{i}|\bar{z}_{i}]^{2h_{i}}}\,e^{i\langle z_{i}|(\theta+\tilde{q}_{i})\cdot\eta_{i}\rangle}\,\frac{\mathsf{\mathsf{T}^{a_{i}}}}{z_{i}\cdot z_{i+1}}.
\end{equation}


\subsubsection{Dimensional Reduction}

We now apply the leaf amplitude formalism to the celestial amplitude
obtained in Eq. (\ref{eq:-148}). In this approach, the sectional
amplitude arises by reducing the pre-moduli supermanifold $\widehat{\mathscr{M}}_{2k+3}$
down to the moduli superspace $\mathscr{M}_{2k+3}$.

Our construction rests on the correspondence between Klein and projective
superspaces reviewed in Subsection \ref{subsec:Sectional-Amplitude}.
There we showed that the (supersymmetric) timelike wedge $W_{s}^{-}\subset\mathbf{K}^{4|8}$
admits coordinates $\big(r,\mathbb{X}^{K}\big)$, where $r$ is an
affine parameter on $\mathscr{R}$ and $\mathbb{X}^{K}=\big(X_{A\dot{A}},\theta_{A}^{\alpha}\big)$
are homogeneous coordinates on $\mathbf{RP}^{3|8}$. A key result
is the decomposition of the Berezin-de Witt measure on $W_{s}^{-}$:
\begin{equation}
d^{4|8}\mathsf{x}\,\big|_{W_{s}^{-}}\,=\,r^{4}d\rho_{r}\wedge D^{3|8}\mathbb{X},\label{eq:-150}
\end{equation}
where $d\rho_{r}\coloneqq d\log r$ is the Haar measure on $\mathscr{R}$. 

Let $\mathsf{W}_{i}^{\sharp I}$ denote the involute dual minitwistor
as defined in Eq. (\ref{eq:-149}). The first step in deriving the
sectional amplitude is to split the celestial amplitude:
\begin{equation}
\widehat{A}_{n;(ab)_{\ell}}^{\mathsf{a_{1}...a_{n}}}\big(\mathsf{W}_{i}^{I}\big)=\widehat{B}_{n;(ab)_{\ell}}^{\mathsf{a_{1}...a_{n}}}\big(\mathsf{W}_{i}^{I}\big)+\widehat{B}_{n;(ab)_{\ell}}^{\mathsf{a_{1}...a_{n}}}\big(\mathsf{W}_{i}^{\sharp I}\big).
\end{equation}
The partial amplitude
\[
\widehat{B}_{n;(ab)_{\ell}}^{\mathsf{a_{1}...a_{n}}}\big(\mathsf{W}_{i}^{I}\big)
\]
is obtained by restricting Eq. (\ref{eq:-148}) to the timelike wedge
$W_{s}^{-}$ and then applying the measure decomposition of Eq. (\ref{eq:-150}).

To express the partial amplitude in closed form, we introduce the
\emph{moduli superspace}:
\begin{equation}
\mathscr{M}_{2k+3}\,\coloneqq\,\mathbf{RP}^{3|8}\times\mathscr{E}_{k+1}.
\end{equation}
This supermanifold is globally charted by:
\begin{equation}
\gamma^{Q}\,\coloneqq\,\big(\mathbb{X}^{K},\tau_{1}^{M_{1}},\tau_{2}^{M_{2}},...,\tau_{k+1}^{M_{k+1}}\big),
\end{equation}
where each $\tau_{\ell}^{M}$ parametrises the factor superspace $\mathcal{P}_{\ell}$.
The abstract index $Q$ runs over the set $\{K,M_{1},...,M_{k+1}\}$.
The canonical $\mathbf{Z}_{2}$-graded volume form on $\mathscr{M}_{2k+3}$
is:
\begin{equation}
\mathcal{D}\boldsymbol{\gamma}\,\coloneqq\,D^{3|8}\mathbb{X}\wedge d^{8|4}\tau_{1}\wedge d^{8|4}\tau_{2}\wedge...\wedge d^{8|4}\tau_{k+1}.\label{eq:-151}
\end{equation}

The partial amplitude is then given by:
\begin{equation}
\widehat{B}_{n;(ab)_{\ell}}^{\mathsf{a_{1}...a_{n}}}\big(\mathsf{W}_{i}^{I}\big)\,=\,\underset{\mathscr{R}\,\,\,}{\int}\,d\rho_{r}\,r^{4}\,\underset{\mathscr{M}_{2k+3}\,\,\,}{\int}\,\mathcal{D}\boldsymbol{\gamma}\,\,\,\mathcal{F}_{(ab)_{\ell}}\big(\gamma^{Q}\big)\,\widehat{\mathcal{T}}^{\mathsf{a_{1}...a_{n}}}\big(\mathsf{W}_{i}^{I}\big).
\end{equation}
Here the trace factor is:
\begin{equation}
\widehat{\mathcal{T}}^{\mathsf{a_{1}...a_{n}}}\big(\mathsf{W}_{i}^{I}\big)\,=\,\mathsf{Tr}\,\bigwedge_{i=1}^{n}\,\frac{\mathcal{C}(2h_{i})}{\langle z_{i}|r\mathcal{R}+\widetilde{\mathcal{Q}}_{i}|\bar{z}_{i}]^{2h_{i}}}\,e^{i\langle z_{i}|(\theta+\tilde{q}_{i})\cdot\eta_{i}\rangle}\,\frac{\mathsf{T^{a_{i}}}}{z_{i}\cdot z_{i+1}}.
\end{equation}

Under the reparametrization:
\begin{equation}
\tau_{\ell}^{M}\,=\,\big(u_{\ell}^{A},v_{\ell}^{B},\tilde{u}_{\ell\dot{A}},\tilde{v}_{\ell\dot{B}},\chi_{\ell}^{\alpha}\big)\,\mapsto\,\tilde{\tau}_{\ell}^{M}\,=\,\big(u_{\ell}^{A},v_{\ell}^{B},r\tilde{u}_{\ell\dot{A}},r\tilde{v}_{\ell\dot{B}},\chi_{\ell}^{\alpha}\big),
\end{equation}
the measure (\ref{eq:-151}) transforms as:
\begin{equation}
\mathcal{D}\boldsymbol{\gamma}\,\mapsto\,r^{4(k+1)}\mathcal{D}\boldsymbol{\gamma}.
\end{equation}
Applying this in the above expression allows the $r$-integral to
decouple and be performed explicitly. One finds:
\begin{equation}
\widehat{B}_{n;(ab)_{\ell}}^{\mathsf{a_{1}...a_{n}}}\big(\mathsf{W}_{i}^{I}\big)\,=\,2\pi\delta\left(\beta_{k+1}\right)\,\mathcal{M}_{n;(ab)_{\ell}}^{\mathsf{a_{1}...a_{n}}}\big(\mathsf{W}_{i}^{I}\big),
\end{equation}
where the \emph{$N^{k+1}$-MHV scaling parameter} is:
\begin{equation}
\beta_{k+1}\big(h_{i}\big)\,\coloneqq\,4(k+2)-2\sum_{i=1}^{n}h_{i}.
\end{equation}
Finally, the sectional/leaf amplitude takes the form:
\begin{equation}
\mathcal{M}_{n;(ab)_{\ell}}^{\mathsf{a_{1}...a_{n}}}\big(\mathsf{W}_{i}^{I}\big)\,=\,\underset{\mathscr{M}_{2k+3}}{\int}\,\mathcal{D}\boldsymbol{\gamma}\,\,\,\mathcal{F}_{(ab)_{\ell}}\big(\gamma^{Q}\big)\,\widetilde{\mathcal{T}}^{\mathsf{a_{1}...a_{n}}}\big(\mathsf{W}_{i}^{I};\gamma^{Q}\big),\label{eq:-152}
\end{equation}
with trace factor:
\begin{equation}
\widetilde{\mathcal{T}}^{\mathsf{a_{1}...a_{n}}}\big(\mathsf{W}_{i}^{I};\gamma^{Q}\big)\,=\,\mathsf{Tr}\,\bigwedge_{i=1}^{n}\,\frac{\mathcal{C}(2h_{i})}{\langle z_{i}|\mathcal{R}+\widetilde{\mathcal{Q}}_{i}|\bar{z}_{i}]^{2h_{i}}}\,e^{i\langle z_{i}|(\theta+\tilde{q}_{i})\cdot\eta_{i}\rangle}\,\frac{\mathsf{T}^{\mathsf{a}_{i}}}{z_{i}\cdot z_{i+1}}.
\end{equation}


\subsubsection{Minitwistor Amplitude}

Substituting the celestial RSVW identity (Eq. (\ref{eq:-20})) into
the sectional amplitude (Eq. (\ref{eq:-152})) yields the following
representation.

\textcompwordmark{}

\paragraph*{Geometric Background.}

Label the external gluons by $i=1,...,n$. On the moduli superspace
$\mathscr{M}_{2k+3}$, we have embedding coordinates $\widetilde{\mathcal{Q}}_{i}^{A\dot{A}}$,
$\tilde{q}_{iA}^{\alpha}$ which depend on the moduli parameters $\tau_{\ell}^{M}$
for $\ell=1,...,k+1$ (cf. Subsection \ref{subsec:Fadde'ev-Popov-Representation-General-Case}). 

The minitwistor superspace $\mathbf{MT}_{s}$ is charted by homogeneous
coordinates:
\[
\mathsf{Z}^{I}\,=\,\big(\lambda^{A},\mu_{\dot{A}},\psi^{\alpha}\big).
\]
For a fixed moduli point:
\[
\gamma^{Q}\,=\,\big(\mathbb{X}^{K},\tau_{1}^{M_{1}},\tau_{2}^{M_{2}},...,\tau_{k+1}^{M_{k+1}}\big)\in\mathscr{M}_{2k+3},
\]
with $\mathbb{X}^{K}\,\coloneqq\,\big(X_{A\dot{A}},\theta_{A}^{\alpha}\big)\in\mathbf{RP}^{3|8}$
and $\mathcal{R}_{A\dot{A}}\coloneqq\left|X\right|^{-1}X_{A\dot{A}}$,
define the minitwistor line $\mathscr{L}_{i}\big(\gamma^{Q}\big)$
by the incidence relations:
\begin{equation}
\begin{cases}
\mu_{\dot{A}}\,=\,\lambda^{A}\,\big(\mathcal{R}_{A\dot{A}}+\widetilde{\mathcal{Q}}_{iA\dot{A}}\big),\\
\psi^{\alpha}\,=\,\lambda^{A}\,\big(\theta_{A}^{\alpha}+\tilde{q}_{iA}^{\alpha}\big).
\end{cases}
\end{equation}
Varying $\gamma^{Q}$ over $\mathscr{M}_{2k+3}$ sweeps out all configurations
of $2k+3$ minitwistor lines, justifying the identification of $\mathscr{M}_{2k+3}$
as the corresponding moduli superspace.

Each line $\mathscr{L}_{i}$ is parametrised by $\mathsf{Y}_{i}^{I}:\mathbf{RP}^{1}\longrightarrow\mathbf{MT}_{s}$
such that:
\begin{equation}
\mathsf{Y}_{i}^{I}\big(\sigma^{A}\big)\,\coloneqq\,\big(\sigma^{A},\sigma^{A}\big(\mathcal{R}_{A\dot{A}}+\widetilde{\mathcal{Q}}_{iA\dot{A}}\big),\sigma^{A}\big(\theta_{A}^{\alpha}+\tilde{q}_{iA}^{\alpha}\big)\big).
\end{equation}
Here $[\sigma^{A}]$ are homogeneous coordinates on $\mathbf{RP}^{1}$.
On each line there is a natural Lie-algebra--valued logarithmic form:
\begin{equation}
\boldsymbol{\omega}^{\mathsf{a}_{i}}\big(\sigma^{A}\big)\,\coloneqq\,\mathsf{T}^{\mathsf{a}_{i}}\frac{D\sigma_{i}}{\sigma_{i}\cdot\sigma_{i+1}}.
\end{equation}


\paragraph*{Minitwistor Amplitude.}

Let $\mathbf{M}^{n}\coloneqq\bigtimes^{n}\mathbf{MT}_{s}$ be the
integration superdomain. The celestial RSVW identity then recasts
the sectional amplitude (Eq. (\ref{eq:-152})) as a multidimensional
minitwistor transform:
\begin{equation}
\mathcal{M}_{n;(ab)_{\ell}}^{\mathsf{a_{1}...a_{n}}}\big(\mathsf{W}_{i}^{I}\big)\,=\,\underset{\mathbf{M}^{n}\,\,\,}{\int}\,\bigwedge_{i=1}^{n}D^{2|4}\mathsf{Z}_{i}\,\Psi_{2h_{i}}\big(\mathsf{Z}_{i}^{I};\mathsf{W}_{i}^{I'}\big)\,\,\,\widetilde{\mathcal{M}}_{n;(ab)_{\ell}}^{\mathsf{a_{1}...a_{n}}}\big(\mathsf{Z}_{i}^{I}\big).
\end{equation}
Moreover, the tree-level $N^{k+1}$-MHV minitwistor amplitude admits
the integral representation:
\begin{equation}
\widetilde{\mathcal{M}}_{n;(ab)_{\ell}}^{\mathsf{a_{1}...a_{n}}}\big(\mathsf{Z}_{i}^{I}\big)\,=\,\underset{\mathscr{M}_{2k+3}}{\int}\,\mathcal{D}\boldsymbol{\gamma}\,\,\,\mathcal{F}_{(ab)_{\ell}}\big(\gamma^{Q}\big)\,\mathsf{Tr}\,\bigwedge_{i=1}^{n}\,\underset{\mathbf{RP}^{1}\,\,\,}{\int}\,\boldsymbol{\omega}^{\mathsf{a}_{i}}\big(\sigma_{i}^{A}\big)\,\overline{\delta}_{(-2h_{i},2h_{i})}^{2|4}\big(\mathsf{Z}_{i}^{I};\mathsf{Y}_{i}^{I'}\big(\sigma_{i}^{A}\big)\big).\label{eq:-153}
\end{equation}


\paragraph*{Conclusion.}

By lifting $\mathcal{F}_{(ab)_{\ell}}$ to a probability distribution
on $\mathscr{M}_{2k+3}$, the minitwistor amplitude $\widetilde{\mathcal{M}}_{n;(ab)_{\ell}}^{\mathsf{a_{1}...a_{n}}}$
acquires a geometric interpretation as a volume integral over this
moduli superspace, weighted by $\mathcal{F}_{(ab)_{\ell}}$. In the
discussion of our minitwistor celestial CFT, we will show that $\widetilde{\mathcal{M}}_{n;(ab)_{\ell}}^{\mathsf{a_{1}...a_{n}}}$
corresponds to the \emph{semiclassical expectation value} of the observable
$\mathcal{F}_{(ab)_{\ell}}$.

The volume form in the integral (\ref{eq:-153}) is localised on the
family of minitwistor lines $\mathscr{L}_{i}$ by the distributions:
\[
\overline{\delta}_{(-2h_{i},2h_{i})}^{2|4}\big(\mathsf{Z}_{i}^{I};\mathsf{Y}_{i}^{I'}\big(\sigma_{i}^{A}\big)\big).
\]
Consequently, the amplitude vanishes whenever the $i$-th gluon does
not lie on its corresponding line $\mathscr{L}_{i}$. This completes
the extension of the $N^{1}$- and $N^{2}$-MHV superamplitude results
to the full tree-level celestial $\mathcal{S}$-matrix.

\section{Minitwistor Wilson Lines\label{sec:Minitwistor-Wilson-Lines}}

We now present a central result of this work: the reformulation of
celestial leaf amplitudes in $\mathcal{N}=4$ SYM theory as expectation
values of Wilson line operators on minitwistor superspace. In the
preceding sections, we showed that the $N^{k}$-MHV celestial amplitudes
localise on a specific set $\Sigma$ of rational curves $\mathcal{L}_{1},\dots,\mathcal{L}_{2k+1}$.
This localisation is not a technical detail but a hint toward a geometric
description of gluon scattering in asymptotically flat spacetimes.
We argue that the language for this description is provided by minitwistor
Wilson lines.

To define these operators, two ingredients are required: (i) a path
along which to compute holonomy, and (ii) a partial connection whose
path-ordered exponential yields the Wilson line. The path is fixed
by our localisation result: it is the union of the minitwistor lines
$\mathcal{L}_{1},\dots,\mathcal{L}_{2k+1}$ supporting the amplitude.
The partial connection is taken to be a pseudoholomorphic structure
on a complex vector bundle over $\mathbf{MT}_{s}$.

For physical insight, we interpret this construction via a minitwistor
sigma model, heuristically referred to as a ``minitwistor string theory.''
Although this model is defined only at the semiclassical level, it
provides a useful conceptual framework. Here, the sigma-model correlation
functions define an effective field theory on $\mathbf{MT}_{s}$,
and one may regard this effective theory (provisionally) as a ``string
field theory'' on minitwistor superspace.

The dynamics of this minitwistor field theory may be understood by
\emph{analogy} with Kodaira-Spencer gravity. In this theory, the gauge
potential parametrises a deformation of the canonical holomorphic
structure on a complex vector bundle. Consequently, the physical degrees
of freedom are encoded in its fieldstrength, the pseudocurvature $(0,2)$-form.
The physical observables are therefore the holonomies of the background
partial connection, and the central result of this correspondence
is that the minitwistor Wilson lines (i.e., the holonomies supported
on the set $\Sigma$ of rational curves where the scattering amplitudes
localise) reproduce the gluonic leaf amplitudes.

\subsection{Holomorphic Wilson Lines}

Our aim in this section is to construct the gauge-invariant observables
of the theory: the holomorphic Wilson lines\footnote{See \citet{mason2010complete} and \citet{bullimore2011holomorphic}.
For the history of the subject, see \citet{atiyah1981green} and \citet{penrose1988topological}.}. We adapt the well-known formalism on projective twistor superspace
$\mathbf{PT}^{3|4}$ to minitwistor superspace $\mathbf{MT}_{s}$.
This extension is needed for defining the holonomy operators that
compute celestial gluon amplitudes. The main technical problem is
to define parallel transport along minitwistor lines in a background
where the canonical holomorphic structure has been deformed by a gauge
field.

\subsubsection{Review: Holomorphic Gauge Theory}

In Appendix \ref{sec:Mini-Introduction-to-Holomorphic}, we provide
a concise, pedagogical introduction to the essentials of holomorphic
gauge theory. Here, we summarise the core definitions employed in
our construction of minitwistor Wilson lines.

Let $\pi\colon E\to\mathbf{MT}_{s}$ be a complex vector superbundle
whose fibres carry the adjoint representation of a gauge Lie superalgebra
$\mathfrak{g}$. In the absence of a gauge field, the physical vacuum
is specified by the canonical holomorphic structure on $E$, given
by the Dolbeault operator $\overline{\partial}^{E}$.

\textcompwordmark{}

\paragraph*{Ground States.}

A matter field $|\psi\rangle\in\Omega^{r,s}(E)$ is called a ground
state if it is holomorphic with respect to the vacuum structure:
\begin{equation}
\overline{\partial}^{E}\!|\psi\rangle=0.
\end{equation}
However, two holomorphic forms that differ by an exact term are gauge-equivalent.
Explicitly, if there exists $|\chi\rangle\in\Omega^{r,s-1}(E)$ such
that
\begin{equation}
|\psi'\rangle=|\psi\rangle+\overline{\partial}^{E}\!|\chi\rangle,
\end{equation}
then $|\psi\rangle$ and $|\psi'\rangle$ represent the same physical
state. Hence the space of distinct ground states is the Dolbeault
cohomology of $\mathbf{MT}_{s}$ with values in $E$. Defining $\mathcal{H}_{p,q}$
as the Hilbert space of ground states of bi-degree $(p,q)$, we have
the isomorphism
\begin{equation}
\mathcal{H}_{p,q}\;\cong\;H^{p,q}(\mathbf{MT}_{s};E).
\end{equation}


\paragraph*{Gauge Potential.}

We now deform the vacuum by introducing a background gauge potential.
Let $\boldsymbol{A}\in\Omega^{0,1}\big(\mathbf{MT}_{s};\mathrm{End}_{\mathbf{C}}(E)\big)$
be a differential $(0,1)$-form valued in endomorphisms of $E$. This
defines a new pseudoholomorphic structure $\mathscr{E}$ via the twisted
Dolbeault operator
\begin{equation}
\overline{\partial}^{\mathscr{E}}\;\coloneqq\;\overline{\partial}^{E}\,+\,\boldsymbol{A}.
\end{equation}
Parallel transport along a minitwistor line $\mathcal{L}$ is then
given by the operator
\begin{equation}
\langle w'|\mathcal{L}|w\rangle\colon\;E|_{w}\;\longrightarrow\;E|_{w'},
\end{equation}
which transports a vector in the fibre over $w$ to the fibre over
$w'$ along $\mathcal{L}$.

\textcompwordmark{}

\paragraph*{Fieldstrength.}

The failure of $\overline{\partial}^{\mathscr{E}}$ to square to zero
defines the fieldstrength (pseudocurvature) of the gauge field:
\begin{equation}
\boldsymbol{F}\;\coloneqq\;\overline{\partial}^{\mathscr{E}}\!\circ\overline{\partial}^{\mathscr{E}}\;=\;\overline{\partial}^{E}\!\boldsymbol{A}\;+\;\boldsymbol{A}\wedge_{\mathfrak{g}}\boldsymbol{A}.
\end{equation}
Here $\boldsymbol{F}$ is a $\mathfrak{gl}(r,\mathbf{C})$-valued
differential $(0,2)$-form on $\mathbf{MT}_{s}$. The term $\overline{\partial}^{E}\!\boldsymbol{A}$
is analogous to the kinetic part of the fieldstrength, while the non-linear
piece $\boldsymbol{A}\wedge_{\mathfrak{g}}\boldsymbol{A}$ encodes
the self-interactions characteristic of non-Abelian gauge theory.

\subsubsection{Parallel Transport; Abelian Case}

In the preceding subsection, we reviewed the geometric framework for
formulating a non-Abelian holomorphic gauge theory on minitwistor
superspace. We now pose the following problem. Let $\pi\colon E\to\mathbf{MT}_{s}$
be a rank-$r$ complex vector bundle endowed with a pseudoholomorphic
structure $\mathscr{E}$ and partial connection $\overline{\partial}^{\mathscr{E}}$.
Moreover, let $\mathcal{L}\subset\mathbf{MT}_{s}$ denote a minitwistor
line. How does one define parallel transport along $\mathcal{L}$
in the background $\mathscr{E}$?

\textcompwordmark{}

\paragraph*{Restricted Bundle.}

To answer this question, we first restrict the bundle $E$ to $\mathcal{L}$.
Let
\begin{equation}
\pi_{\mathcal{L}}\colon E|_{\mathcal{L}}\;\longrightarrow\;\mathcal{L},\quad E|_{\mathcal{L}}\;\coloneqq\;\pi^{-1}(\mathcal{L})
\end{equation}
denote the restricted bundle, where $\pi_{\mathcal{L}}=\pi|_{\pi^{-1}(\mathcal{L})}$.
This restriction is analogous to describing a bulk spacetime from
the viewpoint of a worldline.

The pseudoholomorphic structure $\mathscr{E}$ on $E$ induces a corresponding
structure $\mathfrak{L}$ on $E|_{\mathcal{L}}$. Since $\mathcal{L}$
is a rational curve, choose any nonsingular section $f\in\Gamma(\mathbf{CP}^{1};\mathcal{L})$,
so that $df\colon T(\mathbf{CP}^{1})\to T(\mathcal{L})$ is a bundle
morphism. In terms of this section, the induced Dolbeault operator
on $E|_{\mathcal{L}}$ is defined by
\begin{equation}
\overline{\partial}^{\mathfrak{L}}\;\coloneqq\;\overline{\partial}^{\mathscr{E}}\big|_{\mathcal{L}}\;\coloneqq\;f^{*}\big(\overline{\partial}^{\mathscr{E}}\!\big).
\end{equation}
It is straightforward to verify that this definition is independent
of the choice of $f$.

\textcompwordmark{}

\paragraph*{Holomorphic Frame.}

We next ask whether the restricted bundle $E|_{\mathcal{L}}$ admits
a global holomorphic frame over $\mathcal{L}$. Establishing such
a frame is the key step in constructing the parallel transport operator
along $\mathcal{L}$. To this end, let $\boldsymbol{F}^{\mathfrak{L}}=\overline{\partial}^{\mathfrak{L}}\!\circ\overline{\partial}^{\mathfrak{L}}$
denote the pseudocurvature of $\mathfrak{L}$. By Lemma 1 of Appendix
\ref{sec:Mini-Introduction-to-Holomorphic}, $\boldsymbol{F}^{\mathfrak{L}}$
is $\mathscr{C}^{\infty}$-linear and hence an element of $\Omega^{0,2}\big(\mathcal{L};\mathrm{End}_{\mathbf{C}}(E)\big)$.
Since $\mathcal{L}\cong\mathbf{CP}^{1}$ has complex dimension one,
all $(0,2)$-forms vanish and thus $\boldsymbol{F}^{\mathfrak{L}}=0$.
It follows that $\overline{\partial}^{\mathfrak{L}}$ is integrable
and $\mathfrak{L}$ is holomorphic.

On the other hand, the Birkhoff-Grothendieck theorem\footnote{See \citet{birkhoff1909singular} and \citet{grothendieck1957classification}.
For a pedagogical introduction, cf. \citet[Sec. 1.2]{okonek2013vector}.} states that every holomorphic vector bundle over $\mathbf{CP}^{1}$
splits as a direct sum of line bundles. Applying this to $E|_{\mathcal{L}}$
equipped with the holomorphic structure $\mathfrak{L}$ gives
\begin{equation}
E|_{\mathcal{L}}\;\cong\;\mathcal{O}_{\mathbf{C}}(a_{1})\,\oplus\,\dots\,\oplus\,\mathcal{O}_{\mathbf{C}}(a_{r}).
\end{equation}
In particular, $E|_{\mathcal{L}}$ is topologically trivial. Finally,
Lemma 2 of Appendix \ref{sec:Mini-Introduction-to-Holomorphic} guarantees
that a holomorphic, topologically trivial bundle admits a global holomorphic
frame. Hence there exists a frame $H=(H_{1},\dots,H_{r})$ trivialising
$E|_{\mathcal{L}}$ with
\begin{equation}
\overline{\partial}^{\mathfrak{L}}\!H_{i}\;=\;\big(\overline{\partial}^{\mathscr{E}}\!+\boldsymbol{A}\big)\big|_{\mathcal{L}}\,H_{i}\;=\;0.\label{eq:-158}
\end{equation}

Now that the existence of a global holomorphic frame $H$ on $E|_{\mathcal{L}}$
has been established, the parallel-transport operator between any
two points $w,w'\in\mathcal{L}$ can be written as:
\begin{equation}
\langle w'|\mathcal{L}|w\rangle\;=\;H(w')H^{-1}(w).\label{eq:-193}
\end{equation}


\paragraph*{Abelian Case.}

To derive the parallel-transport operator for a holomorphic gauge
theory on $\mathbf{MT}_{s}$, we begin with the Abelian case $\left(r=1\right)$.
Here, the holomorphic frame reduces to a single component $h\in\Gamma(\mathcal{L};GL(1,\mathbf{C}))$,
which we parametrise by a phase function $\phi\in\mathscr{C}^{\infty}(\mathcal{L};\,\mathfrak{gl}(1,\mathbf{C}))$
via:
\begin{equation}
h=\exp(-\phi).
\end{equation}
Substituting into Eq. (\ref{eq:-158}) gives
\begin{equation}
\overline{\partial}\big|_{\mathcal{L}}\,\phi\;=\;\boldsymbol{A}\big|_{\mathcal{L}},\label{eq:-186}
\end{equation}
where $\overline{\partial}\big|_{\mathcal{L}}$ is the Cauchy-Riemann
(CR) operator on the line $\mathcal{L}$.

To solve Eq. (\ref{eq:-186}), we must invert $\overline{\partial}\big|_{\mathcal{L}}$,
which requires a fundamental solution of the CR operator. To construct
it, let us briefly review the notion of Green differentials on Riemann
surfaces.

Let $\Omega_{\mathfrak{m}}^{p,q}\big(\mathcal{L};\mathcal{O}_{\mathbf{C}}(a)\big)$
denote the space of meromorphic $(p,q)$-forms on $\mathcal{L}$ valued
in the line bundle $\mathcal{O}_{\mathbf{C}}(a)$ (often called \emph{abelian
differentials}). Fix simple poles $w_{1},\dots,w_{k}\in\mathcal{L}$
and assign residues $r_{1},\dots,r_{k}\in\mathbf{C}$. A \emph{Green
differential $\boldsymbol{\tau}\in\Omega_{\mathfrak{m}}^{1,0}$ }with
these poles and residues is defined by the partial differential equation\footnote{See \citet[Ch. 1, Sec. 2]{demailly1997complex} for a mathematically
rigorous discussion of the terms appearing in Eq. (\ref{eq:-188}).} (PDE):
\begin{equation}
\overline{\partial}\big|_{\mathcal{L}}\;\boldsymbol{\tau}(\lambda^{A})\;=\;\sum_{i=1}^{k}\,r_{i}\,\overline{\delta}(w_{i}\!\cdot\!\lambda)\wedge D\lambda,\label{eq:-188}
\end{equation}
where $\overline{\delta}(w_{i}\!\cdot\!\lambda)$ is the $(0,1)$-current
supported at the $i$-th pole\footnote{Recall that $2\pi i\;\overline{\delta}(z)=\overline{\partial}\left(\frac{1}{z}\right)\in\mathscr{D}'_{0,1}$. }.
By Liouville theorem, $\boldsymbol{\tau}$ cannot be holomorphic.
A standard existence-and-uniqueness theorem on Riemann surfaces (see
\citet[Sec. 1.11]{forster1981compact}) then guarantees a unique solution
of Eq. (\ref{eq:-188}). We therefore call $\boldsymbol{\tau}$ the
Green differential for this PDE.

To solve Eq. (\ref{eq:-186}), we introduce the Green differential
$\boldsymbol{k}_{0}(w;\lambda^{A})$ which satisfies the PDE:
\begin{equation}
\frac{1}{2\pi i}\,\overline{\partial}_{\!\lambda}\big|_{\mathcal{L}}\;\boldsymbol{k}_{0}(w;\lambda)\,+\,\overline{\delta}(w\!\cdot\!\lambda)\wedge D\lambda\;=\;0.
\end{equation}
Having defined $\boldsymbol{k}_{0}$, we invert the CR operator on
$\phi$ to obtain:
\begin{equation}
\phi(w)\;=\;\frac{1}{2\pi i}\;\underset{\mathcal{L}\,\,\,\,\,}{\int}\;\boldsymbol{k}_{0}(w;\lambda^{A})\wedge\boldsymbol{A}|_{\mathcal{L}}(\lambda^{A}).
\end{equation}
Substituting this expression for $\phi$ into Eq. (\ref{eq:-193})
yields the Abelian parallel-transport operator:
\begin{equation}
\langle w'|\mathcal{L}|w\rangle\;=\;\exp\left(-\frac{1}{2\pi i}\;\underset{\mathcal{L}\,\,\,\,\,}{\int}\;\boldsymbol{k}(w',w;\lambda^{A})\wedge\boldsymbol{A}|_{\mathcal{L}}(\lambda^{A})\right),
\end{equation}
where $\boldsymbol{k}(w',w;\lambda^{A})\in\Omega_{\mathfrak{m}}^{1,0}(\mathcal{L})$
is the Green differential solving:
\begin{equation}
\frac{1}{2\pi i}\,\overline{\partial}_{\!\lambda}\big|_{\mathcal{L}}\;\boldsymbol{k}(w',w;\lambda^{A})\,+\,\boldsymbol{J}(w',w;\lambda^{A})\wedge D\lambda\;=\;0,\label{eq:-194}
\end{equation}
with the current:
\begin{equation}
\boldsymbol{J}(w',w;\lambda^{A})\;\coloneqq\;\overline{\delta}(w'\!\cdot\!\lambda)-\overline{\delta}(w\!\cdot\!\lambda)\;\in\;\mathscr{D}'_{0,1}\big(\mathcal{L};\,\mathcal{O}_{\mathbf{C}}(-1)\big).
\end{equation}

Since the left-hand side of Eq. (\ref{eq:-194}) belongs to $\Omega_{\mathfrak{m}}^{1,1}\big(\mathcal{L};\mathcal{O}_{\mathbf{C}}(1)\big)$,
the Green differential $\boldsymbol{k}(w',w;\lambda^{A})$ is invariant
under the rescalings $w\mapsto t\,w$, $w'\mapsto t\,w'$ and $\lambda^{A}\mapsto t\,\lambda^{A}$.
This homogeneity condition fixes its analytic form to:
\begin{equation}
\boldsymbol{k}(w',w;\lambda^{A})\;=\;\frac{\langle w',w\rangle}{\langle w',\lambda\rangle\langle\lambda,w\rangle}\,D\lambda.
\end{equation}


\subsubsection{Non-Abelian Parallel Transport; Wilson Operator}

Guided by the Abelian case, we now define the parallel-transport operator
along a minitwistor line $\mathcal{L}$ in a non-Abelian background\footnote{From now on, we fix $w,w'\in\mathcal{L}$ and, to avoid clutter, write
the Green differential $\boldsymbol{k}(w',w;\lambda^{A})$ simply
as $\boldsymbol{k}(\lambda^{A})$.}:
\begin{equation}
\langle w'|\mathcal{L}|w\rangle\;=\;\mathrm{Pexp}\left(-\frac{1}{2\pi i}\,\underset{\mathcal{L\,\,\,\,\,}}{\int}\;\boldsymbol{k}(w',w;\lambda^{A})\wedge\boldsymbol{A}|_{\mathcal{L}}(\lambda^{A})\right).\label{eq:-195}
\end{equation}
Using this expression, we will be able to construct a generating functional
for leaf-gluon amplitudes in MHV configurations, and later, for $N^{k}$-MHV
sectors. Our first task is then to clarify the meaning of the path-ordered
exponential appearing in Eq. (\ref{eq:-195}). 

To simplify notation, let $\{\lambda_{i}^{A}\}_{i=1}^{n}$ denote
points along the line $\mathcal{L}$. Define
\begin{equation}
\boldsymbol{k}_{i}\;\coloneqq\;\boldsymbol{k}(\lambda_{i}^{A}),\quad i=1,\dots,n.
\end{equation}
We then introduce the path-ordered, degree-$n$ monomial in $\boldsymbol{k}$:
\begin{equation}
\mathrm{P}\!\bigwedge_{i=1}^{n}\,\boldsymbol{k}_{i}\;\coloneqq\;\frac{\langle w',w\rangle}{\langle w',\lambda_{1}\rangle\langle\lambda_{1},\lambda_{2}\rangle\dots\langle\lambda_{n-1},\lambda_{n}\rangle\langle\lambda_{n},w\rangle}\,\bigwedge_{i=1}^{n}\,D\lambda_{i}.\label{eq:-196}
\end{equation}
This $(n,0)$-form is invariant under each scaling $\lambda_{i}^{A}\mapsto t\,\lambda_{i}^{A}$.
Hence it defines a \emph{genuine} differential form on the $n$-fold
product $\mathcal{L}^{n}\coloneqq\bigtimes^{n}\mathcal{L}$, not merely
a section of a homogeneous bundle.

For any Lie-algebra-valued $(0,1)$-form $\boldsymbol{B}\in\Omega^{0,1}(\mathcal{L};\mathfrak{g})$
and parameter $g$, we define the path-ordered exponential by the
formal powerseries:
\begin{equation}
\mathrm{Pexp}\left(g\;\underset{\mathcal{L}\,\,\,\,\,}{\int}\;\boldsymbol{k}\wedge\boldsymbol{B}\right)\;\coloneqq\;\mathbb{I}_{\mathfrak{g}}\;+\;\sum_{n\geq1}\;g^{n}\;\underset{\mathcal{L}^{n}\,\,\,\,\,}{\int}\;\mathrm{P}\!\bigwedge_{i=1}^{n}\,\boldsymbol{k}_{i}\wedge\boldsymbol{B}(\lambda_{i}^{A}).
\end{equation}
One can make this rigorous via formal distributions, a standard tool
in vertex-algebra theory (see Chapter 2, $\S\,1$ of \citet{kac1998vertex}).
Substituting into Eq. (\ref{eq:-195}) gives the parallel-transport
operator:
\begin{equation}
\langle w'|\mathcal{L}|w\rangle^{\boldsymbol{A}}\;=\;\mathbb{I}_{\mathfrak{g}}\;+\;\sum_{n\geq1}\;\frac{i^{n}}{(2\pi)^{n}}\;\underset{\mathcal{L}^{n}\,\,\,\,\,}{\int}\;\mathrm{P}\!\bigwedge_{i=1}^{n}\,\boldsymbol{k}_{i}\wedge\boldsymbol{A}|_{\mathcal{L}}(\lambda_{i}^{A}).\label{eq:-198}
\end{equation}
We add the superscript $\boldsymbol{A}$ to the left-hand side of
Eq. (\ref{eq:-198}), since this notation makes explicit its dependence
on the non-Abelian background gauge potential.

Having completed the geometric preliminaries, we now define the Wilson
line operator on minitwistor superspace:
\begin{equation}
\mathbb{W}_{w',w}^{\boldsymbol{A}}[\mathcal{L}]\;\coloneqq\;\mathsf{Tr}_{\mathfrak{g}}\,\langle w'|\mathcal{L}|w\rangle^{\boldsymbol{A}}.
\end{equation}
This non-local, gauge-invariant observable probes the background configuration
induced by the gauge potential $\boldsymbol{A}$.

\textcompwordmark{}

\paragraph*{Physical Interpretation.}

Recall that we view holomorphic gauge theory on the complex vector
bundle $E\to\mathbf{MT}_{s}$ as a theory of deformations of its complex
structure. In this picture, physical states correspond to Dolbeault
cohomology classes of $E$, and the vacuum holomorphic structure is
fixed by the standard Dolbeault operator $\overline{\partial}^{E}$.
Accordingly, we define the minitwistor Wilson line $\mathbb{W}_{w',w}^{\boldsymbol{A}}[\mathcal{L}]$
to measure the deformation of this vacuum induced by an external potential
$\boldsymbol{A}$.

\textcompwordmark{}

We conclude this subsection by deriving a formal powerseries expression
for the minitwistor Wilson line. This result will simplify our derivation
of the generating functional for MHV leaf-gluon amplitudes from the
expectation values of $\mathbb{W}_{w',w}^{\boldsymbol{A}}[\mathcal{L}]$.

First, recall the path-ordered monomial in $\boldsymbol{k}$ introduced
in Eq. (\ref{eq:-196}). One finds
\begin{equation}
\mathrm{P}\!\bigwedge_{i=1}^{n}\,\boldsymbol{k}_{i}\;=\;\mathcal{N}\big(w',w|\lambda_{n},\lambda_{1}\big)\;\bigwedge_{i=1}^{n}\,\frac{D\lambda_{i}}{\lambda_{i}\cdot\lambda_{i+1}},\label{eq:-197}
\end{equation}
where
\begin{equation}
\mathcal{N}\big(w',w|\lambda_{n},\lambda_{1}\big)\;=\;\frac{\langle w',w\rangle\langle\lambda_{n},\lambda_{1}\rangle}{\langle w',\lambda_{1}\rangle\langle\lambda_{n},w\rangle}.
\end{equation}
The function $\mathcal{N}$ is projectively well-defined for $w,w',\lambda_{1},\lambda_{n}\in\mathbf{CP}^{1}$,
since it is invariant under
\begin{equation}
w\mapsto t\,w,\quad w'\mapsto t\,w',\quad\lambda_{1}\mapsto t\,\lambda_{1},\quad\lambda_{n}\mapsto t\,\lambda_{n}.
\end{equation}
Moreover, it obeys the normalisation
\begin{equation}
\mathcal{N}\big(w',w|w',w\big)\;=\;1\quad\text{for all }w,w'\in\mathbf{CP}^{1}.
\end{equation}

Substituting Eq. (\ref{eq:-197}) into the powerseries expansion of
the parallel-transport operator (Eq. (\ref{eq:-198})) gives
\begin{equation}
\langle w'|\mathcal{L}|w\rangle^{\boldsymbol{A}}\;=\;\mathbb{I}_{\mathfrak{g}}\;+\;\sum_{n\geq1}\;\frac{i^{n}}{(2\pi)^{n}}\;\underset{\mathcal{L}^{n}\,\,\,\,\,}{\int}\;\mathcal{N}\big(w',w|\lambda_{n},\lambda_{1}\big)\,\bigwedge_{i=1}^{n}\,\frac{D\lambda_{i}}{\lambda_{i}\cdot\lambda_{i+1}}\wedge\boldsymbol{A}\big|_{\mathcal{L}}(\lambda_{i}^{A}).
\end{equation}
This series provides an expansion of the Wilson line in powers of
the gauge potential:
\begin{equation}
\mathbb{W}_{w',w}^{\boldsymbol{A}}[\mathcal{L}]\;=\;r\;+\;\sum_{n\geq1}\;g^{n}\underset{\mathcal{L}^{n}\,\,\,\,\,}{\int}\;\mathcal{N}\big(w',w|\lambda_{n},\lambda_{1}\big)\;\mathsf{Tr}_{\mathfrak{g}}\left(\;\bigwedge_{i=1}^{n}\,\frac{D\lambda_{i}}{\lambda_{i}\cdot\lambda_{i+1}}\wedge\boldsymbol{A}\big|_{\mathcal{L}}(\lambda_{i}^{A})\;\right)\label{eq:-201}
\end{equation}
where $r\coloneqq\mathrm{rank}_{\mathbf{C}}\,(\mathfrak{g})$ and
$g\coloneqq i/(2\pi)$.

\subsubsection{MHV Leaf-Gluon Amplitudes}

In the preceding subsections, we constructed a theory of minitwistor
Wilson lines. These non-local, gauge-invariant observables probe the
deformation of the holomorphic structure of a complex vector bundle
over minitwistor superspace induced by a background gauge field. Our
formalism is now rich enough to derive the MHV leaf-gluon superamplitude
from the semiclassical expectation value of the minitwistor Wilson
line.

To proceed, we must complete two preliminary tasks. First, we review
the minitwistor description of celestial leaf amplitudes. Second,
we clarify the meaning of semiclassical expectation values for a non-local
operator such as $\mathbb{W}_{w',w}^{\boldsymbol{A}}[\mathcal{L}]$.

\textcompwordmark{}

\paragraph*{Dual Minitwistors and Celestial Supersphere.}

The celestial description of gluonic scattering in $\mathcal{N}=4$
SYM theory assigns to each of the $n$ external gluons a conformal
weight $\Delta_{i}$ and an insertion point on the $\mathcal{N}=4$
celestial supersphere $\mathcal{CS}_{s}\cong\mathbf{CP}^{1|4}$. Following
\citet{Tropper:2024evi}, we cover $\mathcal{CS}_{s}$ with coordinates
$\mathsf{z}\coloneqq(z,\bar{z},\eta^{\alpha})$, where $z$ and $\bar{z}$
are the holomorphic and antiholomorphic coordinates on $\mathbf{CP}^{1}$,
and $\eta^{\alpha}$ are Grassmann variables encoding helicity states.
Thus, the $i$-th gluon's insertion point carries coordinates 
\begin{equation}
\mathsf{z}_{i}\;=\;(z_{i},\bar{z}_{i},\eta_{i}^{\alpha})\;\in\;\mathcal{CS}_{s}.
\end{equation}
From these data, we form the dual minitwistor
\begin{equation}
\mathsf{Z}_{i}^{I}\;\coloneqq\;\big(z_{i}^{A},\bar{z}_{i\dot{A}},\eta_{i}^{\alpha}\big)\;\in\;\mathbf{MT}_{s}^{*},
\end{equation}
where the van der Waerden spinors are $z_{i}^{A}\coloneqq(1,-z_{i})^{T}$
and $\bar{z}_{i\dot{A}}\coloneqq(1,-\bar{z}_{i})^{T}$. Hence, \emph{the
dual minitwistor superspace $\mathbf{MT}_{s}^{*}$ parametrises the
celestial supersphere via}
\begin{equation}
\mathcal{CS}_{s}\;\longrightarrow\;\mathbf{MT}_{s}^{*},\qquad\mathsf{z}_{i}\,\longmapsto\,\mathsf{Z}_{i}^{I}.
\end{equation}

We identify the fixed points $w$ and $w'$ on the minitwistor line
$\mathcal{L}\subset\mathbf{MT}_{s}$ with the endpoint gluons: $w=z_{1}$,
$w'=z_{n}$. This assignment does not restrict the total number of
gluons in our generating functional. Here, the subscript $n$ in $z_{n}$
is merely a label, just as the prime in $w'$ is. 

\textcompwordmark{}

\emph{But what is the physical meaning of the minitwistor line $\mathcal{L}$
on which the Wilson operator $\mathbb{W}_{z_{n},z_{1}}^{\boldsymbol{A}}[\mathcal{L}]$
is supported?} To answer this, we anticipate some ideas from the minitwistor
sigma-model, which we discuss in detail in the next section. We view
the sigma-model as a minitwistor ``string'' with worldsheet the celestial
supersphere $\mathcal{CS}_{s}$ and target manifold the minitwistor
superspace $\mathbf{MT}_{s}$. In this picture, each minitwistor line
$\mathcal{L}$ corresponds to one classical configuration of the sigma-model.
Remarkably, this interpretation leads directly to the expectation
value of the Wilson line operator (see Eq. (\ref{eq:-199}) below),
which serves as the generating functional for leaf-gluon superamplitudes.

\textcompwordmark{}

\paragraph*{Semiclassical Expectation Value of $\mathbb{W}_{z_{n},z_{1}}^{\boldsymbol{A}}[\mathcal{L}]$.}

Our physical picture then interprets the holomorphic gauge theory
on minitwistor superspace probed by $\mathbb{W}_{z_{n},z_{1}}^{\boldsymbol{A}}[\mathcal{L}]$
as the effective field theory induced by the minitwistor sigma-model.
When we regard this sigma-model as a minitwistor ``string theory,''
the holomorphic gauge theory plays the role of its ``string field
theory.''

More formally, the sigma-model describes embeddings of the celestial
supersphere into minitwistor superspace as a family of minitwistor
lines. Its dynamical fields are then holomorphic rational maps
\begin{equation}
\mathbf{CP}^{1|4}\;\longrightarrow\;\mathbf{MT}_{s}
\end{equation}
that embed the worldsheet as a minitwistor line.

Moreover, from Section II, the supersymmetric Hitchin correspondence
on $\mathbf{MT}_{s}$ provides a bijection
\begin{equation}
\mathsf{X}^{K}=\big(X_{A\dot{A}},\,\theta_{A}^{\alpha}\big)\;\in\;\mathbf{H}_{s}\quad\longleftrightarrow\quad\mathcal{L}(X,\theta)\;\subset\;\mathbf{MT}_{s}.
\end{equation}
Thus $\mathbf{H}_{s}$ is the \emph{moduli superspace} of minitwistor
lines. Equivalently, it is the \emph{configuration space} of our minitwistor
string.

We then define
\begin{equation}
\mathcal{I}_{0}[\boldsymbol{A}]\;\coloneqq\;\underset{\mathbf{H}_{s}\,\,\,}{\int}\;D^{3|8}\mathsf{X}\;\mathbb{W}_{z_{n},z_{1}}^{\boldsymbol{A}}\big[\mathcal{L}(X,\theta)\big],\label{eq:-199}
\end{equation}
an integral over the moduli superspace of minitwistor lines. Following
Feynman's pathintegral formalism and the superposition principle,
summing over all embeddings yields an expectation value. In addition,
since our sigma-model is defined only semiclassically, $\mathcal{I}_{0}[\boldsymbol{A}]$
is the \emph{semiclassical expectation value} of the Wilson line operator
probing the background gauge potential $\boldsymbol{A}$ in our minitwistor
string field theory.

\textcompwordmark{}

\paragraph*{Generating Functional.}

We now show that $\mathcal{I}_{0}[\boldsymbol{A}]$ generates the
MHV leaf amplitudes for gluons. First, decompose the gauge potential
$\boldsymbol{A}$ into Fourier modes $\alpha^{\Delta,\mathsf{a}}$
via the minitwistor transform $\mathcal{MT}$ of Section II. To do
so, we project $\boldsymbol{A}$ onto the superwavefunctions $\Psi_{\Delta}\coloneqq\Psi_{\Delta}^{0}$
which satisfy
\begin{equation}
\Psi_{\Delta}(\mathsf{W}^{I};\,\cdot\,)\in\mathscr{D}'_{0,0}\big(\mathbf{MT}_{s}^{*};\,\mathcal{O}_{\mathbf{C}}(-\Delta-2,-\Delta)\big)\quad\text{for fixed }\mathsf{W}^{I}\in\mathbf{MT}_{s},
\end{equation}
and
\begin{equation}
\Psi_{\Delta}(\,\cdot\,;\mathsf{Z}^{I})\in\mathscr{D}'_{0,1}\big(\mathbf{MT}_{s};\,\mathcal{O}_{\mathbf{C}}(\Delta,-\Delta)\big)\quad\text{for fixed }\mathsf{Z}^{I}\in\mathbf{MT}_{s}^{*}.
\end{equation}
We then expand
\begin{equation}
\boldsymbol{A}\big(\mathsf{W}^{I}\big)\;=\;2\pi i\;\underset{\mathbf{MT}_{s}^{*}\,\,\,}{\int}\;\Psi_{\Delta}\big(\mathsf{W}^{I};\mathsf{Z}'{}^{I}\big)\,\alpha^{\Delta,\mathsf{a}}\big(\mathsf{Z}'{}^{I}\big)\,\mathsf{T}^{\mathsf{a}}\wedge D^{2|4}\mathsf{Z}'.\label{eq:-200}
\end{equation}
The factor $2\pi i$ is a convenient normalisation that can be absorbed
into $\alpha^{\Delta,\mathsf{a}}$, and the prime on $\mathsf{Z}'{}^{I}$
marks the integration variable.

How do we interpret this expansion? Its key novelty is the use of
DeWitt notation for the conformal weight $\Delta$. Let $\mathcal{P}$
denote the domain of $\Delta$, equipped with a Stieltjes measure
$d\alpha(\Delta)$. Then, following DeWitt, define
\begin{equation}
\Psi_{\Delta}\big(\mathsf{W}^{I};\mathsf{Z}'{}^{I}\big)\,\alpha^{\Delta,\mathsf{a}}\big(\mathsf{Z}'{}^{I}\big)\;\coloneqq\;\underset{\mathcal{P}\,\,\,\,\,}{\int}\;d\alpha(\Delta)\;\Psi_{\Delta}\big(\mathsf{W}^{I};\mathsf{Z}'{}^{I}\big)\,\alpha_{\Delta}^{\mathsf{a}}\big(\mathsf{Z}'{}^{I}\big),
\end{equation}
so that $\alpha^{\Delta,\mathsf{a}}$ becomes a continuum of mode
functions labeled by $\Delta$.

However, the integral over dual minitwistor superspace in Eq. (\ref{eq:-200})
only makes sense if the integrand
\begin{equation}
\Psi_{\Delta}\big(\mathsf{W}^{I};\mathsf{Z}'{}^{I}\big)\,\alpha^{\Delta,\mathsf{a}}\big(\mathsf{Z}'{}^{I}\big)\,\mathsf{T}^{\mathsf{a}}\wedge D^{2|4}\mathsf{Z}'\;\in\;\Omega^{(2,2)|4}\big(\mathbf{MT}_{s}^{*};\mathfrak{g}\big)
\end{equation}
defines a $\mathfrak{g}$-valued \emph{top-form} on $\mathbf{MT}_{s}^{*}$.
In particular, the mode coefficients must lie in
\begin{equation}
\alpha^{\Delta,\mathsf{a}}\;\in\;\Omega^{1,1}\big(\mathbf{MT}_{s}^{*};\,\mathfrak{g}\otimes\mathcal{O}_{\mathbf{C}}(\Delta-4,\Delta-2)\big).
\end{equation}
It follows that the gauge field itself lives in the direct sum
\begin{equation}
\boldsymbol{A}\;\in\;\bigoplus_{\Delta\in\mathcal{P}}\,\Omega^{0,1}\big(\mathbf{MT}_{s};\,\mathfrak{g}\otimes\mathcal{O}_{\mathbf{C}}(\Delta,-\Delta)\big).
\end{equation}
This may seem unusual, but note that when we restrict $\boldsymbol{A}$
to any minitwistor line $\mathcal{L}$, we have
\begin{equation}
\boldsymbol{A}\big|_{\mathcal{L}}\;\in\;\Omega^{0,1}(\mathcal{L};\,\mathfrak{g}),
\end{equation}
a projectively invariant $\mathfrak{g}$-valued $(0,1)$-form on $\mathcal{L}$.
This matches our requirement for a gauge potential on a minitwistor
line and fits with the definition of the non-Abelian parallel-transport
operator (Eq. (\ref{eq:-195})).

From Eq. (\ref{eq:-200}), the gauge potential induced on the line
$\mathcal{L}$ is:
\begin{equation}
\boldsymbol{A}\big|_{\mathcal{L}}\,(\lambda^{A})\;=\;2\pi i\;\underset{\mathbf{MT}_{s}^{*}\,\,\,}{\int}\;\Psi_{\Delta}\big|_{\mathcal{L}}\,\big(\lambda^{A};\mathsf{Z}'{}^{I}\big)\,\alpha^{\Delta,\mathsf{a}}\big(\mathsf{Z}'{}^{I}\big)\,\mathsf{T}^{\mathsf{a}}\wedge D^{2|4}\mathsf{Z}'.
\end{equation}
Substituting this into the powerseries expansion of the Wilson line
operator (Eq. (\ref{eq:-201})) gives:
\begin{align}
\mathbb{W}_{z_{n},z_{1}}^{\boldsymbol{A}}[\mathcal{L}]\; & =\;r\;+\;\sum_{n\geq1}\;(-1)^{n}\;\underset{\mathcal{L}^{n}\,\,\,\,\,}{\int}\;\mathcal{N}\big(z_{n},z_{1}|\lambda_{n},\lambda_{1}\big)\;\mathsf{Tr}_{\mathfrak{g}}\;\bigwedge_{i=1}^{n}\;\frac{D\lambda_{i}}{\lambda_{i}\cdot\lambda_{i+1}}\\
 & \underset{\mathbf{MT}_{s}^{*}\,\,\,}{\int}\;\Psi_{\Delta_{i}}\big|_{\mathcal{L}}\,\big(\lambda_{i}^{A};\mathsf{Z}_{i}'{}^{I}\big)\,\alpha^{\Delta_{i},\mathsf{a}_{i}}\big(\mathsf{Z}_{i}'{}^{I}\big)\,\mathsf{T}^{\mathsf{a}_{i}}\wedge D^{2|4}\mathsf{Z}_{i}'.\label{eq:-202}
\end{align}
By Fubini's theorem, we can exchange the integrals. Define $\mathbf{X}_{n}^{*}\coloneqq\bigtimes^{n}\mathbf{MT}_{s}^{*}$,
oriented by $\bigwedge_{i=1}^{n}D^{2|4}\mathsf{Z}_{i}'$. Then Eq.
(\ref{eq:-202}) becomes:
\begin{align}
\mathbb{W}_{z_{n},z_{1}}^{\boldsymbol{A}}[\mathcal{L}]\; & =\;r\;+\;\sum_{n\geq1}\;(-1)^{n}\;\underset{\mathbf{X}_{n}^{*}\,\,\,}{\int}\;\bigwedge_{i=1}^{n}\;D^{2|4}\mathsf{Z}_{i}'\wedge\alpha^{\Delta_{i},\mathsf{a}_{i}}\big(\mathsf{Z}_{i}'\big)\;\underset{\mathcal{L}^{n}\,\,\,\,\,}{\int}\;\mathcal{N}\big(z_{n},z_{1}|\lambda_{n},\lambda_{1}\big)\\
 & \mathsf{Tr}_{\mathfrak{g}}\;\bigwedge_{j=1}^{n}\;\frac{D\lambda_{j}}{\lambda_{j}\cdot\lambda_{j+1}}\,\mathsf{T}^{\mathsf{a}_{j}}\wedge\Psi_{\Delta_{j}}\big|_{\mathcal{L}}\,\big(\lambda_{j}^{A};\mathsf{Z}_{j}'{}^{I}\big).
\end{align}
Applying the celestial BMSW identity to the $\mathcal{L}^{n}$ integral
yields:
\begin{align}
\mathbb{W}_{z_{n},z_{1}}^{\boldsymbol{A}}[\mathcal{L}]\; & =\;r\;+\;\sum_{n\geq1}\;(-1)^{n}\;\underset{\mathbf{X}_{n}^{*}\,\,\,}{\int}\;\mathcal{N}\big(z_{n},z_{1}|z'_{n},z'_{1}\big)\;\mathsf{Tr}_{\mathfrak{g}}\;\bigwedge_{i=1}^{n}\;\frac{\mathcal{C}(\Delta_{i})}{\langle z'_{i}|X|\bar{z}'_{i}]^{\Delta_{i}}}\\
 & e^{i\langle z'_{i}|\theta\cdot\eta_{i}\rangle}\;\frac{\mathsf{T}^{\mathsf{a}_{i}}}{z'_{i}\cdot z'_{i+1}}\;\alpha^{\Delta_{i},\mathsf{a}_{i}}\big(\mathsf{Z}_{i}'{}^{I}\big)\wedge D^{2|4}\mathsf{Z}_{i}'.
\end{align}
Hence the integral in Eq. (\ref{eq:-199}) becomes\footnote{The first term in the powerseries expansion of the Wilson line operator
$\mathbb{W}_{z_{n},z_{1}}^{\boldsymbol{A}}[\mathcal{L}]$ equals the
rank $r$ of the Lie algebra $\mathfrak{g}$. Its integral over superspace
$\mathbf{H}_{s}$ vanishes by Berezin integration, $\int d^{0|8}\theta\;r=0$.}:
\begin{align}
\mathcal{I}_{0}[\boldsymbol{A}]\;= & \;\sum_{n\geq1}\;(-1)^{n}\;\underset{\mathbf{X}_{n}^{*}\,\,\,}{\int}\;\mathcal{N}\big(z_{n},z_{1}|z_{n}',z_{1}'\big)\;\underset{\mathbf{H}_{s}\,\,\,}{\int}\;D^{3|8}\mathsf{X}\;\mathsf{Tr}_{\mathfrak{g}}\;\bigwedge_{i=1}^{n}\;\frac{\mathcal{C}(\Delta_{i})}{\langle z_{i}'|X|\bar{z}_{i}']^{\Delta_{i}}}\label{eq:-203}\\
 & e^{i\langle z_{i}'|\theta\cdot\eta_{i}\rangle}\;\frac{\mathsf{T}^{\mathsf{a}_{i}}}{z_{i}'\cdot z_{i+1}'}\;\alpha^{\Delta_{i},\mathsf{a}_{i}}\big(\mathsf{Z}_{i}'{}^{I}\big)\wedge D^{2|4}\mathsf{Z}_{i}'.
\end{align}
As shown in \citet{mol2024comments}, the MHV leaf superamplitude
for gluons is given by:
\begin{equation}
M_{n}^{\mathsf{a}_{1}\dots\mathsf{a}_{n}}\big(\mathsf{Z}_{i}^{I}\big)\;=\;\underset{\mathbf{H}_{s}\,\,\,}{\int}\;D^{3|8}\mathsf{X}\;\mathsf{Tr}_{\mathfrak{g}}\;\bigwedge_{i=1}^{n}\;\frac{\mathcal{C}(2h_{i})}{\langle z_{i}|X|\bar{z}_{i}]^{2h_{i}}}\;e^{i\langle z_{i}|\theta\cdot\eta_{i}\rangle}\;\frac{\mathsf{T}^{\mathsf{a}_{i}}}{z_{i}\cdot z_{i+1}}.
\end{equation}
Here, $h_{i}$ denotes the scaling dimension of the $i$-th gluon.
The wedge product inside the trace arises because the exponential
contains Grassmann-valued spinors $\theta_{A}^{\alpha}$.

Consequently, taking functional derivatives of $\mathcal{I}_{0}[\boldsymbol{A}]$
with respect to the mode coefficients $\alpha^{2h_{i},\mathsf{a}_{i}}\big(\mathsf{Z}_{i}^{I}\big)$
and then setting the background field to zero yields:
\begin{equation}
(-1)^{n}\;\prod_{i=1}^{n}\;\frac{\delta}{\delta\alpha^{2h_{i},\mathsf{a}_{i}}\big(\mathsf{Z}_{i}^{I}\big)}\;\mathcal{I}_{0}[\boldsymbol{A}]\,\Bigg|_{\mathbf{A}=0}\;=\;M_{n}^{\mathsf{a}_{1}\dots\mathsf{a}_{n}}\big(\mathsf{Z}_{i}^{I}\big).
\end{equation}
This completes our demonstration that the semiclassical expectation
value of the Wilson line operator $\mathbb{W}_{z_{n},z_{1}}^{\boldsymbol{A}}[\mathcal{L}]$
serves as the generating functional for the MHV leaf-gluon superamplitudes. 

This result is remarkable: we built the minitwistor Wilson line operator
purely from geometric considerations inspired by holomorphic gauge
theory on minitwistor superspace, guided by our understanding of topological
sigma-models. Importantly, we did not invoke any property specific
to Yang-Mills theory. Nevertheless, we recovered the MHV gluonic superamplitudes.
This indicates that, from the perspective of celestial CFT, our formulation
describes a dual description of gauge theory on asymptotically flat
spacetime.

\subsection{Holonomies and $N^{k}$-MHV Amplitudes}

We extend the holomorphic Wilson line operator $\mathbb{W}[\mathcal{L}]$,
originally defined on a single minitwistor line, to an operator supported
on an algebraic cycle in the minitwistor superspace. This cycle is
built from the family of lines on which the $\mathrm{N}^{k}\text{-MHV}$
minitwistor superamplitude localises.

We then show that the semiclassical expectation value of this operator
provides a generating functional for the tree-level $S$-matrix of
$\mathcal{N}=4$ SYM.

\subsubsection{Summary of Key Steps}

Before proceeding, we review the main steps so far. We studied a holomorphic
gauge theory on a complex vector bundle $E$ over the minitwistor
superspace $\mathbf{MT}_{s}$. To give physical intuition, we drew
an analogy with topological sigma models and field theory, interpreting
the holomorphic gauge theory as the field-theoretic limit of a minitwistor
string, which we will describe in the next section.

Next, we introduced the nonlocal, gauge-invariant observable $\mathbb{W}[\mathcal{L}]$
supported on a minitwistor line $\mathcal{L}$. This operator probes
deformations of the holomorphic vacuum induced by a background gauge
field $\boldsymbol{A}$. By the Hitchin correspondence, the moduli
supermanifold of minitwistor lines coincides with hyperbolic superspace
$\mathbf{H}_{s}$, which we identify with the configuration space
of the minitwistor strings. Hence we interpret
\begin{equation}
\underset{\mathbf{H}_{s}\,\,\,\,\,}{\int}\;D^{3|8}\mathsf{X}\;\mathbb{W}_{z_{n},z_{1}}^{\boldsymbol{A}}\big[\mathcal{L}(X,\theta)\big]\label{eq:-221}
\end{equation}
as the semiclassical expectation value of the Wilson line. Physically,
this expectation value encodes the state of the gauge-field background
and reproduces the generating functional for MHV gluon amplitudes. 

Remarkably, aside from choosing the gauge Lie superalgebra $\mathfrak{g}$
and supersymmetry level $\mathcal{N}=4$, we have not invoked any
other super-Yang-Mills input (no Lagrangian density or equations of
motion), yet we recover dynamical information about SYM theory. This
result supports the conjecture that our CCFT model is dual to four-dimensional
flat-space gauge theory.

\textcompwordmark{}

\subsubsection{Next Steps; Geometrical Motivation}

Now we turn to the problem of identifying an observable in our holomorphic
gauge theory whose semiclassical expectation value reproduces the
$\mathrm{N}^{k}\text{-MHV}$ sectors of leaf-gluon amplitudes. From
Section III we learned that the $\mathrm{N}^{k}\text{-MHV}$ minitwistor
superamplitudes in $\mathcal{N}=4$ SYM localise on a family $\Sigma$
of minitwistor lines $\mathcal{L}_{1},\dots,\mathcal{L}_{2k+1}$.
But whereas the configuration space of a single minitwistor line $\mathcal{L}$
is the hyperbolic superspace $\mathbf{H}_{s}$, the configuration
space of the entire family $\Sigma$ is the \emph{moduli superspace}
\begin{equation}
\mathscr{M}_{2k+1}\;=\;\mathbf{H}_{s}\times\mathcal{P}_{1}\times\mathcal{P}_{2}\times\dots\times\mathcal{P}_{k}.
\end{equation}
Each factor $\mathcal{P}_{\ell}$, for $\ell=1,\dots,k$, is called
a \emph{parameter space} and corresponds to one next-to-MHV particle.
In the split-signature required by the leaf formalism, $\mathcal{P}_{\ell}\cong\mathbf{R}^{8|4}$,
but here we analytically continue these spaces to complex supermanifolds,
$\mathcal{P}_{\ell}\cong\mathbf{C}^{8|4}$. 

We chart each parameter space by
\begin{equation}
\tau_{\ell}^{M}\;=\;\big(u_{\ell}^{A},\,v_{\ell}^{B},\,\widetilde{u}_{\ell\dot{A}},\,\widetilde{v}_{\ell\dot{B}},\,\chi_{\ell}^{\alpha}\big),\label{eq:-223}
\end{equation}
and we denote homogeneous coordinates on $\mathbf{H}_{s}$ by $\mathsf{X}^{K}$.
Thus the full moduli superspace $\mathscr{M}_{2k+1}$ admits the global
chart:
\begin{equation}
\gamma^{Q}\;=\;\big(\mathsf{X}^{K},\tau_{1}^{M_{1}},\tau_{2}^{M_{2}},\dots,\tau_{k}^{M_{k}}\big),\label{eq:-204}
\end{equation}
which we call the \emph{moduli coordinates}. We equip $\mathscr{M}_{2k+1}$
with its standard orientation via the volume superform
\begin{equation}
\mathcal{D}\gamma\;=\;D^{3|8}\mathsf{X}\,\wedge\,d^{8|4}\tau_{1}\,\wedge\,d^{8|4}\tau_{2}\,\wedge\,\dots\,\wedge\,d^{8|4}\tau_{k},\label{eq:-224}
\end{equation}
where each $d^{8|4}\tau_{\ell}$ is the Berezin-DeWitt form on $\mathcal{P}_{\ell}$.
Consequently, when we compute a semiclassical expectation value of
a Wilson-like observable supported on $\Sigma$, we replace the integral
over $\mathbf{H}_{s}$ by an integral over $\mathscr{M}_{2k+1}$ against
the measure $\mathcal{D}\gamma$.

How can we generalise the operator $\mathbb{W}[\mathcal{L}]$, which
is supported on a single minitwistor line $\mathcal{L}$, to an observable
supported on a family $\Sigma$? To address this question, we recall
the geometry of the collection of lines $\Sigma=\{\mathcal{L}_{1},\dots,\mathcal{L}_{2k+1}\}$.
Each line $\mathcal{L}_{m}$ carries \emph{moduli functions}
\begin{equation}
\mathcal{Q}_{m}^{A\dot{A}}=\mathcal{Q}_{m}^{A\dot{A}}\big(\tau_{1}^{M_{1}},\tau_{2}^{M_{2}},\dots,\tau_{k}^{M_{k}}\big),\quad q_{m}^{\alpha A}=q_{m}^{\alpha A}\big(\tau_{1}^{M_{1}},\tau_{2}^{M_{2}},\dots,\tau_{k}^{M_{k}}\big),
\end{equation}
defined on the parameter spaces $\mathcal{P}_{1},\dots,\mathcal{P}_{k}$
inside the moduli superspace $\mathscr{M}_{2k+1}$. Together with
the superspace coordinates $\mathsf{X}^{K}=\big(X_{A\dot{A}},\theta_{A}^{\alpha}\big)$
charting $\mathbf{H}_{s}$, these functions determine the i\emph{ncidence
maps}
\begin{equation}
Y_{m}^{A\dot{A}}(\gamma^{Q})\;\coloneqq\;X^{A\dot{A}}+\mathcal{Q}_{m}^{A\dot{A}}(\tau_{\ell}^{M}),\quad\xi_{m}^{\alpha A}(\gamma^{Q})\;\coloneqq\;\theta^{\alpha A}+q_{m}^{\alpha A}(\tau_{\ell}^{M}).
\end{equation}
In these terms, each minitwistor line $\mathcal{L}_{m}\in\Sigma$
is specified by the incidence relations
\begin{equation}
\mu_{\dot{A}}\;=\;\lambda^{A}Y_{mA\dot{A}}(\gamma^{Q}),\quad\psi^{\alpha}\;=\;\lambda^{A}\xi_{mA}^{\alpha}(\gamma^{Q}).
\end{equation}
In addition, the family $\Sigma$ is an ordered set. Its orientation
is required by the index structure of the dual conformal $R$-invariants
\begin{equation}
R_{n;a_{1}b_{1},a_{2}b_{2},\dots,a_{k}b_{k}}
\end{equation}
which assigns each gluon in the scattering amplitude to a specific
line (or cluster) in $\Sigma$. We implemented this assignment in
Section III via the indicator map $c(i)$, which sends the $i$-th
gluon to its cluster.

Thus, our main problem becomes: how can we formalise a gauge-invariant,
Wilson-like operator that depends on an oriented family of minitwistor
lines, respects the analytic structure of the moduli superspace $\mathscr{M}_{2k+1}$,
and captures the algebraic-geometric character of a minitwistor line,
namely, a conic, rational curve with conormal bundle $\mathcal{O}_{\mathbf{C}}(2)$
embedded in the supersymmetric nonsingular quadric $\mathbf{MT}_{s}$?

We begin with the notion of algebraic cycles from intersection theory.
Recall from Chapter 1, $\S\,3$ of \citet{fulton2013intersection}
that, for an algebraic scheme $S$, a $k$-cycle is a finite formal
sum $\sum_{i}\,a_{i}[T_{i}]$, where each $T_{i}$ is a $k$-dimensional
subvariety of $S$ and each $a_{i}\in\mathbf{Z}$. The group of all
such cycles is $\mathrm{Z}_{k}(S)$, the free Abelian group generated
by the $k$-dimensional subvarieties of $S$.

Now, each minitwistor line $\mathcal{L}_{m}$ is a conic curve on
the nonsingular quadric $\mathbf{MT}_{s}$. Hence the ordered family
$\Sigma=\big\{\mathcal{L}_{1},\,\dots,\,\mathcal{L}_{2k+1}\big\}$
defines an algebraic one-cycle
\begin{equation}
\mathscr{S}_{2k+1}\;\coloneqq\;\sum_{m=1}^{2k+1}\big[\mathcal{L}_{m}\big]\;\in\;\mathrm{Z}_{1}\big(\mathbf{MT}_{s}\big).
\end{equation}
This cycle encodes the orientation of the lines and respects their
algebraic-geometric embedding in $\mathbf{MT}_{s}$. We then introduce
the \emph{holonomy operator} supported on $\mathscr{S}_{2k+1}$ as:
\begin{equation}
\mathrm{Hol}\big[\mathscr{S}_{2k+1}\big]\;=\;\mathrm{P}\left(\langle\mathcal{L}_{1}|u_{1}][u_{1}|\mathcal{L}_{2}|v_{1}][v_{1}|\mathcal{L}_{3}|u_{2}]\,\dots\,[u_{k}|\mathcal{L}_{2k}|v_{k}][v_{k}|\mathcal{L}_{2k+1}\rangle\right),
\end{equation}
where $\mathrm{P}$ denotes path ordering. The basic building blocks
$\langle\mathcal{L}|u]$, $[u|\mathcal{L}|v]$ and $[v|\mathcal{L}\rangle$
will be defined below. This functional acts on the one-cycles in $\mathrm{Z}_{1}(\mathbf{MT}_{s})$
and on the moduli parameters $u_{\ell}^{A},v_{\ell}^{A}$ introduced
in Eq. (\ref{eq:-204}). Finally, we obtain the \emph{Wilson operator}
by tracing over the gauge algebra:
\begin{equation}
\mathbb{W}^{\boldsymbol{A}}\big[\mathscr{S}_{2k+1}\big]\;\coloneqq\;\mathsf{Tr}_{\mathfrak{g}}\,\mathrm{Hol}\big[\mathscr{S}_{2k+1}\big].
\end{equation}
This nonlocal, gauge-invariant observable probes the holomorphic gauge
theory along the cycle $\mathscr{S}_{2k+1}$. In the sigma-model picture,
$\mathscr{S}_{2k+1}$ represents the classical configuration of a
system of $2k+1$ minitwistor strings. We will demonstrate that the
semiclassical expectation value of $\mathbb{W}[\mathscr{S}]$ serves
as the generating functional for the $\mathrm{N}^{k}\text{-MHV}$
leaf-gluon amplitudes.

We now proceed to define the building blocks of the holonomy operator.

\subsubsection{Building Blocks; Product Rules\label{subsec:Product-Rules}}

In the CCFT framework, we regard the holonomy $\mathrm{Hol}[\mathscr{S}]$
as a composite operator in the usual CFT sense. We order its elementary
building blocks by each line's position in the cycle: beginning, intermediate,
or end.

\textcompwordmark{}

\paragraph*{Elementary Blocks.}

For the initial line $\mathcal{L}_{1}$, we define the operator:
\begin{equation}
\langle\mathcal{L}_{1}|u]\;\coloneqq\;\mathrm{Pexp}\left(-\frac{1}{2\pi i}\underset{\mathcal{L}_{1}\,\,\,\,\,}{\int}\boldsymbol{a}(u;\lambda^{A})\wedge\boldsymbol{A}\big|_{\mathcal{L}_{1}}(\lambda^{A})\right).\label{eq:-213}
\end{equation}
The Green differential $\boldsymbol{a}(u;\lambda^{A})\in\Omega_{\mathfrak{m}}^{1,0}\big(\mathcal{L}_{1};\mathcal{O}_{\mathbf{C}}(1)\big)$
satisfies:
\begin{equation}
\frac{1}{2\pi i}\,\overline{\partial}_{\!\lambda}\big|_{\mathcal{L}_{in}}\;\boldsymbol{a}\big(u;\lambda^{A}\big)\;-\;\overline{\delta}(\lambda\!\cdot\!u)\wedge D\lambda\;=\;0,
\end{equation}
and the path-ordered wedge of $\boldsymbol{a}$ over points $\{\lambda_{i}^{A}\}_{i=1}^{a-1}$
is:
\begin{equation}
\mathrm{P}\!\bigwedge_{i=1}^{a-1}\,\boldsymbol{a}\big(u;\lambda_{i}^{A}\big)\;=\;\frac{D\lambda_{1}\wedge D\lambda_{2}\wedge\dots\wedge D\lambda_{a-1}}{\langle\lambda_{1},\lambda_{2}\rangle\langle\lambda_{2},\lambda_{3}\rangle\dots\langle\lambda_{a-2},\lambda_{a-1}\rangle\langle\lambda_{a-1},u\rangle}.\label{eq:-205}
\end{equation}

For a line $\mathcal{L}_{m}$ in the middle of the cycle, we set:
\begin{equation}
[u|\mathcal{L}_{m}|v]\;\coloneqq\;\mathrm{Pexp}\left(-\frac{1}{2\pi i}\underset{\mathcal{L}_{m}\,\,\,\,\,}{\int}\boldsymbol{b}(u,v;\lambda^{A})\wedge\boldsymbol{A}\big|_{\mathcal{L}_{m}}(\lambda^{A})\right).\label{eq:-214}
\end{equation}
The Green differential $\boldsymbol{b}(u,v;\lambda^{A})\in\Omega_{\mathfrak{m}}^{1,0}(\mathcal{L}_{m};\mathbf{C})$
satisfies:
\begin{equation}
\frac{1}{2\pi i}\,\overline{\partial}_{\!\lambda}\big|_{\mathcal{L}_{m}}\;\boldsymbol{b}\big(u,v;\lambda^{A}\big)\;+\;\boldsymbol{K}\big(u,v;\lambda^{A}\big)\wedge D\lambda=0,
\end{equation}
with the $(0,1)$-current:
\begin{equation}
\boldsymbol{K}\big(u,v;\lambda^{A}\big)\;\coloneqq\;\frac{1}{\langle u,v\rangle}\,\big(\overline{\delta}(u\!\cdot\!\lambda)-\overline{\delta}(v\!\cdot\!\lambda)\big)\;\in\;\mathscr{D}'_{0,1}\big(\mathcal{L}_{m};\,\mathcal{O}_{\mathbf{C}}(-1)\big).
\end{equation}
The path-ordered wedge of $\boldsymbol{b}$ over points $\{\lambda_{i}^{A}\}_{i=a}^{b-1}$
is
\begin{equation}
\mathrm{P}\!\bigwedge_{i=a}^{b-1}\,\boldsymbol{b}\big(u,v;\lambda_{i}^{A}\big)\;=\;\frac{D\lambda_{a}\wedge D\lambda_{a+1}\wedge\dots\wedge D\lambda_{b-1}}{\langle u,\lambda_{a}\rangle\langle\lambda_{a},\lambda_{a+1}\rangle\dots\langle\lambda_{b-2},\lambda_{b-1}\rangle\langle\lambda_{b-1},v\rangle}.\label{eq:-206}
\end{equation}

For the final line $\mathcal{L}_{f}\coloneqq\mathcal{L}_{2k+1}$,
we define:
\begin{equation}
[v|\mathcal{L}_{f}\rangle\;\coloneqq\;\mathrm{Pexp}\left(-\frac{1}{2\pi i}\underset{\mathcal{L}_{f}\,\,\,\,\,}{\int}\boldsymbol{c}(v;\lambda^{A})\wedge\boldsymbol{A}\big|_{\mathcal{L}_{f}}(\lambda^{A})\right).\label{eq:-215}
\end{equation}
The Green differential $\boldsymbol{c}(v;\lambda^{A})\in\Omega_{\mathfrak{m}}^{1,0}\big(\mathcal{L}_{f};\,\mathcal{O}_{\mathbf{C}}(1)\big)$
obeys:
\begin{equation}
\frac{1}{2\pi i}\,\overline{\partial}_{\!\lambda}\big|_{\mathcal{L}_{f}}\;\boldsymbol{c}\big(v;\lambda^{A}\big)\;+\;\overline{\delta}(\lambda\!\cdot\!v)\wedge D\lambda\;=\;0.
\end{equation}
Its path-ordered wedge is:
\begin{equation}
\mathrm{P}\!\bigwedge_{i=b}^{n}\,\boldsymbol{c}\big(v;\lambda_{i}^{A}\big)\;=\;\frac{D\lambda_{b}\wedge D\lambda_{b+1}\wedge\dots\wedge D\lambda_{n}}{\langle v,\lambda_{b}\rangle\langle\lambda_{b},\lambda_{b+1}\rangle\dots\langle\lambda_{n-1},\lambda_{n}\rangle\langle\lambda_{n},\lambda_{n+1}\rangle}.\label{eq:-207}
\end{equation}

Combining Eqs. (\ref{eq:-205}), (\ref{eq:-206}) and (\ref{eq:-207}),
path ordering yields:
\begin{equation}
\mathrm{P}\left(\;\bigwedge_{i=1}^{a-1}\,\boldsymbol{a}\big(u;\lambda_{i}^{A}\big)\bigwedge_{i'=a}^{b-1}\,\boldsymbol{b}\big(u,v;\lambda_{i'}^{B}\big)\bigwedge_{i''=b}^{n}\,\boldsymbol{c}\big(v;\lambda_{i''}^{C}\big)\right)\;=\;\mathcal{F}_{ab}^{\lambda}(u,v)\,\bigwedge_{j=1}^{n}\,\frac{D\lambda_{j}}{\lambda_{j}\cdot\lambda_{j+1}},\label{eq:-209}
\end{equation}
where:
\begin{equation}
\mathcal{F}_{ab}^{\lambda}(u,v)\;\coloneqq\;\frac{\langle\lambda_{a-1},\lambda_{a}\rangle\langle\lambda_{b-1},\lambda_{b}\rangle}{\langle\lambda_{a-1},u\rangle\langle u,\lambda_{a}\rangle\langle\lambda_{b-1},v\rangle\langle v,\lambda_{b}\rangle}.\label{eq:-210}
\end{equation}
This reproduces the $\mathcal{F}$-function from Section III in terms
of the spinors $\lambda_{i}^{A}$ instead of the celestial coordinates
$z_{i}$.

\textcompwordmark{}

\paragraph*{Product Rules.}

Here we define how the path-ordering symbol $\mathrm{P}$ organises
a product of Lie-algebra-valued exponentials. The operator $\mathrm{P}$
orders factors according to the index structure of the dual conformal
$R$-invariants. As a result, gluons in the scattering process group
into clusters, where each cluster lies on one of the minitwistor lines
that support the amplitude.

To define the path-ordering operator $\mathrm{P}$, we first introduce
preliminary notation and definitions. In particular, we formalise
the concept of a \emph{path-ordered wedge}, which represents the elementary
building blocks of the holonomy operator introduced above.

Let $\{\mathscr{L}_{i}\}_{i=1}^{N}$ denote a family of rational curves.
We do not assume these curves lie in $\mathbf{MT}_{s}$ or support
$\mathrm{N}^{k}\text{-MHV}$ minitwistor superamplitudes; they may
lie in any algebraic variety. For each line $\mathscr{L}_{i}$, define
its $n$-fold Cartesian product by:
\begin{equation}
\mathscr{L}_{i}^{n}\;\coloneqq\;\underbrace{\mathscr{L}_{i}\,\times\,\dots\,\times\,\mathscr{L}_{i}}_{n\text{-times}}.
\end{equation}

Next, introduce Green differentials $\{\varphi_{i}\}_{i=1}^{N}$ with
$\varphi_{i}\in\Omega_{\mathfrak{m}}^{1,0}(\mathscr{L}_{i})$, and
Lie-algebra-valued $(0,1)$-forms $\{\boldsymbol{B}_{i}\}_{i=1}^{N}$
with $\boldsymbol{B}_{i}\in\Omega^{0,1}(\mathscr{L}_{i};\mathfrak{g})$.
In our application, the $\varphi_{i}$ serve as the fundamental solutions
$\boldsymbol{a},\boldsymbol{b},\boldsymbol{c}$ of the CR operator
$\overline{\partial}\big|_{\mathscr{L}_{i}}$. The $\boldsymbol{B}_{i}$
represent the gauge potential induced on each $\mathscr{L}_{i}$ via
pull-back. Finally, introduce a formal coupling constant $g$.

Then each factor
\begin{equation}
\mathrm{Pexp}\left(\;g\int_{\mathscr{L}_{i}}\varphi_{i}\wedge\boldsymbol{B}_{i}\right)
\end{equation}
is called the \emph{path-ordered wedge} of $\varphi_{i}$ over $\mathscr{L}_{i}$.
In this terminology, the holonomy operator $\mathrm{Hol}[\mathscr{S}]$
arises by applying the path-ordering symbol $\mathrm{P}$ to the product
of these wedges over the minitwistor lines that form the cycle $\mathscr{S}$.

We encode the index structure of the order-$N$ dual conformal invariant
\[
R_{n;a_{1}b_{1},a_{2}b_{2},\dots,a_{N}b_{N}}
\]
into the path-ordering symbol $\mathrm{P}$ via the index family:
\begin{equation}
I_{N}^{n}\;\coloneqq\;\big\{\,\vec{a}=(a_{0},\dots,a_{N})\in\mathbf{Z}^{N+1}\,\big|\,a_{0}=1,\,a_{N}=n+1;\,2\leq a_{1}<a_{2}<\dots<a_{N-1}\leq n-1\,\big\}.
\end{equation}
With this notation, the action of $\mathrm{P}$ on $N$ wedges is:
\begin{equation}
\mathrm{P}\left(\;\prod_{k=1}^{N}\,e^{g\int_{\mathscr{L}_{k}}\varphi_{k}\wedge\boldsymbol{B}_{k}}\right)\;=\;\mathbb{I}_{\mathfrak{g}}\,+\,\sum_{n\geq N+1}\;g^{n}\;\sum_{\vec{a}\in I_{N}^{n}}\;\;\prod_{k=1}^{N}\;\;\underset{\Lambda_{k,\vec{a}}\,\,\,}{\int}\;\mathrm{P}\!\bigwedge_{j\in J_{k,\vec{a}}}\,\varphi_{k}(\lambda_{j}^{A})\wedge\boldsymbol{B}_{k}(\lambda_{j}^{A}).\label{eq:-208}
\end{equation}
Here the integration domain is:
\begin{equation}
\Lambda_{k,\vec{a}}\;\coloneqq\;\mathscr{L}_{k}^{a_{k}-a_{k-1}},
\end{equation}
and the index set for the exterior product reads:
\begin{equation}
J_{k,\vec{a}}\;\coloneqq\;\big\{\,j\in\mathbf{Z}\,\big|\,a_{k-1}\leq j\leq a_{k}-1\,\big\}.
\end{equation}

This completes the specification of the elementary blocks of the holonomy
operator and the action of $\mathrm{P}$ on its defining wedges. We
now turn to the computation of the Wilson line on algebraic one-cycles
and its semiclassical expectation value. We will show that this expectation
value serves as the generating functional for all tree-level $S$-matrix
elements in $\mathcal{N}=4$ SYM theory.

\subsubsection{Holonomy, Wilson Lines and Amplitudes}

The previous subsections extended our Wilson line operator $\mathbb{W}[\mathcal{L}]$,
originally supported on a single minitwistor line $\mathcal{L}\subset\mathbf{MT}_{s}$,
to the observable $\mathbb{W}[\mathscr{S}]$, where $\mathscr{S}\in\mathrm{Z}_{1}(\mathbf{MT}_{s})$
is an algebraic one-cycle. This observable measures how a background
gauge field deforms the holomorphic vacuum along $\mathscr{S}$. We
now compute its expectation value over the family of minitwistor lines
introduced in Section III. From this calculation, we derive a dictionary
that maps these holomorphic gauge theory observables on minitwistor
superspace to the tree-level $\mathcal{S}$-matrix of $\mathcal{N}=4$
SYM theory.

\textcompwordmark{}

\paragraph*{$\mathrm{N}^{1}\text{-}\mathrm{MHV}$ Sector.}

In Subsection III.1, we showed that the $\mathrm{N}^{1}\text{-MHV}$
minitwistor superamplitudes for gluons decompose into partial amplitudes
$\widetilde{M}_{n;ab}^{\mathsf{a}_{1}\dots\mathsf{a}_{n}}\big(\mathsf{W}_{i}^{I}\big)$.
For brevity, we call these the \emph{gluonic minitwistor amplitudes}.
The labels $a,b$ arise from the dual conformal invariant $R_{n;ab}$.
Without loss of generality, we fix
\begin{equation}
2\leq a<b\leq n-2,
\end{equation}
since all other orderings follow by permutation.

Using the celestial RSVW identity, we found that $\widetilde{M}_{n;ab}^{\mathsf{a}_{1}\dots\mathsf{a}_{n}}$
localises on three minitwistor lines $\mathcal{L}_{m}$ for $m=1,2,3$.
The external gluons then group into clusters: the $i$-th gluon lies
on
\begin{equation}
\begin{cases}
\mathcal{L}_{1}, & \text{if }1\leq i\leq a-1;\\
\mathcal{L}_{2}, & \text{if }a\leq i\leq b-1;\\
\mathcal{L}_{3}, & \text{if }b\leq i\leq n.
\end{cases}
\end{equation}
We formalise this assignment by the \emph{indicator function}:
\begin{equation}
c_{ab}(i)\;\coloneqq\;\begin{cases}
1, & i\in[1,a-1];\\
2, & i\in[a,b-1];\\
3, & i\in[b,n].
\end{cases}
\end{equation}
This function assigns each gluon label $i$ to its corresponding cluster
$c_{ab}(i)$.

We associate a set of creation and annihilation operators to each
minitwistor line $\mathcal{L}_{m}$. Denote by $\alpha_{m}^{\Delta,\mathsf{a}}$
the mode functions for gluons on $\mathcal{L}_{m}$. Physically, these
functions give the classical expectation values of the annihilation
operators. Now, combining this observation with Eq. (\ref{eq:-200}),
we decompose the gauge potential induced on $\mathcal{L}_{m}$ as:
\begin{equation}
\boldsymbol{A}\big|_{\mathcal{L}_{m}}\,(\lambda_{i}^{A})\;=\;2\pi i\;\underset{\mathbf{MT}_{s}^{*}\,\,\,}{\int}\;\Psi_{\Delta_{i}}\big|_{\mathcal{L}_{m}}\big(\lambda_{i}^{A};\mathsf{Z}_{i}'{}^{I}\big)\,\alpha_{m}^{\Delta_{i},\mathsf{a}_{i}}\big(\mathsf{Z}_{i}'{}^{I}\big)\,\mathsf{T}^{\mathsf{a}_{i}}\wedge D^{2|4}\mathsf{Z}_{i}'.\label{eq:-217}
\end{equation}

This organisation of mode coefficients and their associated creation
and annihilation operators across the three minitwistor lines provides
the physical motivation for interpreting celestial amplitudes in terms
of a minitwistor sigma-model. We thus identify each line on which
the amplitude localises as a semiclassical configuration of a minitwistor
string. Consequently, an $\mathrm{N}^{k}\text{-MHV}$ amplitude arises
from the correlation functions of a many-body system of $2k+1$ such
strings.

To proceed, we briefly review the geometric interpretation of minitwistor
amplitudes. The family of lines $\{\mathcal{L}_{m}\}_{m=1}^{3}$ is
parametrised by the moduli superspace
\begin{equation}
\mathscr{M}_{3}\;=\;\mathbf{H}_{s}\times\mathcal{P},
\end{equation}
where $\mathcal{P}$ denotes the parameter space for the single next-to-MHV
particle. In Kleinian signature, $\mathcal{P}\cong\mathbf{R}^{8|4}$.
Here, we analytically continue to the complex category, so that $\mathcal{P}\cong\mathbf{C}^{8|4}$.
We chart $\mathcal{P}$ by coordinates:
\begin{equation}
\tau^{M}\;=\;\big(u^{A},\,v^{B},\,\widetilde{u}_{\dot{A}},\,\widetilde{v}_{\dot{B}},\,\chi^{\alpha}\big),
\end{equation}
and orient it with the Berezin-DeWitt form $d^{8|4}\tau$. Thus the
full moduli space admits coordinates
\[
\gamma^{Q}=\big(\mathsf{X}^{K},\tau^{M}\big)\colon\;\mathscr{M}_{3}\;\longrightarrow\;\mathbf{CP}^{3|8}\times\mathbf{C}^{8|4},
\]
where $\mathsf{X}^{K}=(X_{A\dot{A}},\theta_{A}^{\alpha})$ are the
standard superspace coordinates on the complexified hyperboloid $\mathbf{H}_{s}$.
We equip $\mathscr{M}_{3}$ with the canonical orientation given by:
\begin{equation}
\mathcal{D}\gamma\;\coloneqq\;D^{3|8}\mathsf{X}\,\wedge\,d^{8|4}\tau.
\end{equation}
The semiclassical expectation value of the Wilson operator is then
expressed as an integral over $\mathscr{M}_{3}$ against the measure
$\mathcal{D}\gamma$.

As a final preparation for computing the holonomy operator, we explain
how a point $\gamma^{Q}\in\mathscr{M}_{3}$ specifies the family of
lines $\{\mathcal{L}_{m}\}$ via the evaluation map. Each line $\mathcal{L}_{m}$
is determined by moduli functions
\begin{equation}
\mathcal{Q}_{m}^{A\dot{A}}=\mathcal{Q}_{m}^{A\dot{A}}(\tau^{M}),\quad q_{m}^{\alpha A}=q_{m}^{\alpha A}(\tau^{M}),
\end{equation}
with domain $\mathcal{P}$. Combining these with the superspace coordinates
$\mathsf{X}^{K}\in\mathbf{H}_{s}$ defines the incidence maps:
\begin{equation}
Y_{m}^{A\dot{A}}(\gamma^{Q})\;\coloneqq\;X^{A\dot{A}}+\mathcal{Q}_{m}^{A\dot{A}}(\tau^{M}),\quad\xi_{m}^{\alpha A}(\gamma^{Q})\;\coloneqq\;\theta^{\alpha A}+q_{m}^{\alpha A}(\tau^{M}).
\end{equation}
Hence $\mathcal{L}_{m}$ is the locus in $\mathbf{MT}_{s}$ satisfying
$\mu_{m\dot{A}}=\lambda^{A}Y_{mA\dot{A}}(\gamma^{Q})$ and $\psi_{m}^{\alpha}=\lambda^{A}\xi_{mA}^{\alpha}(\gamma^{Q})$.
As $\gamma^{Q}$ varies over $\mathscr{M}_{3}$, these relations sweep
out all possible configurations of the three-line family.

To compute the holonomy operator for the three-line configuration,
we introduce a background gauge potential $\boldsymbol{A}$ on the
holomorphic bundle $E\to\mathbf{MT}_{s}$ and probe how $\boldsymbol{A}$
deforms the vacuum along the algebraic one-cycle:
\begin{equation}
\mathscr{S}_{3}(\gamma^{Q})\;\coloneqq\;\sum_{m=1,2,3}\;[\mathcal{L}_{m}]\;\in\;\mathrm{Z}_{1}(\mathbf{MT}_{s}).
\end{equation}
This cycle is parametrised by $\gamma^{Q}\in\mathscr{M}_{3}$, just
as a single line $\mathcal{L}(X,\theta)$ is parametrised by $\mathsf{X}^{K}\in\mathbf{H}_{s}$. 

The holonomy along $\mathscr{S}_{3}(\gamma^{Q})$ is:
\begin{equation}
\mathrm{Hol}\big[\mathscr{S}_{3}(\gamma^{Q})\big]\;=\;\mathrm{P}\left(\langle\mathcal{L}_{1}|u][u|\mathcal{L}_{2}|v][v|\mathcal{L}_{3}\rangle\right).
\end{equation}
Using the elementary blocks in Eqs. (\ref{eq:-213}), (\ref{eq:-214}),
(\ref{eq:-215}), we write:
\begin{align}
 & \mathrm{Hol}\big[\mathscr{S}_{3}(\gamma^{Q})\big]\;\\
 & =\;\mathrm{P}\left(\;e^{-\frac{1}{2\pi i}\int_{\mathcal{L}_{1}}\boldsymbol{a}(u;\lambda^{A})\wedge\boldsymbol{A}|_{\mathcal{L}_{1}}(\lambda^{A})}e^{-\frac{1}{2\pi i}\int_{\mathcal{L}_{2}}\boldsymbol{b}(u,v;\lambda^{A})\wedge\boldsymbol{A}|_{\mathcal{L}_{2}}(\lambda^{A})}e^{-\frac{1}{2\pi i}\int_{\mathcal{L}_{3}}\boldsymbol{c}(v;\lambda^{A})\wedge\boldsymbol{A}|_{\mathcal{L}_{3}}(\lambda^{A})}\right).
\end{align}
Applying the product rule (Eq. (\ref{eq:-208})) and the monomial
expansion (Eq. (\ref{eq:-209})) yields:
\begin{align}
 & \mathrm{Hol}\big[\mathscr{S}_{3}(\gamma^{Q})\big]\label{eq:-211}\\
 & =\;\mathbb{I}_{\mathfrak{g}}\;+\;\sum_{n\geq4}\;\frac{i^{n}}{(2\pi)^{n}}\;\sum_{2\leq a<b\leq n-1}\;\underset{\mathbf{L}_{n}\,\,\,}{\int}\;\mathcal{F}_{ab}^{\lambda}(\gamma^{Q})\;\bigwedge_{i=1}^{n}\;\frac{D\lambda_{i}}{\lambda_{i}\cdot\lambda_{i+1}}\wedge\boldsymbol{A}\big|_{\mathcal{L}_{c_{ab}(i)}}\,(\lambda_{i}^{A}).
\end{align}
Here $\mathbf{L}_{n}\coloneqq\bigtimes^{n}\mathbf{CP}^{1}$ and $\mathcal{F}_{ab}^{\lambda}$
is defined as in Eq. (\ref{eq:-210}).

We define the Wilson operator by:
\begin{equation}
\mathbb{W}^{\boldsymbol{A}}\big[\mathscr{S}_{3}(\gamma^{Q})\big]\;\coloneqq\;\mathsf{Tr}_{\mathfrak{g}}\;\mathrm{Hol}[\mathscr{S}_{3}(\gamma^{Q})].
\end{equation}
Hence Eq. (\ref{eq:-211}) gives:
\begin{align}
 & \mathbb{W}^{\boldsymbol{A}}\big[\mathscr{S}_{3}(\gamma^{Q})\big]\;=\;r\;+\;\sum_{n\geq4}\;(-1)^{n}\;\sum_{2\leq a<b\leq n-1}\;\underset{\mathbf{L}_{n}\,\,\,}{\int}\;\mathcal{F}_{ab}^{\lambda}(\gamma^{Q})\;\mathsf{Tr}\;\bigwedge_{i=1}^{n}\;\frac{D\lambda_{i}}{\lambda_{i}\cdot\lambda_{i+1}}\\
 & \qquad\underset{\mathbf{MT}_{s}^{*}\,\,\,}{\int}\;\Psi_{\Delta_{i}}\Big|_{\mathcal{L}_{c_{ab}(i)}}\,\big(\lambda_{i}^{A};\mathsf{Z}_{i}'{}^{I}\big)\;\alpha_{c_{ab}(i)}^{\Delta_{i},\mathsf{a}_{i}}\big(\mathsf{Z}_{i}'{}^{I}\big)\,\mathsf{T}^{\mathsf{a}_{i}}\wedge D^{2|4}\mathsf{Z}_{i}'.\label{eq:-212}
\end{align}
Define $\mathbf{X}_{n}^{*}\coloneqq\bigtimes^{n}\mathbf{MT}_{s}^{*}$
with measure $\bigwedge_{i}D^{2|4}\mathsf{Z}_{i}'$. Exchanging the
order of integration by Fubini's theorem, Eq. (\ref{eq:-212}) becomes:
\begin{align}
 & \mathbb{W}^{\boldsymbol{A}}\big[\mathscr{S}_{3}(\gamma^{Q})\big]\;=\;r\;+\;\sum_{n\geq4}\;(-1)^{n}\;\sum_{2\leq a<b\leq n-1}\;\underset{\mathbf{X}_{n}^{*}\,\,\,}{\int}\;\bigwedge_{i=1}^{n}\;D^{2|4}\mathsf{Z}_{i}'\wedge\alpha_{c_{ab}(i)}^{\Delta_{i},\mathsf{a}_{i}}\big(\mathsf{Z}_{i}'{}^{I}\big)\\
 & \qquad\underset{\mathbf{L}_{n}\,\,\,}{\int}\;\mathcal{F}_{ab}^{\lambda}(\gamma^{Q})\;\mathsf{Tr}\;\bigwedge_{j=1}^{n}\;\frac{D\lambda_{j}}{\lambda_{j}\cdot\lambda_{j+1}}\mathsf{T}^{\mathsf{a}_{j}}\wedge\Psi_{\Delta_{j}}\Big|_{\mathcal{L}_{c_{ab}(j)}}\,\big(\lambda_{j}^{A};\mathsf{Z}_{j}'{}^{I}\big).
\end{align}
Finally, applying the celestial BMSW identity gives:
\begin{align}
 & \mathbb{W}^{\boldsymbol{A}}\big[\mathscr{S}_{3}(\gamma^{Q})\big]\;=\;r\;+\;\sum_{n\geq4}\;(-1)^{n}\;\sum_{2\leq a<b\leq n-1}\;\underset{\mathbf{X}_{n}^{*}\,\,\,}{\int}\;\mathcal{F}_{ab}^{z'}(\gamma^{Q})\;\mathsf{Tr}\;\bigwedge_{i=1}^{n}\;\frac{\mathcal{C}(\Delta_{i})}{\langle z_{i}'|Y_{c_{ab}(i)}|\bar{z}_{i}']^{\Delta_{i}}}\label{eq:-216}\\
 & \qquad e^{i\langle z_{i}'|\xi_{c_{ab}(i)}\cdot\eta_{i}\rangle}\;\frac{\mathsf{T}^{\mathsf{a}_{i}}}{z_{i}'\cdot z_{i+1}'}\;\alpha_{c_{ab}(i)}^{\Delta_{i},\mathsf{a}_{i}}\big(\mathsf{Z}_{i}'{}^{I}\big)\wedge D^{2|4}\mathsf{Z}_{i}'.
\end{align}
This completes the explicit expansion of the Wilson operator in our
holomorphic gauge theory.

In the minitwistor sigma-model picture, the gauge field configuration
$\boldsymbol{A}$ probed by the Wilson operator $\mathbb{W}[\mathscr{S}_{3}]$
arises from a system of three minitwistor strings. For each point
$\gamma^{Q}\in\mathscr{M}_{3}$, we associate a semiclassical state
$\mathscr{S}_{3}(\gamma^{Q})$, represented by an algebraic cocycle.
Hence $\mathscr{M}_{3}$ serves as the configuration space of these
strings. Then, by analogy with Feynman's pathintegral, we define 
\begin{equation}
\mathcal{I}_{1}[\boldsymbol{A}]\;\coloneqq\;\underset{\mathscr{M}_{3}\,\,\,}{\int}\;\mathcal{D}\gamma\;\mathbb{W}^{\boldsymbol{A}}\big[\mathscr{S}_{3}(\gamma^{Q})\big],
\end{equation}
which we interpret as the semiclassical expectation value of the non-local
observable $\mathbb{W}^{\boldsymbol{A}}[\mathscr{S}_{3}]$ in our
holomorphic gauge theory\footnote{Recall that in the minitwistor-string framework, this gauge theory
functions as the string field theory.}. 

Integrating Eq. (\ref{eq:-216}) over $\mathscr{M}_{3}$ with the
measure $\mathcal{D}\gamma$ and using the Berezin integral property
$\int d^{0|8}\theta\;r=0$, we get:
\begin{align}
 & \mathcal{I}_{1}[\boldsymbol{A}]\;=\;\sum_{n\geq4}\;(-1)^{n}\;\sum_{2\leq a<b\leq n-1}\;\underset{\mathbf{X}_{n}^{*}\,\,\,}{\int}\;\underset{\mathscr{M}_{3}\,\,\,}{\int}\;\mathcal{D}\gamma\;\mathcal{F}_{ab}^{z'}(\gamma^{Q})\;\mathsf{Tr}\;\bigwedge_{i=1}^{n}\;\frac{\mathcal{C}(\Delta_{i})}{\langle z_{i}'|Y_{c_{ab}(i)}|\bar{z}_{i}']^{\Delta_{i}}}\\
 & \qquad e^{i\langle z_{i}'|\xi_{c_{ab}(i)}\cdot\eta_{i}\rangle}\;\frac{\mathsf{T}^{\mathsf{a}_{i}}}{z_{i}'\cdot z_{i+1}'}\;\alpha_{c_{ab}(i)}^{\Delta_{i},\mathsf{a}_{i}}\big(\mathsf{Z}_{i}'{}^{I}\big)\wedge D^{2|4}\mathsf{Z}_{i}'.
\end{align}
Taking functional derivatives of $\mathcal{I}_{1}$ with respect to
the mode coefficients $\alpha_{c_{ab}(i)}^{2h_{i},\mathsf{a}_{i}}\big(\mathsf{Z}_{i}^{I}\big)$
for fixed $a,b$ satisfying $2\leq a<b\leq n-1$, and then setting
$\boldsymbol{A}=0$, yields:
\begin{equation}
(-1)^{n}\;\prod_{i=1}^{n}\;\frac{\delta}{\delta\alpha_{c_{ab}(i)}^{2h_{i},\mathsf{a}_{i}}\big(\mathsf{Z}_{i}^{I}\big)}\;\mathcal{I}_{1}[\boldsymbol{A}]\Bigg|_{\boldsymbol{A}=0}\;=\;M_{n;ab}^{\mathsf{a}_{1}\dots\mathsf{a}_{n}}\big(\mathsf{Z}_{i}^{I}\big).
\end{equation}
Hence the semiclassical expectation value of the Wilson operator $\mathbb{W}[\mathscr{S}_{3}]$
serves as the generating functional for the $\mathrm{N}^{1}\text{-MHV}$
leaf-gluon amplitudes.

\textcompwordmark{}

\paragraph*{$\mathrm{N}^{2}\text{-}\mathrm{MHV}$ Sector.}

In Subsection \ref{subsec:N2-MHV-Scattering-Amplitude}, we showed
that the $\mathrm{N}^{2}\text{-MHV}$ minitwistor superamplitude for
gluons decomposes into a sum whose individual terms are the partial
amplitudes
\[
\widetilde{\mathcal{M}}_{n;a_{1}a_{2},b_{1}b_{2}}^{\mathsf{a}_{1}\dots\mathsf{a}_{n}}\big(\mathsf{W}_{i}^{I}\big),
\]
as derived in Eq. (\ref{eq:-122}). For brevity, we refer to each
$\widetilde{\mathcal{M}}$ simply as a minitwistor amplitude, with
the $\mathrm{N}^{2}\text{-MHV}$ configuration specified by the subscripts
$a_{\ell},b_{\ell}$ for $\ell=1,2$.

The index structure of each term arises from the order-two dual conformal
invariant $R_{n;a_{1}a_{2},b_{1}b_{2}}$. Henceforth, we restrict
to the index family
\begin{equation}
\mathbf{I}_{n}\;\coloneqq\;\big\{\,\vec{\alpha}=(a_{1},a_{2},b_{1},b_{2})\in\mathbf{Z}^{4}\,\big|\,2\leq a_{1}<a_{2}<b_{2}<b_{1}\leq n-1\,\big\}.
\end{equation}
All subsequent results hold equally for any other ordered choice of
indices by suitable permutation. Throughout this example we assign
the labels $\ell=1,2$ to the two next-to-MHV particles.

In Subsection \ref{subsec:N2-MHV-Scattering-Amplitude}, we saw that
the minitwistor amplitude localises on a family of five lines, $\{\mathcal{L}_{m}\}_{m=1}^{5}\subset\mathbf{MT}_{s}$.
We label these lines by $m=1,\dots,5$ and group the gluons in the
scattering process into clusters, each supported on one line $\mathcal{L}_{m}$.
For a given index vector $\vec{\alpha}\in\mathbf{I}_{n}$, the $i$-th
gluon resides on line $\mathcal{L}_{c_{\vec{\alpha}}(i)}$, where
the indicator function $c_{\vec{\alpha}}(i)$ for an $\mathrm{N}^{2}\text{-MHV}$
configuration is given by:
\begin{equation}
c_{\vec{\alpha}}(i)\;\coloneqq\;\begin{cases}
1, & 1\leq i\leq a_{1}-1;\\
2, & a_{1}\leq i\leq a_{2}-1;\\
3, & a_{2}\leq i\leq b_{2}-1;\\
4, & b_{2}\leq i\leq b_{1}-1;\\
5, & b_{1}\leq i\leq n.
\end{cases}
\end{equation}
Accordingly, we introduce annihilation and creation operators for
gluons on each line $\mathcal{L}_{m}$. The classical expectation
value of the annihilation operator on line $m$ defines the mode function
$\alpha_{m}^{\Delta,\mathsf{a}}\big(\mathsf{Z}^{I}\big)$. The minitwistor-Fourier
expansion of the gauge potential $\boldsymbol{A}\big|_{\mathcal{L}_{m}}$
appears in Eq. (\ref{eq:-217}).

The clustering of gluons described above provides the physical motivation
for interpreting the $\mathrm{N}^{2}\text{-MHV}$ amplitudes as correlation
functions of a many-body system composed of five minitwistor strings.
In this picture, each line $\mathcal{L}_{m}$ represents a semiclassical
configuration of the minitwistor sigma-model. 

With this in mind, we ask: what is the classical configuration space
of this quintet of minitwistor strings? It is the moduli superspace
of the family $\{\mathcal{L}_{m}\}$, namely the supermanifold:
\begin{equation}
\mathscr{M}_{5}\;=\;\mathbf{H}_{s}\times\mathcal{P}_{1}\times\mathcal{P}_{2}.
\end{equation}
Here $\mathcal{P}_{\ell}$ denotes the parameter space associated
with the $\ell$-th next-to-MHV particle. In this section we analytically
continue $\mathcal{P}_{\ell}$ to the complex category, so that $\mathcal{P}_{\ell}\cong\mathbf{C}^{8|4}$.
Globally, each $\mathcal{P}_{\ell}$ is charted by the moduli parameters:
\begin{equation}
\tau_{\ell}^{M}\;=\;\big(u_{\ell}^{A},\,v_{\ell}^{B},\,\widetilde{u}_{\ell\dot{A}},\,\widetilde{v}_{\ell\dot{B}},\,\chi_{\ell}^{\alpha}\big),
\end{equation}
where $u_{\ell}^{A}$, $v_{\ell}^{A}$, $\widetilde{u}_{\ell\dot{A}}$
and $\widetilde{v}_{\ell\dot{A}}$ are bosonic van der Waerden spinors
and $\chi_{\ell}^{\alpha}$ is a Grassmann variable. Hence the full
moduli superspace is globally parametrised by:
\begin{equation}
\gamma^{Q}\;=\;\big(\mathsf{X}^{K},\,\tau_{1}^{M_{1}},\,\tau_{2}^{M_{2}}\big).
\end{equation}

To each point $\gamma^{Q}\in\mathscr{M}_{5}$ we assign the algebraic
one-cycle
\begin{equation}
\mathscr{S}_{5}(\gamma^{Q})\;\coloneqq\;\sum_{m=1}^{5}\;[\mathcal{L}_{m}]\;\in\;\mathrm{Z}_{1}(\mathbf{MT}_{s}),
\end{equation}
which encodes the geometry of the five-line configuration $\{\mathcal{L}_{m}\}$.
When we later reinterpret our construction dynamically via the minitwistor
sigma-model, $\mathscr{S}_{5}(\gamma^{Q})$ will describe the classical
configuration of the five-string many-body system. In the present
context, however, our focus is on probing the holomorphic gauge theory
on $\mathbf{MT}_{s}$ by inserting a Wilson operator supported on
the cycle $\mathscr{S}_{5}(\gamma^{Q})$.

How, then, does a point in $\mathscr{M}_{5}$ map to a one-cycle in
$\mathrm{Z}_{1}(\mathbf{MT}_{s})$? The answer follows from the evaluation
map. Each line $\mathcal{L}_{m}$ is encoded by moduli functions
\begin{equation}
\mathcal{Q}_{m}^{A\dot{A}}=\mathcal{Q}_{m}^{A\dot{A}}\big(\tau_{1}^{M_{1}},\tau_{2}^{M_{2}}\big),\quad q_{m}^{\alpha A}=q_{m}^{\alpha A}(\tau_{1}^{M_{1}},\tau_{2}^{M_{2}})
\end{equation}
defined over the domain $\mathcal{P}_{1}\times\mathcal{P}_{2}$. Their
component expressions appear in Table \ref{tab:Embedding-coordinates--1}.
Together with the superspace coordinates $\mathsf{X}^{K}\in\mathbf{H}_{s}$,
they determine the incidence maps: 
\begin{equation}
Y_{m}^{A\dot{A}}\big(\gamma^{Q}\big)\;=\;X^{A\dot{A}}+\mathcal{Q}_{m}^{A\dot{A}}\big(\tau_{1}^{M_{1}},\tau_{2}^{M_{2}}\big),\quad\xi_{m}^{\alpha A}\big(\gamma^{Q}\big)\;=\;\theta^{\alpha A}+q_{m}^{\alpha A}\big(\tau_{1}^{M_{1}},\tau_{2}^{M_{2}}\big).
\end{equation}
Hence, as in the $\mathrm{N}^{1}\text{-MHV}$ case, the line $\mathcal{L}_{m}$
is the locus in $\mathbf{MT}_{s}$ satisfying the incidence relations
$\mu_{m\dot{A}}=\lambda^{A}Y_{mA\dot{A}}(\gamma^{Q})$ and $\psi_{m}^{\alpha}=\lambda^{A}\xi_{mA}^{\alpha}(\gamma^{Q})$.
As $\gamma^{Q}$ varies over $\mathscr{M}_{5}$, these relations for
$m=1,\dots,5$ sweep out every configuration of the five-line system.

We now compute the Wilson operator supported on the cycle $\mathscr{S}_{5}$.
Introduce a gauge field $\boldsymbol{A}$ for the holomorphic gauge
theory on the bundle $E$ over $\mathbf{MT}_{s}$. The holonomy along
$\mathscr{S}_{5}(\gamma^{Q})$ is:
\begin{equation}
\mathrm{Hol}\big[\mathscr{S}_{5}(\gamma^{Q})\big]\;=\;\mathrm{P}\left(\langle\mathcal{L}_{1}|u_{1}][u_{1}|\mathcal{L}_{2}|v_{1}][v_{1}|\mathcal{L}_{3}|u_{2}][u_{2}|\mathcal{L}_{4}|v_{2}][v_{2}|\mathcal{L}_{5}\rangle\right),
\end{equation}
where the moduli $u_{\ell}^{A},v_{\ell}^{A}$ are all encoded in the
superspace coordinates $\gamma^{Q}$. 

Using the monomial expansion of the Green differentials $\boldsymbol{a},\boldsymbol{b},\boldsymbol{c}$
(Eq. (\ref{eq:-209})), we write:
\begin{align}
 & \mathrm{P}\left(\bigwedge_{i=1}^{a_{1}-1}\boldsymbol{a}(u_{1};\lambda_{i}^{A})\bigwedge_{i=a_{1}}^{a_{2}-1}\boldsymbol{b}(u_{1},v_{1};\lambda_{i}^{A})\bigwedge_{i=a_{2}}^{b_{2}-1}\boldsymbol{b}(v_{1},u_{2};\lambda_{i}^{A})\bigwedge_{i=b_{2}}^{b_{1}-1}\boldsymbol{b}(u_{2},v_{2};\lambda_{i}^{A})\bigwedge_{i=b_{1}}^{n}\boldsymbol{c}(v_{2};\lambda_{i}^{A})\right)\\
 & \qquad=\;\mathcal{F}_{a_{1}a_{2},b_{2}b_{1}}^{\lambda}\big(\gamma^{Q}\big)\;\bigwedge_{i=1}^{n}\,\frac{D\lambda_{i}}{\lambda_{i}\cdot\lambda_{i+1}},
\end{align}
with
\begin{equation}
\mathcal{F}_{a_{1}a_{2},b_{2}b_{1}}^{\lambda}\big(\gamma^{Q}\big)\;\coloneqq\;\mathcal{F}_{a_{1}a_{2}}^{\lambda}\big(\tau_{1}^{M}\big)\,\mathcal{F}_{b_{2}b_{1}}^{\lambda}\big(\tau_{2}^{M}\big).
\end{equation}
Applying the product rule (Eq. (\ref{eq:-208})) gives:
\begin{align}
 & \mathrm{Hol}\big[\mathscr{S}_{5}(\gamma^{Q})\big]\label{eq:-219}\\
 & =\;\mathbb{I}_{\mathfrak{g}}\;+\;\sum_{n\geq6}\;\frac{i^{n}}{(2\pi)^{n}}\;\sum_{\vec{\alpha}\in\mathbf{I}_{n}}\;\underset{\mathbf{L}_{n}\,\,\,}{\int}\;\mathcal{F}_{a_{1}a_{2},b_{2}b_{1}}^{\lambda}\big(\gamma^{Q}\big)\;\bigwedge_{i=1}^{n}\;\frac{D\lambda_{i}}{\lambda_{i}\cdot\lambda_{i+1}}\mathsf{T}^{\mathsf{a}_{i}}\wedge\boldsymbol{A}\big|_{c_{\vec{\alpha}}(i)}\big(\lambda_{i}^{A}\big).\label{eq:-218}
\end{align}
The Wilson operator is its gauge trace,
\begin{equation}
\mathbb{W}^{\boldsymbol{A}}\big[\mathscr{S}_{5}(\gamma^{Q})\big]\;\coloneqq\;\mathsf{Tr}_{\mathfrak{g}}\,\mathrm{Hol}\big[\mathscr{S}_{5}(\gamma^{Q})\big].
\end{equation}
Substituting the minitwistor-Fourier expansion of $\boldsymbol{A}$
(cf. Eq. (\ref{eq:-217})) into the above, one finds:
\begin{align}
 & \mathbb{W}^{\boldsymbol{A}}\big[\mathscr{S}_{5}(\gamma^{Q})\big]\;=\;r\;+\;\sum_{n\geq6}\;(-1)^{n}\;\sum_{\vec{\alpha}\in\mathbf{I}_{n}}\;\underset{\mathbf{X}_{n}^{*}\,\,\,}{\int}\;\bigwedge_{i=1}^{n}\;D^{2|4}\mathsf{Z}_{i}'\wedge\alpha_{c_{\vec{\alpha}}(i)}^{\Delta_{i},\mathsf{a}_{i}}\big(\mathsf{Z}_{i}'{}^{I}\big)\\
 & \qquad\underset{\mathbf{L}_{n}\,\,\,}{\int}\;\mathcal{F}_{a_{1}a_{2},b_{2}b_{1}}^{\lambda}\big(\gamma^{Q}\big)\;\bigwedge_{j=1}^{n}\;\frac{D\lambda_{j}}{\lambda_{j}\cdot\lambda_{j+1}}\mathsf{T}^{\mathsf{a}_{j}}\wedge\Psi_{\Delta_{j}}\Big|_{\mathcal{L}_{c_{\vec{\alpha}}(i)}}\big(\lambda_{j}^{A};\mathsf{Z}_{j}'{}^{I}\big).
\end{align}
Invoking the celestial BMSW identity yields the full expansion,
\begin{align}
 & \mathbb{W}^{\boldsymbol{A}}\big[\mathscr{S}_{5}(\gamma^{Q})\big]\;=\;r\;+\;\sum_{n\geq6}\;(-1)^{n}\;\sum_{\vec{\alpha}\in\mathbf{I}_{n}}\;\underset{\mathbf{X}_{n}^{*}\,\,\,}{\int}\;\mathcal{F}_{a_{1}a_{2},b_{2}b_{1}}^{z'}\big(\gamma^{Q}\big)\\
 & \mathsf{Tr}\;\bigwedge_{i=1}^{n}\;\frac{\mathcal{C}(\Delta_{i})}{\langle z_{i}'|Y_{c_{\vec{\alpha}}(i)}|\bar{z}_{i}']^{\Delta_{i}}}\;e^{i\langle z_{i}'|\xi_{c_{\vec{\alpha}}(i)}\cdot\eta_{i}\rangle}\;\frac{\mathsf{T}^{\mathsf{a}_{i}}}{z_{i}'\cdot z_{i+1}'}\;\alpha_{c_{\vec{\alpha}}(i)}^{\Delta_{i},\mathsf{a}_{i}}\big(\mathsf{Z}_{i}'{}^{I}\big)\wedge D^{2|4}\mathsf{Z}_{i}'.\label{eq:-220}
\end{align}

As in the $\mathrm{N}^{1}\text{-MHV}$ case, the semiclassical expectation
value is:
\begin{equation}
\mathcal{I}_{2}[\boldsymbol{A}]\;\coloneqq\;\underset{\mathscr{M}_{5}\,\,\,\,\,}{\int}\;\mathcal{D}\gamma\;\mathbb{W}^{\boldsymbol{A}}\big[\mathscr{S}_{5}(\gamma^{Q})\big].
\end{equation}
Integrating $\mathbb{W}^{\boldsymbol{A}}$ over $\mathscr{M}_{5}$
and using $\int d^{0|8}\theta\;r=0$ gives:
\begin{align}
 & \mathcal{I}_{2}[\boldsymbol{A}]\;=\;\sum_{n\geq6}\;(-1)^{n}\;\sum_{\vec{\alpha}\in\mathbf{I}_{n}}\;\underset{\mathbf{X}_{n}^{*}\,\,\,}{\int}\;\underset{\mathscr{M}_{5}\,\,\,\,\,}{\int}\;\mathcal{D}\gamma\quad\mathcal{F}_{a_{1}a_{2},b_{2}b_{1}}^{z'}\big(\gamma^{Q}\big)\\
 & \mathsf{Tr}_{\mathfrak{g}}\;\bigwedge_{i=1}^{n}\;\frac{\mathcal{C}(\Delta_{i})}{\langle z_{i}'|Y_{c_{\vec{\alpha}}(i)}|\bar{z}_{i}']^{\Delta_{i}}}\;e^{i\langle z_{i}'|\xi_{c_{\vec{\alpha}}(i)}\cdot\eta_{i}\rangle}\;\frac{\mathsf{T}^{\mathsf{a}_{i}}}{z_{i}'\cdot z_{i+1}'}\;\alpha_{c_{\vec{\alpha}}(i)}^{\Delta_{i},\mathsf{a}_{i}}\big(\mathsf{Z}_{i}'{}^{I}\big)\wedge D^{2|4}\mathsf{Z}_{i}'.
\end{align}

Finally, fixing $\vec{\alpha}\in\mathbf{I}_{n}$ and $\{\mathsf{Z}_{i}^{I}\}\subset\mathbf{MT}_{s}^{*}$,
we take functional derivatives with respect to the mode coefficients
and set $\boldsymbol{A}=0$:
\begin{equation}
(-1)^{n}\;\prod_{i=1}^{n}\;\frac{\delta}{\delta\alpha_{c_{\vec{\alpha}}(i)}^{2h_{i},\mathsf{a}_{i}}\big(\mathsf{Z}_{i}^{I}\big)}\;\mathcal{I}_{2}[\boldsymbol{A}]\Bigg|_{\boldsymbol{A}=0}\;=\;\mathcal{M}_{n;a_{1}a_{2},b_{2}b_{1}}^{\mathsf{a}_{1}\dots\mathsf{a}_{n}}\big(\mathsf{Z}_{i}^{I}\big).
\end{equation}
This result confirms that the semiclassical limit of the nonlocal
gauge-invariant observable $\mathbb{W}[\mathscr{S}]$ in the minitwistor
string field theory generates the tree-level $\mathrm{N}^{2}\text{-MHV}$
leaf-gluon amplitudes.

\textcompwordmark{}

\subsubsection{Tree-level $\mathcal{S}$-Matrix}

Building on our results for the $\mathrm{N}^{1}\text{-}$ and $\mathrm{N}^{2}\text{-MHV}$
gluon amplitudes, we now show how the full tree-level gluon $\mathcal{S}$-matrix
in $\mathcal{N}=4$ SYM theory emerges from the semiclassical expectation
values of $\mathbb{W}[\mathscr{S}]$. This derivation is central to
our work, so we present it in detail. Along the way, we introduce
the operator/state correspondence in celestial CFT. In particular,
mapping each gluon operator to the minitwistor line that supports
its $\mathrm{N}^{k}\text{-MHV}$ amplitude motivates interpreting
those lines as the classical configurations of minitwistor strings.

From Subsection \ref{subsec:General-Case}, the $\mathrm{N}^{k}\text{-MHV}$
minitwistor superamplitude for $n$ gluons can be written as a sum
of partial amplitudes. Each summand has the form $\widetilde{\mathcal{M}}_{n;\vec{a}}^{\mathsf{a}_{1}\dots\mathsf{a}_{n}}\big(\mathsf{Z}_{i}^{I}\big)$,
where
\begin{equation}
\vec{a}\;=\;(a_{1},\dots,a_{k};\,b_{1},\dots,b_{k})\;\in\;\mathbf{I}_{k}^{n},
\end{equation}
and the index set is
\begin{equation}
\vec{}\mathbf{I}_{k}^{n}\;\coloneqq\;\big\{\,(a_{1},\dots,a_{k};b_{1},\dots,b_{k})\in\mathbf{Z}^{2k}\,\big|\,2\leq a_{1}<a_{2}<\dots<b_{2}<b_{1}\leq n-1\,\big\}.
\end{equation}
Let $N\coloneqq2k+1$. The ordering of the pairs $(a_{i},b_{j})$
reflects the structure of the order-$N$ dual conformal invariant
\begin{equation}
R_{n;a_{1}a_{2},a_{3}a_{4},\dots,b_{4}b_{3},b_{2}b_{1}}.
\end{equation}
Since every term in the full $\mathrm{N}^{k}\text{-MHV}$ amplitude
arises by permuting the entries of some $\vec{a}\in\mathbf{I}_{k}^{n}$,
we refer to each $\widetilde{\mathcal{M}}_{n;\vec{a}}^{\mathsf{a}_{1}\dots\mathsf{a}_{n}}$
as a minitwistor gluon amplitude, with its $\mathrm{N}^{k}\text{-MHV}$
configuration specified by $\vec{a}$.

\textcompwordmark{}

\paragraph*{Localisation.}

Subsection \ref{subsec:General-Case} established a localisation theorem:
each partial amplitude $\widetilde{\mathcal{M}}_{n;\vec{a}}^{\mathsf{a}_{1}\dots\mathsf{a}_{n}}$
is supported on a family $\{\mathcal{L}_{m}\}_{m=1}^{N}\subset\mathbf{MT}_{s}$
of minitwistor lines. Here $m=1,\dots,N$ labels the lines, and $\ell=1,\dots,k$
labels one of the next-to-MHV bosons.

The geometric configuration of these lines is parametrised by the
moduli superspace:
\begin{equation}
\mathscr{M}_{N}\;\coloneqq\;\mathbf{H}_{s}\times\mathcal{P}_{1}\times\mathcal{P}_{2}\times\dots\times\mathcal{P}_{k},
\end{equation}
where each $\mathcal{P}_{\ell}\cong\mathbf{R}^{8|4}$ is the parameter
superspace for one external next-to-MHV gluon. The reality of $\mathcal{P}_{\ell}$
follows from the fact that we originally defined the leaf-gluon amplitudes
in split-signature, which forces the momentum-twistor components to
be real.

We then analytically continue to the complex category by setting $\mathcal{P}_{\ell}\cong\mathbf{C}^{8|4}$.
In this way, $\mathscr{M}_{N}$ becomes the moduli superspace of lines
in the complexified minitwistor superspace $\mathbf{MT}_{s}$. The
coordinates $\tau_{\ell}^{M}$ on each $\mathcal{P}_{\ell}$ were
introduced in Eq. (\ref{eq:-223}); the full moduli coordinates $\gamma^{Q}$
were defined in Eq. (\ref{eq:-204}); and the standard orientation
on $\mathscr{M}_{N}$ is given by the measure $\mathcal{D}\gamma$
in Eq. (\ref{eq:-224}).

\textcompwordmark{}

\paragraph*{Evaluation Map and Incidence Relations.}

The moduli space $\mathscr{M}_{N}$ determines the embedding of each
minitwistor line $\mathcal{L}_{m}$ via an evaluation map. In global
coordinates, line $\mathcal{L}_{m}$ carries two sets of moduli functions:
\begin{equation}
\mathcal{Q}_{m}^{A\dot{A}}=\mathcal{Q}_{m}^{A\dot{A}}\big(\tau_{1}^{M_{1}},\tau_{2}^{M_{2}},\dots,\tau_{k}^{M_{k}}\big),\quad q_{m}^{\alpha A}=q_{m}^{\alpha A}\big(\tau_{1}^{M_{1}},\tau_{2}^{M_{2}},\dots,\tau_{k}^{M_{k}}\big)
\end{equation}
whose domain is the full parameter space $\bigtimes_{\ell=1}^{k}\mathcal{P}_{\ell}$.
These moduli functions combine with the standard coordinates $\mathsf{X}^{K}=(X_{A\dot{A}},\theta_{A}^{\alpha})$
on the complexified hyperbolic superspace $\mathbf{H}_{s}$ to define
the incidence maps:
\begin{equation}
Y_{m}^{A\dot{A}}\big(\gamma^{Q}\big)\;=\;X^{A\dot{A}}+\mathcal{Q}_{m}^{A\dot{A}}\big(\tau_{\ell}^{M}\big),\quad\xi_{m}^{\alpha A}\big(\gamma^{Q}\big)\;=\;\theta^{\alpha A}+q_{m}^{\alpha A}\big(\tau_{\ell}^{M}\big).
\end{equation}
Accordingly, the line $\mathcal{L}_{m}(\gamma^{Q})$ is the locus
of points $\mathsf{W}_{m}^{I}=(\lambda_{m}^{A},\mu_{m\dot{A}},\psi_{m}^{\alpha})$
in minitwistor superspace satisfying, for all $[\sigma^{A}]\in\mathbf{CP}^{1}$,
\begin{equation}
\lambda_{m}^{A}=\sigma^{A},\quad\mu_{m\dot{A}}=\sigma^{A}Y_{mA\dot{A}}(\gamma^{Q}),\quad\psi_{m}^{\alpha}=\sigma^{A}\xi_{mA}^{\alpha}(\gamma^{Q}).
\end{equation}


\paragraph*{Operator/State Correspondence.}

Another consequence of the localisation theorem is that, for each
index vector $\vec{a}\in\mathbf{I}_{k}^{n}$,
\begin{equation}
\widetilde{\mathcal{M}}_{n;\vec{a}}^{\mathsf{a}_{1}\dots\mathsf{a}_{n}}\big(\mathsf{W}_{i}^{I}\big)\;=\;0
\end{equation}
whenever the minitwistor $\mathsf{W}_{i}^{I}$ representing the $i$-th
gluon does not lie on the line $\mathcal{L}_{c_{\vec{a}}(i)}$. Here
$c_{\vec{a}}(i)$ is the indicator function defined in Subsection
\ref{subsec:General-Case} that assigns each gluon to its cluster
(for example, $c_{\vec{a}}(i)=1$ for $1\leq i\leq a_{1}-1$, $c_{\vec{a}}(i)=2$
for $a_{1}\leq i\leq a_{2}-1$, and so on).

In celestial CFT, the $i$-th external gluon is described by a conformal
weight $\Delta_{i}$ and an insertion point $\mathsf{z}_{i}=(z_{i},\bar{z}_{i},\eta_{i}^{\alpha})\in\mathcal{CS}_{s}$,
where $z_{i},\bar{z}_{i}\in\mathbf{CP}^{1}$ and $\eta_{i}^{\alpha}$
$\left(\alpha=1,\dots,4\right)$ encodes the helicity. Since the dual
minitwistor superspace $\mathbf{MT}_{s}^{*}$ covers $\mathcal{CS}_{s}$,
we represent $\mathsf{z}_{i}$ by a dual minitwistor $\mathsf{Z}_{i}^{I}$.
We then postulate a field operator
\[
O_{\Delta_{i}}^{\mathsf{a}_{i}}\big(\mathsf{Z}_{i}^{I}\big)
\]
 for the $i$-th gluon in the dual minitwistor representation. The
$\mathcal{MT}$-transform maps this to a minitwistor operator $\widehat{O}_{\Delta_{i}}^{\mathsf{a}_{i}}\big(\mathsf{W}_{i}^{I}\big)$,
in direct analogy with the Fourier transform between momentum and
position representations in quantum mechanics.

Combining this with the localisation theorem, we attach a family of
gluon operators to each line $\mathcal{L}_{m}$. Fix $\vec{a}\in\mathbf{I}_{k}^{n}$.
Then for all $i=1,\dots,n$, the operator $\widehat{O}_{\Delta_{i}}^{\mathsf{a}_{i}}\big(\mathsf{W}_{i}^{I}\big)$
is supported on the line $\mathcal{L}_{c_{\vec{a}}(i)}$. We interpret
the classical expectation values of the gluon annihilation operators
as the mode coefficients $\alpha_{m}^{\Delta,\mathsf{a}_{i}}$ that
appear in the gauge field expansion (see Eq. (\ref{eq:-217})). This
assignment \emph{suggests} that each minitwistor line, together with
its collection of mode functions, admits a semiclassical interpretation
as a minitwistor ``string.''

\textcompwordmark{}

\paragraph*{Holonomy Operator.}

The family of lines $\{\mathcal{L}_{m}\}$ defines an algebraic one-cycle
\begin{equation}
\mathscr{S}_{N}(\gamma^{Q})\;\coloneqq\;\sum_{m=1}^{N}[\mathcal{L}_{m}]\;\in\;\mathrm{Z}_{1}(\mathbf{MT}_{s}),
\end{equation}
where $\gamma^{Q}$ makes its dependence on the moduli superspace
$\mathscr{M}_{N}$ explicit. If each $\mathcal{L}_{m}$ gives a classical
solution of the minitwistor sigma-model, then $\mathscr{S}_{N}$ describes
a many-body state of $N$ such strings in $\mathbf{MT}_{s}$. In the
next section we will show that the correlators of this multi-string
system reproduce the full tree-level gluon $\mathcal{S}$-matrix.

As $\gamma^{Q}$ varies over $\mathscr{M}_{N}$, the cycle $\mathscr{S}_{N}(\gamma^{Q})$
sweeps out every possible $N$-string configuration. This observation
motivates defining the semiclassical expectation value of the Wilson
operator $\mathbb{W}[\mathscr{S}]$ by integrating over $\mathscr{M}_{N}$
against the measure $\mathcal{D}\gamma$.

For now, we compute the holonomy operator supported on $\mathscr{S}_{N}$:
\begin{equation}
\mathrm{Hol}\big[\mathscr{S}_{N}(\gamma^{Q})\big]\;\coloneqq\;\mathrm{P}\left(\langle\mathcal{L}_{1}|u_{1}][u_{1}|\mathcal{L}_{2}|v_{1}][v_{1}|\mathcal{L}_{3}|u_{2}]\dots[u_{k}|\mathcal{L}_{2k}|v_{k}][v_{k}|\mathcal{L}_{2k+1}\rangle\right).
\end{equation}
Here $\mathrm{Hol}$ is a functional on $\mathrm{Z}_{1}(\mathbf{MT}_{s})$,
and its dependence on $\gamma^{Q}\in\mathscr{M}_{N}$ enters through
the moduli parameters $u_{\ell}^{A}$ and $v_{\ell}^{A}$ in each
elementary block $\langle\mathcal{L}_{m}|u][u|\mathcal{L}_{m+1}|v]$.

The monomial expansion and the product rule from Subsection \ref{subsec:Product-Rules}
imply that the holonomy operator admits the expansion:
\begin{align}
 & \mathrm{Hol}\big[\mathscr{S}_{N}(\gamma^{Q})\big]\\
 & =\;\mathbb{I}_{\mathfrak{g}}\;+\;\sum_{n\geq N+1}\;(-1)^{n}\;\sum_{\vec{a}\in\mathbf{I}_{k}^{n}}\;\underset{\mathbf{L}^{n}\,\,\,}{\int}\;\mathcal{F}_{\vec{a}}^{\lambda}\big(\gamma^{Q}\big)\;\bigwedge_{i=1}^{n}\;\frac{D\lambda_{i}}{\lambda_{i}\cdot\lambda_{i+1}}\wedge\boldsymbol{A}\big|_{\mathcal{L}_{c_{\vec{a}}(i)}}\big(\lambda_{i}^{A}\big),
\end{align}
where $\mathbf{L}^{n}\coloneqq\bigtimes^{n}\mathbf{CP}^{1}$ and
\begin{equation}
\mathcal{F}_{\vec{a}}^{\lambda}\big(\gamma^{Q}\big)\;\coloneqq\;\mathcal{F}_{a_{1}a_{2}}^{\lambda}\big(\tau_{1}^{M}\big)\,\mathcal{F}_{a_{3}a_{4}}^{\lambda}\big(\tau_{2}^{M}\big)\,\dots\,\mathcal{F}_{b_{4}b_{3}}^{\lambda}\big(\tau_{k-1}^{M}\big)\,\mathcal{F}_{b_{2}b_{1}}^{\lambda}\big(\tau_{k}^{M}\big).
\end{equation}
Recall that the $\mathcal{F}$-function first appeared in Subsection
\ref{subsec:General-Case} and was redefined in Eq. (\ref{eq:-210}).

Substituting the gauge potential expansion from Eq. (\ref{eq:-217})
and employing Fubini's theorem, we obtain\footnote{Recall that $\mathbf{X}_{n}^{*}\coloneqq\bigtimes^{n}\mathbf{MT}_{s}^{*}$.}:
\begin{align}
 & \mathrm{Hol}\big[\mathscr{S}_{N}(\gamma^{Q})\big]\;=\;\mathbb{I}_{\mathfrak{g}}\;+\;\sum_{n\geq N+1}\;(-1)^{n}\;\sum_{\vec{a}\in\mathbf{I}_{k}^{n}}\;\underset{\mathbf{X}_{n}^{*}\,\,\,}{\int}\;\bigwedge_{i=1}^{n}\;D^{2|4}\mathsf{Z}_{i}'\wedge\alpha_{c_{\vec{a}}(i)}^{\Delta_{i},\mathsf{a}_{i}}\big(\mathsf{Z}_{i}'{}^{I}\big)\\
 & \qquad\underset{\mathbf{L}^{n}\,\,\,}{\int}\;\mathcal{F}_{\vec{a}}^{\lambda}\big(\gamma^{Q}\big)\;\bigwedge_{j=1}^{n}\;\frac{D\lambda_{j}}{\lambda_{j}\cdot\lambda_{j+1}}\mathsf{T}^{\mathsf{a}_{j}}\wedge\Psi_{\Delta_{j}}\Big|_{\mathcal{L}_{c_{\vec{a}}(j)}}\big(\lambda_{j}^{A};\mathsf{Z}_{j}'{}^{I}\big).
\end{align}
Applying the celestial BMSW identity then yields the final form of
this expansion, in terms of the mode coefficients $\alpha_{m}^{\Delta,\mathsf{a}}$
of the gauge potential and the moduli coordinates $\gamma^{Q}$:
\begin{align}
 & \mathrm{Hol}\big[\mathscr{S}_{N}(\gamma^{Q})\big]\;=\;\mathbb{I}_{\mathfrak{g}}\;+\;\sum_{n\geq N+1}\;(-1)^{n}\;\sum_{\vec{a}\in\mathbf{I}_{k}^{n}}\;\underset{\mathbf{X}_{n}^{*}\,\,\,}{\int}\;\mathcal{F}_{\vec{a}}^{z'}\big(\gamma^{Q}\big)\\
 & \qquad\bigwedge_{i=1}^{n}\;\frac{\mathcal{C}(\Delta_{i})}{\langle z_{i}'|Y_{c_{\vec{a}}(i)}|\bar{z}_{i}']^{\Delta_{i}}}\,e^{i\langle z_{i}'|\xi_{c_{\vec{a}}(i)}\cdot\eta_{i}\rangle}\,\frac{\mathsf{T}^{\mathsf{a}_{i}}}{z_{i}'\cdot z_{i+1}'}\;\alpha_{c_{\vec{a}}(i)}^{\Delta_{i},\mathsf{a}_{i}}\big(\mathsf{Z}_{i}'{}^{I}\big)\wedge D^{2|4}\mathsf{Z}_{i}'.\label{eq:-222}
\end{align}


\paragraph*{Wilson Operator; Semiclassical Expectation Value.}

The Wilson operator supported on the algebraic one-cycle is defined
by:
\begin{equation}
\mathbb{W}^{\boldsymbol{A}}\big[\mathscr{S}_{N}(\gamma^{Q})\big]\;\coloneqq\;\mathsf{Tr}_{\mathfrak{g}}\;\mathrm{Hol}\big[\mathscr{S}_{N}(\gamma^{Q})\big],
\end{equation}
where the superscript $\boldsymbol{A}$ emphasise its dependence on
the background gauge field. Physically, $\mathbb{W}[\mathscr{S}]$
is a nonlocal, gauge-invariant probe of the deformation of the holomorphic
vacuum of the gauge theory on $E\to\mathbf{MT}_{s}$ along $\mathscr{S}_{N}$.

We define its semiclassical expectation value by integrating over
the moduli superspace $\mathscr{M}_{N}$ with measure $\mathcal{D}\gamma$:
\begin{equation}
\mathcal{I}_{k}[\boldsymbol{A}]\;\coloneqq\;\left\langle \mathbb{W}^{\boldsymbol{A}}\big[\mathscr{S}_{N}\big]\right\rangle _{0}\;\coloneqq\;\underset{\mathscr{M}_{N}\,\,\,}{\int}\;\mathcal{D}\gamma\;\mathbb{W}^{\boldsymbol{A}}\big[\mathscr{S}_{N}(\gamma^{Q})\big].\label{eq:-225}
\end{equation}
Substituting the expansion of $\mathrm{Hol}$ (see Eq. (\ref{eq:-222}))
and using the Berezin identity $\int d^{0|8}\theta\;r=0$, one finds:
\begin{align}
 & \mathcal{I}_{k}[\boldsymbol{A}]\;=\;\sum_{n\geq N+1}\;(-1)^{n}\;\sum_{\vec{a}\in\mathbf{I}_{k}^{n}}\;\underset{\mathbf{X}_{n}^{*}\,\,\,}{\int}\;\underset{\mathscr{M}_{N}\,\,\,}{\int}\;\mathcal{F}_{\vec{a}}^{z'}\big(\gamma^{Q}\big)\\
 & \qquad\bigwedge_{i=1}^{n}\;\frac{\mathcal{C}(\Delta_{i})}{\langle z_{i}'|Y_{c_{\vec{a}}(i)}|\bar{z}_{i}']^{\Delta_{i}}}\,e^{i\langle z_{i}'|\xi_{c_{\vec{a}}(i)}\cdot\eta_{i}\rangle}\,\frac{\mathsf{T}^{\mathsf{a}_{i}}}{z_{i}'\cdot z_{i+1}'}\;\alpha_{c_{\vec{a}}(i)}^{\Delta_{i},\mathsf{a}_{i}}\big(\mathsf{Z}_{i}'{}^{I}\big)\wedge D^{2|4}\mathsf{Z}_{i}'.
\end{align}
Finally, fixing $\vec{a}\in\mathbf{I}_{k}^{n}$, let $h_{i}$ denote
the scaling dimension of the $i$-th gluon and let $\{\mathsf{Z}_{i}^{I}\}\subset\mathbf{MT}_{s}^{*}$
be the insertion points. Functional differentiation of $\mathcal{I}_{k}[\boldsymbol{A}]$
with respect to the modes $\alpha_{c_{\vec{a}}(i)}^{2h_{i},\mathsf{a}_{i}}\big(\mathsf{Z}_{i}^{I}\big)$,
followed by setting $\boldsymbol{A}=0$, yields:
\begin{equation}
(-1)^{n}\;\prod_{i=1}^{n}\;\frac{\delta}{\delta\alpha_{c_{\vec{a}}(i)}^{2h_{i},\mathsf{a}_{i}}\big(\mathsf{Z}_{i}^{I}\big)}\;\mathcal{I}_{k}[\boldsymbol{A}]\;=\;\mathcal{M}_{n;\vec{a}}^{\mathsf{a}_{1}\dots\mathsf{a}_{n}}\big(\mathsf{Z}_{i}^{I}\big).
\end{equation}
Thus $\mathcal{I}_{k}[\boldsymbol{A}]$ is the generating functional
for the leaf sub-amplitudes in $\mathrm{N}^{k}\text{-MHV}$ configurations
and hence reproduces the full tree-level gluonic $\mathcal{S}$-matrix
in $\mathcal{N}=4$ SYM theory.

\subsection{Discussion}

Now we pause to reflect on the calculations performed so far. In a
more conventional approach to flat-space holography, one may begin
with the $\mathrm{AdS}/\mathrm{CFT}$ correspondence and then take
the $R\to\infty$ limit of anti-de Sitter space to (hopefully) extract
information about the celestial CFT. Here we have adopted a more indirect
strategy. 

In the $\mathrm{AdS}/\mathrm{CFT}$ context, \citet{alday2009null}
examined Wilson loop operators at strong coupling that trace null
segments. They studied the classical equations of motion for a string
with $\mathrm{AdS}_{3}$ as its target space, subject to boundary
conditions that force the worldsheet to end on a null polygon at the
conformal boundary of $\mathrm{AdS}$. Using gauge/gravity duality,
they showed that this construction computes certain eight-gluon scattering
amplitudes.

On the other hand, leaf amplitudes arise by dimensionally reducing
the split-signature celestial amplitudes along the standard hyperbolic
foliation of Klein space, whose leaves are Lorentzian $\mathrm{AdS}_{3}$.
This construction raises the question: if Alday and Maldacena reconstructed
gluon scattering amplitudes from null Wilson loops in $\mathrm{AdS}_{3}$,
can we likewise reconstruct celestial leaf amplitudes from nonlocal,
gauge-invariant observables?

To address this question, we first examined the geometric interpretation
of the tree-level $\mathrm{N}^{k}\text{-MHV}$ minitwistor amplitudes
for gluons discussed in the previous section. We found that these
amplitudes localise on a family $\mathcal{L}_{1},\dots,\mathcal{L}_{2k+1}$
of minitwistor lines. Next, we obtained a field-theoretic interpretation
of the leaf-gluon amplitudes by formulating holomorphic gauge theory
on a complex-vector bundle $E$ over minitwistor superspace $\mathbf{MT}_{s}$.
We then probed the gauge theory by inserting a Wilson-like operator
$\mathbb{W}[\mathscr{S}]$ supported on the algebraic one-cycle $\mathscr{S}=\sum_{m}[\mathcal{L}_{m}]$.
Physically, $\mathbb{W}[\mathscr{S}]$ measures how the background
gauge potential $\boldsymbol{A}$ deforms the holomorphic vacuum.

Finally, we demonstrated that the semiclassical expectation value
of $\mathbb{W}[\mathscr{S}]$ generates the leaf-gluon amplitudes.
We defined this expectation value as an integral over the moduli superspace
$\mathscr{M}_{N}$ (see Eq. (\ref{eq:-225})). A question then arises:
what is the origin of the holomorphic gauge theory on $E\to\mathbf{MT}_{s}$
whose Wilson operator we have used?

Perhaps our boldest proposal is that the holomorphic gauge theory
on $E\to\mathbf{MT}_{s}$ arises as the string field theory limit
of the minitwistor sigma-model introduced in the next section. This
picture departs from the usual celestial-holography dictionary. There,
the flat-space hologram appears as a CFT on the celestial sphere;
here, we instead obtain a holomorphic gauge theory on minitwistor
superspace. This theory emerges as the field-theory limit of a sigma-model
whose target space is $\mathbf{MT}_{s}$ and whose worldsheet is the
Riemann supersphere. 

We then relate our model to the celestial CFT by treating the dual
minitwistor superspace $\mathbf{MT}_{s}^{*}$ as a covering space
of the celestial supersphere $\mathcal{CS}_{s}$. The $\mathcal{MT}$-transform
maps $\mathbf{MT}_{s}^{*}$ back to $\mathbf{MT}_{s}$, thus completing
the correspondence between the sigma-model and the celestial CFT.

\section{Minitwistor String Theory}

This section develops the central idea of our work. We propose a semiclassical
dynamical model for a celestial CFT. The model is holographically
dual to the tree-level gluonic sector of $\mathcal{N}=4$ SYM. We
consider a many-body system of $N$ minitwistor strings, where each
string is realised as a topological sigma-model. The worldsheet is
the $\mathcal{N}=4$ celestial supersphere $\mathcal{CS}_{s}$, and
the target is the minitwistor superspace $\mathbf{MT}_{s}$. We show
that, in the leading-trace semiclassical limit, the $N$-string system
reproduces the tree-level $\mathrm{N}^{k}\text{-MHV}$ leaf amplitudes
for gluons when $N=2k+1$.

To briefly recap the developments so far, in Section II we developed
the formalism of minitwistor superwavefunctions as a toolkit for studying
celestial leaf amplitudes for gluons in $\mathcal{N}=4$ SYM theory.
In Section III we applied this toolkit to the Drummond-Henn solution
of the super-BCFW recursion relations; from that analysis we proved
the localisation theorem. The theorem states that the minitwistor
transform of tree-level leaf-gluon amplitudes in every $\mathrm{N}^{k}\text{-MHV}$
sector localises on a family of minitwistor lines.

We interpret those lines as algebraic one-cycles $\mathscr{S}$ on
$\mathbf{MT}_{s}$. By formulating a holomorphic gauge theory on minitwistor
superspace, we obtain a geometric interpretation of the localisation
theorem: the leaf-gluon amplitudes arise as minitwistor Wilson lines
$\mathbb{W}[\mathscr{S}]$ supported on those one-cycles. We assign
the Fourier modes $\alpha_{m}^{\Delta,\mathsf{a}_{i}}$ of the background
gauge potential $\boldsymbol{A}$ to the minitwistor lines in the
localisation family $\{\mathcal{L}_{m}\}$. Interpreting these modes
as the classical expectation values of gluon annihilation operators
attached to the lines yields a dynamical picture. In this picture,
the lines $\mathcal{L}_{m}\subset\mathbf{MT}_{s}$ in the localisation
family are viewed as minitwistor strings interacting with the background
gauge potential.

We now study a semiclassical many-body system of $N$ minitwistor
strings propagating on the background defined by the holomorphic gauge
theory on $\mathbf{MT}_{s}$. We show that the partition function
of this theory serves as a generating functional for the tree-level
leaf-gluon amplitudes in the $\mathrm{N}^{k}\text{-MHV}$ sector.
From this statement we conclude that the holomorphic gauge theory
introduced in Section IV can be identified with the string-field limit
of the string theory considered here.

We also define vertex operators that encode the worldsheet interactions
of the minitwistor strings. From these vertex operators we construct
the corresponding celestial gluon operators. By setting the background
gauge potential to zero, we isolate the purely worldsheet interactions.
In this limit, the leading-trace, semiclassical correlators of the
celestial gluon operators reproduce the tree-level $\mathrm{N}^{k}\text{-MHV}$
leaf amplitudes. Finally, we show that the OPEs of the celestial gluon
operators close on the $S$-algebra.

Consequently, we obtain the following picture. Each tree-level $\mathrm{N}^{k}\text{-MHV}$
gluonic sector of $\mathcal{N}=4$ SYM is holographically dual to
a semiclassical system of $N$ minitwistor strings, with the integer
$N$ related to the MHV level $k$ by $N=2k+1$. Together with the
other evidence discussed in the final subsection, this observation
leads us to the following proposal.
\begin{conjecture*}
We conjecture the existence of a \emph{fully quantum-mechanical} topological
sigma-model whose worldsheet is the celestial supersphere $\mathcal{CS}_{s}$
and whose target is the minitwistor superspace $\mathbf{MT}_{s}$.
The Hilbert space of this theory decomposes into sectors labelled
by an integer $N$. In the appropriate semiclassical limit, each such
sector is described by the $N$-string system defined here. Moreover,
the full quantum sigma-model realises the celestial CFT dual to the
tree-level maximally supersymmetric YM theory.
\end{conjecture*}
%
The arguments developed in this section are more involved and abstract
than those in the preceding discussion. For pedagogical clarity we
proceed in two steps. In Subsection A we analyse in detail a single-string
system coupled to the holomorphic gauge theory on minitwistor superspace.
We show that this system is dual to the MHV gluonic sector of tree-level
$\mathcal{N}=4$ SYM. The goal of this first discussion is pedagogical
rather than fully rigorous: it introduces the principal physical ideas
and sets our notation and terminology.

In Subsection B we treat the $N$-string system in a more mathematically
rigorous language. There we demonstrate how the $N$-string semiclassical
dynamics reproduce the $\mathrm{N}^{k}\text{-MHV}$ sectors of gauge
theory on flat space.

\subsection{Single-String Model \label{subsec:Classical-Theory-1}}

In this subsection, we study a single minitwistor string formulated
as a topological sigma-model. Its worldsheet is the $\mathcal{N}=4$
celestial supersphere, $\mathcal{CS}_{s}$. Its target is the minitwistor
superspace, $\mathbf{MT}_{s}$. We define this sigma-model at the
semiclassical level only. It will likely develop anomalies upon quantisation.
A fully rigorous treatment, for example via the BV-BRST formalism,
lies beyond our scope. However, treating the sigma-model as a string
theory supplies useful intuition. In particular, it provides a dynamical
interpretation of the Wilson operators $\mathbb{W}[\mathscr{S}]$
introduced above.

We show in Subsection \ref{subsec:Semiclassical-Theory-1} that the
semiclassical partition function of a single minitwistor string, interacting
with a classical ``bath'' modelled by the holomorphic gauge theory,
reproduces the tree-level MHV leaf superamplitudes for gluons in $\mathcal{N}=4$
SYM. This match indicates that the field theory studied in Section
\ref{sec:Minitwistor-Wilson-Lines} can be seen as a minitwistor string-field
limit of the string theory proposed here.

We then construct vertex operators that describe the worldsheet interactions.
From those vertex operators we build celestial gluon operators. We
then compute the leading-trace semiclassical correlators of these
gluon operators and show that they reproduce the tree-level MHV leaf-gluon
amplitudes. This result supports our assertion that the semiclassical
single-string model realises the celestial CFT dual to the MHV sector
of $\mathcal{N}=4$ SYM at tree level.

\subsubsection{Formal Preliminaries\label{subsec:Formal-Preliminaries}}

We now introduce the essential concepts of the theory. To formalise
our sigma-model, we first define the configuration space of the minitwistor
string. Using the supersymmetric Hitchin correspondence (Section II),
we then define the classical moduli superspace that parameterises
the classically allowed configurations of the string. To describe
how the worldsheet $\mathcal{CS}$ is mapped into minitwistor superspace
$\mathbf{MT}_{s}$, we introduce the evaluation maps. These maps characterise
the minitwistor lines $\mathcal{L}\subset\mathbf{MT}_{s}$ through
the associated incidence relations.

Finally, employing the evaluation maps together with their incidence
relations, we construct the embedding maps of $\mathcal{CS}$ into
$\mathbf{MT}_{s}$. In the subsequent subsection, where we define
the dynamics of the model, the evaluation maps will serve as the fundamental
field variables of the theory.

\textcompwordmark{}

\paragraph*{Configuration Space.}

The minitwistor sigma-model is a theory of holomorphic rational maps:
\begin{equation}
\phi\colon\;\mathcal{CS}_{s}\;\longrightarrow\;\mathbf{MT}_{s}.\label{eq:-226}
\end{equation}
Algebraic curves in $\mathbf{CP}^{1}\times\mathbf{CP}^{1}$, and hence
in its supersymmetric extension $\mathbf{MT}_{s}$, are classified
by a bidegree $\beta=(d_{1},d_{2})$. For fixed $\beta$, let 
\begin{equation}
\mathrm{Hol}_{\beta}\big(\mathcal{CS}_{s};\,\mathbf{MT}_{s}\big)
\end{equation}
denote the functor of points parametrising holomorphic maps of bidegree
$\beta$. The automorphism group $\mathrm{Aut}\big(\mathcal{CS}_{s}\big)$
is the superconformal group of $\mathbf{CP}^{1|4}$. Since two maps
(\ref{eq:-226}) differing by a reparametrization of the worldsheet
define the same state, the \emph{physical configuration space} of
a single string of bidegree $\beta$ is:
\begin{equation}
\mathcal{E}_{\beta}\;\coloneqq\;\mathrm{Hol}_{\beta}\big(\mathcal{CS}_{s};\mathbf{MT}_{s}\big)/\mathrm{Aut}\big(\mathcal{CS}_{s}\big).
\end{equation}

Alternatively, each map $\phi\in\mathcal{E}_{\beta}$ defines an algebraic
one-cycle $[\phi(\mathcal{CS}_{s})]\in\mathrm{Z}_{1}(\mathbf{MT}_{s})$.
Two cycles that are rationally equivalent describe the same string
configuration. Thus, the physical configuration space can also be
modelled by the Chow group $\mathrm{A}_{1,\beta}(\mathbf{MT}_{s})$
and the natural forgetful functor: 
\begin{equation}
\mathcal{E}_{\beta}\longrightarrow\mathrm{A}_{1,\beta}\big(\mathbf{MT}_{s}\big),\qquad\phi\;\longmapsto\;\big[\phi(\mathcal{CS}_{s})\big].
\end{equation}
This algebraic viewpoint makes it easier to connect to our earlier
definition of the Wilson operator $\mathbb{W}[\mathscr{S}]$ for algebraic
one-cycles.

\textcompwordmark{}

\paragraph*{Classical String Configurations.}

Since our theory is defined only at the semiclassical level, we must
specify which string configurations are classically allowed and in
terms of which we define expectation values.

We introduced minitwistor strings to give a dynamical derivation of
the localisation theorem. The $\mathrm{N}^{k}\text{-MHV}$ gluon amplitudes
localise on a family of minitwistor lines $\{\mathcal{L}_{m}\}\subset\mathbf{MT}_{s}$.
In the previous section, we saw that these amplitudes come from Wilson
operators $\mathbb{W}[\mathscr{S}]$ supported on the cycle $\mathscr{S}=\sum_{m}[\mathcal{L}_{m}]$.
We also showed that gluon creation and annihilation operators attach
naturally to each $\mathcal{L}_{m}$. Hence we interpret each minitwistor
line $\mathcal{L}\subset\mathbf{MT}_{s}$ as a \emph{classical string
configuration}.

Each minitwistor line is an irreducible curve of bidegree $\beta=(1,1)$.
Translating this geometric fact into dynamics, we define the \emph{classical
configuration space} as $\mathcal{E}_{c}\coloneqq\mathcal{E}_{(1,1)}$,
and write $\mathcal{L}\in\mathcal{E}_{c}$ for any such line. By the
supersymmetric Hitchin correspondence (Section II), the hyperbolic
superspace $\mathbf{H}_{s}$ serves as the moduli space of these lines.
Hence we identify $\mathscr{M}_{c}=\mathbf{H}_{s}$ as the \emph{classical
moduli superspace} for minitwistor strings.

\textcompwordmark{}

\paragraph*{Evaluation Maps.}

A classical string configuration $\mathcal{L}=\mathcal{L}(X,\theta)\in\mathcal{E}_{c}$
is parameterised by a point $\mathsf{X}^{K}=\big(X_{A\dot{A}},\theta_{A}^{\alpha}\big)\in\mathscr{M}_{c}$
through a pair of evaluation maps. Let $\Lambda$ denote the Grassmann
algebra associated to the vector superspace $\mathbf{C}^{0|4}$ and
set $\Lambda[k]\coloneqq\bigwedge^{k}\mathbf{C}^{0|4}$. Let 
\begin{equation}
\Phi_{\dot{A}}\,\in\,\Gamma\big(\mathcal{L};\mathcal{O}(1)\!\oplus\!\mathcal{O}(1)\big),\qquad\varphi^{\alpha}\,\in\,\Lambda[1]\!\otimes\!\Gamma\big(\mathcal{L};\mathcal{O}(1)\big).\label{eq:-277}
\end{equation}
Choose homogeneous coordinates $[\lambda^{A}]$ on $\mathcal{L}$
induced by sections of $H^{0,0}(\mathcal{L};\mathcal{O}(1))$, and
let $\mathsf{W}^{I}$ denote the homogeneous coordinates on $\mathbf{MT}_{s}$
from Section II. The evaluation maps are then:
\begin{equation}
\Phi_{\dot{A}}\big(\lambda^{A}\big)\;=\;\lambda^{A}\,X_{A\dot{A}},\qquad\varphi^{\alpha}\big(\lambda^{A}\big)\;=\;\lambda^{A}\,\theta_{A}^{\alpha}.\label{eq:-227}
\end{equation}
Hence $\mathcal{L}$ appears as the locus of points $\mathsf{W}^{I}=\big(\lambda^{A},\mu_{\dot{A}},\psi^{\alpha}\big)\in\mathbf{MT}_{s}$
satisfying 
\begin{equation}
\mu_{\dot{A}}\;=\;\Phi_{\dot{A}}\big(\lambda^{A}\big),\qquad\psi^{\alpha}\;=\;\varphi^{\alpha}\big(\lambda^{A}\big).\label{eq:-249}
\end{equation}
The evaluation maps thus specify how the moduli $\mathsf{X}^{K}\in\mathscr{M}_{c}$
determine the configuration $\mathcal{L}\in\mathcal{E}_{c}$.

This construction suggests a simple approach to defining the sigma-model
action. We introduce a Lagrangian in which $\Phi_{\dot{A}}$ and $\varphi^{\alpha}$
play the role of fundamental fields. Its Euler-Lagrange equations
then reproduce the incidence relations on the minitwistor line. To
obtain a well-posed variational principle and to apply a saddle-point
approximation in the pathintegral, we seek a Lagrangian quadratic
in $\Phi_{\dot{A}}$ and $\varphi^{\alpha}$.

Now, the evaluation maps are homogeneous of degree one in $\lambda^{A}$:
\begin{equation}
\Phi_{\dot{A}}\big(t\,\lambda^{A}\big)\;=\;t\,\Phi_{\dot{A}}\big(\lambda^{A}\big),\quad\varphi^{\alpha}\big(t\,\lambda^{A}\big)\;=\;t\,\varphi^{\alpha}\big(\lambda^{A}\big),\quad\forall\,t\in\mathbf{C}^{*}.
\end{equation}
Any quadratic form in these fields thus has degree two in $\lambda^{A}$.
But such a form cannot be integrated against the holomorphic measure
\begin{equation}
D\lambda\;\coloneqq\;\left\langle \lambda\,d\lambda\right\rangle \,\in\,\Omega^{1,0}\big(\mathcal{L};\mathcal{O}(2)\big)
\end{equation}
because the integrand would carry excess homogeneity. A straightforward
solution consists in introducing a chart on $\mathcal{L}$ with coordinates
$\sigma^{B}$ that transform as
\begin{equation}
\lambda^{A}\;\mapsto\;t\,\lambda^{A}\quad\implies\quad\sigma^{B}\;\mapsto\;t^{-1}\,\sigma^{B}.
\end{equation}

In terms of the projective coordinates $\lambda^{A}$, the $\sigma$-coordinates
are defined by the transition map:
\begin{equation}
\lambda^{A}=\tau^{A}(\sigma^{B}).\label{eq:-248}
\end{equation}
This transition map specifies how the two patches on $\mathcal{CS}$,
given by the domains of the coordinate functions $\lambda^{A}$ and
$\sigma^{B}$, are glued together. The transition map is a holomorphic
section
\begin{equation}
\tau^{A}\;\in\;\Gamma\big(\mathbf{CP}^{1};\mathcal{O}(-1)\!\oplus\!\mathcal{O}(-1)\big).
\end{equation}
The explicit form of $\tau^{A}$ is constructed from the following
structures. Let $r=1,2$ index a frame field $e_{r}^{A}$ trivialising
the bundle $\mathcal{O}(1)\!\oplus\!\mathcal{O}(1)\to\mathcal{L}$.
Hence any undotted two-component spinor field $\boldsymbol{s}^{A}$
on $\mathcal{L}$ decomposes as
\begin{equation}
\boldsymbol{s}^{A}\;=\;s^{r}\,\epsilon_{r}^{A},\qquad s^{r}\in\mathscr{C}^{\infty}(\mathcal{L};\mathbf{C}).
\end{equation}
Next, let $S_{+}$ be the representation space of $\mathrm{SL}(2;\mathbf{C})$
realising the undotted van der Waerden spinors, and choose a basis
$\iota^{rA}$ for $S_{+}$. Define the component functions
\begin{equation}
s^{r}\big(\sigma^{B}\big)\;\coloneqq\;(-1)^{r}\,\langle\iota^{r},\sigma\rangle,
\end{equation}
and let $\Vert\boldsymbol{\sigma}\Vert\coloneqq-\prod_{r=1,2}s^{r}(\sigma^{B})$.
Accordingly, the transition map (\ref{eq:-248}) is given by
\begin{equation}
\tau^{A}\big(\sigma^{B}\big)\;\coloneqq\;\frac{1}{\Vert\boldsymbol{\sigma}\Vert}\,s^{r}\big(\sigma^{B}\big)\,\epsilon_{r}^{A}.\label{eq:-254}
\end{equation}
From this definition it follows that, under the rescaling $\lambda^{A}\mapsto t\,\lambda^{A}$,
the $\sigma$-coordinates transform as $\sigma^{B}\mapsto t^{-1}\,\sigma^{B}$,
as required.

Therefore, in terms of the $\sigma$-coordinates, the incidence relations
become:
\begin{equation}
\lambda^{A}\;=\;\tau^{A}\big(\sigma^{B}\big),\quad\mu_{\dot{A}}\;=\;\Pi_{\dot{A}}\big(\sigma^{B}\big),\quad\psi^{\alpha}\;=\;\kappa^{\alpha}\big(\sigma^{B}\big).\label{eq:-228}
\end{equation}
Here the new evaluation maps
\begin{equation}
\Pi_{\dot{A}}\in\Gamma\big(\mathcal{L};\mathcal{O}(-1)\!\oplus\!\mathcal{O}(-1)\big),\qquad\kappa^{\alpha}\in\Lambda[1]\!\otimes\!\Gamma\big(\mathcal{L};\mathcal{O}(-1)\big)
\end{equation}
are given by:
\begin{equation}
\Pi_{\dot{A}}\big(\sigma^{B}\big)\;\coloneqq\;\Phi_{\dot{A}}\big(\tau^{A}(\sigma^{B})\big),\qquad\kappa^{\alpha}\big(\sigma^{B}\big)\;\coloneqq\;\varphi^{\alpha}\big(\tau^{A}(\sigma^{B})\big).
\end{equation}
Substituting Eq. (\ref{eq:-249}) into these definitions yields the
explicit form: 
\begin{equation}
\Pi_{\dot{A}}\big(\sigma^{B}\big)\;=\;\frac{\epsilon_{1}^{A}X_{A\dot{A}}}{\langle\sigma,\iota^{2}\rangle}\,-\,\frac{\epsilon_{2}^{A}X_{A\dot{A}}}{\langle\sigma,\iota^{1}\rangle},\qquad\kappa^{\alpha}\big(\sigma^{B}\big)\;=\;\frac{\epsilon_{1}^{A}\theta_{A}^{\alpha}}{\langle\sigma,\iota^{2}\rangle}\,-\,\frac{\epsilon_{2}^{A}\theta_{A}^{\alpha}}{\langle\sigma,\iota^{1}\rangle}.\label{eq:-229}
\end{equation}
These maps are homogeneous of degree $-1$ in $\sigma^{B}$:
\begin{equation}
\Pi_{\dot{A}}\big(t\,\sigma^{B}\big)\;=\;t^{-1}\,\Pi_{\dot{A}}\big(\sigma^{B}\big),\qquad\kappa^{\alpha}\big(\sigma^{B}\big)\;=\;t^{-1}\,\kappa^{\alpha}\big(\sigma^{B}\big).
\end{equation}

In what follows, we call the maps $\Phi_{\dot{A}}$ and $\varphi^{\alpha}$,
which depend on the $\lambda$-coordinates, the \emph{evaluation maps
of the first kind}. Similarly, the maps $\Pi_{\dot{A}}$ and $\kappa^{\alpha}$,
which are parameterised by the $\sigma$-coordinates, will be called
\emph{evaluation maps of the second kind}.

\textcompwordmark{}
\begin{rem}
Embedding Maps.\label{rem:Embedding-Maps.}

The string configuration is given by an embedding map that sends the
worldsheet $\mathcal{CS}$ to a minitwistor line $\mathcal{L}(X,\theta)\subset\mathbf{MT}_{s}$.
We define this embedding map by
\begin{equation}
\mathsf{W}^{I}\big(\lambda^{A}\big)\;\coloneqq\;\big(\lambda^{A},\,\Phi_{\dot{A}}(\lambda^{A}),\,\varphi^{\alpha}(\lambda^{A})\big),\label{eq:-286}
\end{equation}
which is parameterised by the $\lambda$-coordinates. It is constructed
from the first-kind evaluation maps, and so we call the assignment
$\lambda^{A}\mapsto\mathsf{W}^{I}(\lambda^{A})$ the \emph{first-kind
parameterisation} of the minitwistor string.

Using the transition map $\lambda^{A}=\tau^{A}(\sigma^{B})$ defined
in Eq. (\ref{eq:-254}), the string configuration can be equivalently
characterised in terms of the second-kind evaluation maps via the
embedding:
\begin{equation}
\mathsf{Y}^{I}(\sigma^{B})\;\coloneqq\;\big(\tau^{A}(\sigma^{B}),\,\Pi_{\dot{A}}(\sigma^{B}),\,\kappa^{\alpha}(\sigma^{B})\big).\label{eq:-289}
\end{equation}
We refer to the assignment $\sigma^{B}\mapsto\mathsf{Y}^{I}(\sigma^{B})$
as the \emph{second-kind parameterisation} of the minitwistor string. 

The two parameterisations give equivalent representations of the celestial
sphere, and so every physical observable translates between them.
From the target-space viewpoint, the first-kind parameterisation is
more natural because the undotted spinor components of $\mathsf{W}^{I}(\lambda^{A})$
coincide with the projective coordinates $\lambda^{A}$ on $\mathcal{CS}$.
Hence we prefer to express the interaction action $\mathcal{U}$ in
the $\lambda$-coordinates.

By contrast, the second-kind parameterisation simplifies the action
$\mathcal{S}_{0}$ that governs the embedding dynamics (i.e., the
\emph{geometric sector}). The evaluation maps $\Pi_{\dot{A}}$ and
$\kappa^{\alpha}$ carry the homogeneity needed to pair with the holomorphic
measure $D\sigma$. This pairing produces a projectively invariant
top-form on $\mathcal{CS}$, from which one constructs a Lagrangian
quadratic in $\Pi_{\dot{A}}$ and $\kappa^{\alpha}$. Hence the second-kind
parameterisation is better suited for studying the semiclassical regime
of the model via a saddle-point approximation of the path integral.
\end{rem}


\subsubsection{Classical Theory: Geometric Sector}

Our aim is to formulate a variational principle in which the action
$\mathcal{S}$ depends on the fields $\Pi_{\dot{A}}$ and $\kappa^{\alpha}$
as independent variables. The resulting Euler-Lagrange equations must
reproduce the incidence relations in Eq. (\ref{eq:-228}). To this
end, we recast the definitions of the evaluation maps as differential
equations. We then require that these equations follow from $\delta\mathcal{S}=0$
and uniquely recover the explicit maps given in Eq. (\ref{eq:-229}).

\textcompwordmark{}

\paragraph*{Partial Differential Equations; Currents.}

The bosonic evaluation map of the second kind satisfies
\begin{equation}
\frac{1}{2\pi i}\;\overline{\partial}_{\sigma}\,\Pi_{\dot{A}}\big(\sigma^{B}\big)\;+\;\mathcal{J}_{\dot{A}}\big(\sigma^{B};X_{C\dot{C}}\big)\;=\;0.\label{eq:-230}
\end{equation}
Here the bosonic current
\begin{equation}
\mathcal{J}_{\dot{A}}\,\in\,\mathscr{D}'_{0,1}\big(\mathcal{L}(X,\theta);\,\mathcal{O}_{\mathbf{C}}(-1)\!\oplus\!\mathcal{O}_{\mathbf{C}}(-1)\big)
\end{equation}
is the distributional $(0,1)$-form on the minitwistor line defined
by
\begin{equation}
\mathcal{J}_{\dot{A}}\big(\sigma^{B};X_{C\dot{C}}\big)\;\coloneqq\;\overline{\delta}(\sigma\!\cdot\!\iota^{2})\;\epsilon_{1}^{A}X_{A\dot{A}}\;-\;\overline{\delta}(\sigma\!\cdot\,\iota^{1})\;\epsilon_{2}^{A}X_{A\dot{A}}.
\end{equation}

The fermionic evaluation map of the second kind obeys
\begin{equation}
\frac{1}{2\pi i}\;\overline{\partial}_{\sigma}\,\kappa^{\alpha}\big(\sigma^{B}\big)\;+\;\mathcal{K}^{\alpha}\big(\sigma^{B};\theta_{C}^{\gamma}\big)\;=\;0.\label{eq:-231}
\end{equation}
The fermionic current
\begin{equation}
\mathcal{K}^{\alpha}\;\in\;\mathscr{D}'_{0,1}\big(\mathcal{L}(X,\theta);\;\mathbf{C}^{0|4}\!\otimes\!\mathcal{O}_{\mathbf{C}}(-1)\big)
\end{equation}
is
\begin{equation}
\mathcal{K}^{\alpha}\big(\sigma^{B};\theta_{C}^{\gamma}\big)\;\coloneqq\;\overline{\delta}(\sigma\!\cdot\!\iota^{2})\;\epsilon_{1}^{A}\theta_{A}^{\alpha}\;-\;\overline{\delta}(\sigma\!\cdot\!\iota^{1})\;\epsilon_{2}^{A}\theta_{A}^{\alpha}.
\end{equation}

As with the evaluation maps $\Pi_{\dot{A}}$ and $\kappa^{\alpha}$,
both currents are homogeneous of degree $-1$ in the spinor coordinates
$\sigma^{B}$. Explicitly,
\begin{equation}
\mathcal{J}_{\dot{A}}\big(t\,\sigma^{B};X_{C\dot{C}}\big)\;=\;t^{-1}\,\mathcal{J}_{\dot{A}}\big(\sigma^{B};X_{C\dot{C}}\big),\quad\mathcal{K}^{\alpha}\big(t\,\sigma^{B};\theta_{C}^{\gamma}\big)\;=\;t^{-1}\,\mathcal{K}^{\alpha}\big(\sigma^{B};\theta_{C}^{\gamma}\big).
\end{equation}

The existence and uniqueness theorem for linear PDEs on compact Riemann
surfaces (see \citet[ Sec. 1.11]{forster1981compact}) guarantees
that Eqs. (\ref{eq:-230}) and (\ref{eq:-231}) uniquely determine
the evaluation maps. Therefore, any action whose equations of motion
reproduce these PDEs yields the incidence relations as its extremal
equations. Such an action provides a candidate for the classical dynamics
of the minitwistor sigma-model.

\textcompwordmark{}

\paragraph*{Bosonic Sector.}

To define the action for the bosonic sector, we use the monomials:
\begin{equation}
[\Pi\,\overline{\partial}_{\sigma}\,\Pi]\quad\text{and}\quad[\Pi\,\mathcal{J}]\;\in\;\Omega^{0,1}\big(\mathcal{L}(X,\theta);\mathcal{O}_{\mathbf{C}}(-2)\big),
\end{equation}
together with the holomorphic measure:
\begin{equation}
D\sigma\;\coloneqq\;\langle\sigma\,d\sigma\rangle\;\in\;\Omega^{1,0}\big(\mathcal{L}(X,\theta);\,\mathcal{O}_{\mathbf{C}}(2)\big).
\end{equation}
This yields the top-forms:
\begin{equation}
D\sigma\wedge[\Pi\,\overline{\partial}_{\sigma}\,\Pi]\quad\text{and}\quad D\sigma\wedge[\Pi\,\mathcal{J}]\;\in\;\Omega^{1,1}\big(\mathcal{L}(X,\theta)\big).
\end{equation}
We integrate these forms over the minitwistor line $\mathcal{L}(X,\theta)$.
The action functional then reads:
\begin{equation}
\mathcal{S}_{\,\Pi}(X,\theta)\;\coloneqq\;\frac{1}{b}\,\int_{\mathcal{L}(X,\theta)}\,D\sigma\wedge\left(\frac{1}{2\pi i}\;[\Pi\,\overline{\partial}_{\sigma}\,\Pi]\;+\;[\Pi\,\mathcal{J}]\right).\label{eq:-280}
\end{equation}
The parameter $b$ plays a role analogous to the Liouville coupling
in the semiclassical limit.

The integral in Eq. (\ref{eq:-280}) depends on the moduli $\mathsf{X}^{K}\in\mathscr{M}_{c}$
through the integration domain $\mathcal{L}(X,\theta)$ and the current
$\mathcal{J}_{\dot{A}}$, which itself depends on the bosonic projection
$X_{A\dot{A}}$. Varying $\mathcal{S}_{\Pi}$ yields the defining
PDE for the bosonic evaluation map (Eq. (\ref{eq:-230})).

\textcompwordmark{}

\paragraph*{Fermionic Sector.}

To construct the fermionic action, note that the field $\kappa^{\alpha}$
has Grassmann degree one, since it is a section of $\mathcal{L}(X,\theta)$
valued in the vector superspace $\mathbf{C}^{0|4}$. The action itself
must be a real number. Therefore, we can form a Lagrangian polynomial
in $\kappa^{\alpha}$ only by pairing it with another field of Grassmann
degree three and then performing a Berezin integral over the fermionic
directions.

Although the curve $\mathcal{L}(X,\theta)$ lies in the supersymmetric
manifold $\mathbf{MT}_{s}$, it remains bosonic: as a rational curve,
it is biholomorphic to the Riemann sphere, $\mathcal{L}(X,\theta)\cong\mathbf{CP}^{1}$.
To incorporate the full $\mathcal{N}=4$ fermionic structure, we extend
this curve to a minitwistor superline by adjoining four Grassmann
coordinates $\chi^{\alpha}$. We denote the resulting superspace by
$\mathcal{CS}_{s}(X,\theta)$, interpreting it as the embedding of
the celestial supersphere into $\mathbf{MT}_{s}$ as an irreducible
superline of bidegree $(1,1)$:
\begin{equation}
\mathcal{CS}_{s}(X,\theta)\;\cong\;\mathbf{CP}^{1|4}.
\end{equation}
We then combine the bosonic coordinates $\sigma^{A}$ and the fermionic
variables $\chi^{\alpha}$ into the supercoordinates $\mathsf{s}\coloneqq(\sigma^{B},\chi^{\beta})$,
and define the canonical Berezin-DeWitt volume form\footnote{We record a few remarks on integration over $\mathcal{CS}_{s}$. The
$\mathcal{N}=4$ celestial supersphere is the vector superbundle $\mathcal{CS}_{s}\cong\mathbf{CP}^{1}\times\mathbf{C}^{0|4}$
(see Ch. 12 of \citet{rogers2007supermanifolds}). Projective rescalings
act only on the bosonic coordinate $\sigma^{A}$, so that $\sigma^{A}\sim t\,\sigma^{A}$
for all $t\in\mathbf{C}^{*}$. Under this rescaling, the Berezin-DeWitt
superform $D^{1|4}\mathsf{s}=D\sigma\wedge d^{0|4}\chi$ transforms
as $D^{1|4}\mathsf{s}\mapsto t^{2}\,D^{1|4}\mathsf{s}$. To obtain
a projectively invariant top-form on $\mathcal{CS}_{s}$, pair $D^{1|4}\mathsf{s}$
with a $(0,1)$-form
\begin{equation}
\boldsymbol{w}\;\in\;\Omega^{0,1}\big(\mathcal{CS};\;\bigw^{4}\,\mathbf{C}^{0|4}\!\otimes\!\mathcal{O}_{\mathbf{C}}(-2)\big).
\end{equation}
Here $\mathcal{CS}\cong\mathbf{CP}^{1}$ denotes the bosonic base,
and the factor $\bigw^{4}\,\mathbf{C}^{0|4}$ denotes the fermionic
fibres. The form $\boldsymbol{w}$ has Grassmann degree $4$ and homogeneity
$-2$. Hence their wedge product, $D^{1|4}\mathsf{s}\wedge\boldsymbol{w}$,
is a Berezinian-valued top-form on $\mathcal{CS}_{s}$ (cf. $\S\,2.2$
of \citet{voronov1991geometric}). Its projective weight vanishes,
so it can be integrated over $\mathcal{CS}_{s}$.} on $\mathcal{CS}_{s}(X,\theta)$ by:
\begin{equation}
D^{1|4}\mathsf{s}\;\coloneqq\;D\sigma\,\wedge\,d^{0|4}\chi.
\end{equation}

We next introduce a Lagrange multiplier
\begin{equation}
e_{\alpha}\,\in\,\Omega^{0,0}\big(\mathcal{L}(X,\theta);\;\bigw^{3}\,\mathbf{C}^{0|4}\!\otimes\!\mathcal{O}_{\mathbf{C}}(-1)\big).
\end{equation}
We continue to take $\mathcal{L}(X,\theta)$ (and not $\mathcal{CS}_{s}(X,\theta)$)
as the base for this section because the fermionic directions appear
only in the fibre part valued in the exterior superalgebra $\bigw^{3}\,\mathbf{C}^{0|4}$.
We assume that $e_{\alpha}$ is homogeneous of degree $-1$ in the
spinor coordinates $\sigma^{B}$, namely
\begin{equation}
e_{\alpha}\big(t\,\sigma^{B},\chi^{\beta}\big)\;=\;t^{-1}\,e_{\alpha}\big(\sigma^{B},\chi^{\beta}\big).
\end{equation}

Consider the monomials
\begin{equation}
e_{\alpha}\wedge\overline{\partial}_{\sigma}\,\kappa^{\alpha}\quad\text{and}\quad e_{\alpha}\wedge\mathcal{K}^{\alpha}\;\in\;\Omega^{0,1}\big(\mathcal{L}(X,\theta);\,\bigw^{4}\,\mathbf{C}^{0|4}\!\otimes\!\mathcal{O}_{\mathbf{C}}(-2)\big).
\end{equation}
Taking the exterior product of these objects with the Berezin-DeWitt
measure on the superline produces the differential forms:
\begin{equation}
D^{1|4}\mathsf{s}\,\wedge\,e_{\alpha}\,\wedge\,\overline{\partial}_{\sigma}\,\kappa^{\alpha}\quad\text{and}\quad D^{1|4}\mathsf{s}\,\wedge\,e_{\alpha}\,\wedge\,\mathcal{K}^{\alpha}\;\in\;\Omega^{(1,1)|4}\big(\mathcal{CS}_{s}(X,\theta)\big).
\end{equation}
Now the base manifold is the full celestial supersphere $\mathcal{CS}_{s}(X,\theta)$.
These expressions are genuine differential forms valued in the Berezinian
of $\mathcal{CS}_{s}(X,\theta)$ and therefore are integrable.

Accordingly, we take the fermionic sector to be governed by the action:
\begin{equation}
\mathcal{S}_{\,\kappa,e}(X,\theta)\;=\;\frac{1}{b}\,\int_{\mathcal{CS}_{s}(X,\theta)}\;D^{1|4}\mathsf{s}\wedge\left(\frac{1}{2\pi i}\;e_{\alpha}\wedge\overline{\partial}_{\sigma}\,\kappa^{\alpha}\;+\;e_{\alpha}\wedge\mathcal{K}^{\alpha}\right).\label{eq:-281}
\end{equation}
The dependence of the action on the superspace coordinates $\mathsf{X}^{K}=\big(X_{A\dot{A}},\theta_{A}^{\alpha}\big)$
enters through the integration superdomain and through the current
$\mathcal{K}^{\alpha}$, which itself depends on the Grassmann-valued
spinors $\theta_{A}^{\alpha}$. Taking the variation of $\mathcal{S}_{\,\kappa,e}$
with respect to $\kappa^{\alpha}$ and $e_{\alpha}$ yields Eq. (\ref{eq:-231}),
which is the defining PDE for the fermionic incidence relations.

\textcompwordmark{}

\paragraph*{Geometric Sector.}

The evaluation maps of the second kind, $\Pi_{\dot{A}}$ and $\kappa^{\alpha}$,
together with the Grassmann-valued Lagrange multiplier $e_{\alpha}$,
constitute the dynamical variables of the \emph{geometric sector}.
This sector describes the embedding of the celestial supersphere $\mathcal{CS}_{s}$
as a minitwistor string $\mathcal{L}\subset\mathbf{MT}_{s}$. We therefore
collect the fundamental fields of the geometric sector into the multiplet:
\begin{equation}
\varDelta\;\coloneqq\;\big\{\,\Pi_{\dot{A}}(\sigma^{B}),\,\kappa^{\alpha}(\sigma^{B}),\,e_{\alpha}(\sigma^{B},\chi^{\beta})\,\big\}.
\end{equation}
Combining Eqs. (\ref{eq:-280}) and (\ref{eq:-281}), the \emph{geometric
action} reads:
\begin{equation}
\mathcal{S}_{0}[\Delta|X,\theta]\;\coloneqq\;\mathcal{S}_{\,\Pi}(X,\theta)\;+\;\mathcal{S}_{\,\kappa,e}(X,\theta).\label{eq:-238}
\end{equation}

To unify the bosonic and fermionic parts, we introduce a pair of conjugate
superfields and a supercurrent. First, define a vielbein $E_{\dot{A}}^{\;\;\alpha}$
on $\mathcal{CS}_{s}$, normalised by:
\begin{equation}
E_{\dot{A}}^{\;\;\alpha}E_{\;\;\beta}^{\dot{A}}\;=\;\delta_{\;\;\beta}^{\alpha}.
\end{equation}
Next, introduce the \emph{conjugate superfields}:
\begin{equation}
\Sigma_{\dot{A}},\;\Xi^{\dot{A}}\;\in\;\Omega^{0,0}\big(\mathcal{L}(X,\theta);\;\bigw\,\mathbf{C}^{0|4}\!\otimes\!\big(\mathcal{O}_{\mathbf{C}}(-1)\!\oplus\!\mathcal{O}_{\mathbf{C}}(-1)\big)\big)\label{eq:-232}
\end{equation}
with components:
\begin{equation}
\Sigma_{\dot{A}}\big(\sigma^{B},\chi^{\beta}\big)\;\coloneqq\;\chi^{1}\chi^{2}\,\Pi_{\dot{A}}\big(\sigma^{B}\big)\;+\;E_{\dot{A}}^{\;\;\alpha}\,e_{\alpha}\big(\sigma^{B},\chi^{\beta}\big),
\end{equation}
\begin{equation}
\Xi^{\dot{A}}\big(\sigma^{B},\chi^{\beta}\big)\;\coloneqq\;\chi^{3}\chi^{4}\,\Pi^{\dot{A}}\big(\sigma^{B}\big)\;+\;E_{\;\;\alpha}^{\dot{A}}\,\kappa^{\alpha}\big(\sigma^{B}\big).
\end{equation}
In the inclusion relation (\ref{eq:-232}), we treat the base manifold
as the line $\mathcal{L}(X,\theta)$ because the fermionic directions
live in fibres valued in the exterior superalgebra $\bigw\,\mathbf{C}^{0|4}$.
When we wedge the superfields with the measure $D^{1|4}\mathsf{s}$,
the resulting top-forms take values in the Berezinian of $\mathcal{CS}_{s}(X,\theta)$.
Berezin integration then gives:
\begin{equation}
\int\;d^{0|4}\chi\wedge[\Sigma\,\overline{\partial}_{\sigma}\,\Xi]\;=\;[\Pi\,\overline{\partial}_{\sigma}\,\Pi]\;+\;\int\;d^{0|4}\chi\wedge e_{\alpha}\wedge\overline{\partial}_{\sigma}\kappa^{\alpha}.\label{eq:-233}
\end{equation}

Next, define the \emph{supercurrent}:
\begin{equation}
|X,\theta]^{\dot{A}}\,\in\,\mathscr{D}'_{0,1}\big(\mathcal{L}(X,\theta);\;\bigw\,\mathbf{C}^{0|4}\!\otimes\!\big(\mathcal{O}_{\mathbf{C}}(-1)\!\oplus\!\mathcal{O}_{\mathbf{C}}(-1)\big)\big)
\end{equation}
by:
\begin{equation}
|X,\theta]^{\dot{A}}\;\coloneqq\;\chi^{3}\chi^{4}\,\mathcal{J}^{\dot{A}}\big(\sigma^{B};X_{C\dot{C}}\big)\;+\;E_{\;\;\alpha}^{\dot{A}}\,\mathcal{K}^{\alpha}\big(\sigma^{B};\theta_{C}^{\gamma}\big).
\end{equation}
Berezin integration then yields:
\begin{equation}
\int\;d^{0|4}\chi\wedge[\Sigma|X,\theta]\;=\;[\Pi\,\mathcal{J}]\;+\;\int\;d^{0|4}\chi\wedge e_{\alpha}\wedge\mathcal{K}^{\alpha}.\label{eq:-234}
\end{equation}

Combining Eqs. (\ref{eq:-233}) and (\ref{eq:-234}) gives the final
form of the geometric action:
\begin{equation}
\mathcal{S}_{0}[\varDelta|X,\theta]\;\coloneqq\;\frac{1}{b}\,\int_{\mathcal{CS}_{s}(X,\theta)}\,D^{1|4}\mathsf{s}\wedge\left(\frac{1}{2\pi i}\;[\Sigma\,\overline{\partial}_{\sigma}\,\Xi]\;+\;[\Sigma|X,\theta]\right).\label{eq:-235}
\end{equation}


\subsubsection{Classical Theory: Worldsheet CFT\label{subsec:Classical-Theory:-Worldsheet}}

To complete the classical theory, we now specify the auxiliary matter
system that defines the worldsheet CFT. In fully quantum-mechanical
models (e.g., twistor string theories), this matter system contributes
to the total central charge. It also helps cancel or otherwise tame
anomalies that arise on quantisation. 

In the semiclassical framework adopted here, the reason for introducing
the worldsheet CFT is phenomenological: it determines how the minitwistor
string couples to the background potential of the holomorphic gauge
theory on minitwistor superspace.

We consider two worldsheet fermions, $\rho$ and $\rho^{*}$, defined
on the celestial sphere $\mathcal{CS}$. Using the embedding maps
introduced above, we push these fermions forward to the minitwistor
line $\mathcal{L}\subset\mathbf{MT}_{s}$, which represents the classical
string configuration. On $\mathcal{L}$, the fermions couple minimally
to the background gauge potential $\boldsymbol{A}$ introduced in
the preceding section. We now formalise this physical picture.

\textcompwordmark{}

\paragraph*{Holomorphic Gauge Theory.}

Consider holomorphic gauge theory formulated on the complex vector
bundle $\pi\colon\mathbf{E}\to\mathbf{MT}_{s}$. Let $\mathbf{G}$
be a semisimple Lie group and denote by $\mathfrak{g}$ its complexified
Lie algebra. We assume the fibres of $\mathbf{E}$ are isomorphic
to $\mathfrak{g}$.

The classical vacuum of the gauge theory is represented by the canonical
holomorphic structure on $\mathbf{E}$ induced by the Dolbeault operator
$\overline{\partial}^{\mathbf{E}}$. Nontrivial physical configurations
correspond to pseudoholomorphic structures on $\mathbf{E}$, parameterised
by a partial connection. 

To define a $(0,1)$-connection form $\boldsymbol{A}$, we recall
two facts. First, since $\mathfrak{g}$ is semisimple, the adjoint
representation yields an isomorphism $\mathrm{ad}\colon\mathfrak{g}\to\mathrm{Der}_{\mathbf{C}}(\mathfrak{g})$.
Hence $\boldsymbol{A}$ may be taken to be $\mathfrak{g}$-valued.
Second, the Picard group of the bosonic minitwistor space $\mathbf{MT}$
satisfies $\mathrm{Pic}(\mathbf{MT})\cong\mathbf{Z}\!\oplus\!\mathbf{Z}$,
so any differential form valued in the natural homogeneous bundle
of $\mathbf{MT}$ is characterised by a bidegree $\beta=(\Delta_{1},\Delta_{2})$.

Combining these observations, the gauge potential on $\mathbf{E}$
can be parameterised by a $(0,1)$-connection form:
\begin{equation}
\boldsymbol{A}\;\in\;\Omega^{0,1}\big(\mathbf{MT}_{s};\;\mathcal{O}(\Delta_{1},\Delta_{2})\big)\!\otimes\!\mathfrak{g}.
\end{equation}


\paragraph*{Induced Potential.}

Let $\mathcal{L}=\mathcal{L}(X,\theta)$ be a minitwistor line representing
the classical configuration of the string associated to the moduli
$(X,\theta)\in\mathscr{M}_{c}$. If $\Delta_{1}+\Delta_{2}=0$, then
the pullback of $\boldsymbol{A}$ to $\mathcal{L}$ via the restriction
homomorphism satisfies
\begin{equation}
\boldsymbol{A}\big|_{\mathcal{L}}\,\in\,\Omega^{0,1}(\mathcal{L})\!\otimes\!\mathfrak{g}.
\end{equation}
Thus $\boldsymbol{A}|_{\mathcal{L}}$ is a genuine $(0,1)$-form on
the string $\mathcal{L}$, rather than a section of a nontrivial line
bundle. From now on, we restrict attention to parameterisations of
the gauge potential with bidegree obeying $\Delta_{1}+\Delta_{2}=0$,
and we define $\Delta\coloneqq\Delta_{1}$ as the \emph{conformal
weight} assigned to the background field $\boldsymbol{A}$. Under
these conditions, $\boldsymbol{A}|_{\mathcal{L}}$ is the gauge potential
induced on the string $\mathcal{L}$.

Using the first-kind parameterisation (Eq. (\ref{eq:-286})), the
restriction can be written as
\begin{equation}
\boldsymbol{A}\big|_{\mathcal{L}(X,\theta)}\big(\lambda^{A}\big)\;=\;\boldsymbol{A}\big(\mathsf{W}^{I}(\lambda^{A};X,\theta)\big).
\end{equation}
Physically, $\boldsymbol{A}|_{\mathcal{L}}$ is the gauge field seen
by the string propagating on $\mathbf{MT}_{s}$.

\textcompwordmark{}

\paragraph*{Celestial Fermions.}

To couple the string to the background gauge field, we introduce a
fermionic matter system on the worldsheet. We realise this system
by spinor fields supported on the minitwistor line $\mathcal{L}(X,\theta)$.

On a compact complex manifold $\mathbf{S}$, \citet[Prop. 3.2]{atiyah1971riemann}
proved that spin structures are in one-to-one correspondence with
isomorphism classes of holomorphic line bundles $\mathtt{L}$ satisfying
$\mathtt{L}^{2}\cong\mathtt{K}_{\mathbf{S}}$, where $\mathtt{K}_{\mathbf{S}}$
denotes the canonical line bundle of $\mathbf{S}$. By a slight abuse
of notation, we write a choice of such a ``square root'' simply as
$\sqrt{\mathtt{K}_{\mathbf{S}}}$. Using the theory of \citet{leites1980introduction},
this statement extends to compact complex supermanifolds; see also
\citet{giddings1988geometry}. Accordingly, let $\mathtt{K}\cong\mathcal{O}(-2)$
denote the canonical line bundle of the celestial supersphere $\mathcal{CS}_{s}$.

Now pull back $\mathbf{E}$ to the minitwistor line $\mathcal{L}$
via the restriction homomorphism and denote the restricted bundle
by $\mathtt{E}\coloneqq\mathbf{E}|_{\mathcal{L}}$. In the geometric
formulation of gauge theory, matter fields are represented by sections
of vector bundles associated to $\mathtt{E}$ on which the worldsheet
spinors are valued.

So, let $V$ be a complex vector space and let $\mathcal{R}\colon\mathfrak{g}\to\mathfrak{gl}_{\mathbf{C}}(V)$
be a representation. Define the left action $\phi$ of $\mathfrak{g}$
on $\mathtt{E}\!\times\!V$ by:
\begin{equation}
\phi\colon\;\mathfrak{g}\;\longrightarrow\;\mathrm{Aut}_{\mathbf{C}}(\mathtt{E}\!\times\!V),\qquad\phi_{g}(e,v)\,\coloneqq\,\big(\mathrm{ad}_{g}(e),\mathcal{R}_{g}(v)\big),
\end{equation}
for all $g\in\mathfrak{g}$ and $(e,v)\in\mathtt{E}\!\times\!V$.
Using this action, form the quotient
\begin{equation}
\mathtt{F}\coloneqq(E\!\times\!V)/\mathfrak{g},
\end{equation}
and define the surjection $\pi'\colon\mathtt{F}\to\mathcal{L}$ by
$\pi'(\mathfrak{g}\cdot(e,v))\coloneqq\pi(e)$, for all $e\in\mathtt{E}$
and $v\in V$. Then $\mathtt{F}\stackrel{\pi'}{\to}\mathcal{L}$ is
the vector bundle associated to $\mathtt{E}$ with typical fibre isomorphic
to $V$. We denote its dual bundle by $\mathtt{F}^{*}$.

Therefore, the matter content of the worldsheet CFT consists of a
pair of spinor fields
\begin{equation}
\rho\,\in\,\Gamma\big(\mathcal{L};\;\sqrt{\mathtt{K}}\!\otimes\!\mathtt{F}\big),\qquad\rho^{*}\!\in\!\Gamma\big(\mathcal{L};\;\sqrt{\mathtt{K}}\!\otimes\!\mathtt{F}^{*}\big).
\end{equation}


\paragraph*{Dynamics.}

Let $\boldsymbol{a}\in\Omega^{0,1}\big(\mathcal{L};\,\mathfrak{gl}_{\mathbf{C}}(V)\big)$
denote the induced gauge potential on $\mathcal{L}$ acting on the
representation space $V$ of the matter sector:
\begin{equation}
\boldsymbol{a}\coloneqq\mathcal{R}\circ\boldsymbol{A}\big|_{\mathcal{L}}.
\end{equation}
In the first-kind parameterisation of the string, $\boldsymbol{a}$
is given by
\begin{equation}
\boldsymbol{a}(\lambda^{A})\;=\;\mathcal{R}\big[\boldsymbol{A}\big(\mathsf{W}^{I}(\lambda^{A})\big)\big].\label{eq:-287}
\end{equation}
In addition, let $\langle\cdot|\cdot\rangle\colon\mathtt{F}^{*}\!\otimes\!\mathtt{F}\to\mathcal{O}_{\mathbf{CP}^{1}}$
be the canonical pairing. Therefore, we take the dynamics of the matter
CFT to be governed by the action:
\begin{equation}
\mathcal{S}_{\mathrm{CFT}}[\Delta,\rho,\rho^{*}|\boldsymbol{A};X,\theta]\;\coloneqq\;\int_{\mathcal{L}(X,\theta)}\;D\lambda\wedge\left\langle \rho^{*}\big|\big(\overline{\partial}_{\!\lambda}+\boldsymbol{a}(\lambda^{A})\big)\rho\right\rangle .\label{eq:-240}
\end{equation}
Here $\overline{\partial}_{\lambda}$ denotes the CR operator acting
on the $\lambda$-fibres. We have written the action in the $\lambda$-coordinates
because the first-kind parameterisation of the string is more natural
from the target-space perspective. In this parameterisation, the spinor
components of $\mathsf{W}^{I}(\lambda^{A})$ equal $\lambda^{A}$.
It is straightforward to reformulate the action in the $\sigma$-coordinates
using the second-kind parameterisation $\mathsf{Y}^{I}(\sigma^{B})$.

Therefore, the kinetic action is
\begin{equation}
\mathcal{S}_{\mathrm{K}}[\rho,\rho^{*}]\;=\;\int_{\mathcal{CS}}\;D\lambda\wedge\left\langle \rho^{*}\big|\overline{\partial}_{\!\lambda}\rho\right\rangle .
\end{equation}
Similarly, the interaction contribution is
\begin{equation}
\mathcal{U}[\Delta,\rho,\rho^{*}|\boldsymbol{A};X,\theta]\;=\;\int_{\mathcal{L}(X,\theta)}\;D\lambda\wedge\left\langle \rho^{*}\big|\boldsymbol{a}(\lambda^{A})\rho\right\rangle .
\end{equation}
Observe that the dependence on the evaluation maps contained in the
multiplet $\Delta$ enters through the embedding map $\mathsf{W}^{I}(\lambda^{A})$
used to define $\boldsymbol{a}(\lambda^{A})$ in Eq. (\ref{eq:-287}).

\subsubsection{Semiclassical Theory\label{subsec:Semiclassical-Theory-1}}

A semiclassical description applies when some degrees of freedom behave
classically while others require a quantum treatment. A familiar example
is molecular quantum mechanics. There, the centre of mass of the heavy
nuclei and the environmental degrees of freedom follow classical mechanics.
The lighter, faster electrons require a quantum description.

In our setting, the embedding of the celestial sphere into minitwistor
superspace plays the role of the classical centre-of-mass variables.
This embedding is defined by evaluation maps and their incidence relations.
Similarly, the gauge potential $\boldsymbol{A}$ corresponds to the
environmental degrees of freedom and is also treated as classical.
By contrast, the worldsheet fermions are intrinsically quantum, and
their worldsheet CFT couples minimally to the external classical ``bath.'' 

The aim of this subsection is to give a mathematical formulation of
this picture.

\textcompwordmark{}

\paragraph*{Notation.}

Since we employ the path-integral formalism to analyse the semiclassical
theory, we must distinguish field variables from classical solutions
unambiguously. We adopt the following convention: undecorated symbols
denote dynamical variables, while classical solutions carry a tilde.

For example, $\Pi_{\dot{A}}(\sigma^{B})$ and $\kappa^{\alpha}(\sigma^{B})$
denote the dynamical variables that define the second-kind evaluation
maps. The corresponding classical solutions are $\widetilde{\Pi}_{\dot{A}}(\sigma^{B};X,\theta)$
and $\tilde{\kappa}^{\alpha}(\sigma^{B};X,\theta)$. The embedding
map that describes the classical configuration $\mathcal{L}(X,\theta)$
in the $\lambda$-coordinates is given by\footnote{An analogous expression holds for the embedding map $\widetilde{\mathsf{Y}}^{I}$
in the $\sigma$-coordinates; see Eqs. (\ref{eq:-229}) and (\ref{eq:-289}).}:
\begin{equation}
\widetilde{\mathsf{W}}^{I}\big(\lambda^{A};X,\theta\big)=\big(\lambda^{A},\lambda^{A}X_{A\dot{A}},\lambda^{A}\theta_{A}^{\alpha}\big).
\end{equation}


\paragraph{Effective Action.}

We take the dynamics of the minitwistor string propagating on the
background gauge potential $\boldsymbol{A}$ to be governed by the
action:
\begin{equation}
\mathcal{S}_{\mathrm{I}}[\Delta,\rho,\rho^{*}|\boldsymbol{A};X,\theta]\;=\;\mathcal{S}_{0}[\Delta|X,\theta]+\mathcal{S}_{\mathrm{CFT}}[\Delta,\rho,\rho^{*}|\boldsymbol{A};X,\theta].
\end{equation}
Here we denote the action by the subscript $\mathrm{I}$, indicating
that the sigma-model interacts with the background field $\boldsymbol{A}$.
In the next subsection we compute correlators of vertex operators
after setting $\boldsymbol{A}=0$. This choice removes background
contributions and isolates the interactions that arise solely from
worldsheet insertions.

The onshell effective action describing the worldsheet fermions $\rho$
and $\rho^{*}$ interacting with a classical background is defined
by
\begin{equation}
\mathcal{I}[\rho,\rho^{*}|\boldsymbol{A};X,\theta]\;\coloneqq\;\mathcal{S}_{\mathrm{I}}[\Delta,\rho,\rho^{*}|\boldsymbol{A};X,\theta]\,\Big|_{\frac{\delta\mathcal{S}}{\delta\Delta}=0}.
\end{equation}

The effective action is obtained by substituting the classical solutions
that parameterise the string $\mathcal{L}(X,\theta)$ into $\mathcal{S}_{\mathrm{I}}$.
To write $\mathcal{I}$ explicitly, let
\begin{equation}
\widetilde{\boldsymbol{a}}\;\in\;\Omega^{0,1}\big(\mathcal{L}(X,\theta);\;\mathrm{GL}_{\mathbf{C}}(V)\big)
\end{equation}
be the induced potential evaluated at the classical solution:
\begin{equation}
\widetilde{\boldsymbol{a}}(\lambda^{A};X,\theta)\coloneqq\mathcal{R}\big[\boldsymbol{A}\big(\widetilde{\mathsf{W}}^{I}(\lambda^{A};X,\theta)\big)\big].
\end{equation}
Then the onshell effective action becomes:
\begin{equation}
\mathcal{I}[\rho,\rho^{*}|\boldsymbol{A};X,\theta]\;=\;\int_{\mathcal{CS}}\;D\lambda\wedge\left\langle \rho^{*}\big|\big(\overline{\partial}_{\!\lambda}+\widetilde{\boldsymbol{a}}(\lambda^{A};X,\theta)\big)\rho\right\rangle .
\end{equation}


\paragraph*{Saddle-point Approximation.}

The idea behind the saddle-point approximation is as follows. Consider
the semiclassical limit $b\to0$ of a path integral. The integral
runs over the second-kind evaluation maps $\Pi_{\dot{A}},\kappa^{\alpha}$
and the Lagrange multiplier $e_{\alpha}$, and is weighted by $\exp(-\mathcal{S}_{0})$.
In this limit the integral is dominated by the saddle point satisfying
$\delta\mathcal{S}_{0}=0$. 

At that saddle point, the classical equations of motion, which yield
the minitwistor incidence relations that define the line $\mathcal{L}(X,\theta)$,
are imposed on the observables appearing in the integrand. We now
show how this picture is implemented mathematically.

We denote the Feynman ``measure'' by
\begin{equation}
[d\Delta]\;\coloneqq\;[d\Pi\,d\kappa\,de],
\end{equation}
and define the normalisation factor
\begin{equation}
\mathcal{N}_{0}(X,\theta)\;\coloneqq\;\int\;[d\Delta]\;e^{-\mathcal{S}_{0}[\Delta|X,\theta]}.
\end{equation}

Let $\mathrm{F}[\mathsf{W}^{I}(\lambda^{A})]$ be a $c$-number functional
representing an observable that depends only on the string parameterisation.
For simplicity, we use the $\lambda$-coordinates because the first-kind
parameterisation $\lambda^{A}\mapsto\mathsf{W}^{I}(\lambda^{A})$
is more natural from the target-space perspective, as noted above.

Applying the saddle-point approximation to the Euclidean path integral
(see \citet[Ch. 5, Sec. 3]{zinn2021quantum}) and integrating over
$\Pi_{\dot{A}},\kappa^{\alpha}$ and $e_{\alpha}$, we obtain:
\begin{equation}
\lim_{b\to0^{+}}\;\frac{1}{\mathcal{N}_{0}(X,\theta)}\int\;[d\Delta]\;e^{-\mathcal{S}_{\mathrm{I}}[\Delta,\rho,\rho^{*}|\boldsymbol{A};X,\theta]}\;\mathrm{F}\big[\mathsf{W}^{I}(\lambda^{A})\big]\;=\;e^{-\mathcal{I}[\rho,\rho^{*}|\boldsymbol{A};X,\theta]}\;\mathrm{F}\big[\widetilde{\mathsf{W}}^{I}(\lambda^{A};X,\theta)\big].\label{eq:-237}
\end{equation}

Hence, Eq. (\ref{eq:-237}) formalises our intuition. In the limit
$b\to0$, the correlation functions are dominated by the equations
of motion. These equations impose the restriction homomorphism onto
the minitwistor line $\mathcal{L}(X,\theta)$, which describes the
string's classical configuration.

\textcompwordmark{}

\paragraph*{Correlation Functions.}

What is the physical meaning of the right-hand side of Eq. (\ref{eq:-237})?
It evaluates the observable $\mathrm{F}[\mathsf{W}^{I}(\lambda^{A})]$
on the classical string configuration represented by the minitwistor
line $\mathcal{L}(X,\theta)$. Note that the result is weighted by
the inverse of the exponentiated onshell effective action, $e^{-\mathcal{I}}$.
Hence, the semiclassical expectation value of $\mathrm{F}[\mathsf{W}^{I}(\lambda^{A})]$
is obtained by averaging the right-hand side of Eq. (\ref{eq:-237})
over all allowed classical string configurations.

However, because the action $\mathcal{S}_{0}$ is first-order in the
field variables $\Pi_{\dot{A}},\kappa^{\alpha}$, each point $(X,\theta)\in\mathscr{M}_{c}$,
which belongs to the classical moduli superspace of the string, completely
specifies a classical \emph{state}. Thus we may define
\begin{equation}
d\boldsymbol{v}[\rho,\rho^{*};X,\theta]\;\coloneqq\;e^{-\mathcal{I}[\rho,\rho^{*}|\boldsymbol{A};X,\theta]}\;D^{3|8}\mathsf{X}\,[d\rho\,d\rho^{*}]
\end{equation}
as a measure on the system's phase space $\varGamma_{\boldsymbol{A}}$.
The measure space $(\varGamma_{\boldsymbol{A}},d\boldsymbol{v})$
can then be identified with the \emph{semiclassical statistical ensemble}
of a minitwistor string interacting with the background gauge potential
$\boldsymbol{A}$.

It follows that the semiclassical expectation value of the observable
$\mathrm{F}[\mathsf{W}^{I}(\lambda^{A})]$ in the celestial CFT defined
by our minitwistor string is obtained by integrating the right-hand
side of Eq. (\ref{eq:-237}) over the fermions $\rho,\rho^{*}$ and
over the moduli superspace:
\begin{equation}
\lim_{b\rightarrow0}\left\langle \mathscr{F}[\mathsf{W}^{I}]\right\rangle _{\mathcal{CS}}^{\boldsymbol{A}}\;\coloneqq\;\frac{1}{\mathcal{N}_{\mathrm{CFT}}}\;\int_{\mathscr{M}_{c}}D^{3|8}\mathsf{X}\int\;[d\rho\,d\rho^{*}]\;e^{-\mathcal{I}[\rho,\rho^{*}|\boldsymbol{A};X,\theta]}\;\mathrm{F}\big[\widetilde{\mathsf{W}}^{I}(\lambda^{A};X,\theta)\big].
\end{equation}
Here, $\mathscr{F}[\mathsf{W}^{I}]$ denotes the quantum operator
corresponding to the classical observable $\mathrm{F}[\mathsf{W}^{I}]$.
The normalisation factor coming from the worldsheet fermions is
\begin{equation}
\mathcal{N}_{\mathrm{CFT}}\;\coloneqq\;\int\;[d\rho\,d\rho^{*}]\;e^{-\mathcal{S}_{\mathrm{K}}[\rho,\rho^{*}]}.
\end{equation}

The semiclassical correlation function of the worldsheet CFT is defined
by
\begin{equation}
\lim_{b\rightarrow0}\left\langle \mathscr{F}[\mathsf{W}^{I}]\right\rangle _{\mathrm{WS}(X,\theta)}^{\boldsymbol{A}}\;\coloneqq\;\frac{1}{\mathcal{N}(X,\theta)}\;\int\;[d\Delta\,d\rho\,d\rho^{*}]\;e^{-\mathcal{S}_{\mathrm{I}}[\Delta,\rho,\rho^{*}|\boldsymbol{A};X,\theta]}\;\mathrm{F}[\mathsf{W}^{I}(\lambda^{A})],\label{eq:-239}
\end{equation}
where the normalisation factor is
\begin{equation}
\mathcal{N}(X,\theta)\;\coloneqq\;\int\;[d\Delta\,d\rho\,d\rho^{*}]\;e^{-\mathcal{S}_{0}[\Delta|X,\theta]-\mathcal{S}_{\mathrm{K}}[\rho,\rho^{*}]}.
\end{equation}

When the saddle-point approximation is invoked, Eq. (\ref{eq:-239})
yields:
\begin{equation}
\lim_{b\rightarrow0}\left\langle \mathscr{F}[\mathsf{W}^{I}]\right\rangle _{\mathrm{WS}(X,\theta)}^{\boldsymbol{A}}\;=\;\;\frac{1}{\mathcal{N}_{\mathrm{CFT}}}\int\;[d\rho\,d\rho^{*}]\;e^{-\mathcal{I}[\rho,\rho^{*}|\boldsymbol{A};X,\theta]}\;\mathrm{F}\big[\widetilde{\mathsf{W}}^{I}(\lambda^{A};X,\theta)\big].\label{eq:-288}
\end{equation}

Therefore, the semiclassical correlation functions of the celestial
CFT induced by the minitwistor string theory are given by an integral
over the classical moduli superspace $\mathscr{M}_{c}$ of the worldsheet
CFT correlators:
\begin{equation}
\lim_{b\rightarrow0}\left\langle \mathscr{F}[\mathsf{W}^{I}]\right\rangle _{\mathcal{CS}}^{\boldsymbol{A}}\;=\;\lim_{b\rightarrow0}\int_{\mathscr{M}_{c}}D^{3|8}\mathsf{X}\;\left\langle \mathscr{F}[\mathsf{W}^{I}]\right\rangle _{\mathrm{WS}(X,\theta)}^{\boldsymbol{A}}.
\end{equation}
Substituting Eq. (\ref{eq:-239}) into this expression gives the full
semiclassical correlator of the celestial CFT:
\begin{equation}
\lim_{b\rightarrow0}\left\langle \mathscr{F}[\mathsf{W}^{I}]\right\rangle _{\mathcal{CS}}^{\boldsymbol{A}}\;\coloneqq\;\lim_{b\rightarrow0}\;\int_{\mathscr{M}_{c}}\frac{D^{3|8}\mathsf{X}}{\mathcal{N}(X,\theta)}\;\int\;[d\Delta\,d\rho\,d\rho^{*}]\;e^{-\mathcal{S}_{\mathrm{I}}[\Delta,\rho,\rho^{*}|\boldsymbol{A};X,\theta]}\;\mathrm{F}[\mathsf{W}^{I}(\lambda^{A})].
\end{equation}


\paragraph*{Partition Function.}

Finally, the semiclassical partition function of a minitwistor string
propagating on the classical background gauge potential $\boldsymbol{A}$
is defined by
\begin{equation}
\mathscr{Z}[\boldsymbol{A}]\;=\;\lim_{b\rightarrow0}\left\langle 1\right\rangle _{\mathcal{CS}}^{\boldsymbol{A}}.
\end{equation}
Using Eq. (\ref{eq:-239}) we obtain:
\begin{equation}
\mathscr{Z}[\boldsymbol{A}]\;\coloneqq\;\frac{1}{\mathcal{N}_{\mathrm{CFT}}}\;\int_{\mathscr{M}_{c}}D^{3|8}\mathsf{X}\int\;[d\rho\,d\rho^{*}]\;e^{-\mathcal{I}[\rho,\rho^{*}|\boldsymbol{A};X,\theta]}.\label{eq:-241}
\end{equation}
We will shortly demonstrate that $\mathscr{Z}[\mathbf{A}]$ generates
the tree-level MHV leaf-gluon amplitudes. This result will then motivate
a generalisation to multi-string configurations reproducing the $\mathrm{N}^{k}\text{-MHV}$
amplitudes.

\subsubsection{Partition Function and $\text{MHV}$ Amplitudes}

We now evaluate the semiclassical partition function $\mathscr{Z}[\boldsymbol{A}]$
by integrating over the worldsheet fermions $\rho$ and $\rho^{\dagger}$.
We use the chiral determinant method for this functional integral.
For an analytic discussion aimed at string theorists, see Section
7 of \citet{verlinde1987chiral}. For a geometric perspective, see
Section 3 of \citet{alvarez1987geometrical}.

Applying this method to the action $\mathcal{S}_{\mathrm{CFT}}$,
which couples the fermions to the background gauge potential $\boldsymbol{A}$,
we obtain:
\begin{equation}
\int\;[d\rho\,d\rho^{*}]\;e^{-\mathcal{S}_{\mathrm{CFT}}}\;=\;\mathsf{Tr}\log\big(\mathbb{I}_{\mathfrak{\,g}}+\boldsymbol{A}\,\overline{\partial}^{-1}\big)\big|_{\mathcal{L}(X,\theta)}.
\end{equation}
Substituting into Eq. (\ref{eq:-241}) yields:
\begin{equation}
\mathscr{Z}[\boldsymbol{A}]\;=\;\int_{\mathscr{M}_{c}}\;D^{3|8}\mathsf{X}\;\mathsf{Tr}\,\log\big(\mathbb{I}_{\mathfrak{\,g}}+\boldsymbol{A}\,\overline{\partial}^{-1}\big)\big|_{\mathcal{L}(X,\theta)}.\label{eq:-243}
\end{equation}

Next, we expand the integrand using Quillen's determinant line bundle.
As in Subsection 3.3 of \citet{mason2005twistor}, one finds\footnote{For the basic theory, see \citet{quillen1985determinants}. A string
theory perspective appears in \citet{freed1987determinant}. For a
hands-on review with computational examples, consult Subsection 6.3
of \citet{nair2005chern}.}:
\begin{equation}
\mathsf{Tr}\,\log\big(\mathbb{I}_{\mathfrak{g}}+\boldsymbol{A}\,\overline{\partial}^{-1}\big)\big|_{\mathcal{L}(X,\theta)}\;=\;\sum_{n\geq1}\;\frac{(-1)^{n-1}}{n}\;\mathsf{Tr}\underset{\mathbf{L}^{n}\,\,\,}{\int}\;\bigwedge_{i=1}^{n}\;\frac{D\lambda_{i}}{\lambda_{i}\cdot\lambda_{i+1}}\wedge\boldsymbol{A}\big|_{\mathcal{L}(X,\theta)}\big(\lambda_{i}^{A}\big),\label{eq:-242}
\end{equation}
where $\mathbf{L}^{n}\coloneqq\bigtimes^{n}\mathbf{CP}^{1}$. 

We now apply the $\mathcal{MT}$-transform to expand the gauge potential
$\boldsymbol{A}(\mathsf{W}^{I})$ in terms of the minitwistor superwavefunctions
$\Psi_{\Delta}$:
\begin{equation}
\boldsymbol{A}\big(\mathsf{W}^{I}\big)\;=\;\underset{\mathbf{MT}_{s}^{*}\,\,\,}{\int}\Psi_{\Delta}\big(\mathsf{W}^{I};\mathsf{Z}'{}^{I}\big)\,\widetilde{\alpha}^{\Delta,\mathsf{a}}\big(\mathsf{Z}'{}^{I}\big)\,\mathsf{T}^{\mathsf{a}}\wedge D^{2|4}\mathsf{Z}'.\label{eq:-304}
\end{equation}
Here, we adopt DeWitt notation for the conformal weight $\Delta$
(as defined in Section IV). The new mode coefficients relate to those
in the previous section by $\widetilde{\alpha}^{\Delta,\mathsf{a}}=2\pi i\,\alpha^{\Delta,\mathsf{a}}$.
This choice of normalisation makes it more convenient to insert into
the powerseries representation of Quillen's determinant.

Inserting the expansion of $\boldsymbol{A}$ into Eq. (\ref{eq:-242})
and rearranging the integrals via Fubini's theorem, we find:
\begin{align}
 & \mathsf{Tr}\,\log\big(\mathbb{I}_{\mathfrak{g}}+\boldsymbol{A}\,\overline{\partial}^{-1}\big)\big|_{\mathcal{L}(X,\theta)}\;=\;\sum_{n\geq1}\;\frac{(-1)^{n-1}}{n}\;\underset{\mathbf{X}_{n}^{*}\,\,\,}{\int}\;\bigwedge_{i=1}^{n}\,D^{2|4}\mathsf{Z}_{i}'\wedge\widetilde{\alpha}^{\Delta_{i},\mathsf{a}_{i}}\big(\mathsf{Z}_{i}'{}^{I}\big)\\
 & \qquad\mathsf{Tr}_{\mathfrak{g}}\,\underset{\mathbf{L}^{n}\,\,\,}{\int}\;\bigwedge_{j=1}^{n}\,\frac{D\sigma_{j}}{\lambda_{j}\cdot\lambda_{j+1}}\mathsf{T}^{\mathsf{a}_{j}}\wedge\Psi_{\Delta_{j}}\Big|_{\mathcal{L}(X,\theta)}\big(\lambda_{j}^{A};\mathsf{Z}_{j}'{}^{I}\big),\label{eq:-244}
\end{align}
where $\mathbf{X}_{n}^{*}\coloneqq\bigtimes^{n}\mathbf{MT}_{s}^{*}$.
Applying the celestial BMSW identity to this expansion leads to:
\begin{align}
 & \mathsf{Tr}\,\log\big(\mathbb{I}_{\mathfrak{g}}+\boldsymbol{A}\,\overline{\partial}^{-1}\big)\big|_{\mathcal{L}(X,\theta)}\\
 & =\;\sum_{n\geq1}\;\frac{(-1)^{n-1}}{n}\;\underset{\mathbf{X}_{n}^{*}\,\,\,}{\int}\;\mathsf{Tr}_{\mathfrak{g}}\,\bigwedge_{i=1}^{n}\,\frac{\mathcal{C}(\Delta_{i})}{\langle z_{i}'|X|\bar{z}_{i}']^{\Delta_{i}}}\,e^{i\langle z_{i}'|\theta\cdot\eta_{i}\rangle}\,\frac{\mathsf{T}^{\mathsf{a}_{i}}}{z_{i}'\cdot z_{i+1}'}\widetilde{\alpha}^{\Delta_{i},\mathsf{a}_{i}}\big(\mathsf{Z}_{i}'{}^{I}\big)\wedge D^{2|4}\mathsf{Z}_{i}'.
\end{align}
Substituting this result into Eq. (\ref{eq:-243}) yields the final
form of the semiclassical partition function:
\begin{equation}
\mathscr{Z}[\boldsymbol{A}]=\;\sum_{n\geq1}\;\frac{(-1)^{n-1}}{n}\;\underset{\mathbf{X}_{n}^{*}\,\,\,}{\int}\;\underset{\mathscr{M}_{c}\,\,\,}{\int}\;D^{3|8}\mathsf{X}\;\mathsf{Tr}_{\mathfrak{g}}\,\bigwedge_{i=1}^{n}\,\frac{\mathcal{C}(\Delta_{i})}{\langle z_{i}'|X|\bar{z}_{i}']^{\Delta_{i}}}\,e^{i\langle z_{i}'|\theta\cdot\eta_{i}\rangle}\,\frac{\mathsf{T}^{\mathsf{a}_{i}}}{z_{i}'\cdot z_{i+1}'}\widetilde{\alpha}^{\Delta_{i},\mathsf{a}_{i}}\big(\mathsf{Z}_{i}'{}^{I}\big)\wedge D^{2|4}\mathsf{Z}_{i}'.
\end{equation}


\paragraph*{MHV leaf amplitudes.}

Having obtained an explicit form of the semiclassical partition function
$\mathscr{Z}$, we now show that it generates leaf-gluon amplitudes
in MHV configurations.

Consider an $n$-gluon MHV scattering process with celestial scaling
dimensions $h_{i}$. Let $\{\mathsf{Z}_{i}^{I}\}\subset\mathbf{MT}_{s}^{*}$
denote the insertion points. We functionally differentiate with respect
to the mode functions $\widetilde{\alpha}^{2h_{i},\mathsf{a}_{i}}\big(\mathsf{Z}_{i}^{I}\big)$
and then set $\boldsymbol{A}=0$. This gives:
\begin{equation}
\prod_{i=1}^{n}\;\frac{\delta}{\delta\widetilde{\alpha}^{2h_{i},\mathsf{a}_{i}}\big(\mathsf{Z}_{i}^{I}\big)}\,\mathscr{Z}_{\text{sc}}\,\bigg|_{\boldsymbol{A}=0}\;=\;\frac{(-1)^{n-1}}{n}\,\mathcal{M}_{n}^{\mathsf{a}_{1}\dots\mathsf{a}_{n}}\big(\mathsf{Z}_{i}^{I}\big).
\end{equation}
Hence, $\mathscr{Z}$ serves as the generating functional for MHV
leaf-gluon amplitudes in $\mathcal{N}=4$ SYM at tree-level.

\textcompwordmark{}

\paragraph*{Conclusion.}

We showed that the semiclassical partition function $\mathscr{Z}[\boldsymbol{A}]$
of the minitwistor sigma-model, coupled to a background gauge potential
$\boldsymbol{A}$, serves as a generating functional for the tree-level
MHV leaf-gluon amplitudes. The derivation proceeded by expanding $\mathscr{Z}[\boldsymbol{A}]$
in the Fourier modes $\tilde{\alpha}^{\Delta,\mathsf{a}}$ that parameterise
the classical configuration of $\boldsymbol{A}$; by functionally
differentiating with respect to those modes; and by finally evaluating
the result at $\boldsymbol{A}=0$.

This result confirmed the physical expectation that the holomorphic
gauge field theory on minitwistor superspace (which reproduces the
leaf-gluon amplitudes as minitwistor Wilson lines) is the string-field
limit of the semiclassical string theory considered here.

\subsubsection{Vertex Operators\label{subsec:Vertex-Operators-1}}

The final step in our presentation of the semiclassical system with
a single minitwistor string is the construction of the vertex operators
$\mathcal{V}_{\Delta}^{\mathsf{a}}$. We set the background gauge
potential $\boldsymbol{A}=0$ to isolate interactions that arise solely
from worldsheet insertions. In this trivial background, the leading-trace
(large-$N_{c}$) semiclassical celestial correlators of $\mathcal{V}_{\Delta}^{\mathsf{a}}$
reproduce the tree-level MHV leaf amplitudes for gluons. Therefore,
the semiclassical minitwistor string theory provides a holographic
dual to the tree-level MHV gluonic sector of maximally supersymmetric
Yang-Mills theory.

In Subsection \ref{subsec:Vertex-Operators}, we present a more detailed
discussion of vertex operators in minitwistor string theory and of
their algebraic structure. We show that the celestial gluon operators
close on the $S$-algebra, which is a necessary condition for any
candidate celestial CFT dual to flat-space gauge theory. The aim of
the present section is to introduce the essential concepts involved
in the construction of the operator $\mathcal{V}_{\Delta}^{\mathsf{a}}$
in the simpler setting of a single-string system, emphasising physical
intuition rather than formal completeness.

\textcompwordmark{}

\paragraph*{Motivation.}

To motivate the physics, we rewrite the worldsheet CFT action $\mathcal{S}_{\mathrm{CFT}}$
in component form. For this purpose, let $N_{c}\coloneqq\mathrm{dim}_{\mathbf{C}}(\mathfrak{g})$
denote the complex dimension of the gauge Lie algebra, and let $r,s=1,\dots,N_{c}$
index a coordinate basis of the representation space $V$. Introduce
a frame field $(e_{r})$ trivialising the vector bundle $\mathtt{F}\to\mathcal{L}$
and denote by $(e_{r}^{*})$ the dual frame trivialising $\mathtt{F}^{*}\to\mathcal{L}$. 

The worldsheet fermions $\rho$ and $\rho^{*}$ decompose in these
frames as
\begin{equation}
\rho=\rho^{r}\!\otimes\!e_{r},\qquad\rho^{*}=\bar{\rho}^{r}\!\otimes\!e_{r}^{*},\label{eq:-300}
\end{equation}
where the coefficients $\rho^{r},\bar{\rho}^{r}$ are $(0,1)$-forms
on $\mathcal{L}$.

Let $\{\mathsf{T}^{\mathsf{a}}\}$ be a basis of the Lie algebra $\mathfrak{g}$;
the normalisation of this basis will be specified below. The background
gauge potential $\boldsymbol{A}$ on $\mathbf{E}\to\mathbf{MT}_{s}$
decomposes as
\begin{equation}
\boldsymbol{A}\big(\mathsf{W}^{I}\big)=A^{\mathsf{a}}\big(\mathsf{W}^{I}\big)\!\otimes\!\mathsf{T}^{\mathsf{a}},
\end{equation}
with coefficients $A^{\mathsf{a}}\in\Omega^{0,1}(\mathbf{MT}_{s})$.

Next, recall that $\mathcal{R}\colon\mathfrak{g}\to\mathrm{GL}_{\mathbf{C}}(V)$
is the representation of the gauge algebra on $V$, the space in which
the matter fields $\rho$ and $\rho^{*}$ take values. The induced
potential $\boldsymbol{a}\in\Omega^{0,1}(\mathcal{L};\mathrm{GL}_{\mathbf{C}}(V))$
on the line $\mathcal{L}$, acting on the matter sector, is therefore:
\begin{equation}
\boldsymbol{a}(\lambda^{A})\;=\;A^{\mathsf{a}}\big(\mathsf{W}^{I}(\lambda^{A})\big)\!\otimes\!\mathcal{R}[\mathsf{T}^{\mathsf{a}}],\label{eq:-301}
\end{equation}
where $\{\mathcal{R}[\mathsf{T}^{\mathsf{a}_{i}}]\}$ yields a basis
for $\mathrm{GL}_{\mathbf{C}}(V)$. Finally, decomposing $\mathcal{R}[\mathsf{T}^{\mathsf{a}}]$
in the frames $(e_{r})$ and $(e_{r}^{*})$ gives the matrix elements
\begin{equation}
\mathrm{T}_{rs}^{\mathsf{a}}\;\coloneqq\;\langle e_{r}^{*}|\mathcal{R}[\mathsf{T}^{\mathsf{a}}]e_{s}\rangle,\label{eq:-302}
\end{equation}
which are the components of the Lie-algebra generators in the chosen
representation and frame.

Accordingly, substituting the decompositions given by Eqs. (\ref{eq:-300}),
(\ref{eq:-301}) and (\ref{eq:-302}) into the action $\mathcal{S}_{\mathrm{CFT}}$,
we obtain:
\begin{equation}
\mathcal{S}_{\mathrm{CFT}}[\Delta,\rho,\rho^{*}|\boldsymbol{A};X,\theta]\;=\;\int_{\mathcal{L}(X,\theta)}\;D\lambda\wedge\bar{\rho}^{r}\big(\delta_{rs}\,\overline{\partial}_{\!\lambda}+A^{\mathsf{a}}\big(\mathsf{W}^{I}(\lambda^{A})\big)\,\mathrm{T}_{rs}^{\mathsf{a}}\big)\rho^{s}.\label{eq:-303}
\end{equation}
From the second term in Eq. (\ref{eq:-303}), we identify the \emph{classical
worldsheet current}
\begin{equation}
\jmath^{\mathsf{a}}\,\in\,\Gamma\big(\mathcal{L};\,\mathcal{O}(-2)\!\otimes\!\mathfrak{g}\big),\qquad\jmath^{\mathsf{a}}\;\coloneqq\;\bar{\rho}^{r}\,\mathrm{T}_{rs}^{\mathsf{a}}\,\rho^{s}.\label{eq:-291}
\end{equation}
Consequently, the action contribution arising from the coupling of
the matter fields to the gauge potential may be written as
\begin{equation}
\mathcal{U}[A^{\mathsf{a}}]\;=\;\int_{\mathcal{CS}}\;D\lambda\wedge\mathrm{A}^{\mathsf{a}}\big(\mathsf{W}^{I}(\lambda^{A})\big)\,\jmath^{\mathsf{a}}(\lambda^{A}).
\end{equation}

This observation motivates the definition of \emph{minitwistor-string
vertex operators} supported on the celestial sphere $\mathcal{CS}$.
For any $(0,1)$-form
\begin{equation}
\phi\;\in\;\Omega^{0,1}\big(\mathbf{MT}_{s};\;\mathcal{O}(\Delta,-\Delta)\big),
\end{equation}
we define
\begin{equation}
\mathcal{V}^{\mathsf{a}}[\phi]\;\coloneqq\;\int_{\mathcal{CS}}\;D\lambda\wedge\phi\big(\mathsf{W}^{I}(\lambda^{A})\big)\,\jmath^{\mathsf{a}}(\lambda^{A}).\label{eq:-290}
\end{equation}

Observe that the integral in Eq. (\ref{eq:-290}) is well-defined
because, by pulling back the form $\phi$ to the string $\mathcal{L}$
via the restriction homomorphism, one obtains the projectively invariant
$(0,1)$-form $\phi|_{\mathcal{L}}\in\Omega^{0,1}(\mathcal{L})$ given
by:
\begin{equation}
\phi\big|_{\mathcal{L}(X,\theta)}\big(\lambda^{A}\big)\;=\;\phi\big(\mathsf{W}^{I}(\lambda^{A};X,\theta)\big).
\end{equation}
Wedging this form with the holomorphic measure $D\lambda$ and contracting
with the current $\jmath^{\mathsf{a}}(\lambda^{A})$ produce a $\mathfrak{g}$-valued
top form on $\mathcal{L}$.

Recalling the decomposition of the background field $\boldsymbol{A}$
in terms of the family $\{\Psi_{\Delta}\}$ of $p=0$ minitwistor
superwavefunctions (see Eq. (\ref{eq:-304})), we propose the following
candidate vertex operators that generate leaf-gluon amplitudes:
\begin{equation}
\mathcal{V}_{\Delta}^{\mathsf{a}}\big(\mathsf{Z}^{I}\big)\;=\;\int_{\mathcal{CS}}\;D\lambda\wedge\Psi_{\Delta}\big(\mathsf{W}^{I}(\lambda^{A});\mathsf{Z}^{I}\big)\,\jmath^{\mathsf{a}}(\lambda^{A}).\label{eq:-294}
\end{equation}


\paragraph*{Gauge Group.}

For concreteness, we take the gauge group $\mathbf{G}=SO(N_{c})$,
where $N_{c}$ denotes the number of colours of the gauge theory.
We impose the standard gauge-theory normalisations on the generators
$\{\mathsf{T}^{\mathsf{a}}\}$:
\begin{equation}
\mathsf{Tr}\big(\mathsf{T}^{\mathsf{a}}\mathsf{T}^{\mathsf{b}}\big)=2\,\delta^{\mathsf{ab}},\qquad[\mathsf{T}^{\mathsf{a}},\mathsf{T}^{\mathsf{b}}]=if^{\mathsf{abc}}\mathsf{T}^{\mathsf{c}}.
\end{equation}

The matter content of the worldsheet CFT is taken to be $N_{c}$ independent
real fermions $\rho^{r}$. These fermions transform in the vector
representation of $\mathfrak{g}=\mathfrak{so}(N_{c})$. We choose
the representation space used to construct the associated vector bundle
$\mathtt{F}$ to be $V=\mathfrak{so}(N_{c})$ and we take the homomorphism
$\mathcal{R}$ to be the adjoint representation.

The reason for this choice of gauge group and representation is practical:
in the large-$N_{c}$ limit, one eliminates unwanted multi-trace contributions
that arise from current algebra correlators. 

\textcompwordmark{}

\paragraph*{WZNW Current Algebra.}

On an open neighbourhood $\mathscr{U}\subset\mathcal{CS}$ such that
$\lambda^{1}(z)\neq0$ for all $z\in\mathscr{U}$, let $\lambda\coloneqq\lambda^{2}/\lambda^{1}$
denote the affine coordinate. In the operator formalism, the quantum
fields $\hat{\rho}^{r}$ that represent the worldsheet fermions obey
the fundamental OPEs:
\begin{equation}
\hat{\rho}^{r}(\lambda)\,\hat{\rho}^{s}(\lambda')\;\sim\;\frac{\delta^{rs}}{\lambda-\lambda'}.
\end{equation}

Let $J^{\mathsf{a}}$ be the quantum operators representing the worldsheet
currents. By the correspondence principle and by the form of the classical
currents $\jmath^{\mathsf{a}}$ defined in Eq. (\ref{eq:-291}), $J^{\mathsf{a}}$
must be proportional to the normally ordered\footnote{Following \citet[Sec. 6]{francesco1997conformal}, we define normal
ordering as follows. Let $\widehat{O}_{1}(\lambda)$ and $\widehat{O}_{2}(\lambda)$
be a pair of field operators belonging to the worldsheet CFT. Let
$\mathscr{C}(\lambda)$ be a small contour centred at $\lambda$.
The normally ordered operator product of $\widehat{O}_{1}$ and $\widehat{O}_{2}$,
evaluated at $\lambda$, is the quantum observable defined by:
\begin{equation}
\big(\widehat{O}_{1}\widehat{O}_{2}\big)(\lambda)\;\coloneqq\;\oint_{\mathscr{C}(\lambda)}\widehat{O}_{1}(\sigma)\,\boldsymbol{k}(\sigma;\lambda)\,\widehat{O}_{2}(\lambda).
\end{equation}
Here the Green differential $\boldsymbol{k}(\sigma;\lambda)$ is the
Cauchy kernel:
\begin{equation}
\boldsymbol{k}(\sigma;\lambda)\;\coloneqq\;\frac{d\sigma}{2\pi i}\,\frac{1}{\sigma-\lambda}.
\end{equation}
} bilinear $\jmath^{\intercal}\mathrm{T}\jmath$:
\begin{equation}
J^{\mathsf{a}}(\lambda)\;=\;\beta\,\big(\hat{\rho}^{r}\,\mathrm{T}_{rs}^{\mathsf{a}}\,\hat{\rho}^{s}\big)(\lambda).
\end{equation}
Invoking Wick's theorem, we find that the worldsheet currents satisfy
the OPEs:
\begin{equation}
J^{\mathsf{a}}(\lambda)\,J^{\mathsf{b}}(\lambda')\;\sim\;2\beta^{2}\,\frac{\mathsf{Tr}\big(\mathsf{T}^{\mathsf{a}}\mathsf{T}^{\mathsf{b}}\big)}{(\lambda-\lambda')^{2}}+2\beta\,\frac{if^{\mathsf{abc}}\,J^{\mathsf{c}}(\lambda')}{\lambda-\lambda'}.\label{eq:-275}
\end{equation}
Consistency of this OPE with the Ward identity requires $2\beta=1$.
Hence the set $\{J^{\mathsf{a}}\}$ generates the level-one $SO(N_{c})$
WZNW current algebra on the celestial sphere $\mathcal{CS}$.

\textcompwordmark{}

\paragraph*{WZNW Correlator.}

The final ingredient of the worldsheet CFT we require is the correlator.
Let $\mathbb{G}[J^{\mathsf{a}}]$ be an observable that is polynomial
in the worldsheet currents $J^{\mathsf{a}}$, and denote by $\mathrm{G}[\jmath^{\mathsf{a}}]$
the corresponding $c$-number functional. Because we introduced the
semiclassical string using the path-integral formalism, it is convenient
to express the correlator as a functional integral.

Take the functional measure to be given by:
\begin{equation}
[d\rho]\;\coloneqq\;\prod_{r=1}^{N_{c}}\;[d\rho^{r}].
\end{equation}
Since the multiplet $\{\rho^{r}\}$ consists of free fermions, the
action is purely kinetic:
\begin{equation}
\mathcal{S}_{\mathrm{K}}[\rho^{r}]\;=\;\int_{\mathcal{CS}}\;D\lambda\wedge\rho^{r}\,\overline{\partial}_{\!\lambda}\,\rho^{r}.
\end{equation}
Hence we define the correlator of $\mathbb{G}$ by: 
\begin{equation}
\langle\mathbb{G}[J^{\mathsf{a}}]\rangle_{\mathrm{WZNW}}\;\coloneqq\;\frac{1}{\mathcal{N}_{\rho}}\;\int\;[d\rho]\;e^{-\mathcal{S}_{\mathrm{K}}[\rho^{r}]}\,\mathrm{G}[\jmath^{\mathsf{a}}],\label{eq:-293}
\end{equation}
where the normalisation constant is:
\begin{equation}
\mathcal{N}_{\rho}\;\coloneqq\;\int\;[d\rho]\;e^{-\mathcal{S}_{\mathrm{K}}[\rho^{r}]}.
\end{equation}


\paragraph*{Semiclassical Celestial Correlator.}

Recall that our goal in this subsection is to isolate the interactions
arising from worldsheet insertions of vertex operators. To that end,
we set the background gauge potential to $\boldsymbol{A}=0$. The
action of the minitwistor sigma-model then reduces to:
\begin{equation}
\mathcal{S}[\Delta,\rho^{r}|X,\theta]\;=\;\mathcal{S}_{0}[\Delta|X,\theta]+\mathcal{S}_{\mathrm{K}}[\rho^{r}].
\end{equation}

Let $\mathbb{F}=\mathbb{F}[\mathsf{W}^{I};J^{\mathsf{a}}]$ be an
observable that depends on the string parameterisation $\mathsf{W}^{I}(\lambda^{A})$
and on the worldsheet currents $J^{\mathsf{a}}$. We assume that $\mathbb{F}$
is polynomial in the currents $J^{\mathsf{a}}$. Denote by $\mathrm{F}[\mathsf{W}^{I};\jmath^{\mathsf{a}}]$
the corresponding classical functional. The semiclassical celestial
correlator of $\mathbb{F}$ is:
\begin{equation}
\lim_{b\rightarrow0}\,\langle\mathbb{F}\rangle_{\mathcal{CS}}\;=\;\lim_{b\rightarrow0}\,\int_{\mathscr{M}_{c}}\frac{D^{3|8}\mathsf{X}}{\mathcal{N}_{0}(X,\theta)}\;\int\;[d\Delta\,d\rho]\;e^{-\mathcal{S}[\Delta,\rho^{r}|X,\theta]}\,\mathrm{F}[\mathsf{W}^{I}(\lambda^{A});\jmath^{\mathsf{a}}].
\end{equation}

Applying the saddle-point approximation to the $\Delta$-integral
yields the reduced expression:
\begin{equation}
\lim_{b\rightarrow0}\;\langle\mathbb{F}\rangle\;=\;\frac{1}{\mathcal{N}_{\rho}}\int_{\mathscr{M}_{c}}\;D^{3|8}\mathsf{X}\;\int\;[d\rho]\;e^{-\mathcal{S}_{\mathrm{K}}[\rho^{r}]}\;\mathrm{F}[\widetilde{\mathsf{W}}^{I}(\lambda^{A};X,\theta);\jmath^{\mathsf{a}}].\label{eq:-292}
\end{equation}
Here $\widetilde{\mathsf{W}}^{I}(\lambda^{A};X,\theta)$ denotes the
classical solution of the minitwistor sigma-model equations of motion;
it is given by the evaluation maps of the first kind expressed in
the $\lambda$-coordinates. The map $\widetilde{\mathsf{W}}^{I}$
describes the embedding of the worldsheet as the minitwistor line
$\mathcal{L}(X,\theta)\subset\mathbf{MT}_{s}$ corresponding to the
moduli $(X,\theta)\in\mathscr{M}_{c}$.

To simplify Eq. (\ref{eq:-292}), we recast it as an integral of the
WZNW correlator $\langle\dots\rangle_{\mathrm{WZNW}}$ defined in
Eq. (\ref{eq:-293}). Define the restriction of the observable $\mathbb{F}$
to the minitwistor line $\mathcal{L}(X,\theta)$ by
\begin{equation}
\mathbb{F}\big|_{\mathcal{L}(X,\theta)}[J^{\mathsf{a}}]\;\coloneqq\;\mathbb{F}[\widetilde{\mathsf{W}}^{I}(\lambda^{A};X,\theta);J^{\mathsf{a}}].
\end{equation}
We regard $\mathbb{F}|_{\mathcal{L}}$ in two ways. First, it is a
quantum operator that is polynomial in the worldsheet current $J^{\mathsf{a}}$.
Second, it is the classical functional obtained by evaluating $\mathbb{F}$
on the string parameterisation $\widetilde{\mathsf{W}}^{I}(\lambda^{A};X,\theta)$. 

Accordingly, the semiclassical limit of the correlator becomes the
moduli-space integral:
\begin{equation}
\lim_{b\rightarrow0}\;\langle\mathbb{F}[\mathsf{W}^{I};J^{\mathsf{a}}]\rangle\;=\;\int_{\mathscr{M}_{C}}\;D^{3|8}\mathsf{X}\;\big\langle\mathbb{F}\big|_{\mathcal{L}(X,\theta)}[J^{\mathsf{a}}]\big\rangle_{\mathrm{WZNW}}.\label{eq:-295}
\end{equation}


\paragraph*{Tree-level MHV Amplitudes.}

With these preparations, we show that the large-$N_{c}$ semiclassical
limit of celestial correlators of the vertex operators $\mathcal{V}_{\Delta}^{\mathsf{a}}$
reproduces the tree-level MHV leaf-gluon amplitudes.

Let $\mathsf{z}_{i}=(z_{i},\bar{z}_{i},\eta_{i}^{\alpha})\in\mathcal{CS}_{s}$
denote the $i$-th gluon insertion point on the $\mathcal{N}=4$ celestial
supersphere. Recall that the dual minitwistor superspace $\mathbf{MT}_{s}^{*}$
may be regarded as a covering space of $\mathcal{CS}_{s}$. Hence,
for the $i$-th insertion $\mathsf{z}_{i}$, we may choose a representative
\begin{equation}
\mathsf{Z}_{i}^{I}=\big(z_{i}^{A},\bar{z}_{i\dot{A}},\eta_{i}^{\alpha}\big)\in\mathbf{MT}_{s}^{*}.
\end{equation}
To specify the $i$-th gluon state in the celestial CFT, let $\Delta_{i}$
denote its conformal weight. The scaling dimension $h_{i}$ of the
$i$-th gluon is related to the conformal weight and to the helicity
by $2h_{i}+|\eta_{i}|=\Delta_{i}$, where $|\eta_{i}|$ denotes the
expectation value of the helicity operator.

Now, consider the $n$-point correlation function:
\begin{equation}
C_{n}^{\mathsf{a}_{1}\dots\mathsf{a}_{n}}\big(\mathsf{Z}_{i}^{I};\Delta_{i}\big)\;\coloneqq\;\lim_{N\rightarrow\infty}\lim_{b\rightarrow0}\;\left\langle \prod_{i=1}^{n}\mathcal{V}_{2h_{i}}^{\mathsf{a}_{i}}\big(\mathsf{Z}_{i}^{I}\big)\right\rangle _{\mathcal{CS}}.\label{eq:-298}
\end{equation}
Substitute Eq. (\ref{eq:-294}), which defines the minitwistor-string
vertex operator $\mathcal{V}_{\Delta}^{\mathsf{a}}$, into the correlator
above. Pull the superwavefunctions $\Psi_{\Delta}$ outside the WZNW
correlator and reorganise the integrals. One obtains:
\begin{equation}
C_{n}^{\mathsf{a}_{1}\dots\mathsf{a}_{n}}=\lim_{N\rightarrow\infty}\underset{\mathscr{M}_{c}\,\,\,\,\,\,\,}{\int}\;D^{3|8}\mathsf{X}\;\underset{\mathbf{L}_{n}\,\,\,}{\int}\;\bigwedge_{i=1}^{n}\;D\lambda_{i}\wedge\Psi_{2h_{i}}\big(\widetilde{\mathsf{W}}^{I}(\lambda_{i}^{A};X,\theta);\mathsf{Z}_{i}^{I}\big)\quad\left\langle \prod_{j=1}^{n}\;J^{\mathsf{a}_{j}}(\lambda_{j}^{A})\right\rangle _{\mathrm{WZNW}}.\label{eq:-305}
\end{equation}
Composing the superwavefunction with the onshell evaluation map $\widetilde{\mathsf{W}}^{I}$
yields the pullback of $\Psi_{2h_{i}}$ to the classical string configuration
$\mathcal{L}(X,\theta)$. So,
\begin{equation}
\Psi_{2h_{i}}\big|_{\mathcal{L}(X,\theta)}\big(\lambda_{i}^{A};\mathsf{Z}_{i}^{I}\big)\;=\;\Psi_{2h_{i}}\big(\widetilde{\mathsf{W}}^{I}(\lambda_{i}^{A};X,\theta);\mathsf{Z}_{i}^{I}\big).\label{eq:-296}
\end{equation}
We now use the observation of \citet{nair1988current} that, in the
leading-trace (large-$N_{c}$) limit, the current-algebra correlator
produces the Parke-Taylor factor:
\begin{equation}
\left\langle \prod_{i=1}^{n}\;J^{\mathsf{a}_{i}}(\lambda_{i}^{A})\right\rangle _{\mathrm{WZNW}}\;\sim\;\mathsf{Tr}\;\prod_{i=1}^{n}\;\frac{\mathsf{T}^{\mathsf{a}_{i}}}{\lambda_{i}\cdot\lambda_{i+1}}\qquad(N_{c}\rightarrow\infty),\label{eq:-297}
\end{equation}
where the product is taken cyclically and $\lambda_{i}\cdot\lambda_{i+1}$
denotes the natural spinor contraction. 

Substituting the pullback (Eq. (\ref{eq:-296})) and the Parke-Taylor
factor (Eq. (\ref{eq:-297})) into the expression for $C_{n}$ (Eq.
(\ref{eq:-305})) gives the compact form: 
\begin{equation}
C_{n}^{\mathsf{a}_{1}\dots\mathsf{a}_{n}}=\underset{\mathscr{M}_{c}\,\,\,\,\,\,\,}{\int}\;D^{3|8}\mathsf{X}\;\mathsf{Tr}\;\underset{\mathbf{L}_{n}\,\,\,}{\int}\;\bigwedge_{i=1}^{n}\;\frac{D\lambda_{i}}{\lambda_{i}\cdot\lambda_{i+1}}\mathsf{T}^{\mathsf{a}_{i}}\wedge\Psi_{2h_{i}}\big|_{\mathcal{L}(X,\theta)}\big(\lambda_{i}^{A};\mathsf{Z}_{i}^{I}\big).
\end{equation}
Applying the celestial BMSW identity, we obtain:
\begin{equation}
C_{n}^{\mathsf{a}_{1}\dots\mathsf{a}_{n}}\big(\mathsf{Z}_{i}^{I};\Delta_{i}\big)\;=\;\underset{\mathscr{M}_{c}\,\,\,\,\,\,\,}{\int}\;D^{3|8}\mathsf{X}\;\mathsf{Tr}\;\bigwedge_{i=1}^{n}\;\frac{\mathcal{C}(2h_{i})}{\langle z_{i}|X|\bar{z}_{i}]^{2h_{i}}}\,e^{i\langle z_{i}|\theta\cdot\eta_{i}\rangle}\,\frac{\mathsf{T}^{\mathsf{a}_{i}}}{z_{i}\cdot z_{i+1}}.\label{eq:-299}
\end{equation}
We recognise this expression as the tree-level MHV leaf-gluon superamplitude
for $n$ gluons, $\mathcal{M}_{n}^{\mathsf{a}_{1}\dots\mathsf{a}_{n}}\big(\mathsf{Z}_{i}^{I}\big)$.

In our semiclassical minitwistor-string model for the celestial CFT,
the \emph{celestial gluon operator} with conformal weight $\Delta$
and helicity state $\eta^{\alpha}$ is defined by:
\begin{equation}
\mathcal{G}_{\Delta}^{\eta,\mathsf{a}}(z,\bar{z})\coloneqq\mathcal{V}_{\Delta-|\eta|}^{\mathsf{a}}\big(z^{A},\bar{z}_{\dot{A}},\eta^{\alpha}\big).
\end{equation}
Thus Eq. (\ref{eq:-299}) can be written as:
\begin{equation}
\lim_{N_{c}\rightarrow\infty}\lim_{b\rightarrow0}\;\left\langle \prod_{i=1}^{n}\mathcal{G}_{\Delta_{i}}^{\eta_{i},\mathsf{a}_{i}}(z_{i},\bar{z}_{i})\right\rangle _{\mathcal{CS}}\;=\;\mathcal{M}_{n}^{\mathsf{a}_{1}\dots\mathsf{a}_{n}}\big(\mathsf{Z}_{i}^{I}\big).
\end{equation}


\paragraph*{Comment.}

Combining this conclusion with the localisation theorem gives the
physical motivation to generalise the model to a many-body system
of $N$ minitwistor strings. In the next subsection, we propose this
many-body system as the celestial CFT dual to the tree-level $\mathrm{N}^{k}\text{-MHV}$
gluonic subsector whenever $N=2k+1$. In Subsection \ref{subsec:Vertex-Operators},
we also show that the gluon operators close on the $S$-algebra that
any celestial CFT dual to flat-space gauge theory must satisfy. 

\subsection{An $N$-String System Coupled to a Gauge Background\label{subsec:An--String-System}}

We now present the central result of this work. In Subsections \ref{subsec:Classical-Theory}
and \ref{subsec:Worldsheet-CFT}, we generalise the semiclassical
system studied above, which consisted of a single minitwistor string,
to a many-body system of $N$ minitwistor strings interacting with
a background gauge potential on $\mathbf{MT}_{s}$. 

Our primary aim is to show in Subsection \ref{subsec:Semiclassical-Theory}
that the semiclassical partition function of this $N$-string system
serves as a generating functional for the tree-level leaf-gluon amplitudes
in the $\mathrm{N}^{k}\text{-MHV}$ sector of $\mathcal{N}=4$ SYM,
with $N=2k+1$. This realises the interpretation of the holomorphic
gauge theory on $\mathbf{MT}_{s}$ as a ``minitwistor string field
theory'' in the semiclassical regime. Hence we obtain a dynamical
formulation of the localisation theorem: the minitwistor lines on
which the amplitudes localise are identified with the images of the
$N$ strings in the system.

Next, in Subsection \ref{subsec:Vertex-Operators}, we analyse the
model's vertex operators and use them to construct celestial gluon
operators. We show that the leading-trace semiclassical correlators
of these gluon operators reproduce the tree-level $\mathrm{N}^{k}\text{-MHV}$
leaf-gluon amplitudes.

Finally, we establish that the OPEs of the gluon operators close on
the $S$-algebra of the celestial CFT. This confirms that, in the
semiclassical regime, the minitwistor string theory realises the algebraic
structure required of \emph{any} proposed holographic dual to the
$\mathrm{N}^{k}\text{-MHV}$ gluonic sector of $\mathcal{N}=4$ SYM
at tree-level.

\subsubsection{Classical Theory\label{subsec:Classical-Theory}}

We now construct a many-body system of $N$ minitwistor strings coupled
to a background gauge potential. Its correlation functions reproduce
the tree-level leaf amplitudes for gluons in the semiclassical regime.
As a preparation, we briefly review the geometric interpretation of
the minitwistor amplitudes obtained in the previous sections.

Consider scattering of $n$ gluons in $\mathcal{N}=4$ SYM theory.
Fix an integer $k$ with $1\leq k\leq n-1$ and set $N=2k+1$. Assume
the external gluons form an $\mathrm{N}^{k}\text{-MHV}$ configuration,
and label the next-to-MHV gluons by $\ell=1,\dots,k$. 

\textcompwordmark{}

\paragraph*{Geometric Formulation.}

By the localisation theorem of Section III, the minitwistor amplitude
localises on a family of minitwistor lines $\{\mathcal{L}_{m}\}\subset\mathbf{MT}_{s}$,
where $m=1,\dots,N$ indexes each line. The \emph{moduli superspace}
of the configuration $\{\mathcal{L}_{m}\}$ is
\begin{equation}
\mathscr{M}_{N}\;\coloneqq\;\mathbf{H}_{s}\times\mathcal{P}_{1}\times\mathcal{P}_{2}\times\dots\times\mathcal{P}_{k}.
\end{equation}
Here $\mathbf{H}_{s}$ denotes the complexified $(3|8)$-dimensional
anti-de Sitter superspace, and $\mathcal{P}_{\ell}$ is the parameter
space for the $\ell$-th next-to-MHV gluon. In split signature, all
components of the twistor and minitwistor momenta are real, so $\mathcal{P}_{\ell}\cong\mathbf{R}^{8|4}$.
We now perform an analytic continuation to complex parameter spaces,
so $\mathcal{P}_{\ell}\cong\mathbf{C}^{8|4}$.

Next we recall how the moduli superspace $\mathscr{M}_{N}$ parameterises
the geometry of the line family $\{\mathcal{L}_{m}\}$. This review
will clarify how the $\mathrm{N}^{k}\text{-MHV}$ leaf-gluon amplitudes
arise from correlators of $N$-string configurations. 

Each parameter space $\mathcal{P}_{\ell}$ carries global coordinates:
\begin{equation}
\tau_{\ell}^{M}\;=\;\big(u_{\ell}^{A},\,v_{\ell}^{B},\,\widetilde{u}_{\ell\dot{A}},\,\widetilde{v}_{\ell\dot{B}},\,\chi_{\ell}^{\alpha}\big).
\end{equation}
Hence the full moduli superspace is charted by:
\begin{equation}
\gamma^{Q}\;=\;\big(\mathsf{X}^{K},\tau_{1}^{M_{1}},\tau_{2}^{M_{2}},\dots,\tau_{k}^{M_{k}}\big),
\end{equation}
where $\mathsf{X}^{K}=(X_{A\dot{A}},\theta_{A}^{\alpha})$ are the
standard coordinates on $\mathbf{H}_{s}$. We orient $\mathscr{M}_{N}$
using the Berezin-DeWitt form:
\begin{equation}
\mathcal{D}\gamma\;\coloneqq\;D^{3|8}\mathsf{X}\wedge d^{8|4}\tau_{1}\wedge d^{8|4}\tau_{2}\wedge\dots\wedge d^{8|4}\tau_{k}.
\end{equation}

To each line $\mathcal{L}_{m}$ we assign \emph{moduli functions}:
\begin{equation}
\mathcal{Q}_{m}^{A\dot{A}}=\mathcal{Q}_{m}^{A\dot{A}}\big(\tau_{1}^{M_{1}},\tau_{2}^{M_{2}},\,\dots\tau_{k}^{M_{k}}\big),\qquad q_{m}^{\alpha\dot{A}}=q_{m}^{\alpha\dot{A}}\big(\tau_{1}^{M_{1}},\tau_{2}^{M_{2}},\,\dots,\tau_{k}^{M_{k}}\big).
\end{equation}
These live on the product superspace $\bigtimes_{\ell=1}^{k}\mathcal{P}_{\ell}$.
Combining them with $(X_{A\dot{A}},\theta_{A}^{\alpha})$ yields the
\emph{characteristic functions} of the $m$-th line:
\begin{equation}
Y_{m}^{A\dot{A}}\big(\gamma^{Q}\big)\;=\;X^{A\dot{A}}+\mathcal{Q}_{m}^{A\dot{A}}\big(\tau_{\ell}^{M}\big),\qquad\xi_{m}^{\alpha A}\big(\gamma^{Q}\big)\;=\;\theta^{\alpha A}+q_{m}^{\alpha A}\big(\tau_{\ell}^{M}\big).
\end{equation}
We then define the \emph{evaluation maps} on each line\footnote{Recall that $\Lambda$ is the Grassmann algebra associated to the
vector superspace $\mathbf{C}^{0|4}$ and $\Lambda[k]\coloneqq\bigwedge^{k}\mathbf{C}^{0|4}$.}:
\begin{equation}
\Phi_{m\dot{A}}\in\Gamma\big(\mathcal{L}_{m}(\gamma^{Q});\mathcal{O}(1)\!\oplus\!\mathcal{O}(1)\big),\quad\varphi_{m}^{\alpha}\in\Lambda[1]\!\otimes\!\Gamma\big(\mathcal{L}_{m}(\gamma^{Q});\mathcal{O}(1)\big),\label{eq:-306}
\end{equation}
given by:
\begin{equation}
\Phi_{m\dot{A}}\big(\lambda^{A};\gamma^{Q}\big)\;\coloneqq\;\lambda^{A}\,Y_{mA\dot{A}}\big(\gamma^{Q}\big)\qquad\varphi_{m}^{\;\;\alpha}\big(\lambda^{A};\gamma^{Q}\big)\;\coloneqq\;\lambda^{A}\,\xi_{mA}^{\alpha}\big(\gamma^{Q}\big).\label{eq:-261}
\end{equation}
Here $[\lambda^{A}]$ are projective coordinates on $\mathcal{L}_{m}(\gamma^{Q})$. 

Finally, the minitwistor line $\mathcal{L}_{m}(\gamma^{Q})$ appears
as the locus of points $\mathsf{W}_{m}^{I}=(\lambda_{m}^{A},\mu_{m\dot{A}},\psi_{m}^{\alpha})$
satisfying the \emph{incidence relations}:
\begin{equation}
\lambda_{m}^{A}=\lambda^{A},\quad\mu_{m\dot{A}}=\Phi_{m\dot{A}}\big(\lambda^{B};\gamma^{Q}\big),\quad\psi_{m}^{\alpha}=\varphi_{m}^{\;\;\alpha}\big(\lambda^{B};\gamma^{Q}\big).\label{eq:-262}
\end{equation}


\paragraph*{Dynamical Formulation.}

As before, we derive the dynamics for an $N$-string system by posing
a variational problem. Its solutions reproduce the evaluation maps
(Eq. (\ref{eq:-261})) that encode the incidence relations (Eq. (\ref{eq:-261})).
To apply the saddle-point approximation in Feynman's path integral,
we choose a Lagrangian polynomial in the fields.

Recall that $\Phi_{m\dot{A}}$ and $\varphi_{m}^{\alpha}$ are homogeneous
of degree one in $\lambda^{A}$. Any quadratic polynomial in these
fields then has degree two. Such a term cannot combine with the holomorphic
measure $D\lambda=\langle\lambda d\lambda\rangle$ to form a projectively
invariant top-form. So, our strategy is to rewrite the evaluation
maps in terms of coordinates on $\mathcal{L}_{m}(\gamma^{Q})$ that
carry weight $-1$ under the rescaling $\lambda^{A}\mapsto t\,\lambda^{A}$.

We chart $\mathcal{L}_{m}$ by the coordinate functions $\sigma^{B}$,
which are related to $\lambda^{A}$ via the transition map $\lambda^{A}=\tau^{A}(\sigma^{B})$
defined in Eq. (\ref{eq:-254}). In $\sigma$-coordinates we define
the evaluation maps
\begin{equation}
\Pi_{m\dot{A}}\;\in\;\Gamma\big(\mathcal{L}_{m}(\gamma^{Q});\mathcal{O}(-1)\!\oplus\!\mathcal{O}(-1)\big),\quad\kappa_{m}^{\alpha}\;\in\;\Lambda[1]\!\otimes\!\Gamma\big(\mathcal{L}_{m}(\gamma^{Q});\mathcal{O}(-1)\big)\label{eq:-307}
\end{equation}
specified by the relations:
\begin{equation}
\Pi_{m\dot{A}}\big(\sigma^{B};\gamma^{Q}\big)\;\coloneqq\;\Phi_{m\dot{A}}\big(\tau^{A}(\sigma^{B});\gamma^{Q}\big)\;=\;\frac{\epsilon_{1}^{A}\,Y_{mA\dot{A}}\big(\gamma^{Q}\big)}{\langle\sigma,\iota^{2}\rangle}-\frac{\epsilon_{2}^{A}\,Y_{mA\dot{A}}\big(\gamma^{Q}\big)}{\langle\sigma,\iota^{1}\rangle},\label{eq:-255}
\end{equation}
\begin{equation}
\kappa_{m}^{\alpha}\big(\sigma^{B};\gamma^{Q}\big)\;\coloneqq\;\varphi_{m}^{\alpha}\big(\tau^{A}(\sigma^{B});\gamma^{Q}\big)\;=\;\frac{\epsilon_{1}^{A}\,\xi_{mA}^{\alpha}\big(\gamma^{Q}\big)}{\langle\sigma,\iota^{2}\rangle}-\frac{\epsilon_{2}^{A}\,\xi_{mA}^{\alpha}\big(\gamma^{Q}\big)}{\langle\sigma,\iota^{1}\rangle}.\label{eq:-256}
\end{equation}
Parameterising by the coordinates $\sigma^{B}$, the incidence relations
for the minitwistor line $\mathcal{L}_{m}\big(\gamma^{Q}\big)$ read:
\begin{equation}
\lambda_{m}^{A}\;=\;\tau^{A}\big(\sigma^{B}\big),\quad\mu_{m\dot{A}}\;=\;\Pi_{m\dot{A}}\big(\sigma^{B};\gamma^{Q}\big),\quad\psi_{m}^{\alpha}\;=\;\kappa_{m}^{\alpha}\big(\sigma^{B};\gamma^{Q}\big).
\end{equation}

As above, the new evaluation maps arise as the unique solutions of
a system of differential equations. To formulate this system, we define
the currents
\begin{equation}
\mathcal{J}_{m\dot{A}}\in\mathscr{D}'_{0,1}\big(\mathcal{L}_{m}(\gamma^{Q});\mathcal{O}(-1)\!\otimes\!\mathcal{O}(-1)\big),\quad\mathcal{K}_{m}^{\alpha}\in\Lambda[1]\!\otimes\!\mathscr{D}'_{0,1}\big(\mathcal{L}_{m}(\gamma^{Q});\mathcal{O}(-1)\big)
\end{equation}
with local form:
\begin{equation}
\mathcal{J}_{m\dot{A}}\big(\sigma^{B};\gamma^{Q}\big)\;\coloneqq\;\overline{\delta}\big(\sigma\!\cdot\!\iota^{2}\big)\;\epsilon_{1}^{A}\,Y_{mA\dot{A}}\big(\gamma^{Q}\big)\,-\,\overline{\delta}\big(\sigma\!\cdot\!\iota^{1}\big)\;\epsilon_{2}^{A}\,Y_{mA\dot{A}}\big(\gamma^{Q}\big)
\end{equation}
\begin{equation}
\mathcal{K}_{m}^{\alpha}\big(\sigma^{B};\gamma^{Q}\big)\;\coloneqq\;\overline{\delta}\big(\sigma\!\cdot\!\iota^{2}\big)\;\epsilon_{1}^{A}\,\xi_{mA}^{\alpha}\big(\gamma^{Q}\big)\,-\,\overline{\delta}\big(\sigma\!\cdot\!\iota^{1}\big)\;\epsilon_{2}^{A}\,\xi_{mA}^{\alpha}\big(\gamma^{Q}\big).
\end{equation}
Hence the evaluation maps satisfy the linear PDEs:
\begin{equation}
\frac{1}{2\pi i}\;\overline{\partial}_{\sigma}\,\Pi_{m\dot{A}}\big(\sigma^{B};\gamma^{Q}\big)\;+\;\mathcal{J}_{m\dot{A}}\big(\sigma^{B};\gamma^{Q}\big)\;=\;0,\label{eq:-257}
\end{equation}
\begin{equation}
\frac{1}{2\pi i}\;\overline{\partial}_{\sigma}\,\kappa_{m}^{\alpha}\big(\sigma^{B};\gamma^{Q}\big)\;+\;\mathcal{K}_{m}^{\alpha}\big(\sigma^{B};\gamma^{Q}\big)\;=\;0.\label{eq:-258}
\end{equation}
On the minitwistor line $\mathcal{L}_{m}(\gamma^{Q})$, the Cauchy-Riemann
operator $\overline{\partial}_{\sigma}$ acts only on the $\sigma$-fibres.

By the existence and uniqueness theorem for linear PDEs on compact
Riemann surfaces (see $\S\,1.11$ of \citet{forster1981compact}),
the maps $\Pi_{m\dot{A}}$ and $\kappa_{m}^{\alpha}$ defined in Eqs.
(\ref{eq:-255}) and (\ref{eq:-256}) are the unique solutions to
these equations. We now seek an action whose stationarity conditions
reproduce them.

\textcompwordmark{}

\paragraph*{Bosonic Sector.}

For the bosonic sector, we define:
\begin{equation}
\mathcal{S}_{\Pi_{m}}^{N}\big(\gamma^{Q}\big)\;\coloneqq\;\frac{1}{b}\sum_{m=1}^{N}\;\int_{\mathcal{L}_{m}(\gamma^{Q})}\;D\sigma\wedge\left(\frac{1}{2\pi i}\;[\Pi_{m}\,\overline{\partial}_{\sigma}\,\Pi_{m}]\;+\;[\Pi_{m}\,\mathcal{J}_{m}]\right).\label{eq:-259}
\end{equation}
Varying $\mathcal{S}_{\Pi_{m}}^{N}$ with respect to $\Pi_{m}$ and
setting the variation to zero immediately yields Eq. (\ref{eq:-257}).

\textcompwordmark{}

\paragraph*{Fermionic Sector; Celestial Supersphere.}

The fermionic sector requires further attention. The field $\kappa_{m}^{\alpha}$
has Grassmann degree one, while the action must be bosonic. Hence
we pair $\kappa_{m}^{\alpha}$ with another field of Grassmann degree
$3$ and carry out a Berezin integral. However, each minitwistor line
$\mathcal{L}_{m}(\gamma^{Q})$ is bosonic. In fact $\mathcal{L}_{m}(\gamma^{Q})\cong\mathbf{CP}^{1}$.
So, to define a Berezin integral on this line, we extend its coordinates
$\sigma^{B}$ by four fermionic directions $\chi^{\beta}$ associated
with $\mathcal{N}=4$ supersymmetry. 

We denote the resulting \emph{celestial supersphere} by $\mathcal{CS}_{s,m}(\gamma^{Q})$,
the $m$-th copy of the supersymmetric line over which the minitwistor
amplitude localises, parametrised by the moduli point $\gamma^{Q}\in\mathscr{M}_{N}$.
We introduce the $\mathbf{Z}_{2}$-graded coordinate map:
\begin{equation}
\mathsf{s}\coloneqq(\sigma^{B},\chi^{\beta})\colon\;\mathcal{CS}_{s,m}\big(\gamma^{Q}\big)\;\longrightarrow\;\mathbf{CP}^{1}\times\mathbf{C}^{0|4}.
\end{equation}
The natural orientation on the celestial supersphere is given by the
volume superform:
\begin{equation}
D^{1|4}\mathsf{s}\;\coloneqq\;D\sigma\wedge d^{0|4}\chi.
\end{equation}

Now we can define the action for the fermionic sector. Let
\begin{equation}
e_{m\alpha}\in\Lambda[3]\!\otimes\!\Gamma\big(\mathcal{L}_{m}(\gamma^{Q});\mathcal{O}(-1)\big)
\end{equation}
be a Lagrange multiplier of Grassmann degree $3$. In this inclusion
relation, the base manifold remains the bosonic line $\mathcal{L}_{m}(\gamma^{Q})$
because the fermionic directions lie entirely along its fibres.

Consider the projectively invariant top-forms on the celestial supersphere:
\begin{equation}
D^{1|4}\mathsf{s}\wedge e_{m\alpha}\wedge\overline{\partial}_{\sigma}\,\kappa_{m}^{\alpha}\quad\text{and}\quad D^{1|4}\mathsf{s}\wedge e_{m\alpha}\wedge\mathcal{K}_{m}^{\alpha}\;\in\;\Omega^{(1,1)|4}\big(\mathcal{CS}_{s,m}(\gamma^{Q})\big),
\end{equation}
which take values in its Berezinian. We then define the fermionic
action as:
\begin{equation}
\mathcal{S}_{\kappa_{m},e_{m}}^{N}\big(\gamma^{Q}\big)\;=\;\frac{1}{b}\sum_{m=1}^{N}\;\int_{\mathcal{CS}_{s,m}(\gamma^{Q})}\;D^{1|4}\mathsf{s}\wedge\left(\frac{1}{2\pi i}\;e_{m\alpha}\wedge\overline{\partial}_{\sigma}\,\kappa_{m}^{\alpha}\;+\;e_{m\alpha}\wedge\mathcal{K}_{m}^{\alpha}\right).\label{eq:-260}
\end{equation}
Varying $\mathcal{S}_{\pi_{m},e_{m}}^{N}$ with respect to $e_{m\alpha}$
directly yields Eq. (\ref{eq:-258}).

\textcompwordmark{}

\paragraph*{Geometric Sector.}

We denote by the \emph{geometric sector} of the many-body system of
$N$ semiclassical minitwistor strings the sector that governs the
embedding of the celestial supersphere $\mathcal{CS}_{s}$ into minitwistor
superspace $\mathbf{MT}_{s}$. This embedding appears as a family
of minitwistor lines $\mathcal{L}_{1},\dots,\mathcal{L}_{N}$ on which
the minitwistor amplitudes localise. 

The fundamental field variables that define the geometric sector form
the multiplet containing the second-kind evaluation maps together
with the Lagrange multipliers:
\begin{equation}
\digamma\;\coloneqq\;\big\{\Pi_{m\dot{A}}(\sigma^{B}),\,\kappa_{m}^{\alpha}(\sigma^{B}),\,e_{m\alpha}(\sigma^{B},\chi^{\beta})\big\}.\label{eq:-251}
\end{equation}
Combining Eqs. (\ref{eq:-259}) and (\ref{eq:-260}) yields the total
action for the geometric sector:
\begin{equation}
\mathcal{S}_{0}^{N}[\digamma|\gamma^{Q}]\;\coloneqq\;\mathcal{S}_{\Pi_{m}}^{N}\big(\gamma^{Q}\big)\;+\;\mathcal{S}_{\pi_{m},e_{m}}^{N}\big(\gamma^{Q}\big).
\end{equation}

We unify the bosonic and fermionic sectors by introducing two conjugate
superfields, $\Sigma_{m\dot{A}}$ and $\Xi_{m}^{\;\;\dot{A}}$, together
with a supercurrent $|m,\gamma^{Q}]^{\dot{A}}$. The superfields lie
in
\begin{equation}
\Lambda\!\otimes\!\Gamma\big(\mathcal{L}_{m}(\gamma^{Q});\mathcal{O}(-1)\!\oplus\!\mathcal{O}(-1)\big).
\end{equation}
We define:
\begin{equation}
\Sigma_{m\dot{A}}\big(\sigma^{B},\chi^{\beta}\big)\;\coloneqq\;\chi^{1}\chi^{2}\,\Pi_{m\dot{A}}\big(\sigma^{B}\big)\;+\;E_{\dot{A}}^{\;\;\alpha}\,e_{m\alpha}\big(\sigma^{B},\chi^{\beta}\big),
\end{equation}
\begin{equation}
\Xi_{m}^{\;\;\dot{A}}\big(\sigma^{B},\chi^{\beta}\big)\;\coloneqq\;\chi^{3}\chi^{4}\,\Pi_{m}^{\;\;\dot{A}}\big(\sigma^{B}\big)\;+\;E_{\;\;\beta}^{\dot{A}}\,\kappa_{m}^{\beta}\big(\sigma^{B}\big),
\end{equation}
where $E_{\dot{A}}^{\;\;\alpha}$ is the rigid vielbein\footnote{See $\S\,14.1$ of \citet{rogers2007supermanifolds} for a review
of the geometric structures on super Riemann surfaces. That section
discusses the role of the vielbein $E_{\dot{A}}^{\;\;\alpha}$.} on the celestial supersphere introduced earlier. The supercurrent
\begin{equation}
|m,\gamma^{Q}]^{\dot{A}}\in\Lambda\!\otimes\!\mathscr{D}'_{0,1}\big(\mathcal{L}_{m}(\gamma^{Q});\mathcal{O}(-1)\!\oplus\!\mathcal{O}(-1)\big)
\end{equation}
has local form:
\begin{equation}
|m,\gamma^{Q}]^{\dot{A}}\;\coloneqq\;\chi^{3}\chi^{4}\,\mathcal{J}_{m}^{\;\;\dot{A}}\big(\sigma^{B};\gamma^{Q}\big)\;+\;E_{\;\;\alpha}^{\dot{A}}\,\mathcal{K}_{m}^{\alpha}\big(\sigma^{B};\gamma^{Q}\big).
\end{equation}
Hence the \emph{geometric action} becomes\footnote{We use the generalised spinor-helicity bracket:
\begin{equation}
[\omega_{1}\omega_{2}]\;\coloneqq\;\omega_{1\dot{A}}\wedge\omega_{2}^{\;\;\dot{A}}
\end{equation}
for any pair $\omega_{i\dot{A}}$ of Grassmann-valued dotted van der
Waerden spinors.}
\begin{equation}
\mathcal{S}_{0}^{N}[\Delta|\gamma^{Q}]\;=\;\frac{1}{b}\sum_{m=1}^{N}\;\int_{\mathcal{CS}_{s,m}(\gamma^{Q})}\;D^{1|4}\mathsf{s}\wedge\left(\frac{1}{2\pi i}\;[\Sigma_{m}\,\overline{\partial}_{\sigma}\,\Xi_{m}]\;+\;[\Sigma_{m}|m,\gamma^{Q}]\right).\label{eq:-250}
\end{equation}


\paragraph*{Embedding Maps.}

In the discussion above, we covered the holomorphic celestial sphere
$\mathcal{CS}\cong\mathbf{CP}^{1}$ by two coordinate systems. The
first uses homogeneous coordinates $\lambda^{A}$ on $\mathbf{CP}^{1}$.
The second uses coordinates $\sigma^{B}$, which carry homogeneity
weight $-1$. Under the rescaling $\lambda^{A}\mapsto t\,\lambda^{A}$,
the $\sigma$-coordinates transform as $\sigma^{B}\mapsto t^{-1}\,\sigma^{B}$.
The transition map $\lambda^{A}=\lambda^{A}(\sigma^{B})$ between
these patches is given in Eq. (\ref{eq:-254}).

Both coordinate systems on $\mathcal{CS}$ are useful to describe
the field content and dynamics of the minitwistor sigma-model. Accordingly,
the evaluation maps admit two distinct representations. The \emph{first-kind
evaluation maps} depend on the $\lambda$-coordinates and are given
by the sections $\Phi_{m\dot{A}}(\lambda^{A};\gamma^{Q})$ and $\varphi_{m}^{\alpha}(\lambda^{A};\gamma^{Q})$,
introduced in Eqs. (\ref{eq:-306}) and (\ref{eq:-261}). The \emph{second-kind
evaluation maps} depend on the $\sigma$-coordinates and are given
by the sections $\Pi_{m\dot{A}}(\sigma^{B};\gamma^{Q})$ and $\kappa_{m}^{\alpha}(\sigma^{B};\gamma^{Q})$,
defined in Eqs. (\ref{eq:-307}), (\ref{eq:-255}) and (\ref{eq:-256}).

Hence there are two alternative parameterisations of the $m$-th string,
one associated with each kind of evaluation map. Using the first-kind
evaluation maps we parameterise the $m$-th string by
\begin{equation}
\mathsf{W}_{m}^{I}\big(\lambda^{A};\gamma^{Q}\big)\;=\;\big(\lambda^{A},\,\Phi_{m\dot{A}}\big(\lambda^{A};\gamma^{Q}\big),\,\varphi_{m}^{\alpha}\big(\lambda^{A};\gamma^{Q}\big)\big).\label{eq:-245}
\end{equation}
We refer to the assignment $\lambda^{A}\mapsto\mathsf{W}_{m}^{I}(\lambda^{A};\gamma^{Q})$
as the \emph{first-kind parameterisation} of the string $\mathcal{L}_{m}(\gamma^{Q})$. 

Similarly, in terms of the second-kind evaluation maps the $m$-th
string is parameterised by
\begin{equation}
\mathsf{Y}_{m}^{I}\big(\sigma^{B};\gamma^{Q}\big)\;=\;\big(\lambda^{A}\big(\sigma^{B}\big),\,\Pi_{m\dot{A}}\big(\sigma^{B};\gamma^{Q}\big),\,\kappa_{m}^{\alpha}\big(\sigma^{B};\gamma^{Q}\big)\big).\label{eq:-279}
\end{equation}
The assignment $\sigma^{B}\mapsto\mathsf{Y}_{m}^{I}(\sigma^{B};\gamma^{Q})$
is the \emph{second-kind parameterisation} of the string $\mathcal{L}_{m}(\gamma^{Q})$.

When formulating the geometric action $\mathcal{S}_{0}^{N}$, we found
it convenient to employ the $\sigma$-coordinates. However, to define
the worldsheet CFT, it is more practical to use the $\lambda$-coordinates.
In particular, the parameterisation $\lambda^{A}\mapsto\mathsf{W}_{m}^{I}(\lambda^{A};\gamma^{Q})$
is natural from the target-space perspective because the first component
of $\mathsf{W}_{m}^{I}$ is the spinor $\lambda^{A}$.

\subsubsection{Worldsheet CFT\label{subsec:Worldsheet-CFT}}

The phenomenology of minitwistor string theory will rely on auxiliary
matter systems defined on the worldsheet. To reproduce the tree-level
leaf amplitudes for gluons, we will introduce a $2\mathrm{d}\text{ CFT}$
formed by worldsheet fermions. These fermions will couple to an external
gauge potential $\boldsymbol{A}$ on the target superspace $\mathbf{MT}_{s}$.

Integrating out the fermions will produce a chiral determinant. Evaluating
that determinant will yield an effective WZNW action. Consequently,
the coupling of the worldsheet fermions to the background gauge field
will induce a WZNW current algebra, and the correlators of this algebra
will reproduce the Parke-Taylor factors. This mechanism will mirror
the corresponding construction in conventional twistor-string models.

\textcompwordmark{}

\paragraph*{Outline.}

We begin with a pair of worldsheet fermions $\rho,\rho^{*}$ modelled
as spinor fields on the celestial supersphere $\mathcal{CS}_{s}$.
Embedding $\mathcal{CS}_{s}$ into the target superspace $\mathbf{MT}_{s}$
produces a family of minitwistor superlines $\{\mathcal{CS}_{s,m}(\gamma^{Q})\}_{m}$.
This family is described by the evaluation maps $\Pi_{m\dot{A}}$
and $\kappa_{m}^{\alpha}$ via the incidence relations. Under these
evaluation maps, the worldsheet fermions are pushed forward to fermions
supported on the lines $\mathcal{CS}_{s,m}(\gamma^{Q})\subset\mathbf{MT}_{s}$.
Each such line represents a classical configuration of a minitwistor
string in the system. Along every line, the fermions couple minimally
to the background gauge field $\boldsymbol{A}$ on $\mathbf{MT}_{s}$.

Importantly, this construction does not introduce distinct worldsheets
for each minitwistor string. The original celestial supersphere $\mathcal{CS}_{s}$
is a single, fixed object. The different target-space copies of the
celestial supersphere arise from the evaluation maps (and their embeddings),
which map the worldsheet fermions to different minitwistor superlines
in $\mathbf{MT}_{s}$. 

Now, how can we formalise this picture?

\textcompwordmark{}

\paragraph*{A Simple Analogy.}

As an illustration, consider a nonrelativistic system of $N$ spinless
particles. They interact via a potential $V$. Label the position
of the $m$-th particle by $\vec{x}_{m}\in\mathbf{R}^{3}$. Then the
full configuration space is $\mathcal{X}_{N}=\mathbf{R}^{3N}$, and
we chart $\mathcal{X}_{N}$ by the coordinate vector $\vec{X}=(\vec{x}_{1},\dots,\vec{x}_{N})$.
The potential $V(\vec{X})$ that enters the Schr�dinger equation depends
on the \emph{full} set of particle coordinates. Hence we may regard
the potential as a section and write
\begin{equation}
V\,\in\,\Gamma\big(\mathcal{X}_{N};\,\mathcal{X}_{N}\!\times\!\mathbf{R}\big).
\end{equation}


\paragraph*{Embedding Superspace.}

Proceeding by analogy, we take the \emph{configuration space} of the
$N$-string system to be the \emph{embedding superspace}:
\begin{equation}
\mathbf{X}_{N}\;\coloneqq\;\bigtimes{}^{N}\,\mathbf{MT}_{s}.
\end{equation}
As a supermanifold, $\mathbf{X}_{N}$ is globally charted by the \emph{embedding
coordinates} $\big(\mathsf{W}_{m}^{I}\big)_{m=1}^{N}$. Here $\mathsf{W}_{m}^{I}$
denotes the embedding coordinates of the string $\mathcal{L}_{m}$
into $\mathbf{MT}_{s}$.

Fix a parameter in the moduli superspace $\gamma^{Q}\in\mathscr{M}_{N}$.
For each $m$, define $\mathcal{L}_{m}(\gamma^{Q})\subset\mathbf{MT}_{s}$
to be the minitwistor line representing the classical configuration
of the $m$-th string in the localisation family. Observe that $\mathcal{L}_{m}(\gamma^{Q})$
is the image of $\mathbf{CP}^{1}$ under the map $\lambda^{A}\mapsto\mathsf{W}_{m}^{I}(\lambda^{A};\gamma^{Q})$.

With this notation, the classical configuration of the $N$ strings
in the embedding superspace $\mathbf{X}_{N}$ is the Cartesian product
of these images. We denote this configuration by:
\begin{equation}
\mathscr{L}(N;\gamma^{Q})\coloneqq\mathcal{L}_{1}(\gamma^{Q})\times\mathcal{L}_{2}(\gamma^{Q})\times\dots\times\mathcal{L}_{N}(\gamma^{Q})\;\subset\;\mathbf{X}_{N}.
\end{equation}


\paragraph*{The Superpotential.}

We now introduce the superpotential $\boldsymbol{V}=\boldsymbol{V}(\mathsf{W}_{1}^{I},\dots,\mathsf{W}_{N}^{I})$
on $\mathbf{X}_{N}$, which generalises the background gauge field
$\boldsymbol{A}$ discussed in the preceding section. In the elementary
quantum mechanics analogy, $\boldsymbol{V}$ plays the role of the
potential $V(\vec{X})$ in the Schr�dinger equation. 

We define $\boldsymbol{V}$ by its Fourier expansion, using the $\mathcal{MT}$-transform
of Section II. Let $\{\mathsf{T}^{\mathsf{a}}\}$ be a basis of the
gauge Lie algebra $\mathfrak{g}$, and let $\widetilde{\alpha}_{m}^{\Delta,\mathsf{a}}(\mathsf{Z}^{I})$
denote the mode functions of the background gauge field assigned to
the line $\mathcal{L}_{m}$. Then we take\footnote{Here we use DeWitt notation for the conformal weight $\Delta$, so
that
\begin{equation}
\Psi_{\Delta}\big(\mathsf{W}_{m}^{I};\mathsf{Z}'{}^{I}\big)\,\widetilde{\alpha}_{m}^{\Delta,\mathsf{a}}\big(\mathsf{Z}'{}^{I}\big)\;=\;\sum_{m\in\mathbf{Z}}\Psi_{\Delta}\big(\mathsf{W}_{m}^{I};\mathsf{Z}'{}^{I}\big)\,\widetilde{\alpha}_{\Delta,m}^{\mathsf{a}}\big(\mathsf{Z}'{}^{I}\big).
\end{equation}
}:
\begin{equation}
\boldsymbol{V}\big(\mathsf{W}_{1}^{I},\dots,\mathsf{W}_{N}^{I}\big)\;=\;\sum_{m=1}^{N}\;\int_{\mathbf{MT}_{s}^{*}}\;\Psi_{\Delta}\big(\mathsf{W}_{m}^{I};\mathsf{Z}'{}^{I}\big)\,\widetilde{\alpha}_{m}^{\Delta,\mathsf{a}}\big(\mathsf{Z}'{}^{I}\big)\,\mathsf{T}^{\mathsf{a}}\wedge D^{2|4}\mathsf{Z}'.\label{eq:-312}
\end{equation}

Hence we identify the superpotential $\boldsymbol{V}$ as a Lie-algebra-valued
$(0,1)$-form on the natural homogeneous bundle of the embedding superspace
$\mathbf{X}_{N}$. This form extends the gauge potential $\boldsymbol{A}$
on the target superspace $\mathbf{MT}_{s}$ of a single string to
the configuration space $\mathbf{X}_{N}$ of the $N$-string system. 

\textcompwordmark{}

\paragraph*{Induced Potential on Celestial Sphere.}

Applying the restriction homomorphism to the classical configuration
$\mathscr{L}=\mathscr{L}(N;\gamma^{Q})$ yields the \emph{induced
potential }on the holomorphic celestial sphere:
\begin{equation}
\boldsymbol{\upsilon}\in\Omega^{0,1}\big(\mathcal{CS}\big)\!\otimes\!\mathfrak{g},\qquad\boldsymbol{\upsilon}\coloneqq\boldsymbol{V}\big|_{\mathscr{L}}.
\end{equation}
In terms of the embedding coordinates $(\mathsf{W}_{m}^{I})_{m=1}^{N}$
and the first-kind parameterisations of the strings, the induced potential
takes the form
\begin{equation}
\boldsymbol{\upsilon}(\lambda^{A};\gamma^{Q})\;=\;\boldsymbol{V}\big(\mathsf{W}_{1}^{I}(\lambda^{A};\gamma^{Q}),\dots,\mathsf{W}_{N}^{I}(\lambda^{A};\gamma^{Q})\big)\label{eq:-246}
\end{equation}
where $\mathsf{W}_{m}^{I}(\lambda^{A};\gamma^{Q})$ denotes the embedding
map of the $m$-th string.

\textcompwordmark{}

\paragraph*{Celestial Fermions.}

The coupling of the minitwistor strings to the background gauge field
is mediated by the worldsheet fermions $\rho$ and $\rho^{*}$. Physically,
these fermions are pushed forward to spinor fields living on the lines
$\mathcal{L}_{1},\dots,\mathcal{L}_{N}\subset\mathbf{MT}_{s}$ of
the localisation family, where they couple minimally to the gauge
superpotential $\boldsymbol{V}$.

To formalise this intuition, we define $\rho$ and $\rho^{*}$ as
spinor fields on the holomorphic celestial sphere $\mathcal{CS}$
valued in a vector bundle $\mathtt{F}$. The typical fibre of $\mathtt{F}$
is the representation space that models the matter sector of the worldsheet
CFT.

Here our discussion parallels that of the single-string system presented
in Subsection \ref{subsec:Classical-Theory:-Worldsheet}. Consider
a holomorphic gauge field theory formulated on a rank-$N_{c}$ complex
vector bundle $\mathrm{Pr}\colon E\to\mathbf{X}_{N}$. Let $\mathbf{G}$
be a semisimple Lie group with Lie algebra $\mathfrak{g}$, and assume
$\mathrm{Pr}^{-1}(w)\cong\mathfrak{g}$ for all $w\in\mathbf{X}_{N}$. 

Pulling back $E$ to the configuration $\mathscr{L}=\mathscr{L}(N;\gamma^{Q})$
of the $N$-string system via the restriction homomorphism, we obtain
the restricted bundle $\mathtt{E}\coloneqq\mathrm{Pr}^{-1}(\mathscr{L})$
over the holomorphic celestial sphere $\mathcal{CS}$. Since $\mathfrak{g}$
is semisimple, one has $\mathfrak{g}\cong\mathrm{Der}(\mathfrak{g})$.
Hence the induced potential $\boldsymbol{\upsilon}$ defined above
may be identified with a partial connection on $\mathtt{E}\to\mathcal{CS}$.
In particular,
\begin{equation}
\boldsymbol{\upsilon}\in\Omega^{0,1}\big(\mathcal{CS};\mathrm{End}(\mathtt{E})\big).
\end{equation}

Now let $V$ be the complex vector space that carries the representation
of the matter system on the celestial sphere, and let $\mathcal{R}\colon\mathfrak{g}\to\mathrm{GL}(V)$
be a complex representation of the gauge Lie algebra on $V$. Recall
from the previous subsection that we introduced a left action $\varphi\colon\mathfrak{g}\to\mathrm{Aut}(\mathtt{E}\!\times\!V)$
defined by $\varphi_{g}(e,v)\coloneqq(\mathrm{ad}_{g}(e),\mathcal{R}_{g}(v))$.
Using $\varphi$, define an equivalence relation $\simeq$ on $\mathtt{E}\!\times\!V$
by $(e,v)\simeq(e',v')$ iff $\varphi_{g}(e,v)=(e',v')$ for some
$g\in\mathfrak{g}$. With respect to this relation, form the quotient
$\mathtt{F}\coloneqq(\mathtt{E}\!\times\!V)/\mathfrak{g}$. The bundle
$\mathtt{F}$ carries the natural structure of the vector bundle associated
to $\mathtt{E}$, with typical fibre isomorphic to the representation
space $V$.

Finally, let $\mathtt{K}$ denote the canonical line bundle of the
celestial sphere $\mathcal{CS}$. Following \citet{atiyah1971riemann},
choose a spin structure $\sqrt{\mathtt{K}}$ on $\mathcal{CS}$. The
\emph{celestial fermions} are then sections of the corresponding spinor
bundles:
\begin{equation}
\rho\in\Gamma\big(\mathcal{CS};\sqrt{\mathtt{K}}\!\otimes\!\mathtt{F}\big),\qquad\rho^{*}\in\Gamma\big(\mathcal{CS};\sqrt{\mathtt{K}}\!\otimes\!\mathtt{F}^{*}\big).
\end{equation}

The induced potential $\boldsymbol{\upsilon}$ acts on the worldsheet
fermions $\rho$ and $\rho^{*}$ through the partial connection it
defines on the associated bundle $\mathtt{F}$. We denote this partial
connection by $\boldsymbol{\upsilon}^{\sharp}$:
\begin{equation}
\boldsymbol{\upsilon}^{\sharp}\in\Omega^{0,1}\big(\mathcal{CS};\mathrm{GL}(V)\big),\qquad\boldsymbol{\upsilon}^{\sharp}\coloneqq\mathcal{R}\circ\boldsymbol{\upsilon}.
\end{equation}
Using the first-kind parameterisations $\mathsf{W}_{m}^{I}(\lambda^{A};\gamma^{Q})$
of the minitwistor strings, the induced potential on $\mathtt{F}$
can be expressed as
\begin{equation}
\boldsymbol{\upsilon}^{\sharp}(\lambda^{A};\gamma^{Q})\coloneqq\mathcal{R}\big[\boldsymbol{V}\big(\mathsf{W}_{1}^{I}(\lambda^{A};\gamma^{Q}),\dots,\mathsf{W}_{N}^{I}(\lambda^{A};\gamma^{Q})\big)\big].\label{eq:-247}
\end{equation}


\paragraph*{Action.}

With these formal preparations in place, we take the dynamics of the
worldsheet CFT to be governed by the action:
\begin{equation}
\mathcal{S}_{\mathrm{CFT}}[\Delta,\rho,\rho^{*}|\boldsymbol{V};\gamma^{Q}]\;\coloneqq\;\int_{\mathcal{CS}}D\lambda\wedge\big\langle\rho^{*}\big|\big(\overline{\partial}_{\!\lambda}+\boldsymbol{\upsilon}^{\sharp}(\lambda^{A};\gamma^{Q})\rho\big)\big\rangle.\label{eq:-268}
\end{equation}
The dependence of the action functional $\mathcal{S}_{\mathrm{CFT}}$
on both the multiplet $\Delta$ (which contains the evaluation maps)
and on the superpotential $\boldsymbol{V}$ follows from the definition
of the induced potential $\boldsymbol{\upsilon}^{\sharp}$ given in
Eq. (\ref{eq:-247}).

From Eq. (\ref{eq:-268}) the kinetic part of the action reads:
\begin{equation}
\mathcal{S}_{\mathrm{K}}[\rho,\rho^{*}]\;=\;\int_{\mathcal{CS}}D\lambda\wedge\big\langle\rho^{*}\big|\overline{\partial}_{\!\lambda}\rho\big\rangle.\label{eq:-313}
\end{equation}
The term that governs the interaction with the superpotential $\boldsymbol{V}$
is:
\begin{equation}
\mathcal{U}[\Delta,\rho,\rho^{*}|\boldsymbol{V};\gamma^{Q}]\;=\;\int_{\mathcal{CS}}D\lambda\wedge\langle\rho^{*}\big|\boldsymbol{\upsilon}^{\sharp}(\lambda^{A};\gamma^{Q})\rho\rangle.\label{eq:-314}
\end{equation}


\subsubsection{Semiclassical Theory\label{subsec:Semiclassical-Theory}}

We now define the semiclassical theory. A full quantum treatment may
reveal anomalies, and its detailed analysis lies beyond the scope
of this work. Here we adopt the path integral formalism. Because $\mathcal{CS}_{s}\cong\mathbf{CP}^{1|4}$
carries a natural holomorphic structure, we perform an analytic continuation
and work with Euclidean path integrals.

We proceed as follows. We treat the geometric sector of the theory
\emph{classically}. This sector describes the immersion of the celestial
supersphere $\mathcal{CS}_{s}$ into the target superspace $\mathbf{X}_{N}$
as minitwistor strings. The superpotential $\boldsymbol{V}$, which
parameterises the configuration of the holomorphic gauge field theory
on $\mathbf{X}_{N}$, is likewise treated as a classical background.
The worldsheet fermions $\rho$ and $\rho^{*}$, which couple to the
external classical ``bath'' determined by $\boldsymbol{V}$, are retained
as fully \emph{quantum} degrees of freedom. As we shall show in the
next subsection, the path integral over $\rho$ and $\rho^{*}$ produces
the chiral Dirac determinant that yields the integrand of the generating
functional for leaf-gluon amplitudes.

\textcompwordmark{}

\paragraph*{Outline.}

Our strategy to implement the semiclassical theory is to treat the
parameter $b$ that appears in the action $\mathcal{S}_{0}^{N}$ (see
Eq. (\ref{eq:-250})) as a Liouville-like coupling. We then evaluate
the path integral of an observable $\mathrm{F}[\mathsf{W}_{m}^{I}]$,
which depends on the parameterisations $\mathsf{W}_{m}^{I}$ of the
strings, by integrating over the embedding maps $\Delta$ (see Eq.
(\ref{eq:-251})).

In the limit $b\to0$, we apply the saddle-point approximation to
this path integral. The saddle evaluation yields the observable $\mathrm{F}[\mathsf{W}_{m}^{I}]$
evaluated on the classical solutions; these contributions are weighted
by the effective action of the worldsheet fermions propagating on
the classical background superpotential $\boldsymbol{V}$. This construction
defines a measure on the $N$-string system ``phase space'' $\varGamma_{N}$
and hence a corresponding statistical \emph{ensemble}. From that \emph{ensemble}
we obtain the semiclassical correlator for the celestial CFT associated
with the $N$-string system.

\textcompwordmark{}

\paragraph*{Notation. }

The discussion above used the same symbols for two different objects:
the field variables in the geometric sector (the evaluation maps)
and the classical solutions of the sigma-model equations of motion.
In the path-integral formulation that follows, we must distinguish
the dynamical variables from the classical solutions unambiguously. 

We adopt the convention introduced in Subsection \ref{subsec:Semiclassical-Theory-1}.
Undecorated symbols denote the fundamental fields of the theory. Symbols
decorated with a tilde denote the corresponding classical solutions.
For example, the evaluation maps of the first kind,
\begin{equation}
\Phi_{m\dot{A}}\big(\lambda^{A}\big),\qquad\varphi_{m}^{\alpha}\big(\lambda^{A}\big),
\end{equation}
and the embedding map associated with the first-kind parameterisation
of the $m$-th string,
\begin{equation}
\mathsf{W}_{m}^{I}\big(\lambda^{A}\big)\;\coloneqq\;\big(\lambda^{A},\,\Phi_{m\dot{A}}(\lambda^{A}),\,\varphi_{m}^{\alpha}(\lambda^{A})\big),\label{eq:-278}
\end{equation}
refer to field variables expressed in the $\lambda$-coordinates. 

By contrast, the classical solutions stated in Eqs. (\ref{eq:-306})
and (\ref{eq:-261}) are denoted by
\begin{equation}
\widetilde{\Phi}_{m\dot{A}}\big(\lambda^{A};\gamma^{Q}\big),\qquad\widetilde{\varphi}_{m}^{\alpha}\big(\lambda^{A};\gamma^{Q}\big),
\end{equation}
and the classical embedding map for the first-kind parameterisation
of $\mathcal{L}_{m}(\gamma^{Q})$ is
\begin{equation}
\widetilde{\mathsf{W}}_{m}^{I}\big(\lambda^{A};\gamma^{Q}\big)\;\coloneqq\;\big(\lambda^{A},\,\widetilde{\Phi}_{m\dot{A}}(\lambda^{A};\gamma^{Q}),\,\widetilde{\varphi}_{m}^{\alpha}(\lambda^{A};\gamma^{Q})\big).
\end{equation}

The final piece of notation we require is the measure on the moduli
superspace $\mathscr{M}_{N}$ that parameterises the geometric configuration
of the localisation family $\{\mathcal{L}_{m}\}_{m=1}^{N}\subset\mathbf{MT}_{s}$.
This is the collection of marked, irreducible lines of bidegree $\beta=(1,1)$
on which the $\mathrm{N}^{k}\text{-MHV}$ minitwistor sub-amplitude
localises. Recall that the MHV level $k$ and the number of strings
$N$ are related by $N=2k+1$.

Fix the multi-index $\vec{\alpha}=(a_{\ell},b_{\ell})_{\ell=1}^{k}\in\mathbf{Z}^{2k}$
subject to the ordering
\begin{equation}
2\leq a_{1}<a_{2}<\dots<a_{k}<b_{k}<\dots<b_{2}<b_{1}\leq n-1.\label{eq:-267}
\end{equation}
Let
\begin{equation}
S\;\coloneqq\;\big\{\,z_{a_{\ell}-1},\,z_{a_{\ell}},\,z_{b_{\ell}-1},\,z_{b_{\ell}}\,\big\}_{\ell=1}^{k}
\end{equation}
be the set of marked points on $\{\mathcal{L}_{m}\}$. We denote by
$d\boldsymbol{\Omega}_{\vec{\alpha},\mathcal{S}}(\gamma^{Q})$ the
standard measure on the moduli superspace $\mathscr{M}_{N}$ of marked
minitwistor lines with special points $S$.

\textcompwordmark{}

\paragraph*{Actions.}

The first ingredient required for the semiclassical theory is the
full action for the $N$-string system coupled to the background superpotential
$\boldsymbol{V}$. Combining the geometric-sector action with the
worldsheet CFT action, we write:
\begin{equation}
\mathcal{S}_{\mathrm{I}}^{N}[\Delta,\rho,\rho^{*}|\boldsymbol{V};\gamma^{Q}]\;\coloneqq\;\mathcal{S}_{0}^{N}[\Delta|\gamma^{Q}]\;+\;\mathcal{S}_{\mathrm{CFT}}[\Delta,\rho,\rho^{*}|\boldsymbol{V};\gamma^{Q}].
\end{equation}

Let $\mathcal{I}$ denote the effective action that governs the dynamics
of the quantum worldsheet fermions $\rho$ and $\rho^{*}$ propagating
on the classical background $\boldsymbol{V}$ which parameterises
the configuration of the holomorphic gauge theory on $\mathbf{X}_{N}$.
We define $\mathcal{I}$ by evaluating the full action on the classical
solution for the geometric fields $\Delta$:
\begin{equation}
\mathcal{I}[\rho,\rho^{*}|\boldsymbol{V};\gamma^{Q}]\;\coloneqq\;\Big(\,\mathcal{S}^{N}[\,\Delta,\,\rho,\,\rho^{*}|\boldsymbol{V};\gamma^{Q}\,]\,\Big)_{\delta\mathcal{S}^{N}/\delta\Delta=0}.
\end{equation}

We obtain this effective action by substituting into $\mathcal{S}^{N}$
the classical embedding maps $\widetilde{\mathsf{W}}_{m}^{I}$ of
the strings, which follow from the equation of motion $\delta\mathcal{S}_{0}^{N}/\delta\Delta=0$.
Let $\widetilde{\boldsymbol{\upsilon}}^{\sharp}\in\Omega^{0,1}(\mathcal{CS};\mathrm{GL}(V))$
be the induced partial connection on the vector bundle $\mathtt{F}$
associated to the matter sector $\{\rho,\rho^{*}\}$, evaluated at
those classical string configurations. We set:
\begin{equation}
\widetilde{\boldsymbol{\upsilon}}^{\sharp}(\lambda^{A};\gamma^{Q})\;\coloneqq\;\mathcal{R}\big[\boldsymbol{V}\big(\widetilde{\mathsf{W}}_{1}^{I}(\lambda^{A};\gamma^{Q}),\dots,\widetilde{\mathsf{W}}_{N}^{I}(\lambda^{A};\gamma^{Q})\big)\big].
\end{equation}
Hence the effective action takes the explicit form:
\begin{equation}
\mathcal{I}[\rho,\rho^{*}|\boldsymbol{V};\gamma^{Q}]\;=\;\int_{\mathcal{CS}}\;D\lambda\wedge\big\langle\rho^{*}\big|\big(\overline{\partial}_{\!\lambda}+\widetilde{\boldsymbol{\upsilon}}^{\sharp}(\lambda^{A};\gamma^{Q})\big)\rho\big\rangle.
\end{equation}

\textcompwordmark{}

\paragraph*{Saddle-Point Approximation; Semiclassical Statistical Ensemble.}

We now introduce the saddle-point approximation of the Euclidean path
integral for the full action $\mathcal{S}_{\mathrm{I}}^{N}$ with
respect to the geometric fields $\Delta$. This approximation motivates
the introduction of a measure on the formal phase space of the $N$-string
system coupled to the superpotential $\boldsymbol{V}$. Carrying out
the saddle-point analysis leads to a statistical \emph{ensemble} of
minitwistor sigma-models. We propose that the semiclassical correlators
of the resulting \emph{ensemble} provide the celestial correlators
for the minitwistor-string theory.

Treat the parameter $b$ appearing in the geometric-sector action
$\mathcal{S}_{0}^{N}$ as a Liouville-like coupling that controls
the semiclassical expansion. Let $[d\Delta]$ denote the functional
``measure'' over the evaluation maps of the second kind; we take
\begin{equation}
[d\Delta]\;\coloneqq\;\prod_{m=1}^{N}\;[d\Pi_{m}\,d\kappa_{m}\,de_{m}].
\end{equation}
Let $\mathrm{F}[\mathsf{W}_{m}^{I}]$ be a classical functional representing
an observable that depends on the string parameterisations $\mathsf{W}_{m}^{I}(\lambda^{A})$.
Consider the limit $b\to0$ of the path integral of $\mathrm{F}[\mathsf{W}_{m}^{I}]$
over $\Delta$, weighted by $\exp\big(-\mathcal{S}_{\mathrm{I}}^{N}\big)$.
In this limit, the integral is dominated by the stationarity locus
of $\mathcal{S}_{0}^{N}$, where $\delta\mathcal{S}_{0}^{N}/\delta\Delta=0$.
Hence, as computed in Ch. 6 of \citet{schulman2012techniques} or
$\S\,5.3$ of \citet{rivers1988path}, the saddle-point evaluation
yields:
\begin{equation}
\lim_{b\rightarrow0}\;\frac{1}{\mathcal{N}_{0}\big(\gamma^{Q}\big)}\;\int\;[d\Delta]\;e^{-\mathcal{S}^{N}[\Delta,\rho,\rho^{*}|\boldsymbol{V};\gamma^{Q}]}\,\mathrm{F}[\mathsf{W}_{m}^{I}]\;=\;e^{-\mathcal{I}[\rho,\rho^{*}|\boldsymbol{V};\gamma^{Q}]}\,F\big[\widetilde{\mathsf{W}}_{m}^{I}\big(\lambda^{A};\gamma^{Q}\big)\big].\label{eq:-252}
\end{equation}
The normalisation factor $\mathcal{N}_{0}(\gamma^{Q})$ is defined
by
\begin{equation}
\mathcal{N}_{0}\big(\gamma^{Q}\big)\;\coloneqq\;\int\;[d\Delta]\;e^{-\mathcal{S}_{0}^{N}[\Delta|\gamma^{Q}]}.
\end{equation}

The physical interpretation of the right-hand side of the saddle-point
identity (\ref{eq:-252}) is the following. The observable $\mathrm{F}[\mathsf{W}_{m}^{I}]$
is evaluated on the classical configuration $\mathscr{L}(N;\gamma^{Q})$
of the $N$-string system. Using the restriction homomorphism, this
term can be written as:
\begin{equation}
\mathrm{F}\big|_{\mathscr{L}(N;\gamma^{Q})}\big(\lambda^{A}\big)\;=\;\mathrm{F}\big[\widetilde{\mathsf{W}}_{1}^{I}(\lambda^{A};\gamma^{Q}),\dots,\widetilde{\mathsf{W}}_{N}^{I}(\lambda^{A};\gamma^{Q})\big].\label{eq:-308}
\end{equation}
In Eq. (\ref{eq:-252}), the result (\ref{eq:-308}) is weighted by
the inverse of the exponentiated effective action, $e^{-\mathcal{I}}$.
Thus, to obtain the semiclassical vacuum expectation value (VEV) of
the observable $\mathrm{F}[\mathsf{W}_{m}^{I}]$, one must average
the right-hand side of Eq. (\ref{eq:-252}) over all classically allowed
configurations (each parameterised by a point $\gamma^{Q}\in\mathscr{M}_{N}$)
and functionally integrate over the worldsheet fermions $\rho,\rho^{*}$.

To formalise this picture, let $\varGamma_{N}$ denote the ``formal''
phase space\footnote{We treat $\varGamma_{N}$ as a \emph{formal} phase space because we
do not commit to a specific topological manifold underlying $\varGamma_{N}$.
Heuristically, one may write $\varGamma_{N}\cong\mathscr{M}_{N}\!\times\!\mathscr{X}$,
where $\mathscr{M}_{N}$ is the moduli superspace of minitwistor lines
(over which the measure $d\boldsymbol{\Omega}_{\vec{\alpha},S}(\gamma^{Q})$
is defined) and $\mathscr{X}$ is the function space that models the
worldsheet spinor fields $\rho$ and $\rho^{*}$. Naively one might
take
\[
\mathscr{X}=\Gamma\big(\mathbf{CP}^{1};\sqrt{\mathtt{K}}\!\otimes\!\mathtt{F}\big)\times\Gamma\big(\mathbf{CP}^{1};\sqrt{\mathtt{K}}\!\otimes\!\mathtt{F}^{*}\big),
\]
since, at the classical level, $\rho$ is a smooth section of the
bundle $\sqrt{\mathtt{K}}\!\otimes\!\mathtt{F}\to\mathbf{CP}^{1}$
and $\rho^{*}$ is a smooth section of $\sqrt{\mathtt{K}}\!\otimes\!\mathtt{F}^{*}\to\mathbf{CP}^{1}$.
Quantum mechanically, however, this identification is problematic.
As noted by \citet{feynman1965path} in $\S\,7.3$, the trajectories
that dominate functional integrals (in the measure-theoretic sense)
are typically continuous but nowhere differentiable rather than smooth.
Hence a choice of topology on $\mathscr{X}$ that would make the path
integral mathematically well-posed requires functional-analytic input
that goes beyond the present, physics-oriented treatment. For this
reason, we continue to regard $\varGamma_{N}$ as a formal phase space
and refrain from specifying a topology on $\mathscr{X}$ here.} of the $N$-string system coupled to the background gauge field $\boldsymbol{V}$.
On $\varGamma_{N}$ we define the pseudomeasure
\begin{equation}
d\boldsymbol{\mu}_{N}[\rho,\rho^{*};\gamma^{Q}]\;\coloneqq\;\frac{1}{\mathcal{N}_{\mathrm{CFT}}}\,e^{-\mathcal{I}[\rho,\rho^{*}|\boldsymbol{V};\gamma^{Q}]}\;d\boldsymbol{\Omega}_{\vec{\alpha},S}\big(\gamma^{Q}\big)\,[d\rho\,d\rho^{*}],\label{eq:-309}
\end{equation}
where the normalisation factor $\mathcal{N}_{\mathrm{CFT}}$ is
\begin{equation}
\mathcal{N}_{\mathrm{CFT}}\;\coloneqq\;\int\;[d\rho\,d\rho^{*}]\;e^{-\mathcal{S}_{\mathrm{K}}[\rho,\rho^{*}]}.
\end{equation}
Equipping $\varGamma_{N}$ with the pseudomeasure $d\boldsymbol{\mu}_{N}$
yields the \emph{semiclassical statistical ensemble} of $N$ minitwistor
strings coupled to the classical ``bath'' $\boldsymbol{V}$. Semiclassical
correlation functions are then computed as expectation values with
respect to $d\boldsymbol{\mu}_{N}$.

With the structures introduced above, we arrive at the semiclassical
celestial correlator. Let $\mathscr{F}[\mathsf{W}_{m}^{I}]$ denote
the quantum observable corresponding to the classical functional $\mathrm{F}[\mathsf{W}_{m}^{I}]$.
The existence of $\mathscr{F}$ is guaranteed by the correspondence
principle; its semiclassical correlator reads:
\begin{equation}
\lim_{b\rightarrow0}\left\langle \mathscr{F}[\mathsf{W}_{m}^{I}]\right\rangle _{\mathcal{CS}}^{\boldsymbol{V}}\;=\;\frac{1}{\mathcal{N}_{\mathrm{CFT}}}\int_{\mathscr{M}_{N}}\;d\boldsymbol{\Omega}_{\vec{\alpha},S}\big(\gamma^{Q}\big)\;\int\,[d\rho\,d\rho^{*}]\;e^{-\mathcal{I}[\rho,\rho^{*}|\boldsymbol{V};\gamma^{Q}]}\,\mathrm{F}\big[\widetilde{\mathsf{W}}_{m}^{I}(\lambda^{A};\gamma^{Q})\big].\label{eq:-253}
\end{equation}

Define the moduli-dependent, semiclassical vacuum expectation value
of the worldsheet CFT by the limit
\begin{equation}
\lim_{b\to0}\;\left\langle \mathscr{F}[\mathsf{W}_{m}^{I}]\right\rangle _{\mathrm{WS}(\gamma^{Q})}^{\boldsymbol{V}}\;\coloneqq\;\lim_{b\rightarrow0}\;\frac{1}{\mathcal{N}(\gamma^{Q})}\int\;[d\Delta\,d\rho\,d\rho^{*}]\;e^{-\mathcal{S}_{\mathrm{I}}^{N}[\Delta,\rho,\rho^{*}|\boldsymbol{V};\gamma^{Q}]}\,\mathrm{F}\big[\mathsf{W}_{m}^{I}(\lambda^{A})\big],\label{eq:-263}
\end{equation}
with the moduli-dependent normalisation
\begin{equation}
\mathcal{N}\big(\gamma^{Q}\big)\;\coloneqq\;\int\;[d\Delta\,d\rho\,d\rho^{*}]\;e^{-\mathcal{S}_{0}^{N}[\Delta|\gamma^{Q}]-\mathcal{S}_{\mathrm{K}}[\rho,\rho^{*}]}.
\end{equation}
Applying the saddle-point approximation in the $\Delta$-sector yields:
\begin{equation}
\lim_{b\to0}\;\left\langle \mathscr{F}[\mathsf{W}_{m}^{I}]\right\rangle _{\mathrm{WS}(\gamma^{Q})}^{\mathrm{\boldsymbol{V}}}\;=\;\frac{1}{\mathcal{N}_{\mathrm{CFT}}}\int\;[d\rho\,d\rho^{*}]\;e^{-\mathcal{I}[\rho,\rho^{*}|\boldsymbol{V};\gamma^{Q}]}\,\mathrm{F}\big[\widetilde{\mathsf{W}}_{m}^{I}(\lambda^{A};\gamma^{Q})\big].
\end{equation}
Hence Eq. (\ref{eq:-253}) may be rewritten as an integral over the
moduli superspace:
\begin{equation}
\lim_{b\rightarrow0}\;\left\langle \mathscr{F}[\mathsf{W}_{m}^{I}]\right\rangle _{\mathcal{CS}}^{\boldsymbol{V}}\;=\;\lim_{b\to0}\;\int_{\mathscr{M}_{N}}\;d\boldsymbol{\Omega}_{\vec{\alpha},S}\big(\gamma^{Q}\big)\;\left\langle \mathscr{F}[\mathsf{W}_{m}^{I}]\right\rangle _{\mathrm{WS}(\gamma^{Q})}^{\mathrm{\boldsymbol{V}}}.\label{eq:-264}
\end{equation}
Substituting Eq. (\ref{eq:-263}) into Eq. (\ref{eq:-264}) gives
the full semiclassical correlator of the celestial CFT:
\begin{equation}
\lim_{b\rightarrow0}\;\left\langle \mathscr{F}[\mathsf{W}_{m}^{I}]\right\rangle _{\mathcal{CS}}^{\boldsymbol{V}}\;=\;\lim_{b\rightarrow0}\;\int_{\mathscr{M}_{N}}\;\frac{d\boldsymbol{\Omega}_{\vec{\alpha},S}\big(\gamma^{Q}\big)}{\mathcal{N}(\gamma^{Q})}\int\;[d\Delta\,d\rho\,d\rho^{*}]\;e^{-\mathcal{S}_{\mathrm{I}}^{N}[\Delta,\rho,\rho^{*}|\boldsymbol{V};\gamma^{Q}]}\,\mathrm{F}[\mathsf{W}_{m}^{I}].
\end{equation}


\subsubsection{Partition Function and the Tree-level Gluon $\mathcal{S}$-Matrix\label{subsec:Partition-Function}}

We now demonstrate that the semiclassical partition function of the
$N$-string system coupled to the classical background superpotential
$\boldsymbol{V}$ is a generating functional for the tree-level leaf
amplitudes in every $\mathrm{N}^{k}\text{-MHV}$ gluonic sector of
$\mathcal{N}=4$ SYM theory. This identification supports the claim
that the holomorphic gauge theory formulated on minitwistor superspace,
studied in Section IV, arises as the string-field-theory limit of
the semiclassical strings introduced above.

Our strategy is the following. First, we define the semiclassical
partition function of the statistical \emph{ensemble} $\varGamma_{N}$
with respect to the measure $d\boldsymbol{\mu}_{N}$, viewing the
partition function as a functional of the background superpotential
$\boldsymbol{V}$. We denote this functional by $\mathscr{Z}_{N}[\boldsymbol{V}]$.
Second, we recall that $\boldsymbol{V}$ was defined in the previous
subsection (see Eq. (\ref{eq:-312})) via its minitwistor-Fourier
decomposition in terms of the classical mode functions $\upsilon_{m}^{\Delta,\mathsf{a}}(\mathsf{Z}^{I})$.
Physically, these functions are the expectation values of the gluon
annihilation operators, and a set of modes is assigned to each line
$\mathcal{L}_{m}$ in the localisation family. This assignment allows
us to expand the partition function in $\upsilon_{m}^{\Delta,\mathsf{a}}$. 

Finally, we show that functional differentiation of $\mathscr{Z}_{N}[\boldsymbol{V}]$
with respect to $\upsilon_{m}^{\Delta,\mathsf{a}}$, followed by evaluation
at the trivial background $\boldsymbol{V}=0$, reproduces the tree-level
leaf amplitude for gluons in every $\mathrm{N}^{k}\text{-MHV}$ sector
at tree-level.

\textcompwordmark{}

\paragraph*{Partition Function.}

Let $\varGamma_{N}$ be the formal phase space of the $N$-string
system introduced in Subsection \ref{subsec:Semiclassical-Theory}.
Equip $\varGamma_{N}$ with the measure $d\boldsymbol{\mu}_{N}$ given
in Eq. (\ref{eq:-309}). The semiclassical statistical \emph{ensemble}
of $N$ minitwistor strings interacting with the classical external
superpotential $\boldsymbol{V}$ is the pair $(\varGamma_{N},d\boldsymbol{\mu}_{N})$.
Denote the semiclassical partition function of this \emph{ensemble}
by $\mathscr{Z}_{N}[\boldsymbol{V}]$. We regard $\mathscr{Z}_{N}[\boldsymbol{V}]$
as a functional of the classical mode coefficients $\upsilon_{m}^{\Delta,\mathsf{a}}$
that parameterise the configuration of the external gauge field $\boldsymbol{V}$.

The semiclassical partition function in the $d\boldsymbol{\mu}_{N}$-measure
obeys the functional relation:
\begin{equation}
\mathscr{Z}_{N}[\boldsymbol{V}]\;\coloneqq\;\lim_{b\rightarrow0}\;\int_{\:\mathscr{M}_{N}}\;d\boldsymbol{\Omega}_{\vec{\alpha},S}(\gamma^{Q})\;\log\int\;\frac{[d\digamma\,d\rho\,d\rho^{*}]}{\mathcal{N}(\gamma^{Q})}\;e^{-\mathcal{S}_{\mathrm{I}}^{N}[\digamma,\rho,\rho^{*}|\boldsymbol{V};\gamma^{Q}]}.
\end{equation}
Employing Eq. (\ref{eq:-253}), this can be written as the integral
formula:
\begin{equation}
\mathscr{Z}_{N}[\boldsymbol{V}]\;=\;\int_{\:\mathscr{M}_{N}}\;d\boldsymbol{\Omega}_{\vec{\alpha},S}\big(\gamma^{Q}\big)\;\log\int\;\frac{[d\rho\,d\rho^{*}]}{\mathcal{N}_{\mathrm{CFT}}}\;e^{-\mathcal{I}[\rho,\rho^{*}|\boldsymbol{V};\gamma^{Q}]}.\label{eq:-310}
\end{equation}

Next, using Eq. (4) of \citet{witten19822}, integrate over the worldsheet
fermions $\rho$ and $\rho^{*}$. This integration produces the chiral
Dirac determinant. Let $\mathrm{Z}_{R}$ denote the renormalisation
counter-term that isolates the finite, physically relevant contribution.
Hence the chiral determinant admits the path-integral representation:
\begin{equation}
\int\;[d\rho\,d\rho^{*}]\;e^{-\mathcal{I}[\rho,\rho^{*}|\boldsymbol{V};\gamma^{Q}]}\;=\;\mathrm{Z}_{R}\det\big(\mathbb{I}\,+\,\boldsymbol{\upsilon}^{\mathrm{c}}(\lambda^{A};\gamma^{Q})\,\overline{\partial}_{\lambda}^{-1}\big).\label{eq:-311}
\end{equation}

Henceforth we absorb the counter-term $\mathrm{Z}_{R}$ into the normalisation
factor $\mathcal{N}_{\mathrm{CFT}}$. Substituting the identity (\ref{eq:-311})
into Eq. (\ref{eq:-310}) gives the reduced form of the partition
function:
\begin{equation}
\mathscr{Z}_{N}[\boldsymbol{V}]\;=\;\int_{\:\mathscr{M}_{N}}\;d\boldsymbol{\Omega}_{\vec{\alpha},S}\big(\gamma^{Q}\big)\;\mathsf{Tr}\log\big(\mathbb{I}\,+\,\boldsymbol{\upsilon}^{\mathrm{c}}(\lambda^{A};\gamma^{Q})\,\overline{\partial}_{\lambda}^{-1}\big).\label{eq:-266}
\end{equation}

Next, following \citet{boels2007supersymmetric,boels2007twistor},
we \emph{formally} expand the integrand in Eq. (\ref{eq:-266}) as
a power series:
\begin{equation}
\mathsf{Tr}\log\big(\mathbb{I}\,+\,\boldsymbol{\upsilon}^{\mathrm{c}}(\lambda^{A};\gamma^{Q})\,\overline{\partial}_{\lambda}^{-1}\big)\;=\;\sum_{n\geq1}\;\frac{(-1)^{n-1}}{n}\;\mathsf{Tr}\;\int_{\:(\mathbf{CP}^{1})^{\times n}}\;\bigwedge_{i=1}^{n}\;\Bigg(\,\frac{D\lambda_{i}}{\lambda_{i}\cdot\lambda_{i+1}}\wedge\boldsymbol{\upsilon}^{\mathrm{c}}(\lambda_{i}^{A};\gamma^{Q})\,\Bigg).\label{eq:-265}
\end{equation}

We pull back the superpotential $\boldsymbol{V}$ from the embedding
superspace $\mathbf{X}_{N}$ to the celestial sphere $\mathcal{CS}$
via the restriction homomorphism. This induces the partial connection
$\boldsymbol{\upsilon}^{\sharp}$ on the associated vector bundle
$\mathtt{F}\to\mathcal{CS}$, on which the worldsheet spinors $\rho$
and $\rho^{*}$ are represented. Using the minitwistor-Fourier expansion
given by Eq. (\ref{eq:-312}) and evaluating $\boldsymbol{\upsilon}^{\sharp}$
on the classical solutions of the string equations of motion yields:
\begin{equation}
\boldsymbol{\upsilon}^{\mathrm{c}}\big(\lambda_{i}^{A};\gamma^{Q}\big)\;=\;\sum_{m=1}^{N}\;\int_{\:\mathbf{MT}_{s}^{*}}\;\Psi_{\Delta_{i}}\big|_{\mathcal{L}_{m}(\gamma^{Q})}\big(\lambda_{i}^{A};\mathsf{Z}_{i}'{}^{I}\big)\,\upsilon_{m}^{\Delta_{i},\mathsf{a}_{i}}\big(\mathsf{Z}_{i}'{}^{I}\big)\,\mathsf{T}^{\mathsf{a}_{\imath}}\wedge D^{2|4}\mathsf{Z}_{i}'.
\end{equation}

Substituting the induced potential into Eq. (\ref{eq:-265}) and reorganising
the integrals by Fubini's theorem gives the formal expansion:
\begin{align}
 & \mathsf{Tr}\log\big(\mathbb{I}\,+\,\boldsymbol{\upsilon}^{\mathrm{c}}(\lambda^{A};\gamma^{Q})\,\overline{\partial}_{\lambda}^{-1}\big)\;=\;\sum_{n\geq1}\;\frac{(-1)^{n-1}}{n}\;\sum_{m=1}^{N}\;\int_{\:\mathbf{X}_{n}^{*}}\;\bigwedge_{i=1}^{n}\;\left(D^{2|4}\mathsf{Z}_{i}'\wedge\upsilon_{m}^{\Delta_{i},\mathsf{a}_{i}}\big(\mathsf{Z}_{i}'{}^{I}\big)\right)\\
 & \qquad\qquad\mathsf{Tr}\;\int_{\:(\mathbf{CP}^{1})^{\times n}}\;\bigwedge_{j=1}^{n}\;\Bigg(\,\frac{D\lambda_{j}}{\lambda_{j}\cdot\lambda_{j+1}}\mathsf{T}^{\mathsf{a}_{j}}\wedge\Psi_{\Delta_{j}}\big|_{\mathcal{L}_{m}(\gamma^{Q})}\big(\lambda_{j}^{A};\mathsf{Z}_{j}'{}^{I}\big)\,\Bigg).
\end{align}
Invoking the celestial BMSW identity yields:
\begin{align}
 & \mathsf{Tr}\log\big(\mathbb{I}\,+\,\boldsymbol{\upsilon}^{\mathrm{c}}(\lambda^{A};\gamma^{Q})\,\overline{\partial}_{\lambda}^{-1}\big)\;=\;\sum_{n\geq1}\;\frac{(-1)^{n-1}}{n}\;\sum_{m=1}^{N}\;\int_{\:\mathbf{X}_{n}^{*}}\;\bigwedge_{i=1}^{n}\;\left(D^{2|4}\mathsf{Z}_{i}'\wedge\upsilon_{m}^{\Delta_{i},\mathsf{a}_{i}}\big(\mathsf{Z}_{i}'{}^{I}\big)\right)\\
 & \qquad\qquad\qquad\mathsf{Tr}\;\bigwedge_{j=1}^{n}\;\Bigg(\,\frac{\mathcal{C}(\Delta_{j})}{\langle z_{j}'|Y_{m}|\bar{z}_{j}']^{\Delta_{j}}}\exp\left(i\langle z_{j}'|\xi_{m}\cdot\eta_{j}\rangle\right)\;\frac{\mathsf{T}^{\mathsf{a}_{j}}}{z_{j}'\cdot z_{j+1}'}\,\Bigg).
\end{align}
Inserting this expression into Eq. (\ref{eq:-266}) finally gives
the full semiclassical partition function:
\begin{align}
 & \mathscr{Z}_{N}[\boldsymbol{V}]\;=\;\sum_{n\geq1}\;\frac{(-1)^{n-1}}{n}\;\sum_{m=1}^{N}\;\int_{\:\mathbf{X}_{n}^{*}}\;\bigwedge_{i=1}^{n}\;\left(D^{2|4}\mathsf{Z}_{i}'\wedge\upsilon_{m}^{\Delta_{i},\mathsf{a}_{i}}\big(\mathsf{Z}_{i}'{}^{I}\big)\right)\\
 & \qquad\qquad\qquad\int_{\:\mathscr{M}_{N}}\;d\boldsymbol{\Omega}_{\vec{\alpha},S}\big(\gamma^{Q}\big)\;\mathsf{Tr}\;\bigwedge_{j=1}^{n}\;\Bigg(\,\frac{\mathcal{C}(\Delta_{j})}{\langle z_{j}'|Y_{m}|\bar{z}_{j}']^{\Delta_{j}}}\exp\left(i\langle z_{j}'|\xi_{m}\cdot\eta_{j}\rangle\right)\;\frac{\mathsf{T}^{\mathsf{a}_{j}}}{z_{j}'\cdot z_{j+1}'}\,\Bigg).
\end{align}


\paragraph*{Recovering the $\mathrm{N}^{k}\text{-\ensuremath{\mathrm{MHV}}}$
Leaf-Gluon Amplitudes.}

Let $n\geq4$ and fix an integer $1\leq k\leq n-1$. Consider a tree-level
scattering process involving $n$ gluons in an $\mathrm{N}^{k}\text{-MHV}$
configuration. Label the external gluons by $i=1,\dots,n$. In celestial
CFT, the state of the $i$-th gluon is specified by its conformal
weight $\Delta_{i}$ and by its insertion point on the $\mathcal{N}=4$
celestial supersphere $\mathcal{CS}_{s}$. We denote the $i$-th insertion
point by
\begin{equation}
\mathsf{z}_{i}\coloneqq(z_{i},\bar{z}_{i},\eta_{i}^{\alpha})\in\mathcal{CS}_{s}.
\end{equation}
Recall that the dual minitwistor superspace $\mathbf{MT}_{s}^{*}$
is a covering space of the celestial supersphere. Accordingly, we
represent $\mathsf{z}_{i}$ by a dual minitwistor
\begin{equation}
\mathsf{Z}_{i}^{I}\;\coloneqq\;\big(z_{i}^{A},\bar{z}_{i\dot{A}},\eta_{i}^{\alpha}\big)\in\mathbf{MT}_{s}^{*}.
\end{equation}

Let $h_{i}$ denote the scaling dimension of the $i$-th gluon, and
let $|\eta_{i}|$ denote the expectation value of the helicity operator
for that state. Thus the conformal weight $\Delta_{i}$ obeys $2h_{i}+|\eta_{i}|=\Delta_{i}$.

With the preceding remarks we have specified the physics we wish to
analyse. We now derive the leaf-gluon amplitude from the semiclassical
partition function. Fix a multi-index $\vec{\alpha}=(a_{i},b_{i})\in\mathbf{Z}^{2k}$
satisfying the inequality (\ref{eq:-267}). Let $i\mapsto c_{\vec{\alpha}}(i)$
be the indicator function that assigns the $i$-th gluon to a cluster
determined by $\vec{\alpha}$. For example, $c_{\vec{\alpha}}(i)=1$
if $1\leq i\le a_{1}-1$, $c_{\vec{\alpha}}(i)=2$ if $a_{1}\leq i\leq a_{2}-1$,
and so on.

We functionally differentiate the partition function $\mathscr{Z}_{N}[\boldsymbol{V}]$
with respect to the modes $\upsilon_{c_{\vec{\alpha}}(i)}^{2h_{i},\mathsf{a}_{i}}\big(\mathsf{Z}_{i}^{I}\big)$
and evaluate the result on the trivial background:
\begin{align}
 & \Bigg(\,\prod_{i=1}^{n}\;\frac{\delta}{\delta\upsilon_{c_{\vec{\alpha}}(i)}^{2h_{i},\mathsf{a}_{i}}\big(\mathsf{Z}_{i}^{I}\big)}\;\mathscr{Z}_{N}[\boldsymbol{V}]\,\Bigg)_{\!\boldsymbol{V}=0}\\
 & =\;\frac{(-1)^{n-1}}{n}\;\int_{\mathscr{M}_{N}}d\boldsymbol{\Omega}_{\vec{\alpha},S}\big(\gamma^{Q}\big)\;\mathsf{Tr}\;\bigwedge_{i=1}^{n}\;\Bigg(\,\frac{\mathcal{C}(2h_{i})}{\langle z_{i}|Y_{c_{\vec{\alpha}}(i)}|\bar{z}_{i}]^{2h_{i}}}\,\exp\left(i\langle z_{i}|\xi_{c_{\vec{\alpha}}(i)}\cdot\eta_{i}\rangle\right)\,\frac{\mathsf{T}^{\mathsf{a}_{i}}}{z_{i}\cdot z_{i+1}}\,\Bigg).
\end{align}

From Section III we identify the right-hand side as the tree-level
$\mathrm{N}^{k}\text{-MHV}$ leaf-gluon amplitude $\mathcal{M}_{n;\vec{\alpha}}^{\mathsf{a}_{1}\dots\mathsf{a}_{n}}\big(\mathsf{Z}_{i}^{I}\big)$.
Consequently,
\begin{equation}
\Bigg(\,\prod_{i=1}^{n}\;\frac{\delta}{\delta\upsilon_{c_{\vec{\alpha}}(i)}^{2h_{i},\mathsf{a}_{i}}\big(\mathsf{Z}_{i}^{I}\big)}\;\mathscr{Z}_{N}[\boldsymbol{V}]\,\Bigg)_{\!\boldsymbol{V}=0}\;=\;\frac{(-1)^{n-1}}{n}\;\mathcal{M}_{n;\vec{\alpha}}^{\mathsf{a}_{1}\dots\mathsf{a}_{n}}\big(\mathsf{Z}_{i}^{I}\big).
\end{equation}


\paragraph*{Discussion.}

In Section III, we derived a geometric interpretation of the tree-level
$\mathrm{N}^{k}\text{-MHV}$ leaf-gluon amplitudes $\mathcal{M}^{\mathsf{a}_{1}\dots\mathsf{a}_{n}}(\mathsf{Z}_{i}^{I})$
for $\mathcal{N}=4$ SYM as a \emph{localisation theorem}. The statement
is the following. The minitwistor transform of the leaf-gluon amplitudes,
which we denote by $\widetilde{\mathcal{M}}^{\mathsf{a}_{1}\dots\mathsf{a}_{n}}(\mathsf{W}_{i}^{I})$,
is given by an integral over the moduli superspace $\mathscr{M}_{N}$
of marked minitwistor lines $\{\mathcal{L}_{m}\}_{m=1}^{N}$, referred
to as the \emph{localisation family}. Moreover, the minitwistor amplitude
vanishes unless every external gluon participating in the scattering
lies on one of the lines $\mathcal{L}_{1},\dots,\mathcal{L}_{N}$.
Finally, the MHV level $k$ and the number $N$ of lines are related
by $1+2k-N=0$.

Then, in Section IV, we constructed a field-theory interpretation
of tree-level $\mathrm{N}^{k}\text{-MHV}$ leaf-gluon amplitudes as
semiclassical expectation values of nonlocal observables on minitwistor
superspace. These observables are realised as Wilson operators of
a holomorphic gauge field theory on $\mathbf{MT}_{s}$, supported
on the minitwistor lines $\mathcal{L}_{1},\dots,\mathcal{L}_{N}$.
We observed that classical modes of the background gauge potential
localise on the lines in the family $\{\mathcal{L}_{m}\}_{m=1}^{N}$.
These modes are then interpreted physically as the expectation values
of gluon annihilation operators.

We therefore sought a dynamical interpretation of the leaf-gluon amplitudes
in which the lines $\mathcal{L}_{m}$ for $m=1,\dots,N$ are realised
as minitwistor \emph{strings}. To that end, we implemented a many-body
system of semiclassical minitwistor sigma-models. Their worldsheet
is the $\mathcal{N}=4$ celestial supersphere, and their target is
the embedding superspace $\mathbf{X}_{N}$. These sigma-models are
coupled to a classical background gauge superpotential $\boldsymbol{V}$
via a pair of worldsheet spinor fields $\rho$ and $\rho^{*}$.

In this subsection, we demonstrated that the semiclassical partition
function of the model serves as the generating functional for the
leaf-gluon amplitudes at tree-level. This result is consistent with
our picture that the holomorphic gauge theory arises as the string-field-theory
limit of the minitwistor strings presented here.

\subsection{Vertex Operators and the $S$-Algebra\label{subsec:Vertex-Operators}}

\citet{hollands2023operator} take the existence of OPEs as a basic
axiom of QFT. They further argue that a theory's essential properties
follow from its OPE data. From this perspective, a complete description
of a celestial CFT requires three ingredients. One must specify the
vertex operators. One must show that their correlators reproduce the
celestial amplitudes required by the theory. One must also show that
these operators generate the holographic OPEs that appear, for example,
in collinear singularities or as consequences of asymptotic symmetries.

The aim of this subsection is to define the vertex operators associated
with the celestial CFT induced by the semiclassical minitwistor strings.
We first generalise the statement made at the beginning of this section:
the semiclassical correlators of these vertex operators reproduce
the tree-level leaf amplitudes for gluons in any $\mathrm{N}^{k}\text{-MHV}$
sector. We then verify that the celestial OPEs of these operators
close on the gluon $S$-algebra.

\subsubsection{Physical Motivation}

In the preceding discussion, we argued that holomorphic gauge theory
on minitwistor superspace arises as the field-theory limit of minitwistor
strings. The first step of that argument began by analysing the coupling
of the minitwistor sigma-model to a classical background gauge potential.
We then showed that the semiclassical partition function of the sigma-model,
coupled to this gauge field, provides a generating functional for
tree-level gluon amplitudes.

We now present the second step. We identify worldsheet vertex operators
that encode string interactions. We then show that the semiclassical
celestial correlators of these vertex operators, evaluated in the
leading-trace sector, reproduce the tree-level leaf-gluon amplitudes
of $\mathcal{N}=4$ SYM theory.

\textcompwordmark{}

We choose the gauge Lie group to be $\mathbf{G}=\mathrm{SO}(N_{c})$,
where $N_{c}$ denotes the number of colours of the gauge theory.
This choice is convenient because the leading-trace sector of the
celestial correlators is obtained by taking the large-$N_{c}$ limit,
mirroring the familiar limit in conventional gauge/gravity duality.

We take $V=\mathfrak{so}(N_{c})$ as the representation space for
the matter fields $\rho$ and $\rho^{*}$. Accordingly, the vector
bundles $\mathtt{F}$ and $\mathtt{F}^{*}$ over $\mathcal{CS}$,
which carry the worldsheet fermions, are associated to the adjoint
representation of $\mathfrak{so}(N_{c})$. Their construction follows
the procedure described in Subsections \ref{subsec:Classical-Theory:-Worldsheet}
and \ref{subsec:Worldsheet-CFT}.

Henceforth, we index the representation-space components by $r,s=1,\dots,N_{c}$.
For concreteness, let $e_{r}$ be a frame trivialising the vector
bundle $\mathtt{F}\to\mathcal{CS}$, and let $e_{r}^{*}$ be the dual
frame trivialising the bundle $\mathtt{F}^{*}\to\mathcal{CS}$. It
follows that the worldsheet fermions can be written as $\rho=\rho^{r}\!\otimes\!e_{r}$
and $\rho^{*}=\rho^{r}\!\otimes\!e_{r}^{*}$, where $\rho^{r}\in\Gamma(\mathcal{CS};\,\sqrt{\mathtt{K}})$.
Thus, our choice of gauge group and representation space reduces the
matter content of the worldsheet CFT to a set of $N_{c}$ independent
real fermions $\rho^{r}$ valued in the vector representation of $\mathfrak{so}(N_{c})$.

We recall that the action integral $\mathcal{S}_{\mathrm{CFT}}$ (Eq.
(\ref{eq:-268})), which governs the dynamics of the worldsheet CFT,
decomposes into a kinetic action $\mathcal{S}_{\mathrm{K}}$ (Eq.
(\ref{eq:-313})) and an interaction term $\mathcal{U}$ (Eq. (\ref{eq:-314})).
Employing the frames introduced above, the kinetic action assumes
the familiar form:
\begin{equation}
\mathcal{S}_{\mathrm{K}}[\rho^{r}]\;=\;\int_{\:\mathcal{CS}}\;D\lambda\wedge\rho^{r}\,\overline{\partial}_{\lambda}\,\rho^{r}.\label{eq:-273}
\end{equation}
We adopt the strong summation convention for representation-space
indices.

The physically interesting contribution is given by the interaction
term $\mathcal{U}$. To obtain its component form, recall that $\lambda^{A}\mapsto\mathsf{W}_{m}^{I}(\lambda^{A})$
represents the embedding map of the $m$-string $\mathcal{L}_{m}$
expressed in terms of the first-kind evaluation maps $\Phi_{m\dot{A}}$
and $\varphi_{m}^{\alpha}$. Let $\boldsymbol{\upsilon}^{\sharp}$
denote the partial connection in the adjoint representation induced
on the celestial sphere by the pull-back of the superpotential $\boldsymbol{V}$.
Applying the restriction homomorphism to the minitwistor-Fourier decomposition
of $\boldsymbol{V}$ (see Eq. (\ref{eq:-312})) then gives:
\begin{equation}
\boldsymbol{\upsilon}^{\sharp}(\lambda^{A})\;=\;\sum_{m=1}^{N}\;\int_{\:\mathbf{MT}_{s}^{*}}\;\Psi_{\Delta}\big(\mathsf{W}_{m}^{I}(\lambda^{A});\mathsf{Z}'{}^{I}\big)\,\upsilon_{m}^{\Delta,\mathsf{a}}\big(\mathsf{Z}'{}^{I}\big)\,\mathsf{T}^{\mathsf{a}}\wedge D^{2|4}\mathsf{Z}'.
\end{equation}

We substitute this representation of the induced potential into the
definition of $\mathcal{U}$ given in Eq. (\ref{eq:-314}). To this
end, let
\begin{equation}
\jmath^{\mathsf{a}}\;\in\;\Gamma\big(\mathcal{CS};\,\mathtt{K}\!\otimes\!\mathfrak{g}\big)
\end{equation}
denote the \emph{classical worldsheet current}, defined by
\begin{equation}
\jmath^{\mathsf{a}}\;\coloneqq\;\rho^{r}\,\mathsf{T}_{rs}^{\mathsf{a}}\,\rho^{s}.\label{eq:-274}
\end{equation}
Hence the interaction term becomes
\begin{equation}
\mathcal{U}[\digamma,\rho^{r}|\gamma^{Q}]\;=\;\sum_{m=1}^{N}\;\int_{\mathbf{MT}_{s}^{*}}\;\left(\int_{\:\mathcal{CS}}\;D\lambda\wedge\Psi_{\Delta}\big(\mathsf{W}_{m}^{I}(\lambda^{A});\mathsf{Z}'{}^{I}\big)\,\jmath^{\mathsf{a}}(\lambda^{A})\right)\,\upsilon_{m}^{\Delta,\mathsf{a}}\big(\mathsf{Z}'{}^{I}\big)\wedge D^{2|4}\mathsf{Z}'.\label{eq:-270}
\end{equation}
Here $\digamma$ denotes the geometric fields that enter the embedding
maps $\mathsf{W}_{m}^{I}(\lambda^{A})$, which parameterise the strings
in $\lambda$-coordinates.

From Eq. (\ref{eq:-270}) we therefore identify the vertex operator
attached to the $m$-th string $\mathcal{L}_{m}$, carrying conformal
weight $\Delta$ and associated to the dual minitwistor point $\mathsf{Z}^{I}$,
as
\begin{equation}
\mathcal{V}_{\Delta,m}^{\mathsf{a}}\big(\mathsf{Z}^{I}\big)\;\coloneqq\;\int_{\:\mathcal{CS}}\;D\lambda\wedge\Psi_{\Delta}\big(\mathsf{W}_{m}^{I}(\lambda^{A});\mathsf{Z}^{I}\big)\,\jmath^{\mathsf{a}}(\lambda^{A}).\label{eq:-271}
\end{equation}
Consequently, the interaction term reduces to
\begin{equation}
\mathcal{U}[\digamma,\rho^{r}|\gamma^{Q}]\;=\;\sum_{m=1}^{N}\;\int_{\:\mathbf{MT}_{s}^{*}}\;\mathcal{V}_{\Delta,m}^{\mathsf{a}}\big(\mathsf{Z}'{}^{I}\big)\,\upsilon_{m}^{\Delta,\mathsf{a}}\big(\mathsf{Z}'{}^{I}\big)\wedge D^{2|4}\mathsf{Z}'.
\end{equation}


\subsubsection{Leaf Amplitudes from Celestial Correlators}

In this subsection we derive the semiclassical correlation functions
of our dynamical model for the celestial CFT. These correlators encode
interactions that arise exclusively from worldsheet insertions of
the vertex operators $\mathcal{V}_{\Delta,m}^{\mathsf{a}}$. To this
end, we set the external gauge potential to $\boldsymbol{V}=0$.

Hence the action integral governing the dynamics of the celestial
CFT becomes:
\begin{equation}
\mathcal{S}^{N}[\digamma,\rho^{r}|\gamma^{Q}]\;=\;\mathcal{S}_{0}^{N}[\digamma|\gamma^{Q}]\,+\,\mathcal{S}_{\mathrm{K}}[\rho^{r}].
\end{equation}
Here $\mathcal{S}_{0}^{N}$ (see Eq. (\ref{eq:-250})) denotes the
action of the geometric sector, and $\mathcal{S}_{\mathrm{K}}$ (see
Eq. (\ref{eq:-273})) denotes the kinetic action for the worldsheet
fermions. Observe that, with $\boldsymbol{V}=0$, the matter sector
of the worldsheet CFT reduces to a set of $N_{c}$ independent real
\emph{free} fermions $\rho^{r}$ transforming in the vector representation
of $\mathfrak{so}(N_{c})$.

\textcompwordmark{}

\paragraph*{Semiclassical Celestial Correlator.}

The semiclassical correlator associated with the action $\mathcal{S}^{N}$
is introduced as follows. Let $\widehat{O}_{i}$ be a collection of
quantum observables indexed by $i=1,\dots,n$. Suppose each $\widehat{O}_{i}$
depends only on the parameterisations $\mathsf{W}_{m}^{I}(\lambda^{A})$
of the strings. By the correspondence principle there then exists
a set of classical functionals $\mathrm{O}_{i}[\mathsf{W}_{m}^{I}]$
such that:
\begin{equation}
\lim_{b\to0}\;\left\langle \,\prod_{i=1}^{n}\,\widehat{O}_{i}\,\right\rangle _{\mathcal{CS}}\;=\;\lim_{b\to0}\;\int_{\:\mathscr{M}_{N}}\;d\boldsymbol{\Omega}_{\vec{\alpha},S}(\gamma^{Q})\;\int\;\frac{[d\digamma\,d\rho]}{\mathcal{N}(\gamma^{Q})}\;e^{-\mathcal{S}^{N}[\digamma,\rho^{r}|\gamma^{Q}]}\,\prod_{i=1}^{n}\,\mathrm{O}_{i}[\mathsf{W}_{m}^{I}(\lambda^{A})].\label{eq:-269}
\end{equation}
Here the path-integral pseudomeasure of the matter sector is defined
by
\begin{equation}
[d\rho]\;\coloneqq\;\prod_{r=1}^{N_{c}}\,[d\rho^{r}],
\end{equation}
and the normalisation factor by
\begin{equation}
\mathcal{N}(\gamma^{Q})\;=\;\int\;[d\digamma\,d\rho]\;e^{-\mathcal{S}_{0}^{N}[\digamma|\gamma^{Q}]-\mathcal{S}_{\mathrm{K}}[\rho^{r}]}.
\end{equation}


\paragraph*{Physical Setup.}

The physical problem we analyse is the tree-level scattering of $n$
gluons in an $\mathrm{N}^{k}\text{-MHV}$ configuration. In the celestial-CFT
language, let $\Delta_{i}$ denote the conformal weight carried by
the $i$-th gluon. Recall that the dual minitwistor superspace $\mathbf{MT}_{s}^{*}$
covers the celestial supersphere $\mathcal{CS}_{s}$. Choose a dual
minitwistor $\mathsf{Z}_{i}^{I}=(z_{i}^{A},\bar{z}_{i\dot{A}},\eta_{i}^{\alpha})$
that parameterises the insertion point of the $i$-th gluon on $\mathcal{CS}_{s}$.
The scaling dimension $h_{i}$ of the $i$-th gluon and the expectation
value $|\eta_{i}|$ of the helicity operator are related to the conformal
weight by $2h_{i}+|\eta_{i}|=\Delta_{i}$.

Henceforth we fix a multi-index $\vec{\alpha}\in\mathbf{Z}^{2n}$
as in $\S\,$ \ref{subsec:Partition-Function}, and let $i\mapsto c_{\vec{\alpha}}(i)$
be the corresponding indicator function that assigns the $i$-th gluon
to its cluster in $\vec{\alpha}$.

\textcompwordmark{}

\paragraph*{Main Result.}

Consider the semiclassical celestial correlator of the vertex operators
$\mathcal{V}_{2h_{i},c_{\vec{\alpha}}(i)}^{\mathsf{a}_{i}}\big(\mathsf{Z}_{i}^{I}\big)$
in the leading-trace sector:
\begin{equation}
C_{n;\vec{\alpha}}^{\mathsf{a}_{1},\dots,\mathsf{a}_{n}}\big(\mathsf{Z}_{i}^{I};\Delta_{i}\big)\;\coloneqq\;\lim_{N_{c}\to\infty}\,\lim_{b\to0}\;\left\langle \,\prod_{i=1}^{n}\,\mathcal{V}_{2h_{i},c_{\vec{\alpha}}(i)}^{\mathsf{a}_{i}}\big(\mathsf{Z}_{i}^{I}\big)\,\right\rangle _{\mathcal{CS}}.
\end{equation}
Using the defining formula for the celestial correlator (see Eq. (\ref{eq:-269})),
we obtain
\begin{equation}
C_{n;\vec{\alpha}}^{\mathsf{a}_{1},\dots,\mathsf{a}_{n}}\big(\mathsf{Z}_{i}^{I};\Delta_{i}\big)\;=\;\lim_{N_{c}\to\infty}\,\lim_{b\to0}\;\int_{\:\mathscr{M}_{N}}\;d\boldsymbol{\Omega}_{\vec{\alpha},S}(\gamma^{Q})\;\int\;\frac{[d\digamma\,d\rho]}{\mathcal{N}(\gamma^{Q})}\;e^{-\mathcal{S}^{N}[\digamma,\rho^{r}|\gamma^{Q}]}\;\prod_{i=1}^{n}\,\mathcal{V}_{2h_{i},c_{\vec{\alpha}}(i)}^{\mathsf{a}_{i}}\big(\mathsf{Z}_{i}^{I}\big).
\end{equation}
Substituting the vertex-operator definition stated in Eq. (\ref{eq:-271})
and reorganising the integrals by Fubini's theorem gives
\begin{align}
 & C_{n;\vec{\alpha}}^{\mathsf{a}_{1}\dots\mathsf{a}_{n}}\big(\mathsf{Z}_{i}^{I};\Delta_{i}\big)\;=\;\lim_{N_{c}\to\infty}\,\lim_{b\to0}\;\int_{\:\mathscr{M}_{N}}\;d\boldsymbol{\Omega}_{\vec{\alpha},S}\big(\gamma^{Q}\big)\;\int_{\:(\mathbf{CP}^{1})^{\times n}}\;\bigwedge_{i=1}^{n}\;D\lambda_{i}\\
 & \qquad\qquad\qquad\int\;\frac{[d\digamma\,d\rho]}{\mathcal{N}\big(\gamma^{Q}\big)}\;e^{-\mathcal{S}^{N}[\digamma,\rho^{r}|\gamma^{Q}]}\;\bigwedge_{j=1}^{n}\;\left(\,\Psi_{2h_{j}}\big(\mathsf{W}_{c_{\vec{\alpha}}(j)}^{I}(\lambda_{j}^{A});\mathsf{Z}_{j}^{I}\big)\,\jmath^{\mathsf{a}_{j}}\big(\lambda_{j}^{A}\big)\,\right).
\end{align}

Performing the path integral over the geometric fields $\digamma$,
and then taking the semiclassical limit $b\to0$, amounts to replacing
the embedding maps $\mathsf{W}_{m}^{I}(\lambda^{A})$ by the classical
solutions $\widetilde{\mathsf{W}}_{m}^{I}(\lambda^{A};\gamma^{Q})$
of the sigma-model equations of motion. These classical solutions
parameterise the configuration $\mathscr{L}(N;\gamma^{Q})\subset\mathbf{X}_{N}$
of the $N$-string system represented by the point $\gamma^{Q}\in\mathscr{M}_{N}$
in the classical moduli superspace.

Evaluating the minitwistor superwavefunction $\Psi_{\Delta}$ on a
classical solution $\widetilde{\mathsf{W}}_{m}^{I}$ is equivalent
to pulling $\Psi_{\Delta}$ back to the $m$-th string $\mathcal{L}_{m}(\gamma^{Q})\subset\mathbf{MT}_{s}$
via the restriction homomorphism, namely
\begin{equation}
\Psi_{\Delta}\big|_{\mathcal{L}_{m}(\gamma^{Q})}\big(\lambda^{A};\mathsf{Z}^{I}\big)\;=\;\Psi_{\Delta}\big(\widetilde{\mathsf{W}}_{m}^{I}(\lambda^{A};\gamma^{Q});\mathsf{Z}^{I}\big).
\end{equation}
Consequently, letting $\langle\dots\rangle_{\mathrm{WZNW}}$ denote
the current-algebra correlator, the correlation function $C_{n;\vec{\alpha}}^{\mathsf{a}_{1}\dots\mathsf{a}_{n}}$
can be expressed as
\begin{align}
 & C_{n;\vec{\alpha}}^{\mathsf{a}_{1}\dots\mathsf{a}_{n}}\big(\mathsf{Z}_{i}^{I};\Delta_{i}\big)\;=\;\lim_{N_{c}\to\infty}\;\int_{\:\mathscr{M}_{N}}\;d\boldsymbol{\Omega}_{\vec{\alpha},S}\big(\gamma^{Q}\big)\;\\
 & \qquad\qquad\qquad\int_{\:(\mathbf{CP}^{1})^{\times n}}\;\bigwedge_{i=1}^{n}\;\left(D\lambda_{i}\wedge\Psi_{2h_{i}}\big|_{\mathcal{L}_{c_{\vec{\alpha}}(i)}}\big(\lambda_{i}^{A};\mathsf{Z}_{i}^{I}\big)\right)\quad\left\langle \,\prod_{j=1}^{n}\,\hat{\jmath}^{\mathsf{a}_{j}}\big(\lambda_{j}^{A}\big)\,\right\rangle _{\mathrm{WZNW}}.
\end{align}
Restricting to the leading-trace sector by taking the large-$N_{c}$
limit yields
\begin{equation}
C_{n;\vec{\alpha}}^{\mathsf{a}_{1}\dots\mathsf{a}_{n}}\big(\mathsf{Z}_{i}^{I};\Delta_{i}\big)=\int_{\:\mathscr{M}_{N}}\;d\boldsymbol{\Omega}_{\vec{\alpha},S}\big(\gamma^{Q}\big)\;\mathsf{Tr}\int_{\:(\mathbf{CP}^{1})^{\times n}}\bigwedge_{i=1}^{n}\left(\,\frac{D\lambda_{i}}{\lambda_{i}\cdot\lambda_{i+1}}\mathsf{T}^{\mathsf{a}_{i}}\wedge\Psi_{2h_{i}}\big|_{\mathcal{L}_{c_{\vec{\alpha}}(i)}}\big(\lambda_{i}^{A};\mathsf{Z}_{i}^{I}\big)\,\right).
\end{equation}
Invoking the celestial BMSW identity then gives
\begin{equation}
C_{n;\vec{\alpha}}^{\mathsf{a}_{1}\dots\mathsf{a}_{n}}\big(\mathsf{Z}_{i}^{I};\Delta_{i}\big)\;=\;\int_{\:\mathscr{M}_{N}}\;d\boldsymbol{\Omega}_{\vec{\alpha},S}\big(\gamma^{Q}\big)\;\mathsf{Tr}\bigwedge_{i=1}^{n}\left(\frac{\mathcal{C}(2h_{i})}{\langle z_{i}|Y_{c_{\vec{\alpha}}(i)}|\bar{z}_{i}]^{2h_{i}}}\,\exp\left(i\langle z_{i}|\xi_{c_{\vec{\alpha}}(i)}\cdot\eta_{i}\rangle\right)\,\frac{\mathsf{T}^{\mathsf{a}_{i}}}{z_{i}\cdot z_{i+1}}\right)\label{eq:-272}
\end{equation}

By the localisation theorem proven in Section III, the right-hand
side of Eq. (\ref{eq:-272}) is identified with the tree-level $\mathrm{N}^{k}\text{-MHV}$
leaf-gluon amplitude $\mathcal{M}_{n;\vec{\alpha}}^{\mathsf{a}_{1}\dots\mathsf{a}_{n}}$.
Therefore we conclude that
\begin{equation}
\lim_{N_{c}\to\infty}\,\lim_{b\to0}\;\left\langle \,\prod_{i=1}^{n}\,\mathcal{V}_{2h_{i},c_{\vec{\alpha}}(i)}^{\mathsf{a}_{i}}\big(\mathsf{Z}_{i}^{I}\big)\,\right\rangle _{\mathcal{CS}}\;=\;\mathcal{M}_{n;\vec{\alpha}}^{\mathsf{a}_{1}\dots\mathsf{a}_{n}}\big(\mathsf{Z}_{i}^{I}\big).
\end{equation}

To conclude, define the celestial gluon operator $\mathcal{G}_{\Delta,m}^{\eta,\mathsf{a}_{i}}$
with helicity state $\eta^{\alpha}$ and conformal weight $\Delta$
attached to the $m$-th string by
\begin{equation}
\mathcal{G}_{\Delta}^{\eta,\mathsf{a}_{i}}(z,\bar{z})\;=\;\mathcal{V}_{\Delta-\left|\eta\right|,m}^{\mathsf{a}_{i}}\big(z^{A},\bar{z}_{\dot{A}},\eta^{\alpha}\big),\label{eq:-276}
\end{equation}
where $|\eta_{i}|$ denotes the expectation value of the helicity
operator and $2h_{i}=\Delta_{i}-|\eta_{i}|$.

Thus, the leading-trace ($k\rightarrow0$), semiclassical ($b\rightarrow0$)
correlator of the celestial gluon operators reproduces the leaf-gluon
amplitude:
\begin{equation}
\lim_{k\rightarrow0}\,\lim_{b\rightarrow0}\;\left\langle \prod_{i=1}^{n}\;\mathcal{G}_{2h_{i},c_{\vec{a}}(i)}^{\eta_{i},\mathsf{a}_{i}}\big(z_{i},\bar{z}_{i}\big)\right\rangle _{\mathcal{CS}}\;=\;\mathcal{M}_{n;\vec{\alpha}}^{\mathsf{a}_{1}\dots\mathsf{a}_{n}}\big(\mathsf{Z}_{i}^{I}\big).
\end{equation}

\begin{quotation}
This formula is the central result of the paper: the leading-trace,
semiclassical celestial correlators of the gluon vertex operators
reproduce the tree-level leaf-gluon amplitudes in every $\mathrm{N}^{k}\text{-MHV}$
sector. In this way we have provided a bottom-up realisation of the
celestial CFT for $\mathcal{N}=4$ SYM as a many-body system of semiclassical
minitwistor strings.
\end{quotation}

\subsubsection{The $S$-Algebra\label{subsec:S-Algebra}}

Asymptotic symmetries, together with the structure of gauge-theory
collinear singularities, impose a constraint on \emph{any} celestial
CFT that is proposed to be holographically dual to gauge theory on
asymptotically flat spacetimes. The CFT must contain primary fields
that generate the gluon $S$-algebra\footnote{Cf. \citet{fotopoulos2019primary,guevara2021holographic,pate2021celestial,himwich2022celestial}.}.
Therefore, to test the consistency of our proposal we verify that
the gluon operators $\mathcal{G}_{\Delta}^{\mathsf{a}}$ close on
the $S$-algebra\footnote{See \citet{banerjee2024all} for a comprehensive discussion.}.

We restrict our attention to the celestial CFT modelled by a single
semiclassical string. As discussed above, the single-string system
is dual only to the MHV gluonic subsector of the gauge theory at tree-level.
Nevertheless, this simple model suffices to generate the $S$-algebra.
In practice, working with a single string means that all vertex operators
are assigned to the same minitwistor line. Therefore we may omit the
index $m$ that labels the vertex operators in Eq. (\ref{eq:-271}).

\textcompwordmark{}

\paragraph*{Current Algebra.}

The first step in computing the OPE of the string vertex operators
is to promote the classical worldsheet current $\jmath^{\mathsf{a}}$
(Eq. (\ref{eq:-274})) to a quantum operator $J^{\mathsf{a}}$. The
correspondence principle implies that $J^{\mathsf{a}}$ should be
proportional to the normally-ordered product $(\rho^{\intercal}\mathsf{T}^{\mathsf{a}}\rho)$.
Denote the proportionality constant by $\beta$. 

Applying Wick's theorem yields the OPE for $J^{\mathsf{a}}$, and
the result is displayed in Eq. (\ref{eq:-275}) of Subsection \ref{subsec:Vertex-Operators-1}.
Requiring consistency of that equation with the Ward identity fixes
$2\beta=1$. Hence $J^{\mathsf{a}}$ generates a level-one $\mathrm{SO}(N_{c})$
WZNW current algebra on the celestial sphere.

Now, since we are concerned only with the leading-trace sector, we
will work with the OPE in the large-$N_{c}$ limit,
\begin{equation}
J^{\mathsf{a}}(\lambda_{1})J^{\mathsf{b}}(\lambda_{2})\widesim\;\frac{if^{\mathsf{abc}}J^{c}(\lambda_{2})}{\lambda_{1}-\lambda_{2}}\qquad(N_{c}\gg1).\label{eq:-315}
\end{equation}


\paragraph*{Integrated Vertex Operator.}

The next step is to derive the integrated form of the vertex operator
$\mathcal{V}_{\Delta}^{\mathsf{a}}$. Let $\Phi_{\dot{A}}$ and $\varphi^{\alpha}$
denote the evaluation maps of the first kind introduced in Eq. (\ref{eq:-227})
of Subsection \ref{subsec:Formal-Preliminaries}. Let $\lambda^{A}\mapsto\mathsf{W}^{I}(\lambda^{A})$
be the embedding map of the minitwistor string $\mathcal{L}$ introduced
in Eq. (\ref{eq:-278}) of Remark \ref{rem:Embedding-Maps.}. 

Composing the superwavefunction $\Psi_{\Delta}$ with this first-kind
parameterisation pulls $\Psi_{\Delta}$ back to the minitwistor line
$\mathcal{L}$. Hence:
\begin{equation}
\Psi_{\Delta}\big(\mathsf{W}^{I}(\lambda^{B});\mathsf{Z}^{I}\big)\;=\;\overline{\delta}_{-\Delta}\big(z^{A},\lambda^{B}\big)\,\frac{\mathcal{C}(\Delta)}{[\Phi(\lambda)\,\bar{z}]^{\Delta}}\,\exp\left(i\frac{\langle z\iota\rangle}{\langle\lambda\iota\rangle}\varphi(\lambda)\cdot\eta\right).
\end{equation}
Substituting this expression into the definition of $\mathcal{V}_{\Delta}^{\mathsf{a}}$
(see Eq. (\ref{eq:-271})) and integrating over $\lambda^{A}$ gives
\begin{equation}
\mathcal{V}_{\Delta}^{\mathsf{a}_{i}}\big(\mathsf{Z}^{I}\big)\;=\;\frac{\mathcal{C}(\Delta)}{[\Phi(z)\,\bar{z}]^{\Delta}}\,e^{i\varphi(z)\cdot\eta}\,J^{\mathsf{a}_{i}}(z).
\end{equation}
Consider now the following operator product:
\begin{equation}
\mathcal{V}_{d_{1}}^{\mathsf{a}}\big(\mathsf{Z}_{1}^{I}\big)\,\mathcal{V}_{d_{2}}^{\mathsf{b}}\big(\mathsf{Z}_{2}^{I}\big)\;=\;\frac{\mathcal{C}(d_{1})}{[\Phi(z_{1})\,\bar{z}_{1}]^{d_{1}}}\frac{\mathcal{C}(d_{2})}{[\Phi(z_{2})\,\bar{z}_{2}]^{d}}\,e^{i(\varphi(z_{1})\cdot\eta_{1}+\varphi(z_{2})\cdot\eta_{2})}\,J^{\mathsf{a}}(z_{1})J^{\mathsf{b}}(z_{2}).\label{eq:-274-1}
\end{equation}


\paragraph*{Definition.}

Let $z_{12}\coloneqq z_{1}-z_{2}$ and $\bar{z}_{12}\coloneqq\bar{z}_{1}-\bar{z}_{2}$.
We study the OPE implied by Eq. (\ref{eq:-274-1}) in the \emph{large-$N_{c}$
holomorphic collinear limit}. This limit is obtained by holding $\eta_{1}^{\alpha}$,
$\eta_{2}^{\alpha}$ and $\bar{z}_{12}$ fixed while $N_{c}\gg1$
and $z_{12}\to0$. 

\textcompwordmark{}

\paragraph*{Main Result.}

Accordingly, Eqs. (\ref{eq:-315}) and (\ref{eq:-274-1}) imply:
\begin{equation}
\mathcal{V}_{d_{1}}^{\mathsf{a}}\big(\mathsf{Z}_{1}^{I}\big)\,\mathcal{V}_{d_{2}}^{\mathsf{b}}\big(\mathsf{Z}_{2}^{I}\big)\widesim\;\frac{if^{\mathsf{abc}}}{z_{12}}\;\frac{\mathcal{C}(d_{1})}{[\Phi(z_{2})\,\bar{z}_{1}]^{d_{1}}}\,\frac{\mathcal{C}(d_{2})}{[\Phi(z_{2})\,\bar{z}_{2}]^{d_{2}}}\,e^{i\varphi(z_{2})\cdot(\eta_{1}+\eta_{2})}\,J^{\mathsf{c}}(z_{2})\quad(N_{c}\gg1).\label{eq:-275-1}
\end{equation}
Next, we use the identity
\begin{equation}
\frac{\mathcal{C}(d_{1})}{[\mu\bar{z}_{1}]^{d_{1}}}\,\frac{\mathcal{C}(d_{2})}{[\mu\,\bar{z}_{2}]^{d_{2}}}\;=\;\sum_{k\geq0}\;\frac{\bar{z}_{12}^{k}}{k!}\,B(d_{1}+k,d_{2})\,\partial_{\bar{z}_{2}}^{k}\,\frac{\mathcal{C}(d_{1}+d_{2})}{[\mu\bar{z}_{2}]^{d_{1}+d_{2}}}.
\end{equation}
Substituting this formula into Eq. (\ref{eq:-275-1}) gives the OPE
of the vertex operators:
\begin{equation}
\mathcal{V}_{d_{1}}^{\mathsf{a}}\big(\mathsf{Z}_{1}^{I}\big)\,\mathcal{V}_{d_{2}}^{\mathsf{b}}\big(\mathsf{Z}_{2}^{I}\big)\widesim\;\frac{if^{\mathsf{abc}}}{z_{12}}\;\sum_{k\geq0}\;\frac{\bar{z}_{12}^{k}}{k!}\,B(d_{1}+k,d_{2})\,\partial_{\bar{z}_{2}}^{k}\,\mathcal{V}_{d_{1}+d_{2}}^{\mathsf{c}}(z_{2},\bar{z}_{2},\eta_{1}+\eta_{2})\quad(N_{c}\gg1).\label{eq:-277-1}
\end{equation}


\paragraph*{Corollary.}

Recall that the celestial gluon operators $\mathcal{G}_{\Delta}^{\eta,\mathsf{a}}(z,\bar{z})$
are defined in terms of the vertex operators $\mathcal{V}_{\Delta}^{\mathsf{a}}(\mathsf{Z}^{I})$
(see Eq. (\ref{eq:-276})). Using Eq. (\ref{eq:-277-1}) then yields
the following OPEs.

For two positive-helicity gluons with conformal weights $\Delta_{i}$
at points $\mathsf{z}_{i}\in\mathcal{CS}$:
\begin{equation}
\mathcal{G}_{\Delta_{1}}^{+,\mathsf{a}}(z_{1},\bar{z}_{1})\,\mathcal{G}_{\Delta_{2}}^{+,\mathsf{b}}(z_{2},\bar{z}_{2})\;\sim\;\frac{if^{\mathsf{abc}}}{z_{12}}\;\sum_{k\geq0}\;\frac{\bar{z}_{12}^{k}}{k!}\,B(\Delta_{1}+k-1,\Delta_{2}-1)\,\partial_{\bar{z}_{2}}^{k}\,\mathcal{G}_{\Delta_{1}+\Delta_{2}-1}^{+,\mathsf{c}}(z_{2},\bar{z}_{2}).\label{eq:-284}
\end{equation}

For one positive-helicity gluon and one negative-helicity gluon:
\begin{equation}
\mathcal{G}_{\Delta_{1}}^{+,\mathsf{a}}(z_{1},\bar{z}_{1})\,\mathcal{G}_{\Delta_{2}}^{-,\mathsf{b}}(z_{2},\bar{z}_{2})\;\sim\;\frac{if^{\mathsf{abc}}}{z_{12}}\;\sum_{k\geq0}\;\frac{\bar{z}_{12}^{k}}{k!}\,B(\Delta_{1}+k-1,\Delta_{2}+1)\,\partial_{\bar{z}_{2}}^{k}\,\mathcal{G}_{\Delta_{1}+\Delta_{2}-1}^{-,\mathsf{c}}(z_{2},\bar{z}_{2}).\label{eq:-285}
\end{equation}

\begin{rem}
We present these OPEs under the simplifying assumption that all gluons
are outgoing. This hypothesis is natural in Klein-space kinematics.
The corresponding OPEs for all incoming or mixed incoming/outgoing
kinematics are obtained by the obvious substitutions.
\end{rem}

The primary contributions read:
\begin{equation}
\mathcal{G}_{\Delta_{1}}^{+,\mathsf{a}}(z_{1},\bar{z}_{1})\,\mathcal{G}_{\Delta_{2}}^{+,\mathsf{b}}(z_{2},\bar{z}_{2})\;\sim\;\frac{if^{\mathsf{abc}}}{z_{12}}\,B(\Delta_{1}-1,\Delta_{2}-1)\,\mathcal{G}_{\Delta_{1}+\Delta_{2}-1}^{+,\mathsf{c}}(z_{2},\bar{z}_{2})\;+\;O(\bar{z}_{12}),
\end{equation}
\begin{equation}
\mathcal{G}_{\Delta_{1}}^{+,\mathsf{a}}(z_{1},\bar{z}_{1})\,\mathcal{G}_{\Delta_{2}}^{-,\mathsf{b}}(z_{2},\bar{z}_{2})\;\sim\;\frac{if^{\mathsf{abc}}}{z_{12}}\,B(\Delta_{1}-1,\Delta_{2}+1)\,\mathcal{G}_{\Delta_{1}+\Delta_{2}-1}^{-,\mathsf{c}}(z_{2},\bar{z}_{2})\;+\;O(\bar{z}_{12}),
\end{equation}

These leading terms reproduce results obtained from asymptotic-symmetry
analysis and from the gauge-theory study of collinear singularities.
The difference is conceptual: our derivation arises from a dynamical
model for the celestial CFT, rather than from a kinematical reparameterisation
of gauge theory on flat spacetime\footnote{See, for a different route, \citet{adamo2022celestial}.}.
The model is realised as a theory of semiclassical minitwistor strings,
and we therefore propose it as a holographic dual to $\mathcal{N}=4$
SYM on flat space at tree level across all $\mathrm{N}^{k}\text{-MHV}$
sectors.
\begin{rem}
An alternative approach to extract the gluon $S$-algebra from string
vertex operators dressed by worldsheet currents is to introduce a
level-$k$ WZNW current algebra on the celestial sphere $\mathcal{CS}$,
and then consider the limit $k\to0$. However, this limit is ill-defined
at the quantum level because $k$ is quantised rather than continuously
variable. Thus, strictly speaking, one would have to define the current
algebra with $k=0$ from the outset, which in turn forces the OPE
to vanish, $J^{\mathsf{a}}J^{\mathsf{b}}\sim0$. 

A further complication is that the level contributes to the total
central charge of the sigma model. Consequently, $k$ is not a freely
adjustable parameter that can be sent to zero at will. It is constrained
by the requirement of quantum anomaly cancellation.

In principle, these issues can be avoided by noting that our celestial
CFT model is defined only at the semiclassical level. Nevertheless,
we prefer the approach adopted above: keep $k\neq0$ and restrict
attention to the leading-trace sector by taking the large-$N_{c}$
limit. This prescription is more physical, as it mirrors the standard
large-$N$ limit used in AdS/CFT. Moreover, retaining a nonzero level
may prove useful in future work on the fully quantum minitwistor string,
where $k$ is expected to enter anomaly-cancellation conditions.
\end{rem}


\subsection{Discussion}

The preceding sections developed a many-body theory of semiclassical
minitwistor strings propagating on a background gauge potential on
$\mathbf{MT}_{s}$. The physical motivation for this theory is the
localisation theorem. That theorem states that the minitwistor transform\footnote{Defined in Subsection \ref{subsec:Minitwistor-Fourier-Transform}.}
$\mathcal{MT}$ of tree-level $\mathrm{N}^{k}\text{-MHV}$ leaf-gluon
amplitudes localises on a family of minitwistor lines $\{\mathcal{L}_{m}\}_{m=1}^{N}$,
where $N=2k+1$.

In Section \ref{sec:Minitwistor-Wilson-Lines} we showed that these
amplitudes admit a geometric interpretation as semiclassical expectation
values of Wilson line operators $\mathbb{W}[\mathscr{S}]$. Here $\mathscr{S}$
denotes an algebraic one-cycle on $\mathbf{MT}_{s}$ constructed from
the localisation family $\{\mathcal{L}_{m}\}$. Writing the generating
functional for leaf-gluon amplitudes as an expectation value of $\mathbb{W}[\mathscr{S}]$
leads to weighted volume integrals over the moduli superspace $\mathscr{M}_{N}$
of minitwistor lines.

To evaluate those integrals we expanded the background gauge potential
$\boldsymbol{A}$ on $\mathbf{MT}_{s}$ (and subsequently its associated
superpotential $\mathbb{A}$ on $\mathbf{X}_{N}$) in the basis of
minitwistor superwavefunctions $\{\Psi_{\Delta}\}$,
\begin{equation}
\boldsymbol{A}\big|_{\mathcal{L}_{m}}\,(\lambda_{i}^{A})\;=\;2\pi i\;\underset{\mathbf{MT}_{s}^{*}\,\,\,}{\int}\;\Psi_{\Delta_{i}}\big|_{\mathcal{L}_{m}}\big(\lambda_{i}^{A};\mathsf{Z}_{i}'{}^{I}\big)\,\alpha_{m}^{\Delta_{i},\mathsf{a}_{i}}\big(\mathsf{Z}_{i}'{}^{I}\big)\,\mathsf{T}^{\mathsf{a}_{i}}\wedge D^{2|4}\mathsf{Z}_{i}'.\label{eq:-217-1}
\end{equation}
This expansion follows from the $\mathcal{MT}$ transform introduced
in Section II. The result is a collection of modes $\alpha_{m}^{\Delta,\mathsf{a}}$
labelled by $m=1,\dots,N$, so that the modes are naturally associated
to the lines $\mathcal{L}_{m}$ in the localisation family.

Interpreting each $\alpha_{m}^{\Delta,\mathsf{a}}$ as the classical
VEV of a gluon annihilation operator suggested a \emph{dynamical}
interpretation of the localisation theorem. In that interpretation,
each line $\mathcal{L}_{m}$ was the image of a ``minitwistor string''
propagating on the background gauge field on $\mathbf{MT}_{s}$. The
(celestial) gluon operators then arose from the worldsheet vertex-operator
algebra of these minitwistor strings. This identification supplied
a route from the geometry of localisation to a semiclassical, many-string
description of the corresponding $\mathrm{N}^{k}\text{-MHV}$ amplitudes.

\textcompwordmark{}

\paragraph*{Vertex Operators.}

We proceeded by defining worldsheet vertex operators $\mathcal{V}_{\Delta,m}^{\mathsf{a}}$.
The Picard group of the bosonic component of the target space,
\begin{equation}
\mathrm{Pic}(\mathbf{MT})\;\cong\;\mathbf{Z}\!\otimes\!\mathbf{Z},
\end{equation}
endows each vertex operator with a conformal weight $\Delta$. We
identified $\Delta$ with the celestial conformal weight of the corresponding
primary in the celestial CFT. This construction yielded a denumerable
family of vertex operators labelled by the string index $m=1,\dots,N$.
We therefore regard each $\mathcal{V}_{\Delta,m}^{\mathsf{a}}$ as
being ``attached'' to the minitwistor line $\mathcal{L}_{m}\subset\mathbf{MT}_{s}$
that represents the classical configuration of the $m$-th string.
To realise the dynamical picture described above, we built the (celestial)
gluon operators $\mathcal{G}_{\Delta,m}^{\eta,\mathsf{a}}$ from the
vertex operators $\mathcal{V}_{\Delta,m}^{\mathsf{a}}$.

Setting the background gauge field to zero isolates the interactions
that arise from worldsheet insertions. In this background-free limit
we computed the leading-trace semiclassical correlators of the $\mathcal{G}_{\Delta,m}^{\eta,\mathsf{a}}$
and reproduced the tree-level $\mathrm{N}^{k}\text{-MHV}$ leaf amplitude
for gluons. From this result we concluded that, for each MHV level
$k$, the semiclassical system of $N$ minitwistor strings is holographically
dual to the tree-level $\mathrm{N}^{k}\text{-MHV}$ gluonic sector
of $\mathcal{N}=4$ SYM whenever $N=2k+1$.

\textcompwordmark{}

\paragraph*{$S$-algebra.}

\citet{fotopoulos2019primary}, \citet{guevara2021holographic}, \citet{pate2021celestial}
and \citet{himwich2022celestial} showed that two independent arguments
imply that the vertex operator algebra generated by the primary fields
of the celestial CFT must obey an algebraic structure known as the
$S$-algebra\footnote{\citet{banerjee2023all} later classified all $S$-invariant celestial
OPEs for outgoing positive-helicity gluons; they also identified Knizhnik-Zamolodchikov-type
null states in theories that obey the $S$-algebra. }. One argument follows from the asymptotic-symmetry analysis of gauge
theory on flat space; the other follows from the structure of gluonic
collinear singularities studied by \citet{bern1999multi}. 

So, any dynamical model proposed as the holographic dual to flat-space
gauge theory must contain a vertex operator algebra whose primary
fields close on the $S$-algebra, irrespective of model-specific details.
In Subsection \ref{subsec:S-Algebra} we demonstrated that the celestial
gluon operators $\mathcal{G}_{\Delta,m}^{\eta,\mathsf{a}_{i}}$, constructed
from the minitwistor-string vertex operators $\mathcal{V}_{\Delta,m}^{\mathsf{a}_{i}}$,
realise the $S$-algebra at the level of primary contributions. These
operators furthermore contain a tower of corrections; their explicit
form is given in Eqs. (\ref{eq:-284}) and (\ref{eq:-285}). These
corrections coincide with those obtained by rewriting the celestial
leaf amplitudes as Feynman-Witten diagrams for massless scalars propagating
on $\mathrm{AdS}_{3}$, using the formalism of \citet{casali2022celestial}.

\textcompwordmark{}

\paragraph*{Speculations.}

From these observations it is natural to conjecture the existence
of a fully quantum-mechanical minitwistor sigma-model whose Hilbert
space decomposes into a denumerable family of sectors labelled by
$N$. We speculate that these sectors may arise either as fundamental
eigenstates or, perhaps, as coherent states; semiclassically each
such sector is approximated by the $N$-string system analysed above.
In this conjectural theory each sector $N$ of the full quantum sigma-model
would provide the holographic dual to the tree-level $\mathrm{N}^{k}\text{-MHV}$
gluonic sector of $\mathcal{N}=4$ SYM. 

An important feature of the $N$-string system is that the worldsheet
does not split into $N$ distinct components. Instead the worldsheet
remains a \emph{single} object, the celestial supersphere $\mathcal{CS}_{s}$.
Different values of $N$ are realised by varying the number of evaluation
maps and their associated embeddings into $\mathbf{MT}_{s}$ obtained
via the incidence relations. Thus the multiplicity of minitwistor
strings is encoded in the evaluation maps rather than in disconnected
worldsheet topologies.

One logical possibility is to introduce an infinite tower of evaluation
maps $\Pi_{m\dot{A}},\kappa_{m}^{\alpha}$ $(m\in\mathbf{N})$, and
to take the full geometric action of the theory to be given by:
\begin{equation}
\mathcal{S}^{\mathrm{\infty}}\big(\gamma^{Q}\big)\;=\;\sum_{m=1}^{\infty}\;\int_{\mathcal{CS}_{s,m}(\gamma^{Q})}\;D^{1|4}\mathsf{s}\wedge\left(\frac{1}{2\pi i}\;[\Sigma_{m}\,\overline{\partial}_{\sigma}\,\Xi_{m}]\;+\;[\Sigma_{m}|m,\gamma^{Q}]\right).
\end{equation}
Such a construction would produce an infinite tower of vertex operators
$\mathcal{V}_{\Delta,m}^{\mathsf{a}}$ $(m\in\mathbf{N})$. From these
one could then form the gluon vertex operators $\{\mathcal{G}_{\Delta,m}^{\eta,\mathsf{a}}\}_{m=1}^{\infty}$.
One may consider the leading-trace semiclassical correlators of these
gluon operators, which have the form:
\begin{equation}
\lim_{k\rightarrow0}\,\lim_{b\rightarrow0}\;\left\langle \prod_{i=1}^{n}\;\mathcal{G}_{2h_{i},c_{\vec{a}}(i)}^{\eta_{i},\mathsf{a}_{i}}\big(z_{i},\bar{z}_{i}\big)\right\rangle _{\mathcal{CS}},
\end{equation}
and which would generate tree-level gluon amplitudes at MHV level
$k$. In this formulation, those correlators are thought of as exciting
only the first $N$ entries of the full, countably-infinite tower
of minitwistor-string sectors contained in $\text{\ensuremath{\mathcal{S}^{\infty}}}$. 

Equivalently, one can regard the proposal as consisting of a single
worldsheet, the celestial supersphere $\mathcal{CS}_{s}$, together
with embedding maps into the target superspace
\begin{equation}
\mathbf{X}_{\infty}\;=\;\bigtimes_{m\in\mathbf{N}}\;\mathbf{MT}_{s,m}.
\end{equation}
In this picture, a finite-amplitude calculation would correspond to
restricting attention to the first $N$ factors of $\mathbf{X}_{\infty}$,
while the full action $\mathcal{S}^{\infty}$ encodes the infinite
geometric structure of the model.

Applying the BV-BRST formalism to a theory with worldsheet reparameterisation
invariance and an infinite family of fields will lead to serious difficulties.
This quantisation will require an infinite tower of gauge-fixing ghosts
and the corresponding antighosts. The resulting spectrum will likely
be non-unitary. Moreover, since each ghost system will contribute
to the total central charge, the theory will likely suffer from anomalies.
Hence, this renders the construction artificial.

Given these difficulties, a near-term goal is to construct a fully
quantum-mechanical, topological sigma-model on $\mathcal{CS}_{s}$
whose target is simply the minitwistor superspace $\mathbf{MT}_{s}$.
One should then verify whether the model admits a denumerable family
of sectors labelled by $N\in\mathbf{N}$, and check whether the sectors
of this putative sigma-model map onto the corresponding $\mathrm{N}^{k}\text{-MHV}$
sectors of SYM.

There is an indication that this picture can be realised. The celestial
correlator obtained above can be written as an integral over the moduli
superspace $\mathscr{M}_{N}$ of $N$ marked minitwistor lines as
follows:
\begin{equation}
\left\langle \prod_{i=1}^{n}\;\mathcal{G}_{2h_{i},c_{\vec{a}}(i)}^{\eta_{i},\mathsf{a}_{i}}\big(z_{i},\bar{z}_{i}\big)\right\rangle _{\mathcal{CS}}\;=\;\underset{\mathscr{M}_{N}\,\,\,\,\,}{\int}\;d\boldsymbol{\Omega}_{\vec{\alpha},S}\big(\gamma^{Q}\big)\;\left\langle \prod_{i=1}^{n}\;\mathcal{G}_{2h_{i},c_{\vec{a}}(i)}^{\eta_{i},\mathsf{a}_{i}}\big(z_{i},\bar{z}_{i}\big)\right\rangle _{\mathrm{WS}(\gamma^{Q})}.\label{eq:-236}
\end{equation}
Here $\langle\cdot\rangle_{\mathrm{WS}(\gamma^{Q})}$ denotes the
correlator of the worldsheet theory, which depends on the moduli $\gamma^{Q}\in\mathscr{M}_{N}$.

Now, a standard computation of the scattering matrix in bosonic string
theory, using the gauge-fixing procedure in the Polyakov pathintegral
(see \citet[Ch. 5]{polchinski1998string}), suggests the following
formal analogy. Using Polchinski's notation, one recalls that
\begin{equation}
S(1;\dots;n)\;\propto\;\sum_{\mathrm{topologies}}\;e^{-\lambda\chi_{E}}\;\underset{F\,\,\,\,\,}{\int}\;\frac{d^{m}t}{n_{R}}\;\left\langle \prod_{k=1}^{m}B_{k}\,\prod_{i=1}^{n}\hat{\mathscr{V}}_{i}\right\rangle _{\mathrm{WS}}.\label{eq:-282}
\end{equation}
Here $B_{k}$ denotes the $b$-ghost insertions, and $\hat{\mathscr{V}}_{i}$
stands for $\tilde{c}c\mathscr{V}_{\mathrm{m}}$ in the closed-string
case and $t_{\mathsf{a}}c^{\mathsf{a}}\mathscr{V}_{\mathrm{m}}$ in
the open-string case. Thus the amplitude is written as a moduli-space
integral of worldsheet correlators.

There is, therefore, a strong formal analogy between Eqs. (\ref{eq:-236})
and (\ref{eq:-282}). For our purposes this analogy suggests a useful
hint: the celestial leaf amplitudes obtained above as moduli-superspace
integrals of worldsheet correlators may admit a derivation from a
full quantum-mechanical minitwistor-sigma model.

To make this speculation a concrete research direction, we point out
the possibility of gauging the original twistor-string theory proposed
by \citet{berkovits2004alternative} by means of minitwistor rescaling
transformations. To simplify notation, let $\sigma$ denote an affine
coordinate on a local patch of the holomorphic celestial sphere $\mathcal{CS}\cong\mathbf{CP}^{1}$.
Introduce holomorphic, rational maps
\begin{equation}
\mathsf{Y}_{I},\mathsf{W}^{I}\;\colon\quad\mathbf{CP}^{1}\;\longrightarrow\mathbf{MT}_{s}
\end{equation}
which we take to be canonically conjugate field variables. In the
coordinate $\sigma$ these maps are parameterised by their component
fields,
\begin{equation}
\mathsf{Y}_{I}(\sigma)\;=\;\big(\omega_{A}(\sigma),\,\pi^{\dot{A}}(\sigma),\,\zeta_{\alpha}(\sigma)\big),\qquad\mathsf{W}^{I}(\sigma)\;\coloneqq\;\big(\lambda^{A}(\sigma),\,\mu_{\dot{A}}(\sigma),\,\psi^{\alpha}(\sigma)\big).
\end{equation}
The kinetic sector of the Berkovits action can then be written, in
spinor-helicity notation, as
\begin{equation}
\mathcal{I}_{0}[\mathsf{Y}_{I},\mathsf{W}^{I}]\;=\;\underset{\mathcal{CS}\,\,\,\,\,\,\,}{\int}\;d\sigma\wedge\big(\langle\omega\overline{\partial}_{\sigma}\lambda\rangle+[\pi\overline{\partial}_{\sigma}\mu]+\zeta\cdot\overline{\partial}_{\sigma}\psi\big).
\end{equation}
The minitwistor gauge transformations act on these fields by
\begin{equation}
\lambda^{A}\longmapsto t_{1}\,\lambda^{A},\qquad\mu_{\dot{A}}\longmapsto t_{2}\,\mu_{\dot{A}},\qquad\psi^{\alpha}\longmapsto t_{1}\,\psi^{\alpha}
\end{equation}
\begin{equation}
\omega_{A}\longrightarrow t_{1}^{-1}\,\omega_{A},\qquad\pi^{\dot{A}}\longrightarrow t_{2}^{-1}\,\pi^{\dot{A}},\qquad\zeta_{\alpha}\longmapsto t_{1}^{-1}\,\zeta_{\alpha}.
\end{equation}
Thus, to gauge the kinetic action we define the worldsheet $(0,1)$-currents
\begin{equation}
\boldsymbol{j}^{\sigma}\coloneqq\langle\omega\lambda\rangle+\zeta\cdot\psi,\qquad\boldsymbol{k}^{\sigma}\coloneqq[\pi\mu],
\end{equation}
and introduce the worldsheet connection $(0,1)$-forms $\boldsymbol{a}_{\sigma}$
and $\boldsymbol{b}_{\sigma}$. These gauge potentials transform under
a minitwistor gauge transformation as
\begin{equation}
\boldsymbol{a}_{\sigma}\longmapsto\boldsymbol{a}_{\sigma}-\overline{\partial}_{\sigma}\log t_{1},\qquad\boldsymbol{b}_{\sigma}\longmapsto\boldsymbol{b}_{\sigma}-\overline{\partial}_{\sigma}\log t_{2}.
\end{equation}
Hence we propose that a Berkovits-like minitwistor string theory may
be governed by the action
\begin{equation}
\mathcal{I}[\mathsf{Y}_{I},\mathsf{W}^{I};\boldsymbol{a}_{\sigma},\boldsymbol{b}_{\sigma}]\;=\;\mathcal{I}_{0}[\mathsf{Y}_{I},\mathsf{W}^{I}]\,+\,\underset{\mathcal{CS}\,\,\,\,\,\,\,}{\int}\;d\sigma\wedge\big(\boldsymbol{a}_{\sigma}\boldsymbol{j}^{\sigma}+\boldsymbol{b}_{\sigma}\boldsymbol{k}^{\sigma}\big)\,+\,\mathcal{I}_{\mathrm{CFT}}.\label{eq:-283}
\end{equation}
Here $\mathcal{I}_{\mathrm{CFT}}$ denotes an auxiliary matter CFT
that models the theory's phenomenology, contributes to the total central
charge, and is likely necessary to cancel or tame anomalies arising
from quantisation.

We further speculate that this putative celestial CFT, which would
serve as the full holographic dual to $\mathcal{N}=4$ SYM, need not
be formulated strictly on the celestial supersphere $\mathcal{CS}_{s}$.
Instead, it may be natural to take as worldsheet a supersymmetric
generalisation of the weighted projective line\footnote{See \citet{dolgachev2006weighted}.}.
Such a choice would assign to each gluon insertion point $z_{i},\bar{z}_{i}$
a conformal weight $\Delta_{i}$. We therefore leave to future work
the investigation of the full quantum-mechanical treatment of a minitwistor
sigma-model governed by the action (\ref{eq:-283}), where the worldsheet
is a weighted projective superline\footnote{Drawing from \citet{ogievetsky1987principal} and \citet{schubring2020sigma}
that sigma-models can be formulated with target spaces given by fibre
bundles, it is also possible to consider target spaces beyond the
minitwistor superspace. One may consider holomorphic vector superbundles
over $\mathbf{MT}_{s}$ as alternative targets.}.

\section{One-Loop Gauge Theory Amplitude}

\subsection{Introduction}

Consider a scattering process with $N\coloneqq m+n$ gluons in an
MHV configuration in $\mathcal{N}=4$ SYM theory. We study the one-loop
quantum correction to this process. This correction arises from an
MHV Feynman constructed via the off-shell celestial CSW prescription.

\paragraph*{External Gluons.}

The external gluons split into two sets: left $\left(L\right)$ and
right $\left(R\right)$.

Left gluons are labelled by $i=1,...,m$. Their spinor momenta are
$\lambda_{i}^{A}$ and $\bar{\lambda}_{i\dot{A}}$, and their helicity
states are encoded in Grassmann variables $\eta_{i}^{\alpha}$. Right
gluons are labelled by $j=1,...,n$. Their spinor momenta are $\nu_{j}^{A}$
and $\bar{\nu}_{j\dot{A}}$, and their helicity states are the fermionic
variables $\tilde{\eta}_{j}^ {}$
\begin{equation}
P_{L}^{A\dot{A}}\,\coloneqq\,\sum_{i=1}^{m}\,\lambda_{i}^{A}\bar{\lambda}_{i}^{\dot{A}},\,\,\,P_{R}^{A\dot{A}}\,\coloneqq\,\sum_{j=1}^{n}\,\nu_{j}^{A}\bar{\nu}_{j}^{\dot{A}}.\label{eq:-56}
\end{equation}


\paragraph*{Loop Configuration.}

To describe the momenta of the virtual gluons circulating in the loop
of the MHV Feynman diagram under consideration, we introduce the convention
that the two lines joining the MHV vertices are labeled by $k\in\{1,2\}$,
with associated four-momenta denoted by $L_{k}^{A\dot{A}}$. The \emph{celestial
CSW prescription }is as follows. Let the fixed auxiliary spinors be
defined by $\boldsymbol{n}^{A}\coloneqq\left(0,1\right)$ and $\bar{\boldsymbol{n}}_{\dot{A}}\coloneqq\left(0,1\right)$.
Then, the off-shell four-momentum $L_{k}^{A\dot{A}}$ is decomposed
as:
\begin{equation}
L_{k}^{A\dot{A}}\,\coloneqq\,s_{\ell_{k}}\ell_{k}^{A}\bar{\ell}_{k}^{\dot{A}}+r_{k}\boldsymbol{n}_{k}^{A}\bar{\boldsymbol{n}}_{k}^{\dot{A}},\label{eq:-57}
\end{equation}
where $s_{\ell_{k}},r_{k}\in\mathbf{R}_{+}^{n}$ and $\{\ell_{k}^{A},\bar{\ell}_{k\dot{A}}\}$
is a normalised spinor basis defined by $\ell_{k}^{A}\coloneqq\big(z_{\ell_{k}},-1\big)$
and $\bar{\ell}_{k\dot{A}}\coloneqq\big(1,-\bar{z}_{k}\big)$. Physically,
the parameters $s_{\ell_{k}}$ and $r_{k}$ play a role analogous
to frequency, while the variables $z_{\ell_{k}}$ and $\bar{z}_{\ell_{k}}$
parametrise the insertion point of the virtual gluon on the celestial
sphere in the framework of celestial CFT. In addition, we denote by
$\eta_{\ell_{k}}$ the helicity state of the $k$-th virtual gluon.

\paragraph*{Phase Superspace.}

An elementary calculation shows that the Lebesgue measure $d^{4}L_{k}$
on the phase space of the $k$-th virtual gluon, when normalised by
the massless scalar propagator $1L_{k}^{2}$, may be written in the
language of the celestial CSW prescription as:
\begin{equation}
\frac{d^{4}L_{k}}{L_{k}^{2}}\,=\,d^{2}\ell_{k}\wedge\frac{dr_{k}}{r_{k}}\wedge\frac{ds_{\ell_{k}}}{s_{\ell_{k}}}\,s_{\ell_{k}}^{2}.
\end{equation}
Note that the four-momentum $L_{k}^{A\dot{A}}$ and the helicity state
$\eta_{\ell_{k}}$ of the $k$-th virtual gluon can be combined into
a superchart with coordinate functions $\mathbb{L}_{k}^{Q}\coloneqq\big(L_{k}^{A\dot{A}},\eta_{k}^{\alpha}\big)$,
which parametrise the phase superspace $\mathbf{R}^{4|4}$. The orientation
on $\mathbf{R}^{4|4}$ is defined by the Berezin-de Witt volume superform:
\begin{equation}
d^{4|4}\mathbb{L}_{k}\,\coloneqq\,d^{4}L_{k}\wedge d^{0|4}\eta_{\ell k}.
\end{equation}


\paragraph*{Scattering Superamplitude.}

Let $\mathcal{A}$ denote the colour-stripped one-loop gluonic scattering
amplitude for the process under consideration. Applying the Feynman
rules to the MHV vertices yields:
\begin{equation}
\mathcal{A}\,=\,\left(2\pi\right)^{4}\,\delta^{4|0}\big(P_{L}^{A\dot{A}}+P_{R}^{A\dot{A}}\big)\underset{\mathbf{R}^{4|4}\times\mathbf{R}^{4|4}\,\,\,}{\int}\frac{d^{4|4}\mathbb{L}_{1}}{L_{1}^{2}}\wedge\frac{d^{4|4}\mathbb{L}_{2}}{L_{2}^{2}}\,\,\,\delta^{4|0}\big(L_{2}^{A\dot{A}}-L_{1}^{A\dot{A}}+P_{L}^{A\dot{A}}\big)\,\,\,\mathcal{A}_{L}\mathcal{A}_{R},\label{eq:-53}
\end{equation}
where the Parke-Taylor factors associated with the left and right
gluons are given by:
\begin{equation}
\mathcal{A}_{L}\,\coloneqq\,\frac{\delta^{0|8}\big(\mathcal{Q}_{L}^{\alpha A}\big)}{\left\langle \lambda_{1},\lambda_{2}\right\rangle \left\langle \lambda_{2},\lambda_{3}\right\rangle ...\left\langle \lambda_{m-1},\lambda_{m}\right\rangle \left\langle \lambda_{m},\ell_{2}\right\rangle \left\langle \ell_{2},\ell_{1}\right\rangle \left\langle \ell_{1},\lambda_{1}\right\rangle },\label{eq:-54}
\end{equation}
\begin{equation}
\mathcal{A}_{R}\,\coloneqq\,\frac{\delta^{0|8}\big(\mathcal{Q}_{R}^{\alpha A}\big)}{\left\langle \nu_{1},\nu_{2}\right\rangle \left\langle \nu_{2},\nu_{2}\right\rangle ...\left\langle \nu_{n-1},\nu_{n}\right\rangle \left\langle \nu_{n},\ell_{1}\right\rangle \left\langle \ell_{1},\ell_{2}\right\rangle \left\langle \ell_{2},\nu_{1}\right\rangle }.\label{eq:-55}
\end{equation}
The supercharges for the left and right gluons are defined as:
\begin{equation}
\mathcal{Q}_{L}^{\alpha A}\,\coloneqq\,\sum_{i=1}^{m}\,\lambda_{i}^{A}\eta_{i}^{\alpha}+\ell_{2}^{A}\eta_{\ell_{2}}^{\alpha}-\ell_{1}^{A}\eta_{\ell_{1}}^{\alpha},\label{eq:-58}
\end{equation}
\begin{equation}
\mathcal{Q}_{R}^{\alpha A}\,\coloneqq\,\sum_{j=1}^{n}\,\nu_{j}^{A}\tilde{\eta}_{j}^{\alpha}-\ell_{2}^{A}\eta_{\ell_{2}}^{\alpha}+\ell_{1}^{A}\eta_{\ell_{1}}^{\alpha}.\label{eq:-59}
\end{equation}

By combining Eqs. (\ref{eq:-53}), (\ref{eq:-54}) and (\ref{eq:-55}),
we obtain a single expression for the one-loop MHV amplitude. Expanding
the Berezin-de Witt volume superforms into their bosonic and fermionic
components, the amplitude takes the form:
\begin{align}
\mathcal{A} & \,=\,\left(2\pi\right)^{4}\,\delta^{4|0}\big(P_{L}^{A\dot{A}}+P_{R}^{A\dot{A}}\big)\,\,\,\underset{\mathbf{R}^{4}\times\mathbf{R}^{4}\,\,\,}{\int}\,\frac{d^{4}L_{1}}{L_{1}^{2}}\wedge\frac{d^{4}L_{2}}{L_{2}^{2}}\,\,\,\delta^{4|0}\big(L_{2}^{A\dot{A}}-L_{1}^{A\dot{A}}+P_{L}^{A\dot{A}}\big)\label{eq:-52}\\
 & \underset{\mathbf{R}^{0|4}\times\mathbf{R}^{0|4}\,\,\,\,\,}{\int}\,d^{0|4}\eta_{\ell_{1}}\wedge d^{0|4}\eta_{\ell_{2}}\,\,\,\delta^{0|4}\left(\mathcal{Q}_{L}^{\alpha A}\right)\wedge\delta^{0|4}\left(\mathcal{Q}_{R}^{\alpha A}\right)\,\,\,\mathcal{H}\big(\lambda_{i}^{A},\nu_{j}^{B},\ell_{k}^{C}\big)
\end{align}
Here, the holomorphic function $\mathcal{H}\coloneqq\mathcal{H}\big(\lambda_{i}^{A},\nu_{j}^{B},\ell_{k}^{C}\big)$
is obtained by multiplying the Parke-Taylor factors corresponding
to each MHV vertex. This function may be expressed compactly by introducing
the notation:
\begin{equation}
\lambda_{m+1}^{A}\coloneqq\ell_{2}^{A},\,\,\,\nu_{n+1}^{A}\coloneqq\ell_{1}^{A},\,\,\,\ell_{3}^{A}\coloneqq\ell_{1}^{A}.
\end{equation}
Thus,
\begin{equation}
\mathcal{H}\big(\lambda_{i}^{A},\nu_{j}^{B},\psi^{\alpha}\big)\,=\,\frac{1}{\left\langle \lambda_{1},\ell_{1}\right\rangle \left\langle \nu_{1},\ell_{2}\right\rangle }\,\prod_{i=1}^{m}\,\frac{1}{\lambda_{i}\cdot\lambda_{i+1}}\,\prod_{j=1}^{n}\,\frac{1}{\nu_{k}\cdot\nu_{k+1}}.\label{eq:-68}
\end{equation}


\subsection{Celestial Amplitude}

In the preceding section, we specified the gluon configuration and
detailed the structure of the Feynman diagram to be computed. We now
translate from the conventional plane-wave representation of scattering
amplitudes to their formulation in terms of celestial conformal primaries
via the half-Mellin transform. Consequently, we derive an expression
for the one-loop celestial superamplitude $\widehat{\mathcal{A}}_{\ell(1)}$
constructed from MHV vertices.

As anticipated from the behaviour of quantum field theories containing
massless vector bosons, the amplitude $\widehat{\mathcal{A}}_{\ell(1)}$
exhibits infrared divergences. To isolate the finite contributions,
we introduce a regularisation scheme \emph{analogous} to dimensional
regularisation, but adapted for celestial amplitudes. In our approach,
regularisation is effected by introducing an infinitesimal celestial
conformal weight $\Delta_{\varepsilon}$ for the virtual gluons circulating
in the loop $\ell_{1}\cup\ell_{2}$, rather than by analytically continuing
the spacetime dimension. Once the regularised celestial amplitude
$:\widehat{\mathcal{A}}_{\ell(1)}:$ is obtained, we proceed to apply
the leaf amplitude formalism in the subsequent subsection.

\subsubsection{Preliminaries}

\paragraph*{Momentum-Conserving Delta-Functions.}

The first step in our calculation is to invoke the integral representation
of the four-dimensional delta-function:
\begin{equation}
\delta^{4|0}\big(P^{A\dot{A}}\big)\,=\,\underset{\mathbf{R}^{4}\,\,\,}{\int}\,\frac{d^{4}x}{\left(2\pi\right)^{4}}\,\,\,\exp\big(ix\cdot P\big),\,\,\,x\cdot P\,\coloneqq\,x_{A\dot{A}}P^{A\dot{A}}.
\end{equation}
Using the definitions provided in Eq. (\ref{eq:-56}), we derive the
following representation for the bosonic momentum-conserving delta-function
for the external gluons appearing in Eq. (\ref{eq:-52}):
\begin{equation}
\delta^{4|0}\big(P_{L}^{A\dot{A}}+P_{R}^{A\dot{A}}\big)\,=\,\underset{\mathbf{R}^{4}\,\,\,}{\int}\,\frac{d^{4}x}{\left(2\pi\right)^{4}}\,\,\,\prod_{i=1}^{m}\,\exp\big(i\langle\lambda_{i}|x|\bar{\lambda}_{i}]\big)\,\prod_{j=1}^{n}\,\exp\big(i\langle\nu_{j}|x|\bar{\nu}_{j}]\big).\label{eq:-61}
\end{equation}
Here, the use of the product symbol $\prod_{i}$ is justified because
the arguments of the exponentials are entirely bosonic. 

Similarly, recalling Eq. (\ref{eq:-57}) for the loop momenta $L_{k}^{A\dot{A}}$
$\left(k\in\{1,2\}\right)$, the momentum-conserving delta-function
inside the loop of the MHV Feynman diagram is given by: 
\begin{align}
 & \delta^{4|0}\big(L_{2}^{A\dot{A}}-L_{1}^{A\dot{A}}+P_{L}^{A\dot{A}}\big)\label{eq:-64}\\
 & \,=\,\underset{\mathbf{R}^{4}\,\,\,}{\int}\,\frac{d^{4}\tilde{x}}{\left(2\pi\right)^{4}}\,\,\,\exp\left(is_{\ell_{2}}\langle\ell_{2}|\tilde{x}|\bar{\ell_{2}}]-is_{\ell_{1}}\langle\ell_{1}|\tilde{x}|\bar{\ell}_{1}]+i\left(r_{2}-r_{1}\right)\langle n|\tilde{x}|\bar{n}]\right)\prod_{i=1}^{m}\exp\left(i\langle\lambda_{i}|\tilde{x}|\bar{\lambda}_{i}]\right).\label{eq:-65}
\end{align}


\paragraph{Supercharge Conservation.}

Next, we address the fermionic delta functions associated with supercharge
conservation. Given a Grassmann-valued van der Waerden spinor variable
$\chi_{A}^{\alpha}$, the fermionic delta function $\delta^{0|8}\big(\chi_{A}^{\alpha}\big)$
is formally defined by the expansion:
\begin{equation}
\delta^{0|8}\big(\chi_{A}^{\alpha}\big)\,\coloneqq\,\frac{1}{2^{4}}\,\bigwedge_{\alpha=1}^{4}\,\varepsilon^{AB}\,\chi_{A}^{\alpha}\,\wedge\,\chi_{B}^{\alpha}\,.\label{eq:-60}
\end{equation}
Because $\chi_{A}^{\alpha}$ is an element of a $\mathbf{Z}_{2}$-graded
exterior algebra, its multiplication is understood in terms of the
wedge product. 

By substituting the definitions from Eqs. (\ref{eq:-58}) and (\ref{eq:-59})
for the supercharges $\mathcal{Q}_{L}^{\alpha A}$ and $\mathcal{Q}_{R}^{\alpha A}$
into Eq. (\ref{eq:-60}), we deduce:
\begin{equation}
\delta^{0|8}\big(\mathcal{Q}_{L}^{\alpha A}\big)\,=\,\int\,d^{0|8}\theta\,\,\,\exp\left(i\langle\ell_{2}|\theta\cdot\eta_{\ell_{2}}\rangle-i\langle\ell_{1}|\theta\cdot\eta_{\ell_{1}}\rangle\right)\,\bigwedge_{i=1}^{m}\,\exp\big(i\langle\lambda_{i}|\theta\cdot\eta_{i}\rangle\big),\label{eq:-63}
\end{equation}
and:
\begin{equation}
\delta^{0|8}\big(\mathcal{Q}_{R}^{\alpha A}\big)\,=\,\int\,d^{0|8}\tilde{\theta}\,\,\,\exp\big(i\langle\ell_{1}|\tilde{\theta}\cdot\eta_{\ell_{1}}\rangle-i\langle\ell_{2}|\tilde{\theta}\cdot\eta_{\ell_{2}}\rangle\big)\,\bigwedge_{j=1}^{n}\,\exp\big(i\langle\nu_{j}|\tilde{\theta}\cdot\tilde{\eta}_{j}\rangle\big).\label{eq:-62}
\end{equation}


\subsubsection{Integral Representation of One-Loop Superamplitude}

Substituting Eqs. (\ref{eq:-61})--(\ref{eq:-62}) into Eq. (\ref{eq:-52})
yields an integral representation for $\mathcal{A}_{\ell(1)}$. The
resulting expression is lengthy but can be Mellin-transformed, which
is required to derive the corresponding celestial superamplitude $\widehat{\mathcal{A}}_{\ell(1)}$.
Owing to its complexity, we begin by describing the integration superdomain
over which $\mathcal{A}_{\ell(1)}$ is defined, which organises the
expression in a more systematic way.

\paragraph{Integration Superdomain.}

First, observe that the bosonic integration variables $x_{A\dot{A}}$
and the fermionic components $\theta_{A}^{\alpha}$ appearing in the
momentum- and supercharge-conserving delta functions (Eqs. (\ref{eq:-61})
and (\ref{eq:-63})) can be combined into a single set of superspace
coordinates, $\mathsf{x}^{K}\coloneqq\big(x_{A\dot{A}},\theta_{A}^{\alpha}\big)$,
which parametrise the integration superdomain $\mathbf{R}^{4|8}$.
Here, the abstract index $K$ runs over $\{(A\dot{A}),(\alpha A)\}$.
The orientation on $\mathbf{R}^{4|8}$ is given by the Berezin-de
Witt volume superform:
\begin{equation}
d^{4|8}\mathsf{x}\,\coloneqq\,d^{4}x\wedge d^{0|8}\theta.\label{eq:-66}
\end{equation}
Similarly, the variables $\tilde{x}_{A\dot{A}}$ and $\tilde{\theta}_{A}^{\alpha}$,
which arises from the loop-momentum-conserving and the right-gluon
supercharge-conserving delta functions (Eqs. (\ref{eq:-65}) and (\ref{eq:-62})),
are unified into the superspace coordinates $\tilde{\mathsf{x}}^{K}\coloneqq\big(\tilde{x}_{A\dot{A}},\tilde{\theta}_{A}^{\alpha}\big)$,
with orientation on $\widetilde{\mathbf{R}}^{4|8}$ defined by:
\begin{equation}
d^{4|8}\tilde{\mathsf{x}}\,\coloneqq\,d^{4}\tilde{x}\wedge d^{0|8}\tilde{\theta}.
\end{equation}
As we shall show, performing a dimensional reduction on the coordinates
$\mathsf{x}^{K}$ and $\tilde{\mathsf{x}}^{K}$ that parametrise the
integration superdomain, following the prescriptions of the leaf amplitude
formalism, yields the moduli parameters on $\mathbf{RP}^{3|8}\times\widetilde{\mathbf{RP}}^{3|8}\times\mathbf{R}_{+}$
that characterise the minitwistor lines on which the one-loop celestial
superamplitudes localise.

To continue, the loop spinor-momenta $\ell_{k}^{A}$ and $\bar{\ell}_{k\dot{A}}$,
together with the helicity state $\eta_{k}^{\alpha}$ of the $k$-th
virtual gluon, may be assembled into a dual minitwistor variable,
$\mathsf{W}_{\ell_{k}}^{I}\,\coloneqq\,\big(\ell_{k}^{A},\bar{\ell}_{k\dot{A}},\eta_{k}^{\alpha}\big)$,
which is associated with the $\mathbf{Z}_{2}$-graded volume form:
\begin{equation}
d^{2|4}\mathsf{W}_{\ell_{k}}\,=\,d\ell_{k}\wedge d\bar{\ell}_{k}\wedge d^{0|4}\eta_{\ell_{k}}.\label{eq:-67}
\end{equation}
Note that $d\ell_{k}\wedge d\bar{\ell}_{k}$ is the Lebesgue measure
on $\mathbf{R}^{2}$, rather than the canonical holomorphic measure
on $\mathbf{RP}^{1}$. This definition is necessary because we integrate
over all \emph{off-shell} loop momenta $L_{1}^{A\dot{A}}$ and $L_{2}^{A\dot{A}}$
in accordance with our celestial version of the CSW prescription.
Moreover, our integration domain is real rather than complex since,
in preparing for the application of the leaf amplitude formalism,
we have assumed Kleinian signature instead of Minkowski signature.

Finally, the volume superform used to express the amplitude $\mathcal{A}_{\ell(1)}$
is obtained by combining all the previously defined measures (Eqs.
(\ref{eq:-66}) and (\ref{eq:-67})) into:
\begin{equation}
d\mathcal{V}\,\coloneqq\,d^{4|8}\mathsf{x}\wedge d^{4|8}\tilde{\mathsf{x}}\wedge d^{2|4}\mathsf{W}_{\ell_{1}}\wedge d^{2|4}\mathsf{W}_{\ell_{2}},
\end{equation}
which defines the orientation of the integration superdomain:
\begin{equation}
\mathscr{D}\coloneqq\mathbf{R}^{4|8}\times\widetilde{\mathbf{R}}^{4|8}\times\mathbf{R}^{2|4}\times\mathbf{R}^{2|4}.
\end{equation}


\paragraph{Explicit Expression for $\mathcal{A}_{\ell(1)}$.}

With these structures in place, we find that the superamplitude can
be formulated as:
\begin{align}
 & \mathcal{A}_{\ell(1)}\,=\,\frac{1}{\left(2\pi\right)^{4}}\,\underset{\mathscr{D}\,\,\,\,\,}{\int}\,d\mathcal{V}\,\,\,\Sigma\left(\tilde{x}\right)\,\mathcal{G}\big(\tilde{x}|\ell_{k},\bar{\ell_{k}}\big)\,\mathcal{H}\big(\lambda_{i},\nu_{j},\ell_{k}\big)\,\exp\left(i\langle\ell_{1}|\theta_{\ell_{1}}\cdot\eta_{\ell_{1}}\rangle+i\langle\ell_{2}|\theta_{\ell_{2}}\cdot\eta_{\ell_{2}}\rangle\right)\label{eq:-69}\\
 & \bigwedge_{i=1}^{m}\,\,\,\exp\big(i\langle\lambda_{i}|x+\tilde{x}|\bar{\lambda}_{i}]+i\langle\lambda_{i}|\theta\cdot\eta_{i}\rangle\big)\,\,\,\bigwedge_{j=1}^{n}\,\,\,\exp\big(i\langle\nu_{j}|x|\bar{\nu}_{j}]+i\langle\nu_{j}|\tilde{\theta}_{j}\cdot\tilde{\eta}_{j}\rangle\big).
\end{align}
In this expression, the function $\mathcal{H}$ of the spinor momenta
is as defined in Eq. (\ref{eq:-68}). The new objects are introduced
as follows. 

First, the helicity state $\eta_{\ell_{1}}^{\alpha}$ of the virtual
gluon associated with the line $\ell_{1}$ couples to the Grassmann-valued
two-component spinor variable defined by $\theta_{\ell_{1}}^{\alpha A}\coloneqq\tilde{\theta}^{\alpha A}-\theta^{\alpha A}$.
Similarly, the helicity state $\eta_{\ell_{2}}^{\alpha}$ of the virtual
gluon circulating along the line $\ell_{2}$ couples to $\theta_{\ell_{2}}^{\alpha A}\coloneqq-\theta_{\ell_{1}}^{\alpha A}$. 

The quantity $\Sigma$ is defined as a function of the coordinates
$\tilde{x}_{A\dot{A}}$ by: 
\begin{equation}
\Sigma\big(\tilde{x}_{A\dot{A}}\big)\,\coloneqq\,\underset{\mathbf{R}_{+}^{2}}{\int}\,\frac{dr_{1}}{r_{1}}\wedge\frac{dr_{2}}{r_{2}}\,\,\,\exp\left(i\left(r_{2}-r_{1}\right)\langle\boldsymbol{n}|\tilde{x}|\bar{\boldsymbol{n}]}\right),
\end{equation}
and this integral \emph{diverges}. In fact, $\Sigma$ contains the
infrared divergence of the amplitude $\mathcal{A}_{\ell(1)}$ that
is anticipated from the behaviour of gauge field theories. In conventional
approaches, such divergences are isolated from the finite contributions
via dimensional regularisation. In our scheme, which may be better
suited for celestial holography, we introduce an infinitesimal celestial
conformal weight $\Delta_{\varepsilon}$ (with the understanding that
$\Delta_{\varepsilon}\rightarrow0^{+}$ as $\varepsilon\rightarrow0$)
associated with the virtual gluons circulating in the loop $\ell_{1}\cup\ell_{2}$,
so that $\Sigma$ is replaced by the $\varepsilon$-regularised expression:
\begin{equation}
\Sigma_{\varepsilon}\big(\tilde{x}\big)\,\coloneqq\,\underset{\mathbf{R}_{+}^{2}}{\int}\,\frac{dr_{1}}{r_{1}}\wedge\frac{dr_{2}}{r_{2}}\,\,\,r_{1}^{\Delta_{\varepsilon}}r_{2}^{\Delta_{\varepsilon}}\,\exp\left(i\left(r_{2}-r_{1}\right)\langle\boldsymbol{n}|\tilde{x}|\bar{\boldsymbol{n}}]\right).\label{eq:-71}
\end{equation}
A straightforward evaluation of this integral yields:
\begin{equation}
\Sigma_{\varepsilon}\big(\tilde{x}\big)\,=\,\frac{\Gamma\left(\Delta_{\varepsilon}\right)^{2}}{\langle\boldsymbol{n}|\tilde{x}|\bar{\boldsymbol{n}]^{2\Delta_{\varepsilon}}}}.
\end{equation}
In what follows, we denote by $\mathcal{A}_{\ell(1)}^{\varepsilon}$
the $\varepsilon$-regularised one-loop gluonic superamplitude, obtained
by replacing $\Sigma(\tilde{x})$ with $\Sigma_{\varepsilon}(\tilde{x})$
in Eq. (\ref{eq:-69}).

The function $\mathcal{G}$ is defined by:
\begin{equation}
\mathcal{G}\big(\tilde{x}_{A\dot{A}}|\ell_{k}^{A},\bar{\ell}_{k\dot{A}}\big)\,\coloneqq\,\underset{\mathbf{R}_{+}^{2}\,\,\,}{\int}\,\frac{ds_{\ell_{1}}}{s_{\ell_{1}}}\wedge\frac{ds_{\ell_{2}}}{s_{\ell_{2}}}\,\,\,s_{\ell_{1}}^{2}s_{\ell_{2}}^{2}\,\exp\big(is_{\ell_{2}}\langle\ell_{2}|\tilde{x}|\bar{\ell}_{2}]-is_{\ell_{1}}\langle\ell_{1}|\tilde{x}|\bar{\ell}_{2}]\big).\label{eq:-72}
\end{equation}
Note that $\mathcal{G}$ assumes the standard form of a Mellin transform
when the variables $s_{\ell_{1}}$ and $s_{\ell_{2}}$ are interpreted
as frequency-like parameters. In this case, the integral can be evaluated
straightforwardly, yielding:
\begin{equation}
\mathcal{G}\left(\tilde{x}|\ell_{k},\bar{\ell}_{k}\right)\,=\,\frac{\mathcal{C}\left(2\right)}{\langle\ell_{1}|\tilde{x}|\bar{\ell}_{1}]^{2}}\frac{\mathcal{C}\left(2\right)}{\langle\ell_{2}|\tilde{x}|\bar{\ell}_{2}]^{2}}.
\end{equation}


\subsubsection{Half-Mellin Transform}

We now proceed to derive the $\varepsilon$-regularised one-loop celestial
superamplitude $\widehat{\mathcal{A}}_{\ell(1)}^{\varepsilon}$ by
applying the Mellin transform with respect to the dotted spinor momenta
$\bar{\lambda}_{i\dot{A}}$ (for $1\leq i\leq n$) and $\bar{\nu}_{i\dot{A}}$
(for $1\leq j\leq m$). 

\paragraph*{Preliminaries.}

To this end, consider the $i$-th gluon in the left subset $L$. Let
$\{z_{i}^{A},\bar{z}_{i\dot{A}}\}$ denote a normalised basis of van
der Waerden spinors that parametrises the insertion point of the $i$-th
gluon on the holomorphic celestial sphere $\mathcal{CS}\simeq\mathbf{CP}^{1}$.
Furthermore, let $s_{i}\in\mathbf{R}_{+}$ represent the frequency
associated with this gluon. Thus, we perform the substitution on $\mathcal{A}_{\ell(1)}^{\varepsilon}$
(as given in Eq. (\ref{eq:-69})):
\begin{equation}
\lambda_{i}^{A}\mapsto z_{i}^{A},\,\,\,\bar{\lambda}_{i\dot{A}}\mapsto s_{i}\bar{z}_{i\dot{A}}.
\end{equation}
Recall that, in the context of celestial CFT, the quantum state of
the gluon $i\in L$ is completely specified by $z_{i}^{A}$, $\bar{z}_{i\dot{A}}$,
and the Grassmann variable $\eta_{i}^{\alpha}$ encoding its helicity.
These quantum numbers are naturally assembled into the dual minitwistor:
\begin{equation}
\mathsf{W}_{\text{L},i}^{I}\,\coloneqq\,\big(z_{i}^{A},\bar{z}_{i\dot{A}},\eta_{i}^{\alpha}\big).
\end{equation}

Analogously, consider the $j$-th gluon in the right subset $R$.
We denote by $\{w_{j}^{A},\bar{w}_{j\dot{A}}\}$ the normalised spinor
basis that characterises the $j$-th insertion point on the celestial
sphere $\mathcal{CS}$, and let $t_{j}\in\mathbf{R}_{+}$ denote the
frequency of the $j$-th gluon. We then substitute in $\widehat{\mathcal{A}}_{\ell(1)}^{\varepsilon}$:
\begin{equation}
\nu_{j}^{A}\mapsto w_{j}^{A},\,\,\,\bar{\nu}_{j\dot{A}}\mapsto t_{j}\bar{w}_{j\dot{A}}.
\end{equation}
Moreover, in celestial CFT, the configuration of the gluon $j\in R$
is specified by the ordered pair $\big(w_{j}^{A},\bar{w}_{j\dot{A}}\big)$
of van der Waerden spinors, and the fermionic variable $\tilde{\eta}_{j}^{\alpha}$
describing its helicity, which defines the dual minitwistor:
\begin{equation}
\mathsf{W}_{\text{R},j}^{I}\,\coloneqq\,\big(w_{i}^{A},\bar{w}_{j\dot{A}},\eta_{j}^{\alpha}\big).
\end{equation}

Finally, to fully characterise a gluon in celestial CFT, one must
specify its celestial conformal weight. For a left-gluon $i\in L$,
let $\Delta_{\text{L},i}$ denote its conformal weight and $\epsilon_{\text{L},i}$
its helicity expectation value. The scaling dimension of the $i$-th
left-gluon is defined as:
\begin{equation}
h_{\text{L},i}\,\coloneqq\,\frac{\Delta_{\text{L},i}+\epsilon_{\text{L},i}}{2}.
\end{equation}
Similarly, for a right-moving gluon $j\in R$, let $\Delta_{\text{R},j}$
and $\epsilon_{\text{R},j}$ represent its conformal weight and helicity
expectation value, respectively. The scaling dimension of the $j$-th
right-gluon is given by:
\begin{equation}
h_{\text{R},j}\,\coloneqq\,\frac{\Delta_{\text{R},j}+\epsilon_{\text{R},j}}{2}.
\end{equation}


\paragraph*{Celestial Amplitude.}

Accordingly, the $\varepsilon$-regularised one-loop celestial superamplitude
$\widehat{\mathcal{A}}_{\ell(1)}^{\varepsilon}$ is defined via the
half-Mellin transform as:
\begin{equation}
\widehat{\mathcal{A}}_{\ell(1)}^{\varepsilon}\big(\mathsf{W}_{L,i}^{I};\mathsf{W}_{R,j}^{J}\big)\,\coloneqq\,\underset{\mathbf{R}_{+}^{N}\,\,\,}{\int}\,d\boldsymbol{\mu}(s_{i},t_{j})\,\,\,\mathcal{A}_{\ell(1)}^{\varepsilon}\big(z_{i}^{A},s_{i}\bar{z}_{i\dot{A}},\eta_{i}^{\alpha};w_{j}^{A},t_{j}\bar{w}_{j\dot{A}},\tilde{\eta}_{j}^{\alpha}\big),\label{eq:-73}
\end{equation}
where the integration measure over $\mathbf{R}_{+}^{N}$ (with $N=n+m$)
is specified by:
\begin{equation}
d\boldsymbol{\mu}(s_{i},t_{j})\,\coloneqq\,\bigwedge_{i=1}^{n}\,\,\,\frac{ds_{i}}{s_{i}}\,s_{i}^{2h_{\text{L},i}}\,\bigwedge_{j=1}^{m}\,\,\,\frac{dt_{j}}{t_{j}}\,t_{i}^{2h_{\text{R},j}}.
\end{equation}
Notice that the factor $\bigwedge_{i=1}^{n}\frac{ds_{i}}{s_{i}}$
defines the Haar measure on the direct product group $\prod_{i=1}^{n}\mathbf{R}_{+}^{\times}$,
where $\mathbf{R}_{+}^{\times}\coloneqq\left(\mathbf{R}_{+},\cdot\right)$
denotes the multiplicative group of positive real numbers.

Additionally, the renormalised celestial amplitude is defined as the
asymptotic limit:
\begin{equation}
:\widehat{\mathcal{A}}_{\ell(1)}^{\varepsilon}:\big(\mathsf{W}_{\text{L},i}^{I};\mathsf{W}_{\text{R},j}^{J}\big)\,=\,\lim_{\varepsilon\rightarrow0^{+}}\,\frac{\widehat{\mathcal{A}}_{\ell(1)}^{\varepsilon}\big(\mathsf{W}_{\text{L},i}^{I};\mathsf{W}_{\mathbf{\text{R}},j}^{J}\big)}{\Gamma(\Delta_{\varepsilon})^{2}}.\label{eq:-70}
\end{equation}


\paragraph*{Integral Representation.}

By performing the integral transform specified in Eq. (\ref{eq:-73})
and substituting the explicit expressions for $\Sigma_{\varepsilon}$,
$\mathcal{G}$ and $\mathcal{H}$ from Eqs. (\ref{eq:-71}), (\ref{eq:-72})
and (\ref{eq:-68}) into Eq. (\ref{eq:-70}), we obtain an integral
representation for the celestial amplitude $\widehat{\mathcal{A}}_{\ell(1)}\big(\mathsf{W}_{\text{L},i};\mathsf{W}_{\text{R},j}\big)$
that we now describe.

To define the integration superdomain, we introduce the normalised
$\mathbf{Z}_{2}$-graded volume forms:
\begin{equation}
\mathcal{D}^{2|4}\mathsf{W}_{\ell_{1}}\,\coloneqq\,\frac{1}{z_{1}\cdot\ell_{1}}\,d^{2|4}\mathsf{W}_{\ell_{1}},\,\,\,\mathcal{D}^{2|4}\mathsf{W}_{\ell_{2}}\,\coloneqq\,\frac{1}{w_{1}\cdot\ell_{2}}\,d^{2|4}\mathsf{W}_{\ell_{2}}.
\end{equation}
Notice that these measures are invariant under the rescalings $\ell_{1}^{A}\mapsto t_{1}\ell_{1}^{A}$
and $\ell_{2}^{A}\mapsto t_{2}\ell_{2}^{A}$, respectively. Then,
we redefine the volume element on the integration superdomain $\mathscr{D}$
as:
\begin{equation}
d\boldsymbol{V}\,\coloneqq\,d^{4|8}\mathsf{x}\wedge d^{4|8}\tilde{\mathsf{x}}\wedge\mathcal{D}^{2|4}\mathsf{W}_{\ell_{1}}\wedge\mathcal{D}^{2|4}\mathsf{W}_{\ell_{2}}.
\end{equation}

Consequently, the renormalised celestial superamplitude $:\widehat{\mathcal{A}}_{\ell(1)}:$
can be written as:
\begin{equation}
:\widehat{\mathcal{A}}_{\ell(1)}:\big(\mathsf{W}_{\text{L},i}^{I};\mathsf{W}_{\text{R},j}^{J}\big)\,=\,\frac{1}{\left(2\pi\right)^{4}}\underset{\mathcal{\mathscr{D}}\,\,\,\,\,}{\int}\,d\boldsymbol{V}\,\,\,\mathcal{E}_{L}\big(\mathsf{W}_{\text{L},i}^{I}\big)\mathcal{E}_{R}\big(\mathsf{W}_{\text{R},j}^{J}\big)\mathcal{E}_{\ell}\big(\mathsf{W}_{\ell_{k}}^{K}\big).
\end{equation}
The integrand is expressed in terms of the following auxiliary functions:
\begin{equation}
\mathcal{E}_{L}\big(\mathsf{W}_{\text{L},i}^{I}\big)\,\coloneqq\,\bigwedge_{i=1}^{m}\,\frac{\mathcal{C}\left(2h_{\text{L},i}\right)}{\langle z_{i}|x+\tilde{x}|\bar{z}_{i}]^{2h_{\text{L},i}}}\,e^{i\langle z_{i}|\theta\cdot\eta_{i}\rangle}\,\frac{1}{z_{i}\cdot z_{i+1}},\,\,\,\label{eq:-80}
\end{equation}
\begin{equation}
\mathcal{E}_{R}\big(\mathsf{W}_{\text{R},j}^{I}\big)\,\coloneqq\,\bigwedge_{j=1}^{n}\,\frac{\mathcal{C}(2h_{\text{R},j})}{\langle w_{j}|x|\bar{w}_{j}]^{2h_{\text{R},j}}}\,e^{i\langle w_{j}|\tilde{\theta}\cdot\tilde{\eta}_{j}\rangle}\,\frac{1}{w_{j}\cdot w_{j+1}},\label{eq:-81}
\end{equation}
\begin{equation}
\mathcal{E}_{\ell}\big(\mathsf{W}_{\ell_{k}}^{I}\big)\,\coloneqq\,\bigwedge_{k\in\{1,2\}}\,\frac{\mathcal{C}\left(2\right)}{\langle\ell_{k}|\tilde{x}|\bar{\ell_{k}}]^{2}}\,e^{i\langle\ell_{k}|\theta_{\ell_{k}}\cdot\eta_{\ell_{k}}\rangle}\,\frac{1}{\ell_{k}\cdot\ell_{k+1}}.\label{eq:-82}
\end{equation}
In other words, each auxiliary function $\mathcal{E}$ is the product
of an $\mathcal{N}=4$ supersymmetric gluonic celestial wavefunction
and the corresponding Parke-Taylor factor.

\subsection{Leaf Amplitude Formulation}

Finally, we apply the leaf amplitude formalism to the renormalised
celestial superamplitude $\widehat{\mathcal{A}}_{\ell(1)}\big(\mathsf{W}_{\text{L},i};\mathsf{W}_{\text{R},j}\big)$.
From a physical perspective, this procedure is equivalent to performing
a dimensional reduction on the superspace coordinates $\mathsf{x}^{K}$
and $\tilde{\mathsf{x}}^{K}$ that parametrise the integration superdomain
$\mathbf{R}^{4|8}\times\widetilde{\mathbf{R}}^{4|8}$. This reduction
yields an integral representation for the celestial amplitude in terms
of coordinates $\mathbb{X}^{K}$, $\widetilde{\mathbb{X}}^{K}$ and
$v$ that parametrise the integration superspace:
\begin{equation}
\mathscr{M}_{\ell(1)}\,\coloneqq\,\mathbf{RP}^{3|8}\times\widetilde{\mathbf{RP}}^{3|8}\times\mathbf{R}_{+}.
\end{equation}
Moreover, when the amplitude is expressed in terms of minitwistor
superwavefunctions $\Psi_{\Delta}^{p}\big(\mathsf{Z};\mathsf{W}\big)$,
the superspace $\mathscr{M}_{\ell(1)}$ is identified with the moduli
space of a closed configuration of minitwistor superlines, over which
the celestial leaf amplitude localises. Thus, the transition from
physical celestial amplitudes to leaf amplitudes corresponds to a
dimensional reduction of the moduli space of nodal twistor lines,
resulting in a closed configuration of minitwistor lines.

\subsubsection{Kleinian and Projective Superspaces}

Before applying the leaf formalism, it is important to define precisely
the geometric structures on the supermanifolds upon which the celestial
leaf superamplitudes are to be constructed.

Thus, let $X_{A\dot{A}}$ denote homogeneous coordinates on the three-dimensional
real projective space $\mathbf{RP}^{3}$, and let $u$ denote an affine
parameter on the multiplicative group of positive reals $\mathbf{R}_{+}^{\times}$.
In addition, define the ``projective'' coordinates on $\mathbf{RP}^{3}$
by: 
\begin{equation}
\mathcal{R}_{A\dot{A}}\,\coloneqq\,\frac{X_{A\dot{A}}}{\left|X\right|},
\end{equation}
which are invariant under rescalings $X_{A\dot{A}}\mapsto tX_{A\dot{A}}$. 

\paragraph*{Volume Forms.}

Recall that Klein space $\mathbf{R}^{(2,2)}$ is naturally partitioned
into the lightcone $\Lambda$ and the timelike and spacelike wedges,
denoted by $W^{-}$ and $W^{+}$, respectively. In particular, $W^{-}$
is the set of all points $x_{A\dot{A}}\in\mathbf{K}^{4}$ such that
$x^{2}\coloneqq x_{A\dot{A}}x^{A\dot{A}}<0$. We now introduce a coordinate
system on $W^{-}$ by defining $\mathcal{X}\coloneqq\big(u,\mathcal{R}_{A\dot{A}}\big)$,
so that for every point $p\in W^{-}$ we have:
\begin{equation}
\mathcal{X}(p)\,\coloneqq\,\big(u\left(p\right),\mathcal{R}_{A\dot{A}}\left(p\right)\big)\in\mathbf{R}_{+}\times\mathbf{RP}^{3}.
\end{equation}
The coordinates $\mathcal{X}$ are related to the global rectangular
coordinates $x_{A\dot{A}}$ on $\mathbf{K}^{4}$ via:
\begin{equation}
x_{A\dot{A}}(p)\,=\,u(p)\,\mathcal{R}_{A\dot{A}}(p),\,\,\,\text{for all}\,\,\,p\in W^{-}.
\end{equation}
In terms of $\mathcal{X}$, the restriction of the Lebesgue measure
$d^{4}x$ to $W^{-}$ decomposes as:
\begin{equation}
d^{4}x\big|_{W^{-}}\,=\,\frac{D^{3}X}{\left|X\right|^{4}}\wedge\frac{du}{u}\,u^{4}.\label{eq:-75}
\end{equation}
Moreover, since the standard measure on $\mathbf{R}^{4|8}$ is given
by $d^{4|8}\mathsf{x}=d^{4}x\wedge d^{0|8}\theta$, the volume element
on the supersymmetric extension of the timelike wedge $\mathbf{W}^{-}\subset\mathbf{R}^{4|8}$
takes the form:
\begin{equation}
d^{4|8}\mathsf{x}|_{\mathbf{W}^{-}}\,=\,\frac{D^{3}X}{\left|X\right|^{4}}\wedge d^{0|8}\theta\wedge\frac{du}{u}\,u^{4}.
\end{equation}


\paragraph*{Projective Superspace.}

Now, let $\mathbf{RP}^{3|8}$ denote the $(3|8)$-dimensional real
projective superspace, which we identify with the trivial vector superbundle
$\mathbf{RP}^{3}\times\mathbf{R}^{0|8}$, with typical fibre isomorphic
to the vector superspace spanned by the Grassmann-valued van der Waerden
spinors $\theta_{A}^{\alpha}$ $\left(1\leq\alpha\leq4\right)$. A
trivialisation of this fibration is achieved by introducing the homogeneous
coordinates $\mathbb{X}^{K}\coloneqq\big(X_{A\dot{A}},\theta_{A}^{\alpha}\big)$
that chart the total space $\mathbf{RP}^{3|8}$. The canonical orientation
on $\mathbf{RP}^{3|8}$ is then defined by the volume superform:
\begin{equation}
D^{3|8}\mathbb{X}\,\coloneqq\,\frac{D^{3}X}{\left|X\right|^{4}}\wedge d^{0|8}\theta.\label{eq:-74}
\end{equation}
Note that, under the superspace rescaling transformation:
\[
\mathbb{X}^{K}=\big(X_{A\dot{A}},\theta_{A}^{\alpha}\big)\,\mapsto\,\mathbb{X}'{}^{K}=\big(sX_{A\dot{A}},t\theta_{A}^{\alpha}\big),
\]
this measure transforms as:
\[
D^{3|8}\mathbb{X}\,\mapsto\,D^{3|8}\mathbb{X}'=t^{-8}D^{3|8}\mathbb{X}.
\]
Moreover, the supersymmetric extension of Eq. (\ref{eq:-75}) becomes:
\begin{equation}
d^{4|8}\mathsf{x}|_{\mathbf{W}^{+}}\,=\,D^{3|8}\mathbb{X}\wedge\frac{du}{u}u^{4}.
\end{equation}


\subsubsection{Pre-Leaf Amplitude}

Performing the dimensional reduction on the supercoordinates $\mathsf{x}^{K}$
according to the procedure developed by Sharma et. al., we find that
the renormalised celestial superamplitude may be expressed as a sum
of partial amplitudes:
\begin{equation}
:\widehat{\mathcal{A}}_{\ell(1)}:\big(\mathsf{W}_{\text{L},i};\mathsf{W}_{\text{R},j}\big)=\mathcal{B}_{0}\big(z_{i},\bar{z}_{i},\eta_{i};w_{j},\bar{w}_{j},\tilde{\eta}_{j}\big)+\mathcal{B}_{0}\big(z_{i},i\bar{z}_{i},\eta_{i};w_{j},i\bar{w}_{j};\tilde{\eta}_{j}\big).
\end{equation}
Notice that the second term is obtained from the first by the replacements:
\begin{equation}
\bar{z}_{i\dot{A}}\mapsto i\bar{z}_{i\dot{A}},\,\,\,\bar{w}_{i\dot{A}}\mapsto i\bar{w}_{i\dot{A}}.
\end{equation}

The partial superamplitude $\mathcal{B}_{0}\big(\mathsf{W}_{\text{L},i}^{I};\mathsf{W}_{\text{R},j}^{J}\big)$
is obtained by replacing the full integral over the supercoordinates
$\mathsf{x}^{K}$, which parametrise Klein superspace, with an integral
over the coordinates $\mathbb{X}^{K}$ that chart the supersymmetric
extension of the timelike wedge, $\mathbf{W}^{-}\subset\mathbf{R}^{4|8}$.
In particular, one has:
\begin{equation}
\mathcal{B}_{0}\big(\mathsf{W}_{\text{L},i}^{I};\mathsf{W}_{\text{R},j}^{I'}\big)\,=\,\frac{1}{\left(2\pi\right)^{4}}\,\int\,\mathcal{D}^{2|4}\mathsf{W}_{\ell_{1}}\wedge\mathcal{D}^{2|4}\mathsf{W}_{\ell_{2}}\underset{\mathscr{D}_{0}\,\,\,}{\int}\,D^{3|8}\mathbb{X}\wedge d^{4|8}\tilde{\mathsf{x}}\,\,\,\,\,\mathcal{I}\big(\mathsf{W}_{\text{L},i}^{I};\mathsf{W}_{\text{R},j}^{I'};\mathsf{W}_{\ell_{k}}^{I''}\big),\label{eq:-76}
\end{equation}
where the integrals are taken over the dual minitwistors $\mathsf{W}_{\ell_{1}}^{I}$
and $\mathsf{W}_{\ell_{2}}^{I}$, which parametrise the internal (virtual)
gluon states, and over the superdomain:
\begin{equation}
\mathscr{D}_{0}\,\coloneqq\,\mathbf{RP}^{3|8}\times\mathbf{R}^{4|8}.
\end{equation}
The function $\mathcal{I}$ is represented by an integral over the
multiplicative group of positive reals $\mathscr{R}\simeq\big(\mathbf{R}_{+},\cdot\big)$,
endowed with the Haar measure $dH_{u}\coloneqq d\log u$, and is defined
as:
\begin{equation}
\mathcal{I}\big(\mathsf{W}_{\text{L},i}^{I};\mathsf{W}_{\text{R},j}^{I'};\mathsf{W}_{\ell_{k}}^{I''}\big)\,=\,\underset{\mathscr{R}\,\,\,\,\,\,\,}{\int}\,dH_{u}\,\,\,u^{4}\,\mathcal{E}_{L}\big(\mathsf{W}_{\text{L},i}^{I}\big)\,\mathcal{E}_{R}\big(\mathsf{W}_{\text{R},j}^{I'}\big)\,\mathcal{E}_{\ell}\big(\mathsf{W}_{\ell_{k}}^{I''}\big).\label{eq:-77}
\end{equation}
Expressed in terms of the new integration variables, the auxiliary
functions $\mathcal{E}_{L}$ and $\mathcal{E}_{R}$ take the forms:
\begin{equation}
\mathcal{E}_{L}\big(\mathsf{W}_{\text{L},i}^{I}\big)\,=\,\bigwedge_{i=1}^{m}\,\frac{\mathcal{C}\big(2h_{\text{L},i}\big)}{\langle z_{i}|u\mathcal{R}+\tilde{x}|\bar{z}_{i}]^{2h_{\text{L},i}}}\,e^{i\langle z_{i}|\theta\cdot\eta_{i}\rangle}\,\frac{1}{z_{i}\cdot z_{i+1}},\label{eq:-78}
\end{equation}
and:
\begin{equation}
\mathcal{E}_{\text{R}}\big(\mathsf{W}_{\text{R},j}^{I}\big)\,=\,\bigwedge_{j=1}^{n}\,\frac{\mathcal{C}\big(2h_{\text{R},j}\big)}{\langle w_{j}|u\mathcal{R}|\bar{w}_{j}]^{2h_{\text{R},j}}}\,e^{i\langle w_{j}|\tilde{\theta}\cdot\tilde{\eta}_{j}\rangle}\,\frac{1}{w_{j}\cdot w_{j+1}}.\label{eq:-79}
\end{equation}
Meanwhile, $\mathcal{E}_{\ell}$ remains as specified in Eq. (\ref{eq:-82}).

Next, consider the superspace rescaling:
\[
\tilde{\mathsf{x}}^{K}=\big(\tilde{x}_{A\dot{A}},\tilde{\theta}_{A}^{\alpha}\big)\,\mapsto\,\big(u\tilde{x}_{A\dot{A}},\tilde{\theta}_{A}^{\alpha}\big).
\]
Under this transformation, the Berezin-de Witt volume superform transforms
as:
\[
d^{4|8}\tilde{\mathsf{x}}\,\mapsto\,u^{4}\,d^{4|8}\tilde{\mathsf{x}}.
\]
Substituting these transformations into Eqs. (\ref{eq:-76})--(\ref{eq:-79})
allows the affine parameter $u$ to factorise and be integrated, yielding:
\begin{equation}
\mathcal{B}_{0}\big(\mathsf{W}_{\text{L},i}^{I};\mathsf{W}_{\text{R},j}^{I'}\big)\,=\,\frac{1}{\left(2\pi\right)^{3}}\,\delta\big(\beta_{\ell(1)}\big)\,\mathcal{B}_{1}\big(\mathsf{W}_{\text{L},i}^{I};\mathsf{W}_{\text{R},j}^{I'}\big),
\end{equation}
where the quantity $\beta_{\ell(1)}$, which encodes the \emph{total
scaling dimension} of the scattering process, takes the form:
\begin{equation}
\beta_{\ell(1)}\,\coloneqq\,4-2\sum_{i=1}^{m}h_{\text{L},i}-2\sum_{j=1}^{n}h_{\text{R},j}.
\end{equation}

The new partial amplitude $\mathcal{B}_{1}\coloneqq\mathcal{B}_{1}\big(\mathsf{W}_{\text{L},i}^{I};\mathsf{W}_{\text{R},j}^{J}\big)$,
which we term the \emph{pre-leaf amplitude}, is defined by:
\begin{equation}
\mathcal{B}_{1}\big(\mathsf{W}_{\text{L},i}^{I};\mathsf{W}_{\text{R},j}^{I'}\big)\,=\,\underset{\mathbf{R}^{4|8}\,\,\,}{\int}\,\mathcal{D}^{2|4}\mathsf{W}_{\ell_{1}}\wedge\mathcal{D}^{2|4}\mathsf{W}_{\ell_{2}}\underset{\mathscr{D}_{0}\,\,\,}{\int}\,D^{3|8}\mathbb{X}\wedge d^{4|8}\tilde{\mathsf{x}}\,\,\,\mathcal{F}_{\text{L}}\big(\mathsf{W}_{\text{L},i}^{I}\big)\,\text{\ensuremath{\mathcal{F}}}_{\text{R}}\big(\mathsf{W}_{\text{R},j}^{I'}\big)\,\mathcal{E}_{\ell}\big(\mathsf{W}_{\ell_{k}}^{I''}\big).\label{eq:-87}
\end{equation}
Here, the new auxiliary functions are represented by:
\begin{equation}
\mathcal{F}_{\text{L}}\big(\mathsf{W}_{\text{L},i}^{I}\big)\,\coloneqq\,\bigwedge_{i=1}^{m}\,\frac{\mathcal{C}\big(2h_{\text{L},i}\big)}{\langle z_{i}|\mathcal{R}+\tilde{x}|\bar{z}_{i}]^{2h_{\text{L},i}}}\,e^{i\langle z_{i}|\theta\cdot\eta_{i}\rangle}\,\frac{1}{z_{i}\cdot z_{i+1}},
\end{equation}
and:
\begin{equation}
\mathcal{F}_{\text{R}}\big(\mathsf{W}_{\text{R},j}^{I}\big)\,\coloneqq\,\bigwedge_{j=1}^{n}\,\frac{\mathcal{C}\big(2h_{\text{R},j}\big)}{\langle w_{j}|\mathcal{R}|\bar{w}_{j}]^{2h_{\text{R},j}}}\,e^{i\langle w_{j}|\tilde{\theta}\cdot\eta_{j}\rangle}\,\frac{1}{w_{j}\cdot w_{j+1}}.
\end{equation}

These expressions complete the dimensional reduction from integration
over the supercoordinates $\mathsf{x}^{K}$, which parametrise $\mathbf{R}^{4|8}$,
to the coordinates $\mathbb{X}^{K}$ that parametrise $\mathbf{RP}^{3|8}$.
The final step in obtaining the leaf amplitude is to carry out the
reduction with respect to the remaining integration supervariables,
namely $\tilde{\mathsf{x}}^{K}\in\widetilde{\mathbf{R}}^{4|8}$.

\subsubsection{Leaf Amplitude}

Next, we address the final step required to derive the (renormalised,
one-loop) \emph{leaf} superamplitude for gluons in celestial CFT.

\paragraph*{Notation.}

For clarity, we denote by $\widetilde{\mathbf{R}}^{4|8}$ the integration
superdomain parametrised by $\tilde{\mathsf{x}}^{K}$, thus distinguishing
it from the domain over which $\mathsf{x}^{K}$ is defined. Similarly,
the projective superspace obtained by dimensionally reducing $\widetilde{\mathbf{R}}^{4|8}$
is denoted by $\widetilde{\mathbf{RP}}^{3|8}$.

\paragraph*{Preliminaries.}

Proceeding analogously to the argument developed in the preceding
section, let $\widetilde{X}_{A\dot{A}}$ denote homogeneous coordinates
on $\widetilde{\mathbf{RP}}^{3|8}$, and define the corresponding
projective coordinates by:
\begin{equation}
\widetilde{\mathcal{R}}_{A\dot{A}}\,\coloneqq\,\frac{\widetilde{X}_{A\dot{A}}}{|\widetilde{X}|}.
\end{equation}
Next, on the timelike wedge $\widetilde{W}^{-}$ of $\widetilde{\mathbf{R}}^{4|8}$,
we introduce the coordinates $\widetilde{\mathcal{X}}\coloneqq\big(v,\widetilde{\mathcal{R}}_{A\dot{A}}\big)$,
which are related to the global rectangular coordinates $\tilde{x}_{A\dot{A}}$
via:
\begin{equation}
\tilde{x}_{A\dot{A}}(p)\,=\,v(p)\,\widetilde{\mathcal{R}}_{A\dot{A}}(p),\,\,\,\text{for all}\,p\in\widetilde{W}^{-}.
\end{equation}
Thus, the Lebesgue measure $d^{4}\tilde{x}$ restricted to $\widetilde{W}^{-}$
decomposes as:
\begin{equation}
d^{4}\tilde{x}\,|_{\widetilde{W}^{-}}\,=\,\frac{D^{3}\widetilde{X}}{|\widetilde{X}|^{4}}\wedge\frac{dv}{v}\,v^{4}.\label{eq:-83}
\end{equation}

Turning to the supersymmetric extension of the above structures, let
$\widetilde{\mathbb{X}}^{K}\coloneqq\big(\widetilde{X}_{A\dot{A}},\tilde{\theta}_{A}^{\alpha}\big)$
denote a global trivialisation of the vector superbundle $\widetilde{\mathbf{RP}}^{3|8}\simeq\widetilde{\mathbf{RP}}^{3}\times\mathbf{R}^{0|8}$.
The total space is endowed with the orientation defined by the volume
superform:
\begin{equation}
D^{3|8}\widetilde{\mathbb{X}}\,\coloneqq\,\frac{D^{3}\widetilde{X}}{|\widetilde{X}|^{4}}\wedge d^{0|8}\tilde{\theta}.
\end{equation}

Recall that the standard measure on Klein superspace is given by $d^{4|8}\tilde{\mathsf{x}}=d^{4}\tilde{x}\wedge d^{0|8}\tilde{\theta}$,
and let $\widetilde{\mathbf{W}}^{-}$ denote the supersymmetric extension
of the timelike wedge. Hence, Eq. (\ref{eq:-83}) generalises to:
\begin{equation}
d^{4|8}\tilde{\mathsf{x}}\,|_{\widetilde{\mathbf{W}}^{-}}\,=\,D^{3|8}\widetilde{\mathbb{X}}\wedge\frac{dv}{v}\,v^{4}.
\end{equation}


\paragraph*{Dimensional Reduction.}

We now apply the leaf formalism to dimensionally reduce the integral
over $\tilde{\mathsf{x}}^{K}$ in Eq. (\ref{eq:-87}). In this procedure,
we find that the function $\mathcal{B}_{1}$ decomposes into a sum
of partial amplitudes:
\begin{equation}
\mathcal{B}_{1}\big(\mathsf{W}_{\text{L},i}^{I};\mathsf{W}_{\text{R},j}^{J}\big)\,=\,\mathcal{M}_{\ell(1)}\big(z_{i},\bar{z}_{i},\eta_{i};w_{j},\bar{w}_{j},\tilde{\eta}_{j}\big)-i\mathcal{M}_{\ell(1)}\big(z_{i},i\bar{z}_{i},\eta_{i};w_{j},i\bar{w}_{j},\tilde{\eta}_{j}\big).
\end{equation}
Here, the arguments of the second term are obtained from those of
the first by the replacements $\bar{z}_{i\dot{A}}\mapsto i\bar{z}_{i\dot{A}}$
and $\bar{w}_{i\dot{A}}\mapsto i\bar{w}_{i\dot{A}}$.

The partial superamplitude $\mathcal{M}_{\ell(1)}\coloneqq\mathcal{M}_{\ell(1)}\big(\mathsf{W}_{\text{L},i}^{I};\mathsf{W}_{\text{R},j}^{I'}\big)$
is precisely the renormalised, one-loop corrected celestial \emph{leaf}
superamplitude describing the scattering of gluons in configurations
characterised by maximal-helicity-violation in $\mathcal{N}=4$ SYM
theory. We now describe the explicit formulation of $\mathcal{M}_{\ell(1)}$.

Let the \emph{moduli space} be defined by the supermanifold:
\begin{equation}
\mathscr{M}_{\ell(1)}\,\coloneqq\,\mathbf{RP}^{3|8}\times\widetilde{\mathbf{RP}}^{3|8}\times\mathbf{R}_{+},
\end{equation}
which is charted by the ($\mathbf{Z}_{2}$-graded) coordinate map:
\begin{equation}
p\in\mathscr{M}_{\ell(1)}\mapsto\tau^{M}(p)\coloneqq\big(\mathbb{X}^{K}(p),\widetilde{\mathbb{X}}^{K'}(p),v(p)\big),
\end{equation}
with $M$ serving as the abstract index for the supercoordinates.
The standard orientation on $\mathscr{M}_{\ell(1)}$ is provided by
the volume superform: 
\begin{equation}
d\boldsymbol{\tau}\,\coloneqq\,D^{3|8}\mathbb{X}\,\wedge\,D^{3|8}\widetilde{\mathbb{X}}\,\wedge\,\frac{dv}{v}.
\end{equation}
Thus, the leaf superamplitude is expressed as:
\begin{equation}
\mathcal{M}_{\ell(1)}\big(\mathsf{W}_{\text{L},i}^{I};\mathsf{W}_{\text{R},j}^{I'}\big)\,=\,\int\,\mathcal{D}^{2|4}\mathsf{W}_{\ell_{1}}\wedge\mathcal{D}^{2|4}\mathsf{W}_{\ell_{2}}\,\underset{\mathscr{M}_{\ell(1)}\,\,\,}{\int}\,d\boldsymbol{\tau}\,\,\,\mathcal{F}_{\text{L}}\big(\mathsf{W}_{\text{L},i}^{I}\big)\mathcal{F}_{\text{R}}\big(\mathsf{W}_{\text{R},j}^{I'}\big)\mathcal{F}_{\ell}\big(\mathsf{W}_{\ell_{k}}^{I''}\big),\label{eq:-84}
\end{equation}
where the auxiliary functions are given by:
\begin{equation}
\mathcal{F}_{\text{L}}\big(\mathsf{W}_{\text{L},i}^{I}\big)\,=\,\bigwedge_{i=1}^{m}\,\frac{\mathcal{C}\big(2h_{\text{L},i}\big)}{\langle z_{i}|\mathcal{R}+v\widetilde{\mathcal{R}}|\bar{z}_{i}]^{2h_{\text{L},i}}}\,e^{i\langle z_{i}|\theta\cdot\eta_{i}\rangle}\,\frac{1}{z_{i}\cdot z_{i+1}},
\end{equation}
\begin{equation}
\mathcal{F}_{\text{R}}\big(\mathsf{W}_{\text{R},j}^{I}\big)\,=\,\bigwedge_{j=1}^{n}\,\frac{\mathcal{C}\big(2h_{\text{R},j}\big)}{\langle w_{j}|\mathcal{R}|\bar{w}_{j}]^{2h_{\text{R},j}}}\,e^{i\langle w_{j}|\tilde{\theta}\cdot\tilde{\eta}_{j}\rangle}\,\frac{1}{w_{j}\cdot w_{j+1}},
\end{equation}
\begin{equation}
\mathcal{F}_{\ell}\big(\mathsf{W}_{\ell_{k}}^{I}\big)\,=\,\bigwedge_{k=1}^{2}\,\frac{\mathcal{C}(2)}{\langle\ell_{k}|\widetilde{\mathcal{R}}|\bar{\ell}_{k}]^{2}}\,e^{i\langle\ell_{k}|\theta_{\ell_{k}}\cdot\eta_{\ell_{k}}\rangle}\,\frac{1}{\ell_{k}\cdot\ell_{k+1}}.
\end{equation}


\subsection{One-loop Minitwistor Superamplitude}

\paragraph{Celestial/Minitwistor Correspondence.}

The renormalised one-loop celestial leaf amplitude $\mathcal{M}_{\ell(1)}$
was derived in the preceding section (cf. Eq. (\ref{eq:-84})). This
amplitude is a function of the dual minitwistors:
\[
\mathsf{W}_{\text{L},i}^{I}\,=\,\big(z_{i}^{A},\bar{z}_{i\dot{A}},\eta_{i}^{\alpha}\big)\,\,\,\text{and}\,\,\,\mathsf{W}_{\text{R},j}^{I'}\,=\,\big(w_{j}^{A},\bar{w}_{j\dot{A}},\tilde{\eta}_{j}^{\alpha}\big),
\]
which encode the quantum states of the left-moving gluon $i\in L$
and the right-moving gluon $j\in R$ from the perspective of celestial
CFT. In this context, the ordered pairs $\big(z_{i}^{A},\bar{z}_{i\dot{A}}\big)$
and $\big(w_{j}^{A},\bar{w}_{j\dot{A}}\big)$ of van der Waerden spinors
parametrise the insertion points of the $i$-th and $j$-th gluons
on the celestial sphere $\mathcal{CS}$, while the Grassmann variables
$\eta_{i}^{\alpha}$ and $\tilde{\eta}_{j}^{\alpha}$ encode their
helicity degrees of freedom. In other words, describing the state
of a gluon using the dual minitwistors $\mathsf{W}_{\text{L},i}^{I}$
and $\mathsf{W}_{\text{R},j}^{I'}$ is formally \emph{analogous} to
describing the state of a nonrelativistic particle in the \emph{position}
basis in elementary quantum mechanics.

Recall that, in elementary quantum mechanics, one may equivalently
describe the state of a particle in momentum space by means of the
Fourier transform. In our setting, we transition from the description
in terms of celestial supercoordinates $\big(z^{A},\bar{z}_{\dot{A}},\eta^{\alpha}\big)$
to a representation in terms of minitwistor superspace coordinates
$\mathsf{Z}^{I}=\big(\lambda^{A},\mu_{\dot{A}},\psi^{\alpha}\big)$
via the minitwistor transform.

\paragraph{Celestial RSVW Representation.}

To that end, we begin with the celestial RSVW identity. Let $f\coloneqq f\big(\sigma^{A}\big)$
be a homogeneous holomorphic function of an undotted van der Waerden
spinor $\sigma^{A}$ that satisfies:
\begin{equation}
f\big(t\sigma^{A}\big)=t^{-2}f\big(\sigma^{A}\big),\,\,\,\text{for all}\,t\in\mathbf{C}^{*}.
\end{equation}
 Furthermore, let $\mathcal{L}\coloneqq\mathcal{L}\left(X,\theta\right)\subset\mathbf{MT}_{s}$
denote the minitwistor line defined by the embedding:
\[
\mathsf{Z}_{\mathcal{L}}^{I}:\mathbf{CP}^{1}\longrightarrow\mathbf{MT}_{s},
\]
whose parametrisation in terms of homogeneous coordinates $\sigma^{A}$
is given by:
\begin{equation}
\mathsf{Z}_{\mathcal{L}}^{I}\big(\sigma^{A}\big)\,\coloneqq\,\big(\sigma^{A},\sigma^{A}X_{A\dot{A}},\sigma^{A}\theta_{A}^{\alpha}\big).
\end{equation}
Then, the celestial RSVW formula may be expressed as:
\begin{equation}
\frac{\mathcal{C}\big(\Delta\big)}{\langle z|X|\bar{z}]^{\Delta}}\,e^{i\langle z|\theta\cdot\eta\rangle}\,f\left(z\right)\,=\,\underset{\mathbf{MT}_{s}\,\,\,}{\int}\,D^{2|4}\mathsf{Z}\,\,\,\Psi_{\Delta}\big(\mathsf{Z}^{I};\mathsf{W}^{I'}\big)\,\,\,\underset{\mathbf{CP}^{1}\,\,\,}{\int}\,D\sigma\,\,\,\overline{\delta}_{\left(-\Delta,\Delta\right)}^{2|4}\big(\mathsf{Z}^{I};\mathsf{Z}_{\mathcal{L}}^{I'}\left(\sigma\right)\big)\,f\left(\sigma\right).
\end{equation}

Substituting this identity into Eq. (\ref{eq:-84}), we obtain the
following result. Let:
\[
\mathsf{Z}_{\text{L}}^{I},\mathsf{Z}_{\text{R}}^{I},\mathsf{Z}_{\ell}^{I}:\mathbf{CP}^{1}\longrightarrow\mathbf{MT}_{s},
\]
be a family of embeddings of minitwistors lines, defined respectively
by:
\begin{equation}
\mathsf{Z}_{\text{L}}^{I}\big(\sigma^{A}\big)\,\coloneqq\,\big(\sigma^{A},\sigma^{A}\big(\mathcal{R}_{A\dot{A}}+v\widetilde{\mathcal{R}}_{A\dot{A}}\big),\sigma^{A}\theta_{A}^{\alpha}\big),\label{eq:-85}
\end{equation}
\begin{equation}
\mathsf{Z}_{\text{R}}^{I}\big(\sigma^{A}\big)\,\coloneqq\,\big(\sigma^{A},\sigma^{A}\mathcal{R}_{A\dot{A}},\sigma^{A}\tilde{\theta}_{A}^{\alpha}\big),
\end{equation}
\begin{equation}
\mathsf{Z}_{\ell}^{I}\big(\sigma^{A}\big)\,\coloneqq\,\big(\sigma^{A},\sigma^{A}\widetilde{\mathcal{R}}_{A\dot{A}},\sigma^{A}\theta_{\ell_{k}A}^{\alpha}\big).\label{eq:-86}
\end{equation}
Here, the bosonic variables $\mathcal{R}_{A\dot{A}}$, $\widetilde{\mathcal{R}}_{A\dot{A}}$,
and $v$, along with the fermionic coordinates $\theta^{\alpha A}$,
$\tilde{\theta}^{\alpha A}$, and $\theta_{\ell_{k}}^{\alpha A}$,
serve as the moduli parameters that characterise the system (\ref{eq:-85})--(\ref{eq:-86})
of minitwistor lines. This observation justifies our interpretation
of the supervariables $\tau^{M}$ charting $\mathscr{M}_{\ell(1)}$
as the moduli parameters and of the superspace $\mathscr{M}_{\ell(1)}$
as the corresponding moduli space.

\paragraph*{Minitwistor Amplitude.}

Consequently, the leaf superamplitude becomes a \emph{minitwistor
transform}:
\begin{equation}
\mathcal{M}_{\ell(1)}\big(\mathsf{W}_{\text{L},i}^{I};\mathsf{W}_{\text{R},j}^{I'}\big)\,=\,\int\,\mathcal{D}^{2|4}\mathsf{W}_{\ell_{1}}\wedge\mathcal{D}^{2|4}\mathsf{W}_{\ell_{2}}\,\underset{\mathscr{E}^{M}\,\,\,\,\,}{\int}\,d\boldsymbol{\nu}\,\,\,\widetilde{\mathcal{M}}_{\ell(1)}\big(\mathsf{Z}_{\text{L},i}^{I};\mathsf{Z}_{\text{R},j}^{I'};\mathsf{Z}_{\ell_{k}}^{I''}\big),
\end{equation}
where the integration superdomain:
\[
\mathscr{E}^{M}\,\coloneqq\,\prod_{i=1}^{M}\,\mathbf{MT}_{s},\,\,\,M\coloneqq m+n+2,
\]
is endowed with an orientation determined by the volume superform:
\begin{equation}
d\boldsymbol{\nu}=\bigwedge_{i=1}^{m}D^{2|4}\mathsf{Z}_{\text{L},i}\Psi_{2h_{\text{L},i}}\big(\mathsf{Z}_{\text{L},i};\mathsf{W}_{\text{L},i}\big)\bigwedge_{j=1}^{n}D^{2|4}\mathsf{Z}_{\text{R},j}\Psi_{2h_{\text{R},j}}\big(\mathsf{Z}_{\text{R},j};\mathsf{W}_{\text{R},j}\big)\bigwedge_{k=1}^{2}D^{2|4}\mathsf{Z}_{\ell_{k}}\Psi_{2}\big(\mathsf{Z}_{\ell_{k}};\mathsf{W}_{\ell_{k}}\big).
\end{equation}
Accordingly, the (renormalised, one-loop) \emph{minitwistor superamplitude}
encoding the scattering of $m+n$ gluons in $\mathcal{N}=4$ SYM theory
is given by:
\begin{equation}
\widetilde{\mathcal{M}}_{\ell(1)}\big(\mathsf{Z}_{\text{L},i}^{I};\mathsf{Z}_{\text{R},j}^{I'};\mathsf{Z}_{\ell_{k}}^{I''}\big)\,=\,\underset{\mathscr{M}_{\ell(1)}\,\,\,}{\int}\,d\boldsymbol{\tau}\,\,\,\mathcal{G}_{\text{L}}\big(\mathsf{Z}_{\text{L},i}^{I}\big)\,\mathcal{G}_{\text{R}}\big(\mathsf{Z}_{\text{R},j}^{I'}\big)\,\mathcal{G}_{\ell(1)}\big(\mathsf{Z}_{\ell_{k}}^{I''}\big),
\end{equation}
where the integrand functions are defined as:
\begin{equation}
\mathcal{G}_{\text{L}}\big(\mathsf{Z}_{\text{L},i}^{I}\big)\,=\,\underset{\mathscr{L}^{m}\,\,\,}{\int}\,\bigwedge_{i=1}^{m}\,\boldsymbol{\omega}\big(\sigma_{i}^{A}\big)\,\,\,\overline{\delta}_{\left(-2h_{\text{L},i},2h_{\text{L},i}\right)}^{2|4}\big(\mathsf{Z}_{\text{L},i}^{I};\mathsf{Z}_{\text{L}}^{I'}\big(\sigma_{i}^{A}\big)\big),
\end{equation}
\begin{equation}
\mathcal{G}_{\text{R}}\big(\mathsf{Z}_{\text{R},j}^{I}\big)\,=\,\underset{\mathscr{L}^{n}\,\,\,}{\int}\,\bigwedge_{j=1}^{n}\,\boldsymbol{\omega}\big(\sigma_{j}^{A}\big)\,\,\,\overline{\delta}_{\left(-2h_{\text{R},j},2h_{\text{R},j}\right)}^{2|4}\big(\mathsf{Z}_{\text{R},j}^{I};\mathsf{Z}_{\text{R}}^{I'}\big(\sigma_{j}^{A}\big)\big),
\end{equation}
\begin{equation}
\mathcal{G}_{\ell}\big(\mathsf{Z}_{\ell_{k}}^{I}\big)\,=\,\underset{\mathscr{L}^{2}\,\,\,}{\int}\,\bigwedge_{k=1}^{2}\,\boldsymbol{\omega}\big(\sigma_{k}^{A}\big)\,\,\,\overline{\delta}_{(-2,2)}^{2|4}\big(\mathsf{\mathsf{Z}}_{\ell_{k}}^{I};\mathsf{Z}_{\ell}^{I'}\big(\sigma_{k}^{A}\big)\big).
\end{equation}
Here, the integration domain is:
\[
\mathscr{L}^{m}\,\coloneqq\,\prod_{i=1}^{m}\,\mathbf{CP}^{1},
\]
and the logarithmic differential form is given by:
\begin{equation}
\boldsymbol{\omega}\big(\sigma_{i}^{A}\big)\,\coloneqq\,\frac{D\sigma_{i}}{\sigma_{i}\cdot\sigma_{i+1}}.
\end{equation}


\paragraph*{Geometric Interpretation.}

It follows that the integrand of the minitwistor superamplitude is
localised on the system (\ref{eq:-85})--(\ref{eq:-86}) of minitwistor
lines. Consequently, the full superamplitude $\mathcal{M}_{\ell(1)}$
admits a geometric interpretation as an integral over the moduli space
parametrising this system of lines in minitwistor superspace. Furthermore,
the celestial amplitude vanishes whenever the configuration $\big\{\mathsf{W}_{\text{L},i}^{I},\mathsf{W}_{\text{R},j}^{I'}\big\}$
of external gluons does not lie along the minitwistor lines: 
\begin{equation}
\mathscr{L}_{\text{L}}\big(\mathcal{R},\widetilde{\mathcal{R}},\theta\big)\coloneqq\big\{\big(\lambda^{A},\mu_{\dot{A}},\psi^{\alpha}\big)\in\mathbf{MT}_{s}\,\big|\,\mu_{\dot{A}}=\lambda^{A}\big(\mathcal{R}_{A\dot{A}}+v\widetilde{\mathcal{R}}_{A\dot{A}}\big),\,\,\,\psi^{\alpha}=\lambda^{A}\theta_{A}^{\alpha}\big\},
\end{equation}
\begin{equation}
\mathscr{L}_{\text{R}}\big(\mathcal{R},\tilde{\theta}\big)\coloneqq\big\{\big(\lambda^{A},\mu_{\dot{A}},\psi^{\alpha}\big)\in\mathbf{MT}_{s}\,\big|\,\mu_{\dot{A}}=\lambda^{A}\mathcal{R}_{A\dot{A}},\,\,\,\psi^{\alpha}=\lambda^{A}\tilde{\theta}_{A}^{\alpha}\big\}.
\end{equation}


\appendix

\section{Mini-Introduction to Holomorphic Gauge Theory\label{sec:Mini-Introduction-to-Holomorphic}}

As shown in Section \ref{sec:Minitwistor-Wilson-Lines}, every tree-level
celestial gluon amplitude in $\mathcal{N}=4$ SYM theory, including
all $N^{k}$-MHV sectors, can be reconstructed from holomorphic Wilson
line operators on minitwistor superspace. We interpret this result
as evidence that these Wilson lines are gauge-invariant observables
of a ``minitwistor string field theory.'' Such a field theory arises
as the effective description of a minitwistor sigma-model.

This geometric perspective on scattering amplitudes suggests a new
mathematical framework for gauge theory in asymptotically flat spacetimes.
To prepare the reader, we now give a concise introduction to holomorphic
gauge theory (HGT) on minitwistor superspace. We emphasise physical
intuition over mathematical rigour and explain how familiar gauge-theory
concepts emerge from \emph{complex} differential geometry.

We focus on the concept of a \emph{pseudoholomorphic structure }along
with its related notions of partial connections and pseudocurvature.
The idea of a partial connection was introduced by \citet{bott2006lectures}
and \citet{rawnsley1979flat} in the context of holomorphic foliations.
Our exposition follows the pedagogical treatments of \citet{donaldson1997geometry}
and the reviews by \citet{donaldson2006mathematical} and \citet{guichard2018introduction}.
For a full, rigorous account, see \citet{kobayashi2014differential},
\citet{chern1967complex} and \citet{moroianu2007lectures}.

\subsection{Basic Structures}

In this subsection, we introduce the framework for constructing minitwistor
Wilson lines. We begin with an algebraic formulation. Then we incorporate
the analytic structure of a holomorphic gauge theory, reviewing the
necessary concepts as we proceed.

\textcompwordmark{}
\begin{notation}
Let $\pi\colon E\to\mathbf{MT}_{s}$ be a complex vector bundle. We
describe physical fields as sections of $E$, possibly tensored with
exterior powers of the cotangent bundle. We define the space of smooth
$E$-valued differential forms of bidegree $(r,s)$ as:
\begin{equation}
\Omega^{r,s}(\mathbf{MT}_{s};E)\;\coloneqq\;\Gamma\big(\mathbf{MT}_{s};\bigw^{r,s}\,T^{*}(\mathbf{MT}_{s})\!\otimes\!E\big).
\end{equation}
We denote the space of smooth vector fields by:
\begin{equation}
\mathscr{X}(\mathbf{MT}_{s})\;\coloneqq\;\Gamma\big(\mathbf{MT}_{s};\,T(\mathbf{MT}_{s})\big).
\end{equation}
In particular, the dual space satisfies:
\begin{equation}
\mathscr{X}^{*}(\mathbf{MT}_{s})\;\cong\;\Omega^{1}(\mathbf{MT}_{s};\mathbf{C}).
\end{equation}
\end{notation}

To introduce a gauge theory, we require that each bundle fibre carry
a gauge Lie-algebra structure. Specifically, for any point $\mathsf{w}\in\mathbf{MT}_{s}$,
we demand $E|_{\mathsf{w}}\cong\mathfrak{g}$, where $\mathfrak{g}$
is a complexified Lie algebra. This identification endows each fibre
with the Lie bracket $[\cdot,\cdot]_{\mathfrak{g}}$.

We then unify this bracket with the exterior calculus of differential
forms by defining a generalised wedge product $\wedge_{\mathfrak{g}}$.
For two $\mathfrak{g}$-valued forms $\alpha$ and $\beta$, $\alpha\wedge_{\mathfrak{g}}\beta$
computes their pointwise Lie bracket in the fibre and antisymmetrises
their form components. Later, we shall express the fieldstrength tensor
using $\wedge_{\mathfrak{g}}$.
\begin{defn}
Let $\alpha\in\Omega^{r,s}(\mathbf{MT}_{s};E)$ and $\beta\in\Omega^{k,\ell}(\mathbf{MT}_{s};E)$
be two $\mathfrak{g}$-valued forms. We define their generalised wedge
product 
\begin{equation}
\alpha\wedge_{\mathfrak{g}}\beta\;\in\;\Omega^{r+k,s+\ell}(\mathbf{MT}_{s};E)
\end{equation}
by its contraction with vector fields. For any $X_{i}\in\mathscr{X}(\mathbf{MT}_{s})$,
set $p=r+s$ and $q=k+\ell$ and write:
\begin{equation}
i_{X_{1}\wedge\dots\wedge X_{p+q}}\big(\alpha\wedge_{\mathfrak{g}}\beta\big)\;\coloneqq\;\sum_{\upsilon\in S_{p+q}}\,\frac{(-1)^{|\nu|}}{p!\,q!}\,\boldsymbol{\big[}\alpha(X_{\upsilon(1)},\dots,X_{\upsilon(p)}),\beta(X_{\upsilon(p+1)},\dots,X_{\upsilon(p+q)})\boldsymbol{\big]}_{\mathfrak{g}},\label{eq:-156}
\end{equation}
where $i_{X}$ denotes the interior product with $X$, the sum runs
over all $\upsilon\in S_{p+q}$, and $|\upsilon|$ is the parity of
$\upsilon$.
\end{defn}

The basic object in HGT is the gauge potential (partial connection)
$\boldsymbol{A}\big(\mathsf{W}^{I}\big)$. It is a $(0,1)$-form whose
values are linear operators on the fibres of $E$. Hence
\begin{equation}
\boldsymbol{A}\;\in\;\Omega^{0,1}\big(\mathbf{MT}_{s};\,\mathrm{End}_{\mathbf{C}}(E)\big)\;\cong\;\Omega^{0,1}\big(\mathbf{MT}_{s};\,\mathfrak{gl}(r,\mathbf{C})\big).
\end{equation}
This partial connection deforms the canonical holomorphic structure
of the bundle. Its fieldstrength $\boldsymbol{F}$ captures the pseudocurvature
of the theory. 

Our final algebraic tool defines how the connection $\boldsymbol{A}$
acts on $E$-valued fields. Since $\boldsymbol{A}$ is an $\mathrm{End}_{\mathbf{C}}(E)$-valued
form, we build its action from the natural evaluation map:
\begin{equation}
\varepsilon\colon\;\mathrm{End}_{\mathbf{C}}(E)\!\otimes\!E\;\longrightarrow\;E,\qquad\varepsilon\big(T\!\otimes\!|e\rangle\big)\coloneqq T|e\rangle,
\end{equation}
for any operator $T\in\mathrm{End}_{\mathbf{C}}(E)$ and vector $|e\rangle\in E$.
We then promote this to a second exterior product $\wedge_{\varepsilon}$.
\begin{defn}
Let
\begin{equation}
\mathcal{T}\,\in\,\Omega^{r,s}\big(\mathbf{MT}_{s};\,\mathrm{End}_{\mathbf{C}}(E)\big)\quad\text{and}\quad|\boldsymbol{e}\rangle\,\in\,\Omega^{k,\ell}\big(\mathbf{MT}_{s};E\big),
\end{equation}
and set $p=r+s$, $q=k+\ell$. For any vector fields $X_{i}\in\mathscr{X}(\mathbf{MT}_{s})$,
\begin{equation}
i_{X_{1}\wedge\dots\wedge X_{p+q}}\big(\mathcal{T}\wedge_{\varepsilon}\boldsymbol{e}\big)\;\coloneqq\;\sum_{\upsilon\in S_{p+q}}\,\frac{(-1)^{|\nu|}}{p!\,q!}\,\varepsilon\big(\mathcal{T}(X_{\upsilon(1)},\dots,X_{\upsilon(p)})\!\otimes\!\big|\boldsymbol{e}(X_{\upsilon(p+1)},\dots,X_{\upsilon(p+q)})\big\rangle\big).
\end{equation}
\end{defn}

This construction defines the covariant action of $\boldsymbol{A}$.
For an $E$-valued field $|\psi\rangle$, the gauge interaction term
is:
\begin{equation}
\boldsymbol{A}\wedge_{\varepsilon}|\psi\rangle,
\end{equation}
so the pseudoholomorphic covariant derivative reads:
\begin{equation}
\overline{\partial}^{\mathscr{E}}\!|\psi\rangle=\overline{\partial}^{E}\!|\psi\rangle+\boldsymbol{A}\wedge_{\varepsilon}|\psi\rangle.
\end{equation}


\subsection{Partial Connections and Holomorphic Gauge Potentials}

We identify the vacuum of HGT with the canonical holomorphic structure
on the bundle $E\to\mathbf{MT}_{s}$. Nontrivial field configurations,
such as semiclassical states, arise from deformations of this vacuum.
To make this precise, we now define a pseudoholomorphic structure.

This concept generalises the Cauchy-Riemann operator $\overline{\partial}$
on a complex manifold to the bundle $E$. A \emph{pseudoholomorphic
structure} (PHS) $\mathscr{E}$ on $E$ is given by a $\mathbf{C}$-linear
operator:
\begin{equation}
\overline{\partial}^{\mathscr{E}}\colon\quad\Omega^{r,s}(\mathbf{MT}_{s};E)\;\longrightarrow\;\Omega^{r,s+1}(\mathbf{MT}_{s};E).
\end{equation}
This operator satisfies the graded Leibniz rule:
\begin{equation}
\overline{\partial}^{\mathscr{E}}\!(\alpha\wedge|\boldsymbol{\psi}\rangle)\;=\;\overline{\partial}\alpha\wedge|\boldsymbol{\psi}\rangle\;+\;(-1)^{r+s}\,\alpha\wedge\overline{\partial}^{\mathscr{E}}\!|\boldsymbol{\psi}\rangle,\label{eq:-157}
\end{equation}
for any $\alpha\in\Omega^{r,s}(\mathbf{MT}_{s};\mathbf{C})$ and $|\boldsymbol{\psi}\rangle\in\Omega^{p,q}(\mathbf{MT}_{s};E)$.

Physically, a section $|\boldsymbol{\psi}\rangle\in\Omega^{p,q}(\mathbf{MT}_{s};E)$
represents a matter field. We call $|\boldsymbol{\psi}\rangle$ holomorphic
(a BPS-like sate) with respect to $\mathscr{E}$ if
\begin{equation}
\overline{\partial}^{\mathscr{E}}\!|\boldsymbol{\psi}\rangle=0.
\end{equation}

\textcompwordmark{}

\paragraph*{A Technical Aside.}

In practice, one often defines a PHS by first specifying its action
on $\Gamma(\mathbf{MT}_{s};E)$ and then extending to all forms. Let
\begin{equation}
\mathscr{D}\colon\quad\Gamma(\mathbf{MT}_{s};E)\;\longrightarrow\;\Omega^{0,1}(\mathbf{MT}_{s};E)
\end{equation}
be a $\mathbf{C}$-linear operator satisfying the Leibniz rule:
\begin{equation}
\mathscr{D}(f\!\cdot\!|\boldsymbol{\phi}\rangle)\;=\;\overline{\partial}\!f\!\otimes\!|\boldsymbol{\phi}\rangle\;+\;f\!\cdot\mathscr{D}|\boldsymbol{\phi}\rangle,
\end{equation}
for any $f\in\mathscr{C}^{\infty}(\mathbf{MT}_{s})$ and section $|\boldsymbol{\phi}\rangle\in\Gamma(\mathbf{MT}_{s};E)$.
Then $\mathscr{D}$ extends \emph{uniquely} to a pseudoholomorphic
operator (partial connection) $\overline{\partial}^{\mathscr{D}}$
on all $E$-valued $(r,s)$-forms by:
\begin{equation}
\overline{\partial}^{\mathscr{D}}\!(\alpha\!\otimes\!|\boldsymbol{\phi}\rangle)\;\coloneqq\;\overline{\partial}\alpha\!\otimes\!|\boldsymbol{\phi}\rangle\;+\;(-1)^{r+s}\alpha\wedge\mathscr{D}|\boldsymbol{\phi}\rangle.
\end{equation}

To verify that this extension obeys the general Leibniz rule (Eq.
(\ref{eq:-157})), consider a decomposable form:
\begin{equation}
|\boldsymbol{\psi}\rangle=\beta\!\otimes\!|\boldsymbol{\phi}\rangle,
\end{equation}
with $\beta\in\Omega^{p,q}(\mathbf{MT}_{s};\mathbf{C})$ and $|\boldsymbol{\phi}\rangle\in\Gamma(\mathbf{MT}_{s};E)$.
Then
\begin{align}
\overline{\partial}^{\mathscr{D}}\!\big(\alpha\!\wedge\!|\boldsymbol{\psi}\rangle\big)\; & =\;\overline{\partial}^{\mathscr{D}}\!(\alpha\!\wedge\!(\beta\!\otimes|\!\boldsymbol{\phi}\rangle))\\
 & =\;\overline{\partial}(\alpha\wedge\beta)\otimes|\boldsymbol{\phi}\rangle\;+\;(-1)^{p+q}\,(\alpha\wedge\beta)\otimes\mathscr{D}|\boldsymbol{\phi}\rangle\\
 & =\;\overline{\partial}\alpha\wedge|\boldsymbol{\psi}\rangle\;+\;(-1)^{p}\,\alpha\wedge\big(\overline{\partial}\beta\otimes|\boldsymbol{\phi}\rangle+(-1)^{q}\beta\otimes\mathscr{D}|\boldsymbol{\phi}\rangle\big)\\
 & =\;\overline{\partial}\alpha\wedge|\boldsymbol{\psi}\rangle\;+\;(-1)^{p}\,\alpha\wedge\mathscr{D}|\boldsymbol{\phi}\rangle.
\end{align}
Here $p\coloneqq\deg(\alpha)$ and $p\coloneqq\deg(\beta)$. This
computation shows that defining $\mathscr{D}$ on sections suffices
to extend it uniquely to all $E$-valued forms.

\textcompwordmark{}

\paragraph*{Semiclassical States.}

With these definitions, we obtain a simple physical result. It formalises
our view of the canonical holomorphic structure on $E$ as the classical
vacuum. Deforming this ground state by adiabatically turning on a
background gauge potential produces semiclassical states. The point
is that these deformations lie in a moduli space.

The above result can be stated as follows. Let $\mathscr{E}_{1}$
and $\mathscr{E}_{2}$ be two pseudoholomorphic structures on the
same bundle $E$. Then they differ by an $\mathrm{End}_{\mathbf{C}}(E)$-valued
$(0,1)$-form:
\begin{equation}
\overline{\partial}^{\mathscr{E}_{2}}\!-\overline{\partial}^{\mathscr{E}_{1}}\;\in\;\Omega^{0,1}\big(\mathbf{MT}_{s};\mathrm{End}_{\mathbf{C}}(E)\big).
\end{equation}
Conversely, fix a state $\mathscr{E}$. Let
\begin{equation}
\boldsymbol{A}\;\in\;\Omega^{0,1}\big(\mathbf{MT}_{s};\,\mathrm{End}_{\mathbf{C}}(E)\big),
\end{equation}
and let $\hbar$ be a (formal) deformation parameter. We then define
\begin{equation}
\overline{\partial}^{\mathscr{E}(\hbar)}\;=\;\overline{\partial}^{\mathscr{E}}+\hbar\,\boldsymbol{A}\wedge_{\varepsilon}.
\end{equation}
This operator defines a new, deformed PHS. Physically, $\boldsymbol{A}$
represents the background gauge field being turned on.

The proofs are straightforward. First, define the difference operator
\begin{equation}
\mathcal{T}\coloneqq\overline{\partial}^{\mathscr{E}_{1}}\!-\overline{\partial}^{\mathscr{E}_{2}}.
\end{equation}
For any smooth function $f$ and section $|\boldsymbol{\psi}\rangle$,
the Leibniz rule for each $\overline{\partial}^{\mathscr{E}_{i}}$
gives
\begin{equation}
\mathcal{T}(f\!\cdot\!|\boldsymbol{\psi}\rangle)=f\!\cdot\!\mathcal{T}|\boldsymbol{\psi}\rangle.
\end{equation}
Thus $\mathcal{T}$ is $\mathscr{C}^{\infty}$-linear. Hence it defines
an $\mathrm{End}_{\mathbf{C}}(E)$-valued $(0,1)$-form. 

Next, set:
\begin{equation}
\overline{\partial}^{\mathscr{E}(\hbar)}=\overline{\partial}^{\mathscr{E}}+\hbar\,\boldsymbol{A}\wedge_{\varepsilon}.
\end{equation}
We check the Leibniz rule on sections $|\boldsymbol{\psi}\rangle\in\Gamma(\mathbf{MT}_{s};E)$:
\begin{align}
\overline{\partial}^{\mathscr{E}(\hbar)}\!\big(f\!\cdot\!|\boldsymbol{\psi}\rangle\big)\; & =\;\overline{\partial}^{\mathscr{E}}\big(f\!\cdot\!|\boldsymbol{\psi}\rangle)\;+\;\hbar\,f\!\cdot\!(\boldsymbol{A}\wedge_{\varepsilon}|\boldsymbol{\psi}\rangle\big)\\
 & =\;\overline{\partial}f\!\otimes\!|\boldsymbol{\psi}\rangle\;+\;f\!\cdot\!\big(\overline{\partial}^{\mathscr{E}}|\boldsymbol{\psi}\rangle+\hbar\,\boldsymbol{A}\wedge_{\varepsilon}|\boldsymbol{\psi}\rangle\big)\\
 & =\;\overline{\partial}f\!\otimes\!|\boldsymbol{\psi}\rangle\;+\;f\!\cdot\overline{\partial}^{\mathscr{E}(\hbar)}|\boldsymbol{\psi}\rangle.
\end{align}
By our earlier argument, this operator extends uniquely to all $E$-valued
forms. \textbf{QED.}

The conceptual upshot is that the set of all pseudoholomorphic structures
on the bundle $E\to\mathbf{MT}_{s}$ forms an affine space modelled
on
\begin{equation}
\Omega^{0,1}\big(\mathbf{MT}_{s};\,\mathrm{End}_{\mathbf{C}}(E)\big).
\end{equation}
Physically, this means any semiclassical state can be reached from
any other by adding a suitable background field configuration $\boldsymbol{A}$.
Thus the space of semiclassical states is realised geometrically as
the moduli of deformations of the vacuum holomorphic structure. 

\subsection{Pseudocurvature and Fieldstrength}

The state of HGT is encoded by the geometry of a complex vector bundle
$E\to\mathbf{MT}_{s}$. The vacuum is the state with no field ``excitations.''
We say a nondegenerate classical vacuum exists when $E$ admits a
holomorphic structure. In that case, the vacuum corresponds to the
canonical holomorphic structure on $E$, defined by the Dolbeault
operator:
\begin{equation}
\overline{\partial}^{E}\colon\quad\Omega^{p,q}(\mathbf{MT}_{s};E)\;\longrightarrow\;\Omega^{p,q+1}(\mathbf{MT}_{s};E).
\end{equation}
Its integrability condition,
\begin{equation}
\overline{\partial}^{E}\!\circ\overline{\partial}^{E}\;=\;0,
\end{equation}
implies the fieldstrength tensor vanishes. Physically, this condition
signifies a particle-free state. In this case, we call the potential
$\boldsymbol{A}$ pure gauge. A key question is when such a vacuum
state exists.

In supersymmetric theories, BPS states play a central role. These
states correspond to a PHS $\mathscr{E}$ on the bundle $E$. This
structure comes with a partial connection $\overline{\partial}^{\mathscr{E}}$,
whose pseudocurvature
\begin{equation}
\boldsymbol{F}^{\mathscr{E}}\coloneqq\overline{\partial}^{\mathscr{E}}\!\circ\overline{\partial}^{\mathscr{E}}
\end{equation}
represents the \emph{fieldstrength tensor} of the theory.

\textcompwordmark{}

\paragraph*{Physical Motivation.}

To understand the physical meaning of identifying the fieldstrength
with the pseudocurvature of a deformed holomorphic structure $\mathscr{E}$,
we may contrast it to curvature in Einstein's gravity.

In pseudo-Riemannian geometry, curvature manifests as tidal accelerations.
If the Riemann tensor vanishes on a neighbourhood $U$, spacetime
is locally flat there. Physically, this flatness implies no tidal
acceleration between any two nearby freely falling bodies. Experimentally,
one tests this by establishing the existence of a local inertial frame.
Mathematically, one introduces frame fields $\boldsymbol{e}_{\;\;\mu}^{a}$
on $U$ that satisfy $\nabla_{\mu}\,\boldsymbol{e}_{\;\;\nu}^{a}=0$
under the Levi-Civita connection $\nabla$. 

The gauge-theoretic analogue of flatness follows the same pattern
as in gravity. The key geometric objects are replaced by their gauge
theory counterparts:
\begin{enumerate}
\item The spacetime worldvolume $U$ is replaced by a trivialising neighbourhood
$\mathscr{U}\subset\mathbf{MT}_{s}$. Over $\mathscr{U}$, the bundle
$E$ becomes locally trivial:
\begin{equation}
E\big|_{\mathscr{U}}\;\cong\;U\times\mathbf{C}^{r}.
\end{equation}
\item The Levi-Civita connection $\nabla$, which defines parallel transport,
is replaced by the twisted Dolbeault operator:
\begin{equation}
\overline{\partial}^{\mathscr{E}}\!=\overline{\partial}^{E}+\boldsymbol{A}.
\end{equation}
This operator measures how the canonical holomorphic structure $\overline{\partial}^{E}$
is deformed by the background gauge potential $\boldsymbol{A}$.
\item The inertial frame fields $\boldsymbol{e}_{\;\;\mu}^{a}$ correspond
to a local holomorphic frame $\boldsymbol{s}=\{\boldsymbol{s}_{i}\}_{i=1}^{r}$.
These are local holomorphic sections of $E$ over the patch $\mathscr{U}$
that trivialise the bundle.
\end{enumerate}
%
A nonzero pseudocurvature $\boldsymbol{F}^{\mathscr{E}}$ obstructs
the existence of a frame whose sections remain holomorphic under the
deformed structure. Conversely, if $\boldsymbol{F}^{\mathscr{E}}=0$,
then there exists a frame $\{\boldsymbol{s}_{i}\}_{i=1}^{r}$ on an
open set $\mathscr{U}\subset\mathbf{MT}_{s}$ satisfying:
\begin{equation}
\overline{\partial}^{\mathscr{E}}\!\boldsymbol{s}_{i}\big|_{\mathscr{U}}\;=\;\big(\overline{\partial}^{E}\!+\boldsymbol{A}\big)\boldsymbol{s}_{i}\big|_{\mathscr{U}}\;=\;0,\qquad\text{for }i=1,\dots,r.
\end{equation}
This condition is the direct analogue of setting up an inertial frame
in a region of vanishing gravitational field.

\textcompwordmark{}

\subsubsection{Holomorphic Frames}

To analyse the local physics of a gauge theory on a rank-$r$ complex
vector bundle $E$, we introduce a local frame field over an open
set $\mathscr{U}\subset\mathbf{MT}_{s}$. A local frame $\boldsymbol{s}$
is an ordered collection of $r$ sections,
\begin{equation}
\boldsymbol{s}=(\boldsymbol{s}_{1},\dots,\boldsymbol{s}_{r}),\qquad\boldsymbol{s}_{i}\in\Gamma\big(\mathscr{U};\,E\big|_{\mathscr{U}}\big).
\end{equation}
At each point $\mathsf{w}\in\mathscr{U}$, the vectors $\{\boldsymbol{s}_{i}(\mathsf{w})\}$
form a basis of the fibre $E\big|_{\mathsf{w}}$. 

Physically, choosing a frame is equivalent to selecting a local basis
for the internal degrees of freedom.

\textcompwordmark{}

\paragraph*{Holomorphicity.}

When does a gauge theory on the vector bundle $E$ admit a classical
vacuum? In our geometric formulation, a vacuum is a field configuration
with vanishing pseudocurvature. Hence $E$ admits a classical vacuum
when it admits a holomorphic structure.

A local criterion is the following:
\begin{lem}
The bundle $E\to\mathbf{MT}_{s}$ is holomorphic if and only if each
open patch $\mathscr{U}$ of a chosen trivialising cover admits a
local holomorphic frame.
\end{lem}

\begin{proof}
Assume the theory admits a nondegenerate classical vacuum. By definition,
this means that the bundle $E$ is holomorphic. Let $\mathscr{U}\subset\mathbf{MT}_{s}$
be a trivialising neighbourhood. Then there exists a holomorphic trivialisation
map
\begin{equation}
\tau_{\mathscr{U}}\colon\;\pi^{-1}(\mathscr{U})\;\longrightarrow\;\mathscr{U}\times\mathbf{C}^{r}.
\end{equation}
Choose the standard basis $\{\boldsymbol{e}_{i}\}_{i=1}^{r}$ of the
typical fibre $\mathbf{C}^{r}$. Pull this basis back by the inverse
trivialisation $\tau_{\mathscr{U}}^{-1}$. For each $\mathsf{w}\in\mathscr{U}$
define the local sections:
\begin{equation}
\boldsymbol{s}_{i}(\mathsf{w})\coloneqq\tau_{\mathscr{U}}^{-1}(\mathsf{w},\boldsymbol{e}_{i}).
\end{equation}
The differential of $\tau_{\mathscr{U}}$ is an isomorphism on tangent
spaces. Hence $\{\boldsymbol{s}_{i}(\mathsf{w})\}$ forms a basis
of the fibre $E\big|_{\mathsf{w}}$ at every $\mathsf{w}\in\mathscr{U}$.
Since $\tau_{\mathscr{U}}$ is biholomorphic, each map $\mathsf{w}\mapsto\boldsymbol{s}_{i}(\mathsf{w})$
is holomorphic. Thus the sections $\{\boldsymbol{s}_{i}\}$ provide
a local holomorphic frame on $\mathscr{U}$.

To prove the converse, we examine the consistency conditions satisfied
by the transition functions that glue local patches. Let $\mathscr{U}$
and $\mathscr{U}'$ be two overlapping trivialising neighbourhoods.
Let $\boldsymbol{s}=(\boldsymbol{s}_{i})$ and $\boldsymbol{s}'=(\boldsymbol{s}_{i}')$
be local frames on $\mathscr{U}$ and $\mathscr{U}'$, respectively.
Assume each frame is holomorphic with respect to the partial connection
$\overline{\partial}^{\mathscr{E}}$. In physical terms, on each patch
we choose a basis of field sections with no antiholomorphic dependence
relative to the background complex structure.

On the overlap $\mathscr{U}\cap\mathscr{U}'$ the frames are related
by a smooth, invertible transition map:
\begin{equation}
g_{i}^{\;\;j}\colon\;\mathscr{U}\cap\mathscr{U}'\;\longrightarrow\;GL(r,\mathbf{C}),\qquad\boldsymbol{s}_{i}'=g_{i}^{\;\;j}\boldsymbol{s}_{j}.
\end{equation}
Apply the Leibniz rule for $\overline{\partial}^{\mathscr{E}}$ on
$\mathscr{U}\cap\mathscr{U}'$. This gives
\begin{equation}
\overline{\partial}^{\mathscr{E}}\!\boldsymbol{s}'_{i}\;=\;\overline{\partial}g_{i}^{\;\;j}\!\otimes\!\boldsymbol{s}_{j}\;+\;g_{i}^{\;\;j}\,\overline{\partial}^{\mathscr{E}}\!\boldsymbol{s}_{j}.
\end{equation}
By hypothesis $\overline{\partial}^{\mathscr{E}}\!\boldsymbol{s}_{i}=0$
and $\overline{\partial}^{\mathscr{E}}\!\boldsymbol{s}_{i}'=0$. Hence
$\overline{\partial}g_{i}^{\;\;j}=0$ on $\mathscr{U}\cap\mathscr{U}'$.
Thus each transition function $g_{i}^{\;\;j}$ is holomorphic on the
overlap.

Holomorphic transition functions give a holomorphic atlas for $E$.
By definition, this equips the bundle $E\to\mathbf{MT}_{s}$ with
a global holomorphic structure. Therefore $E$ is a holomorphic vector
bundle.
\end{proof}

\subsubsection{Existence of a Classical Vacuum}

We are now in a position to state an important physical observation.
For a holomorphic gauge theory defined on a complex vector bundle
$E\to\mathbf{MT}_{s}$, the bundle is holomorphic if and only if there
exists a partial connection $\overline{\partial}^{\mathscr{E}}$ on
$E$ whose pseudocurvature, $\boldsymbol{F}\coloneqq\overline{\partial}^{\mathscr{E}}\!\circ\overline{\partial}^{\mathscr{E}}$,
vanishes. In more physical terms, this asserts that the gauge theory
admits a classical vacuum if and only if one can find a gauge potential
for which the associated field-strength tensor is zero everywhere.

To understand why this must be true, we first consider the case where
$E$ is holomorphic. We can simply choose our pseudoholomorphic structure
$\mathscr{E}$ to be the canonical holomorphic structure that is naturally
induced by the Dolbeault operator on $E$. In this case, $\overline{\partial}^{\mathscr{E}}\!=\overline{\partial}^{E}$,
and the integrability condition $\overline{\partial}^{\mathscr{E}}\!\circ\overline{\partial}^{\mathscr{E}}=0$
holds by definition, guaranteeing a vanishing field strength.

For the more involved converse, let us assume we have a pseudoholomorphic
structure $\mathscr{E}$ on our vector bundle whose partial connection
$\overline{\partial}^{\mathscr{E}}$ is integrable, meaning its pseudocurvature
vanishes. Our goal is to demonstrate that this implies $\pi\colon E\to\mathbf{MT}_{s}$
is holomorphic. To do this, we will explicitly construct a local holomorphic
frame on an arbitrary trivialising neighbourhood $\mathscr{U}\subset\mathbf{MT}_{s}$
and then invoke our previous result. Let $\boldsymbol{s}\coloneqq(\boldsymbol{s}_{1},\dots,\boldsymbol{s}_{r})$
be any local frame on $\mathscr{U}$; we do not assume this frame
is holomorphic, so in general $\overline{\partial}^{\mathscr{E}}\!\boldsymbol{s}_{i}\neq0$.

The key observation is that for each basis section $\boldsymbol{s}_{i}$,
the quantity $\overline{\partial}^{\mathscr{E}}\boldsymbol{s}_{i}$
is a section of the bundle of $(0,1)$-forms on $\mathscr{U}$ with
values in $E$; that is, $\overline{\partial}^{\mathscr{E}}\!\boldsymbol{s}_{i}\in\Omega^{0,1}(\mathscr{U};E|_{\mathscr{U}})$.
Since the set $\{\boldsymbol{s}_{1},\dots,\boldsymbol{s}_{r}\}$ is
complete, we can expand the differential $\overline{\partial}^{\mathscr{E}}\!\boldsymbol{s}_{i}$
in terms of the frame itself, which means there must exist a set of
complex-valued differential $(0,1)$-forms, $\alpha_{i}^{\;j}\in\Omega^{0,1}(\mathscr{U})$,
such that $\overline{\partial}^{\mathscr{E}}\!\boldsymbol{s}_{i}=\alpha_{i}^{\;j}\otimes\boldsymbol{s}_{j}$.
Now, we apply the operator $\overline{\partial}^{\mathscr{E}}$ again
and use the Leibniz rule:
\begin{equation}
(\overline{\partial}^{\mathscr{E}}\!\circ\overline{\partial}^{\mathscr{E}})\boldsymbol{s}_{i}\;=\;\big(\overline{\partial}\alpha_{i}^{\;j}-\alpha_{i}^{\;k}\wedge\alpha_{k}^{\;j}\big)\otimes\boldsymbol{s}_{j}.
\end{equation}
Our initial assumption was that the partial connection is ``flat,''
meaning the field-strength vanishes, so the left-hand side is zero.
This forces the term in the parenthesis to be zero as well:
\begin{equation}
\overline{\partial}\alpha_{i}^{\;j}\;=\;\alpha_{i}^{\;k}\wedge\alpha_{k}^{\;j}.
\end{equation}
This is precisely the Maurer-Cartan equation. Therefore, the Frobenius
integrability condition guarantees the existence of a set of complex-valued
differentiable functions $f_{i}^{\;j}$ on some open subset $\mathscr{U}'\subseteq\mathscr{U}$
that satisfy $f^{ik}f_{jk}=\delta_{\;j}^{i}$ and, most importantly,
solve the differential equation $\alpha_{i}^{\;j}+f_{i}^{\;k}\overline{\partial}f_{k}^{\;j}=0$,
where we use the Kronecker delta $\delta_{\;j}^{i}$ to lower and
raise the internal indices $i,j,\dots$

Finally, we can use these functions to define a new frame field $\boldsymbol{s}'_{i}=f_{i}^{\;j}\boldsymbol{s}_{j}$
on the sub-patch $\mathscr{U}'$. A straightforward calculation shows
that this new frame is indeed holomorphic:
\begin{equation}
\overline{\partial}^{\mathscr{E}}\!\boldsymbol{s}'_{i}\;=\;\big(\overline{\partial}f_{i}^{\;k}+f_{i}^{\;j}\alpha_{j}^{\;k}\big)\otimes\boldsymbol{s}_{k}.
\end{equation}
Contracting this expression with $f_{\ell}^{\;i}$ yields:
\begin{equation}
f_{\ell}^{\;i}\overline{\partial}^{\mathscr{E}}\boldsymbol{s}'_{i}\;=\;\big(\alpha_{\ell}^{\;k}+f_{\ell}^{\;i}\overline{\partial}f_{i}^{\;k}\big)\otimes\boldsymbol{s}_{k}\;=\;0.
\end{equation}
This vanishes because of the way we defined the functions $f_{i}^{\;j}$
as solutions to the Frobenius integrability condition. We have thus
successfully constructed a local holomorphic frame, $\boldsymbol{s}'_{i}$.
Since we can do this for any trivialising neighbourhood, our preceding
result implies that the bundle $E$ must be holomorphic, which completes
the argument.

This theorem provides the rigorous justification for our initial definition,
identifying the canonical holomorphic structure of the vector bundle
with the classical vacuum of the gauge theory. The key insight is
that, for a generic pseudoholomorphic connection $\overline{\partial}^{\mathscr{E}}$,
there might not be a full basis of solutions to the equation $\overline{\partial}^{\mathscr{E}}\!\boldsymbol{s}=0$.
The integrability condition, $\overline{\partial}^{\mathscr{E}}\!\circ\overline{\partial}^{\mathscr{E}}=0$,
which physically signifies the vanishing of the field-strength tensor,
$\boldsymbol{F}=0$, is the necessary and sufficient condition for
the existence of a maximal number of independent, locally holomorphic
sections, which in turn defines the nondegenerate classical vacuum
of the theory.

\subsubsection{Properties of Pseudocurvature}

Let us begin the final part of our review by recapitulating the essential
features of the formalism developed in the preceding subsections.
We have formulated a holomorphic gauge theory on minitwistor superspace
as a theory of pseudoholomorphic structures, $\mathscr{E}$, on a
complex vector bundle $E$. In this setting, each physical state of
the theory corresponds to a particular choice of structure $\mathscr{E}$.
The associated partial connection, $\overline{\partial}^{\mathscr{E}}$,
is identified with the gauge potential, and its field-strength tensor
is given by the pseudocurvature, a $(0,2)$-form defined as $\boldsymbol{F}\coloneqq\overline{\partial}^{\mathscr{E}}\!\circ\overline{\partial}^{\mathscr{E}}$.

We then established that such a theory admits a nondegenerate classical
vacuum if and only if the underlying vector bundle, $\pi\colon E\to\mathbf{MT}_{s}$,
is holomorphic. The vacuum state was identified with the canonical
holomorphic structure $\overline{\partial}^{E}$ that exists on any
such bundle. The consistency of this framework was confirmed by demonstrating
that a configuration of vanishing field strength ($\boldsymbol{F}=0$)
exists precisely when the bundle $E$ is holomorphic, which aligns
with the physical requirement of a vanishing field strength in vacuum.

While this geometric formulation is elegant, it is instructive to
connect it explicitly to the familiar language of conventional gauge
theory. To this end, we now derive some properties of the field-strength
tensor (e.g., the Bianchi identity) directly from the definition of
the pseudocurvature. This exercise will serve to solidify the identification
of $\boldsymbol{F}$ as the field-strength tensor of the theory and
demonstrate the utility of the formalism. 

\textcompwordmark{}

\paragraph*{Smooth Linearity.}

Our first task is to clarify the analytic nature of the pseudocurvature
$\boldsymbol{F}$, which we have identified with the field-strength
tensor. A genuine classical field-strength should act locally on matter
fields; that is, its value at a point $\mathsf{w}\in\mathbf{MT}_{s}$
should depend only on the value of the matter field at $\mathsf{w}$.
Mathematically, this property is captured by the condition of $\mathscr{C}^{\infty}$-linearity.

Thus, we now show that the pseudocurvature $\boldsymbol{F}$ is indeed
a $\mathscr{C}^{\infty}(\mathbf{MT}_{s})$-linear operator. This means
that $\boldsymbol{F}$ can be regarded as an \emph{operator-valued
differential form} that acts pointwise on the fibres of the vector
bundle $E$, 
\begin{equation}
\boldsymbol{F}\;\in\;\Omega^{0,2}\big(\mathbf{MT}_{s};\,\mathrm{End}_{\mathbf{C}}(E)\big).
\end{equation}
To see why this must be true, we examine how $\boldsymbol{F}$ acts
on a matter field $|\phi\rangle\in\Gamma(\mathbf{MT}_{s};E)$ that
has been multiplied by an arbitrary complex-valued smooth function
$f\in\mathscr{C}^{\infty}(\mathbf{MT}_{s})$. Applying the definition
of the pseudocurvature and repeatedly using the Leibniz rule for the
partial connection $\overline{\partial}^{\mathscr{E}}$, we find:
\begin{equation}
\boldsymbol{F}\big(f\!\cdot\!|\phi\rangle\big)\;=\;\overline{\partial}^{\mathscr{E}}\big(\overline{\partial}^{\mathscr{E}}(f\!\cdot\!|\phi\rangle)\big)\;=\;\overline{\partial}^{\mathscr{E}}\big(\overline{\partial}f\otimes|\phi\rangle\big)+\overline{\partial}^{\mathscr{E}}\big(f\!\cdot\!\overline{\partial}^{\mathscr{E}}|\phi\rangle\big)\;=\;f\!\cdot\!\boldsymbol{F}|\phi\rangle.
\end{equation}
This result confirms that $\boldsymbol{F}$ is indeed $\mathscr{C}^{\infty}$-linear.
Physically, this is the precise condition ensuring that the field-strength
acts \emph{locally}, as a well-behaved classical observable.

This pointwise, linear action on the fibres motivates a convenient
and physically suggestive notation. For any operator-valued form $\boldsymbol{T}\in\Omega^{r,s}\big(\mathbf{MT}_{s};\mathrm{End}_{\mathbf{C}}(E)\big)$
and any matter field $|\phi\rangle\in\Gamma(\mathbf{MT}_{s};E)$,
we will denote their action as
\begin{equation}
\boldsymbol{T}|\phi\rangle\;\coloneqq\;\boldsymbol{T}\wedge_{\varepsilon}|\phi\rangle.\label{eq:-159}
\end{equation}
This notation is deliberately reminiscent of the action of an observable
on a state-vector in quantum mechanics, reinforcing the picture of
$\boldsymbol{F}$ as the field-strength observable acting on a matter
field.

\textcompwordmark{}

\paragraph*{Structure Equation.}

A central question in any gauge theory is to understand how the field
content of the theory changes when the system is perturbed. Let us
consider a holomorphic gauge theory formulated on a complex vector
bundle $E$ over minitwistor superspace. The states of this theory,
$\mathscr{E}$, are described by pseudoholomorphic structures, which
are specified by a partial connection $\overline{\partial}^{\mathscr{E}}$
and an associated pseudocurvature $(0,2)$-form $\boldsymbol{F}^{\mathscr{E}}$.
Suppose we begin with a known state $\mathscr{E}_{1}$ and perturb
it to a new state $\mathscr{E}_{2}$ by adiabatically turning on a
background gauge potential, which we represent by an endomorphism-valued
$(0,1)$-form $\boldsymbol{A}\in\Omega^{0,1}\big(\mathbf{MT}_{s};\,\mathrm{End}_{\mathbf{C}}(E)\big)$.
How is the field-strength $\boldsymbol{F}^{\mathscr{E}_{2}}$ of the
new state related to the original field-strength $\boldsymbol{F}^{\mathscr{E}_{1}}$
and the perturbing field $\boldsymbol{A}$? It turns out that the
new field-strength is completely determined by the initial state and
the perturbation via the well-known decomposition:
\begin{equation}
\boldsymbol{F}^{\mathscr{E}_{2}}\;=\;\boldsymbol{F}^{\mathscr{E}_{1}}+\overline{\partial}^{\mathscr{E}_{1}}\!\boldsymbol{A}+\boldsymbol{A}\wedge_{\varepsilon}\boldsymbol{A}.\label{eq:-190}
\end{equation}

To see why this relation holds, we begin by recalling that the set
of all pseudoholomorphic structures on the bundle $E$ forms an affine
space over  the infinite-dimensional vector space $\Omega^{0,1}\big(\mathbf{MT}_{s};\,\mathrm{End}_{\mathbf{C}}(E)\big)$.
This affine structure implies that the difference between any two
partial connections, $\overline{\partial}^{\mathscr{E}_{2}}\!-\overline{\partial}^{\mathscr{E}_{1}}$,
is a $\mathscr{C}^{\infty}$-linear operator. As such, it can be identified
with the endomorphism-valued $(0,1)$-form $\boldsymbol{A}$ that
we have interpreted physically as the background gauge potential.
This allows us to write the ``perturbed'' partial connection directly
in terms of the original one,
\begin{equation}
\overline{\partial}^{\mathscr{E}_{2}}\!=\overline{\partial}^{\mathscr{E}_{1}}\!+\boldsymbol{A}.\label{eq:-183}
\end{equation}
Physically, this equation states that the new gauge derivative is
simply the old one plus a term accounting for the newly introduced
background field.

To find the corresponding field strength, we examine its action on
an arbitrary matter field, which we represent as a section $|\phi\rangle\in\Gamma(\mathbf{MT}_{s};E)$.
A direct computation using Eq. (\ref{eq:-183}), the $\mathbf{C}$-linearity
of $\overline{\partial}^{\mathscr{E}_{2}}$ and its graded Leibniz
rule yields:
\begin{align}
\boldsymbol{F}^{\mathscr{E}_{2}}\!|\phi\rangle & \;=\;\overline{\partial}^{\mathscr{E}_{2}}\!\big(\overline{\partial}^{\mathscr{E}_{2}}\!|\phi\rangle\big)\;=\;\overline{\partial}^{\mathscr{E}_{2}}\!\big(\overline{\partial}^{\mathscr{E}_{1}}\!|\phi\rangle\big)\,+\,\overline{\partial}^{\mathscr{E}_{2}}\big(\boldsymbol{A}|\phi\rangle\big)\\
 & \;=\;\overline{\partial}^{\mathscr{E}_{1}}\big(\overline{\partial}^{\mathscr{E}_{1}}|\phi\rangle\big)+\boldsymbol{A}\big(\overline{\partial}^{\mathscr{E}_{1}}\!|\phi\rangle\big)+\overline{\partial}^{\mathscr{E}_{1}}\!\big(\boldsymbol{A}|\phi\rangle\big)+\boldsymbol{A}\wedge_{\varepsilon}\big(\boldsymbol{A}\wedge_{\varepsilon}|\phi\rangle\big)\label{eq:-189}\\
 & \;=\;\big(\boldsymbol{F}^{\mathscr{E}_{1}}+\overline{\partial}^{\mathscr{E}_{1}}\boldsymbol{A}+\boldsymbol{A}\wedge_{\varepsilon}\boldsymbol{A}\big)|\phi\rangle.\label{eq:-184}
\end{align}
A brief technical remark is in order regarding the derivation of the
final line. In manipulating expressions involving both differential
forms and sections, we adopted a quantum mechanics-like notation (refer
to Eq. (\ref{eq:-159})) where the wedge product, $\wedge_{\varepsilon}$,
is implicitly understood. However, one must be careful to respect
the graded nature of the exterior algebra. For example, the term $\overline{\partial}^{\mathscr{E}_{1}}\!(\boldsymbol{A}|\phi\rangle)$
is shorthand for $\overline{\partial}^{\mathscr{E}_{1}}(\boldsymbol{A}\wedge_{\varepsilon}|\phi\rangle)=\overline{\partial}^{\mathscr{E}_{1}}\!\boldsymbol{A}\wedge_{\varepsilon}|\phi\rangle-\boldsymbol{A}\wedge_{\varepsilon}\overline{\partial}^{\mathscr{E}_{1}}\!|\phi\rangle$.
Rearranging this gives the operator relation:
\begin{equation}
\big(\overline{\partial}^{\mathscr{E}_{1}}\!\boldsymbol{A}\big)|\phi\rangle=\overline{\partial}^{\mathscr{E}_{1}}\!\big(\boldsymbol{A}|\phi\rangle\big)+\boldsymbol{A}\big(\overline{\partial}^{\mathscr{E}_{1}}\!|\phi\rangle\big),
\end{equation}
which was used in Eq. (\ref{eq:-189}). Since the final expression,
Eq. (\ref{eq:-184}), holds for an arbitrary matter field $|\phi\rangle$,
we can abstract it to the operator equation given in Eq. (\ref{eq:-190}).

An immediate consequence arises when we apply this result to a \emph{supersymmetric}
minitwistor gauge theory. Such theories admit a natural, nondegenerate
classical vacuum state, which corresponds to the canonical holomorphic
structure on the vector superbundle $E\to\mathbf{MT}_{s}$ induced
by the standard Dolbeault operator, $\overline{\partial}^{E}$. In
this vacuum state, the field-strength is, by definition, zero: $\boldsymbol{F}=\overline{\partial}^{E}\!\circ\overline{\partial}^{E}=0$.
Let us now consider a semiclassical BPS-state $\mathscr{E}$ that
arises from adiabatically perturbing this vacuum by turning on a background
gauge potential $\boldsymbol{A}$. By specialising Eq. (\ref{eq:-190})
to this case (setting $\mathscr{E}_{1}=E$ and $\mathscr{E}_{2}=\mathscr{E}$)
, we obtain the \emph{structure equation} for the field-strength tensor:
\begin{equation}
\boldsymbol{F}^{\mathscr{E}}\;=\;\overline{\partial}^{E}\!\boldsymbol{A}\,+\,\boldsymbol{A}\wedge_{\varepsilon}\boldsymbol{A}.
\end{equation}
This result is the analogue of the familiar equation $F=dA+A\wedge A$
in standard YM theory, adapted to the holomorphic setting of minitwistor
superspace.

\textcompwordmark{}

\paragraph*{Bianchi Identity.}

Having established how the field-strength tensor transforms under
perturbations, we now ask if there are any intrinsic constraints it
must satisfy. Indeed, the field-strength $\boldsymbol{F}$ of any
pseudoholomorphic structure $\mathscr{E}$ on $E$ must obey a consistency
condition known as the Bianchi (differential) identity. To formulate
this, we first note that the partial connection $\overline{\partial}^{\mathscr{E}}$
on $E$ naturally induces a pseudoholomorphic structure on the associated
vector bundle of endomorphisms, $\mathrm{End}_{\mathbf{C}}(E)\to\mathbf{MT}_{s}$.
This allows us to act with $\overline{\partial}^{\mathscr{E}}$ on
$\mathrm{End}_{\mathbf{C}}(E)$-valued differential forms, such as
the pseudocurvature $\boldsymbol{F}$ itself,
\begin{equation}
\overline{\partial}^{\mathscr{E}}\!\boldsymbol{F}\;\in\;\Omega^{0,3}\big(\mathbf{MT}_{s};\,\mathrm{End}_{\mathbf{C}}(E)\big).
\end{equation}
The Bianchi identity is the statement that this action yields identically
zero,
\begin{equation}
\overline{\partial}^{\mathscr{E}}\!\boldsymbol{F}\;=\;0.\label{eq:-192}
\end{equation}

In our geometric framework, the argument for this identity is quite
simple. We again consider the action of the differential operators
$\overline{\partial}^{\mathscr{E}}$ and $\boldsymbol{F}$ on an arbitrary
matter field $|\phi\rangle\in\Gamma(\mathbf{MT}_{s};E)$. Using the
graded Leibniz rule, we find:
\begin{equation}
\overline{\partial}^{\mathscr{E}}\!\big(\boldsymbol{F}|\phi\rangle\big)\;=\;\big(\overline{\partial}^{\mathscr{E}}\boldsymbol{F}\big)|\phi\rangle+\boldsymbol{F}\big(\overline{\partial}^{\mathscr{E}}\!|\phi\rangle\big),\label{eq:-185}
\end{equation}
where, as before, we use a compact notation that implicitly includes
the necessary anti-symmetrisation. On the other hand, we can evaluate
the left-hand side directly using the definition of the field-strength
and the associativity of operator composition:
\begin{equation}
\overline{\partial}^{\mathscr{E}}\!\big(\boldsymbol{F}|\phi\rangle\big)\;=\;\overline{\partial}^{\mathscr{E}}\!\big((\overline{\partial}^{\mathscr{E}}\!\circ\overline{\partial}^{\mathscr{E}})|\phi\rangle\big)\;=\;\big(\overline{\partial}^{\mathscr{E}}\!\circ\overline{\partial}^{\mathscr{E}}\big)\overline{\partial}^{\mathscr{E}}\!|\phi\rangle\;=\;\boldsymbol{F}\big(\overline{\partial}^{\mathscr{E}}\!|\phi\rangle\big).\label{eq:-191}
\end{equation}
Comparing Eqs. (\ref{eq:-185}) and (\ref{eq:-191}), we see that
$(\overline{\partial}^{\mathscr{E}}\boldsymbol{F})|\phi\rangle=0$.
Since this holds for any matter field $|\phi\rangle$, the operator
$\overline{\partial}^{\mathscr{E}}\!\mathbf{F}$ must be zero, which
is the Bianchi identity given in Eq. (\ref{eq:-192}).

For the state of a theory defined by a perturbation $\boldsymbol{A}$
around a trivial vacuum, with partial connection $\overline{\partial}^{E}$,
it is often more useful to express the Bianchi identity in terms of
these fundamental fields. By substituting $\overline{\partial}^{\mathscr{E}}\!=\overline{\partial}^{E}\!+\boldsymbol{A}$
into Eq. (\ref{eq:-192}), we arrive at an equivalent form of the
Bianchi identity:
\begin{equation}
\overline{\partial}^{E}\!\boldsymbol{F}\,+\,\boldsymbol{A}\wedge_{\varepsilon}\boldsymbol{F}\;=\;0.\label{eq:-187}
\end{equation}
From a physical standpoint, the Bianchi identity plays the role of
a local conservation law. This is analogous to other areas of field
theory. In gravitation, for example, the contracted Bianchi identity
on the Einstein tensor, $\nabla_{\mu}\,G^{\mu\nu}=0$, leads directly
to the statement of local energy-momentum conservation, $\nabla_{\mu}\,T^{\mu\nu}=0$,
through the Einstein field equation. Similarly, in the context of
holomorphic gauge theory, Eq. (\ref{eq:-187}) can be cast in the
form of a conservation law, $\overline{\partial}^{E}\boldsymbol{F}=\star\boldsymbol{J}$,
where the $\mathrm{End}_{\mathbf{C}}(E)$-valued $(0,3)$-form $\star\boldsymbol{J}\coloneqq-\boldsymbol{A}\wedge_{\varepsilon}\boldsymbol{F}$
can be interpreted as the conserved ``charge'' current density. This
current arises, via Noether's theorem, from the underlying gauge symmetry
of the theory, that is, the invariance of the physics under changes
of local trivialisation of the vector bundle $E$.

\bibliographystyle{../TTCFT/revtex-tds/bibtex/bst/revtex/aipnum4-2}
\bibliography{CCFT2}

\end{document}
